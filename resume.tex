\pdfbookmark[0]{resume}{resume}
    \thispagestyle{empty}
\chapter*{}
\markboth{resume-Abstract}{}
\section*{resume}
essai1


\subsubsection*{Mots-clefs}
toto



  \vspace{0.5cm}



 \newpage

  \section*{Abstract}

Ecological modelling is nowadays a leading topic. the models used in environmental science turn to be more and more complex. Driven by technological advancement, the models used in environmental sciences are increasingly complex. The complexity of models is a need for a variety of studies, but it can also be a source of misunderstanding, or even errors. How to manage its implementation is  therefore necessary. With this in mind, two complementary approaches have been studied in this thesis.

On the one hand, it is possible to manage the complexity \ textit {a priori},  by directly constraining the construction assumptions and formalism of the model using a theoretical framework. An illustration of the use of a theoretical framework, the theory of Dynamic Energy Budgets, shows how an accurate description of the effect of ultraviolet was added to a model of scleractinian corals. This study enlightened their possible role in coral bleaching events.

  Managing complexity can also be carried out \ textit {a posteriori}, ie once the construction phase is done. Thus, a simplification methodology using a statistical analysis of the model outputs was established . As an example, this method was applied on a micro-scale model of the mesopelagic ecosystem.

Eventually, not being able of pursuing an analytical approach of the model is not inevitable for those who want to still mastering their model, it exists a multitude of tool who brings equally interesting informations. Indeed, such a control may be induced from both the use of a theory to build the model and a statistical study of its functioning of

No longer have the opportunity to undertake an analytical approach to the model is not inevitable for those who want a master tool for a multitude of approaches provide information equally interesting. Indeed, this control may be induced by stress from a theory of the system and is modeled by a statistical study of the functioning of the model.



  \subsection*{Keywords} Complexity, Simplification, Dynamic Energy Budget Theory, Coral, Mesopelagic.

\thispagestyle{empty}
