\chapter[Complementary data]{Complementary data
\chaptermark{Complementary data}}
\chaptermark{Complementary data}
\minitoc

\section{Bacterial strain from ANTARES station}
Conjointly to the isolation of \textit{Photobacterium phosphoreum} ANT-2200, a second bacterial strain was isolated. It has been described as \textit{Acinetobacter lwoffii} (Kingdom:Bacteria, Phylum:Proteobacteria, Class: Gammaproteobacteria, Order:Pseudomonadales, Family:	Moraxellaceae). \\
In Chapter 4, describing the diversity in samples from March and October 2011, \textit{Acinetobacter} have been detected up to 1.9 and 2.3\% respectively. \\  

\begin{figure}[!h]
\begin{small}
\begin{center}
\includegraphics[scale=0.5]{contam.pdf}
\caption{Phylogenetic tree for \textit{Acinetobacter lwoffii} isolated at the ANTARES station. }
\label{}
\end{center}
\end{small}
\end{figure} 

\section{Oxygen time series}

Oxygen time series has been analyzed in a similar way as bioluminescence, temperature, salinity and current speed time series. However, due to a pronounced trend in the data with no clear explanation, as well as no pertinent results this variable has been taken off this study. Figure \ref{oxyIMF} represents the Hilbert-Huang decomposition filtered into 10 IMFs and Figure \ref{oxyHHG} is the spectrogram crossing both oxygen and bioluminescence Hilbert-Huang frequency decomposition.\\

\begin{figure}[!h]
\linespread{1} 
\centering
\includegraphics[width=15cm]{oxyIMF.jpeg}
\caption[Hilbert-Huang decomposition for the oxygen signal into 10 Intrinsic Mode Functions.]{Hilbert-Huang decomposition for the oxygen signal into 10 intrinsic mode functions, from C1 to C10. These Intrinsic Mode Functions are first order stationary. }
\label{oxyIMF}
\end{figure}

\begin{figure}[!h]
\linespread{1} 
\centering
\includegraphics[width=12cm]{oxy_hhg.pdf}
\caption[Cross spectrogram between bioluminescence and oxygen.]{Cross spectrogram between bioluminescence and oxygen. The cross correlation coefficients (represented between -0.2 and 0.8) are plotted as isolines and color scale for bioluminescence with oxygen. X and y-axis represent frequencies for each variable.}
\label{oxyHHG}
\end{figure}

%\section{communiqu�s de presse}
%\begin{figure}[!h]
%\linespread{1} 
%\centering
%\includegraphics[width=15cm]{compress.pdf}
%\caption[]{}
%\label{compress}
%\end{figure} 
