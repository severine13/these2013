\chapter{Conclusions and Perspectives }
\minitoc
\chaptermark{}{}
%\label{concl}
\newpage
\newpage
\section{Conclusions}

The aim of this thesis was to investigate if bioluminescence can be defined as a proxy of biological activity in the deep sea. To reach this goal, we follow two main axes\string: \\ 

\textbf{\textit{(i)} Bioluminescence is described as "weak" in the deep sea, compared to the ocean surface, but are there variations in light intensity over time and how to explain them?}\\

\textbf{\textit{(ii)} In the deep sea, what is the part of bacterial bioluminescence in the emission of light \textit{in situ}?}\\

The first axis has been mainly developed in Chapter 2 and 3, the second axis in Chapter 4 and 5. Moreover, in Chapter 1, (Introduction part) the interest of taking into account the different scales to defined processes has been highlighted. In this "Conclusions and Perspectives" section, results will be summarized, at various scales, into a more global view in order to answer to the general question.\\

\subsection{Is bioluminescence a proxy ?}

A proxy is an intermediate with the ability to mimic the behavior of something else. In this work, the bioluminescence was investigated as a proxy of biogeochemical processes, as well as a proxy of prokaryotic communities.\\

Photomultipliers have been used as detectors of 'natural' bioluminescence. In this work, the ability of such apparatus to record \textit{in situ} duration and intensity of bioluminescence events, with continuous, automated, and high-frequency data, has been validated \citep{tamburini2013}. Only few detectors dedicated to biological-variable recording exist, and such instrumentation is of importance to the global network of ocean observation. Moreover, the instrumented line, used at the ANTARES, station led to the conjoint record of physical variables, at high frequency and continuously. Discrete water samplings have also been performed over one-year survey for both physical and biological descriptions.\\

% Following the ergodicity therory, over time, it exists only one replicate of these processes, thereby, \\

At a regional scale, due to time dependence of those data, only one replicate (time series) is available to study those processes. Well-adapted methods were needed in order to analyze and interpret those multivariate time series. For such an aim, the Hilbert-Huang and wavelet methods have been used to interpret these time series. Our results demonstrated that the bioluminescence is, at all time and frequency, linked to current speed due to mechanical stimulation. Moreover, we highlighted that in peculiar conditions, the bioluminescence is strongly linked to changes in water-mass properties and this is mainly observed for the highest bioluminescence intensity. Such high bioluminescence events are mainly linked to current speed faster than 19 cm s$^{-1}$, temperature above 12.92 $\degres$ C and salinity higher than 38.479 and explained by deep convection flowing through the ANTARES observatory. During these events, water masses could carry to the site significant amounts of bioluminescent organisms and/or could fuel the deep sea with particulate and dissolved organic matters. These hydrological movements could induce, at the end, an increase in the bioluminescence activity. Moreover, those threshold values are of great interest for the prediction of events of high bioluminescence activity in the future. For astrophysicists, these results are also of interest to protect photomultipliers from too high light activity, that could damage these photon detectors. The survey of these water-mass changes requests a more global survey at the Mediterranean Sea level since the convection of newly formed water masses, described in the Gulf of Lion, will spread into the deep-sea and impact the ANTARES station few days later.\\

At a local scale, in Chapter 4, discrete water samplings have been performed every 2 months and over a one-year period in 2011, depending on the meteorological conditions. This sampling protocol was well adapted for the local description of the ANTARES site, as we observed only few variations in hydrological variables and prokaryotic communities. Moreover, the year 2011 was characterized by no deep convection reaching the seafloor (contrary to 2010) and represents a relatively "calm" period in term of bioluminescence activity, at the ANTARES station. Besides, during this survey, the sampled bioluminescence (integrated over 7 days for discrete information) and the bacterial abundance were too limited discrete observations to be correlated and therefore to interpret bioluminescence as a proxy. To reach this information, there is a need of higher frequency or rather longer survey, including a sampling during newly formed deep sea waters, to evaluate the increase in bioluminescent bacteria. Indeed, longer survey describing variations in bioluminescence intensity, as well as in bacterial abundances, will give more reliability on a possible correlation.\\

Finally, bioluminescence has been efficiently used as a proxy at the regional scale for hydrological processes. However, to validate the use of bioluminescence as a proxy of bacterial abundance at a local scale, longer investigations are needed. Consequently to these results, the observation scale (local or regional) was strongly dependent on the process to be described (water masses and bacterial communities). So is the use of bioluminescence as a proxy.\\

The ANTARES station, at first, was dedicated to astrophysics. Therefore, environmental sciences only have a minor weight within the ANTARES collaboration which sometimes leads to go against scientific decisions for oceanographic goals and to the loss of the necessary continuity of time series. This is a major limit for recording long time series. For example, the IL07 instrumented line has been shut down in October 2010 and it was not possible to re-immerse it before March 2013. Another major constraint is the huge amount of stored data to be analyzed. For discrete water samplings, major limits are the meteorology uncertainties as well as the low frequency records. The development of automatic water sampling and filtration would be a great improvement for scientific community. As an example, the deep-ESP (deep Environmental Sample Processor), developed at the MBARI (Monterey Bay Research Aquarium), is dedicated to automatic samplings that permit water filtration at relatively high frequency into bathypelagic environment. Moreover, this automated molecular biology laboratory can detect microorganisms using their genetic material. Such device can transmit informations in real time. For \textit{in situ} observatories, the sampling frequency of oceanic observations might be adapted, using real-time monitoring. This will lead to define oceanographic processes acting at various observation scales and to limit the quantity of data stored.\\

\subsection{Is bioluminescence linked to biological activity ?}

Bioluminescence is a reaction produced by both Eukaryotes and Bacteria. We investigated this phenomenon as a biological activity correlated with the physiological state of these organisms.\\

At first, from time series records at the regional scale, this work describes \textit{in situ} variations of bioluminescence over time. Three main events of high bioluminescence activity (in March 2009, December 2009 and March 2010) were detected. These bioluminescence events have been defined as intense compared to the global fluctuations recorded over time. Such high variations were unexpected. \\

Furthermore, looking at the local scale, in Chapter 5, prokaryotic survey over the year 2011 led to a first link of bioluminescence with the activity of bioluminescent bacteria. This was based on \textit{lux} genes quantification. In particular, this work demonstrates that bioluminescent bacteria (belonging to \textit{Photobacterium} genus) were active. This was observed even during a period of low bioluminescence at the ANTARES site.\\ 

In Chapter 4, at the microscale in the laboratory, bioluminescence has also been demonstrated to be involved in bacterial activity. The bacterial strain used as a model, \textit{Photobacterium phosphoreum} ANT-2200, has been isolated during an event of high bioluminescence activity in 2005. In \cite{martini2013}, we demonstrated that the bioluminescence of \textit{Photobacterium phosphoreum} ANT-2200 strain is influenced by environmental growth conditions (carbon availability, temperature, and pressure). Higher activity of bioluminescence has been recorded at high pressure \textit{vs.} atmospheric pressure, showing that it does not have an inhibitory effect on the bioluminescence activity. On the contrary, bioluminescent bacteria potentially involved in the bioluminescence activity are adapted to extreme conditions encountered in the deep sea. This shows that bioluminescence is an important physiological response for bacterial activity.\\

All these results allowed to validate that bioluminescence is associated to the physiological state of these organisms. However, questions remain concerning the type of organisms emitting light and recorded by photomultipliers. Indeed, photomultipliers are still inefficient to discriminate bacterial and eukaryotic bioluminescence. In this work, we focused on bioluminescent bacteria due to the lack of \textit{in situ} informations in the literature. The wide diversity of bioluminescent organisms involved in the bioluminescence activity detected by the ANTARES telescope, has to be further explored. Bathyphotometers have been developed to enhance turbulence and mechanical stimulation on eukaryotic bioluminescent organisms. Photomultipliers, detect \textit{a priori} all light emission. The combined use of both devices should allow to measure the global and the eukaryotic bioluminescence and, therefore, to estimate the prokaryotic part in the signal. An efficient protocol has to be developed given the different sampling strategies between those instruments. Comparison between records over time will probably lead to the description of organisms, as well as the ability and limits of those complementary methods.\\

Moreover, the use of adapted video cameras nearby a photomultiplier could give interesting information describing each bioluminescent organisms. The successive record of images using a camera with an active lighting, followed by red lighting (invisible for most of organisms) could permit to identify organisms. The photomultiplier will measure the bioluminescence intensity and shape of emission over time. These couples of characteristics could help to discriminate automatically bioluminescent organisms crossing the ANTARES observatory. Detection methods are widely developed and image processing easily automatized as shown by \cite{stemmann2008}, \cite{aguzzi2009} and the Eye-in-the-Sea project \underline{www.mbari.org/mars/science/eits.html}. By the end, using a network of about 885 photomultipliers will be a fascinating way to detect organisms crossing the entire 3D-telescope over space and time in the deep sea. Indeed, spatial analyzes can be performed using all photomultipliers within a total volume of 0.1 km$^3$ and would permit to estimate spatial dynamics of bioluminescent organisms through the observatory. \\ 

\subsection{Does bioluminescence play a major role in the deep sea ?}

About 95\% of the hydrosphere (deeper than 200-300 m) is characterized by its darkness and the deep sea (under 1,000 m depth) is also defined by high hydrostatic pressure. The interest of the question "Does bioluminescence play a major role in the deep sea ?" is to evaluate possible roles of bioluminescence in such environment and to investigate how high pressure would impact organisms in the deep sea.\\

At the regional scale, in Chapter 2, we demonstrated the input of newly formed deep waters from the surface to the deep, increasing, in an unexpected way, the bioluminescence activity (see \cite{tamburini2013} for details). Moreover, its has been observed that seasonal spring blooms of chlorophyll-\textit{a} occur at the surface of the ANTARES station. Such blooms are known to involve an increase in POC and DOC sinking into the deep sea. Moreover, as we described all over this work, bioluminescent bacteria can be attached to particles and therefore be transported to the deep. Such blooms can also involve bioluminescent eukaryotic organisms, sinking into the deep sea. In this work, first investigations do not permit to identify links between bioluminescence and surface seasonal spring blooms. However, in further work, using a suitable sampling survey, this phenomenon and its impact on bioluminescence activity and, by the end on bioluminescent organisms, has to be distinguished from newly formed deep-water-mass effects.\\

At the local scale, in Chapter 4, we have estimated between 0.1 to 1\%, the part of bioluminescent bacteria among the total bacterial community at 2,000 m depth. This percentage proved the presence of such bacteria at the deep ANTARES station and do not exclude them as a part of the bioluminescence signal. But this survey has been done out of intense-bioluminescence-activity periods and further investigation during these events are needed. Sinking particles are known to be hot spot of microbial activity in the dark ocean \citep{cho1988,turley2002,long2001,azam2001}. Indeed, prokaryotic abundance in particles is up to 3 order of magnitude higher than the abundance of free-living procaryotes in the same volume of sea water \citep{turley1995}. Focusing on sinking particles or marine snow for investigating \textit{lux} genes might be of interest for defining the role of bioluminescent bacteria into the carbon cycle, and in the deep sea.\\ 

At the microscale, in Chapter 3, bioluminescent bacterial physiology has been investigated in relation to temperature and pressure, as the two main characteristics of the deep Mediterranean Sea. Their effects have been analyzed on both growth rate and maximum population density for the bioluminescent bacterial strain \textit{Photobacterium phosphoreum} ANT-2200. This study defined an innovative way to describe bacterial optimal conditions for growth and permitted to characterized this strain as moderately piezophile. Moreover, we highlighted the increase in bioluminescence for this bacterial strain under high pressure conditions. Then, the first results also demonstrated that under high pressure conditions, this model strain allocated a better yield for oxygen consumption than under atmospheric pressure.\\ 

To conclude from the results of this thesis work, we investigated the bioluminescence as a proxy of biological activity using two sets of indicators: scales and levels. On the one hand, observation scales were ranging from micro to local and to regional. On the other hand, ecological levels were defined from population to communities and to ecosystem. We demonstrated that bioluminescence can be used as a proxy of biological activity for some levels (population and ecosystem) and some scales (micro and regional). For local scale and community level, we can not yet conclude and several perspective works have been proposed in this conclusion part.\\


\section{Perspectives}

A first perspective developed hereafter proposes to investigate the cultivation of the bacterial strain in a bioreactor, under controlled and monitored conditions. Using this method, physiological parameters described in Chapter 4 could be improved at atmospheric pressure before going back to high pressure system. Finally, a project concerning the implementation of video cameras on the ANTARES site and the concomitant analysis of data from photomultipliers is presented.\\

\subsection[Physiological parameters determination using bioreactor culture]{Physiological parameters determination using bioreactor culture
\sectionmark{Bioreactor culture}}
\sectionmark{Bioreactor culture}
\label{bioreactor}
\subsubsection*{Bioreactor platform}

The \textit{Photobacterium phosphoreum} ANT-2200 pre-culture is inoculated into a bioreactor fairmentec (total volume 2.2 L and working volume of 1.5 L) with an erthalite culture chamber (biologically and chemically neutral mater of white color and reflecting light). The bioreactor is autoclaved within 121\degres C for 20 min into an autoclave system Federagi FVG. The bioreactor is equipped with sensors controlling automatically the temperature (Julabo F25, France), the pH (InPro 3253, Mettler Toledo, Suisse), the redox-potential (InPro3253, Mettler Toledo, Suisse), and the dissolved oxygen (InPro6800, Mettler Toledo, Suisse). The transmitter for both pH and redox electrods is a Mettler Toledo PH2100e. The transmitter for oxygen is a Mettler Toledo O2 4100e. Sensors are calibrated before each experiment and a verification is done at the end, to validate measurements. On the outlet gas stream line, a CO$_2$ probe (Vaisala GMT 221, Finland) was fitted downstream from the glass exhaust water cooler to enable on-line measurement of CO$_2$ concentration. To control the gaz-mix for the oxygenation into the bioreactor, a debimeter (Bronkhrost, el flow, Netherlands) has been used. The debit injected was from 0 to 200 mL min$^{-1}$. An HPLC (shimadzu RID6A refractometer and colonne HPX87H) allow the control of glucose consumption throughout the experiment. The bioluminescence is recorded using an optical fiber (FVP600660710 Photon Lines) plugged to the bioreactor end cap and connected to a photomultiplier (H7155, Hammamatsu) linked to a counting box (C8855). Photons counting rate is recorded automatically on a computer, with an integrating rate of 10 s. In order to simplify the photon acquisition, experiments have been performed into dark room.\\

\begin{figure}[!h]
\linespread{1} 
\centering
\includegraphics[scale=0.5]{fermenteur.pdf}
\caption[The bioreactor platform with pH, O$_2$, temperature and CO$_2$ data acquisition and control in real time and high frequency sub-sampling.]{The bioreactor platform with pH, O$_2$, temperature and CO$_2$ data acquisition and control in real time and high frequency sub-sampling. Glycerol consumption and Optical Density (OD) are measured with culture sub-sampling over time. The culture room was completely dark during bioluminescence measurements.}
\label{fermenteur}
\end{figure} 

Bioreactor liquid volume and NaOH consumption,
regulating the pH, were monitored by two balances (respectively Combics 1 and BP 4100, Sartorius, France). Temperature, pH,
gas flow rates and stirrer speed were regulated through control units (local loops). All this equipment was connected to an automat (Wago, France) via either a serial link (RS232/RS485), a 4-20 mA analog loop, or a digital signal. The automat was connected to a computer and a BatchPro software (Decobecq Automatismes, France)
was used to monitor, acquire data and manage the processes with
good flexibility and accuracy. \\

\subsubsection*{Data acquisition into bioreactor}

The bioreactor is an optimal system for the bacterial cultivation under controlled conditions and to the record of variables simultaneously. Moreover such instrumentation is dedicated to the cultivation of a large volume of bacterial culture. In this experiment, \textit{Photobacterium phosphoreum} ANT-2200 has been cultivated for 70 h. Dissolved oxygen concentration was set at 100 \% of saturation (oxygen dissolved in the medium) to avoid limitation, temperature regulated at 13\degres C and pH at 7.5. During the growth, air input (O$_2$) (by bacterial consumption), glycerol consumption (by bacterial metabolism), bioluminescence emission, CO$_2$ (by bacterial respiration), and Optical Density (OD$_{600nm}$) (linked to bacterial biomass) are recorded (see Figure \ref{fermdata}). Contamination has been tested, from culture sub-sampling dedicated to the OD measurement, by microscopy observation. High OD$_{600nm}$ values have been recorded due to stirring conditions, bioreactor volume and air input.\\

\begin{figure}[!h]
\linespread{1} 
\centering
\includegraphics[scale=0.6]{fermenteur_data.pdf}
\caption[Data acquisition from the bioreactor. \textit{Photobacterium posphoreum} ANT-2200 strain has been cultivated in an ONR7a growth medium  under controlled conditions.]{Data acquisition from the bioreactor. \textit{Photobacterium posphoreum} ANT-2200 strain has been cultivated in an ONR7a growth medium  under controlled conditions. Oxygen concentration has been controlled to be non-limitant at 100\% of saturation (red dots), temperature is 13\degres C and pH is 7.5. CO$_2$ (blue dots), air input to the medium (green dots), bioluminescence activity (yellow dots) are recorded at high frequency and in real time. Culture has been sampled each 4 to 12 hours for both optical density (OD$_{600nm}$, black dots) and glycerol measurements (brown line). Black lines are the data continuous fit for each variable. The OD$_{600nm}$ has been plotted using a logistic model (see \citealp{martini2013} and \citealp{verhulst1838}). Grey area represents the bacterial growth duration.}
\label{fermdata}
\end{figure} 
The chemical equation for glycerol respiration is defined as :
\begin{eqnarray}
C_3H_8O_3 + 3.5 O_2 \longrightarrow 3 CO_2 + 4 H_2O + Biomasse
\end{eqnarray}
The glycerol concentration shows a limitation in carbon content occurring at 65 h of growth (reaching 0 g L$^{-1}$). This limitation involves the end of bacterial population growth with the end of exponential phase (OD$_{600nm}$ value about 5), CO$_2$ release and O$_2$ consumption, as well as a spontaneous stop of the bioluminescence activity.\\

Thanks to this continuous record and variable acquisition, oxygen consumption, bioluminescence emission, growth, and glycerol consumption helped to determine the medium limitations and to a first estimation of physiological parameters. \cite{del1998} defined Bacterial Growth Efficiency (BGE) as the amount of new bacterial biomass produced per unit of organic C substrate assimilated. BGE can be estimated using the bacterial Biomass Production (BP) and the Bacterial Respiration (BR) with CO$_2$ production as BGE=BP/(BP+BR). In natural ecosystems, BGE is estimated between 0.01 and 0.6 for oligotrophic to eutrophic environments \citep{del1998}. These BGE values are higher in batch setup at laboratory with BGE values between 0.3 and 0.9 \citep{del1998}. In this work, the experimental BGE value is 0.86 showing an important biomass production per unit of organic carbon. In this work, the experimental BGE value can be estimated with both glycerol and O$_2$ consumption, the OD$_{600nm}$ / dry weight ratio and the CO$_2$ production. However, this value has to be discussed with caution because the computation is based on only one experiment. There is a need of replicates for robust interpretation. Moreover, further investigations with several replicates will lead to the computation of physiological parameters for this bacterial strain.\\

%several physiological parameters can be estimated.\\

%\begin{table}
%\center
%\caption{Summary of physiological parameters from bio-reactor experiments into ONR7a medium, at 13\degres C and atmospheric pressure. BGE: Bacterial Growth Efficiency.}
%\begin{tabular}{ccc}
%\textbf{Parameter} & \textbf{Formula}& \textbf{Experimental value}\\
%\hline
%\textbf{CO$_2$ production}& (CO$_2$)$_{Tf}$ - (CO$_2$)$_{T0}$&15.7 mM\\
%\textbf{O$_2$ consumption}& (O$_2$)$_{Tf}$ - (O$_2$)$_{T0}$&19.1 mM\\
%\textbf{OD$_{600nm}$/dry weight}& weighted & 3.2 g OD$_{unit}$ $^{-1}$\\
%\textbf{Biomass}& OD$_{max}$ / 3.2 & 1.2 g L$^{-1}$\\
%\textbf{Glycerol consumption}& glycerol$_{Tf}$ - glycerol$_{T0}$&14.4 mM\\
%\hline
%\textbf{Glycerol respiration} & C$_3$H$_8$O$_3$ + 3.5 O$_2$ $\rightarrow$ 3 CO$_2$ + 4 H$_2$O&\\
%\hline
%\textbf{CO$_2$/O$_2$ ratio}& theoretical = 0.86 & 0.82\\
%\textbf{BGE} & BP /(BP+BR) & 0.86\\
%\textbf{Biomass C content} &  & 0.5\\
%\textbf{\% O$_2$ for bioluminescence}& see \ref{oxypercent} & 4.8 \%\\
%\end{tabular}
 %\label{calculs} 
%\end{table}    

%In Table \ref{calculs}, the values of gaz exchanges and physiological parameters have been determined before the end of the growth, between T$_0$=0 h and T$_f$=67 h. Under these experimental conditions, the total CO$_2$ consumption, estimated as the difference between the initial and the final CO$_2$ concentrations, is 15.7 mM. The O$_2$ consumption determined as the cumulative oxygen input to the system is about 19.1 mM. The OD$_{600nm}$ / dry weight ratio of 3.9 has been determined by desiccation of 1.5 L of bacterial culture, the biomass can also be estimated from this ratio. The glycerol consumption is defined as the difference between the initial glycerol input in the growth medium and the final glycerol concentration at the end of the bacterial growth (T$_f$=67 h). During this experiment, the glycerol consumption has been measured about 14.4 mM. Moreover, the Bacterial Growth Efficiency (BGE) is defined by \cite{del1998} as the amount of new bacterial biomass produced per unit of organic C substrate assimilated. BGE can be estimated using the bacterial Biomass Production (BP) and the Bacterial Respiration (BR) with CO$_2$ production as BGE=BP /(BP+BR). In natural ecosystems, BGE is estimated between 0.01 and 0.6 for oligotrophic to eutrophic environments \citep{del1998}. These BGE values are higher in batch setup at laboratory. In this work, the experimental BGE value is 0.86 showing an important biomass production per unit of organic carbon.\\

%In order to determine the part of oxygen consumption involved into the increase of bacterial biomass, the theoretical and experimental CO$_2$ / O$_2$ ratios have been estimated. 
%The chemical equation for glycerol respiration is :
%\begin{eqnarray}
%C_3H_8O_3 + 3.5 O_2 \longrightarrow 3 CO_2 + 4 H_2O
%\end{eqnarray}
%From this stoechiometric relation, the theoretical $\frac{CO_{2resp}}{O_{2resp}}$ ratio is about 0.86 compared to an experimental value of 0.82. %These two values of experimental O$_2$ consumption and theoretical value assumed by the glycerol respiration give a potential of 4.8\% of oxygen dedicated to the bioluminescence reaction (see \ref{oxypercent} for computation details). This value  has to be discussed with caution because indirectly estimated, using theoretical and experimental values. However, this percentage is close to values estimated by \cite{makemson1986,nealson1979} and \cite{dunlap1984}. For improvement, several methods can be developed for this bacterial strain cultivated into bioreactor. Indeed, the estimated values from the literature have been achieved using inhibitors of the bioluminescence reaction such as CCCP or cyanide (see \ref{inhib}) components. Similar experiments can be performed with \textit{Photobacterium phosphoreum} ANT-2200 to confirm this percentage.\\ 

\subsection[Proposal for the detection of bioluminescent organisms]{Proposal for the detection of bioluminescent organisms
\sectionmark{Proposal}}
\sectionmark{Proposal}


\subsubsection*{Importance of bioluminescence into the carbon cycle}

Following this work, we propose to deploy autonomous PMTs through the water column in order to check bioluminescent organisms at the upper and lower level of the mesopelagic zone as well as "bioluminescent" fecal pellets flux.\\

\subsubsection*{PMTs acquisition and calibration in the laboratory.} 

The Centre de physique des Particules de Marseille (CPPM) is one of the main institutes of the ANTARES collaboration, in charge of ANTARES PMTs development. Recently, in the framework of the MEUST project (Mediterranean Eurocentre for underwater sciences and technologies), the CPPM develops autonomous PMTs dedicated to the survey of a new site for the future km$^3$ neutrino telescope. In close collaboration with the CPPM, we could ask the development of specific PMTs dedicated to the bioluminescence study. Indeed, using this first experiment, we are able to define the need of two different data sampling rates, in order to increase memory space acquisition and the efficiency of PMTs data acquisition.\\

\subsubsection*{First essays of PMTs deployments on mooring line.} 

In the framework of the site survey dedicated to the MEUST project, a deep mooring line has already been deployed. We plan to reproduce the schema of this mooring line also in the upper and lower part of the mesopelagic zone, adding UVP and LISST instruments. First essays will be done to choose the best implementation of all instruments, first data acquisition and cross-data calibration. More precisely, two sampling rates have to be improved with first deployments. Firstly, a low frequency sampling rate (about minutes) will be dedicated to the record of bioluminescence intensity linked to environmental variables. Then, at other periods, higher frequency sampling rate will be dedicated to the organisms' recognition spectra. Modulation in frequency sampling rate is a good compromise between data acquisition interest and data storage. \\

\subsubsection*{Time-frequency analyzes of bioluminescence time-series and oceanographic variables (temperature, salinity, currents, oxygen, POC fluxes)}

One of the approaches proposed for further investigations, is to understand links between bioluminescent organisms and carbon supply in the mesopelagic and bathypelagic environment. To do so, we could use the previously described mooring line (PMTs, UVP, LISS, CTD, current meter at different depths). Time series analysis using statistically adapted and developed mathematical methods is a crucial point to access to a better ecological understanding. Indeed, such time series are statistically defined as non-linear and non-stationary. If the Fourier method has been traditionally used for time-series decomposition it is not well adapted to non-stationary data. Since the 80's two different methods named the wavelet and the Huang-Hilbert decomposition methods have been developed to counteract this gap \citep{Huang1998,Torrence1998}. These decompositions methods are defined in the time-frequency domain. From the ANTARES time-series analysis, using both the wavelets and Hilbert-Huang methods, bioluminescence intensity has been found to be clearly sea-current-intensity dependent but also highly correlated with modification of deep-water masses (using salinity and temperature as proxies) \citep{martini2013b}. These mathematical methods represent an appropriate way to provide new insight from time-series and would be applied in future work.\\


\subsubsection*{Bioluminescence spectra analysis and automatic classification of organisms}

Using this instrumented-mooring line, a perspective of ecological interest in the mesopelagic and bathypelagic could also be approached. As already described, bioluminescence is a phenomenon produced by very wide range of organisms. We focused on links between biology and the environmental changes aiming at the definition of populations or organisms involved. Bioluminescence spectra analysis is a way to determine what kind of organisms are present close to the mooring line and, eventually, to detect some shifts in populations over time. Indeed, different organisms have already been described to emit different light pattern (see Figure \ref{proposal}, \citealp{widder1989,nealson1986}).\\

\begin{figure}[!h]
\linespread{1} 
\centering
\includegraphics[scale=0.6]{proposal.pdf}
\caption[Example of bioluminescence emission patterns. A) From the ANTARES observatory, B) In the literature.]{Example of bioluminescence emission patterns. A) From the ANTARES observatory, B) In the literature \citep{hastings1991}.}
\label{proposal}
\end{figure} 
 
These signatures can be used to describe populations crossing the mooring line. Bioluminescence spectra as a signature of organisms has been approached by \cite{nealson1986}, which only used ten species. In this part, the proposed approach is the use of mathematical unsupervised classification for population detection. Emission intensity can be recorded as curves for individuals crossing the detector. The shape, length and area of the curves are several potential descriptors that can be used for the classification method. Depending on the variability between these criteria, each bioluminescence emission curve is attributed to one class. At the end of the process, each class defines similar curves with the same pattern of light emission, potentially, the same class of organisms.\\

This classification method is already used into several fields such as cytometry \citep{malkassian2011} or hydrological prediction \citep{nerini2000}. In a final aim, crossing this automatic classification method for bioluminescence intensity (recorded at high-frequency sampling rates) with the UVP, we should define an image of organisms for each class defined. In order to recognize organisms that have been classified with video camera and bioluminescence spectra, collaborations with international experts in bioluminescent organisms have to be envisaged. This support would be of great importance to determine higher level identification for observed organisms.\\ 


