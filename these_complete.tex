\documentclass[a4paper,12pt,twoside,openright]{book}
%\documentclass[chap,11pt]{thesis}
%\documentclass[twoside,12pt]{Classes/aesm_edspia}
\usepackage[a4paper,top=4.5cm,bottom=3.5cm,left=3cm,right=2cm]{geometry}
\usepackage[Sonny]{fncychap}
%\geometry{left=3.6cm,right=3.6cm,top=3cm} 
%\usepackage[utf8x]{inputenc}
\usepackage[latin1]{inputenc}% accents dans le source
%\usepackage{varioref}
\usepackage[T1]{fontenc}
\usepackage{amsmath}
\usepackage{lipsum} %Pour faire des essais
% Choix de la langue
\usepackage[francais]{babel}
\usepackage{natbib}
%% Pour une thse en franais et en anglais
%\usepackage{titlesec}
% Lmodern et substitution des petites capitales grasses manquantes
\usepackage{lmodern}
\rmfamily
\DeclareFontShape{T1}{lmr}{b}{sc}{<->ssub*cmr/bx/sc}{}
\DeclareFontShape{T1}{lmr}{bx}{sc}{<->ssub*cmr/bx/sc}{}
% Diffrents paquets pour les maths
\usepackage{amssymb,amsmath,amsthm,amscd}
\usepackage{mathrsfs}

\usepackage{float}
% Pour les figures
\usepackage{subfig}
%Pour la page de garde
\usepackage{tabularx} % Permet d'utiliser l'environnement tabularx
\usepackage{calc} % Pour pouvoir donner des formules dans les dfinitions de longueur
\usepackage{graphicx} % Pour inclure des graphiques 
% Attention : pour inclure des .jpg comme dans l'exemple (ou des .png ou .pdf)
% il faut compiler directement en pdf (commande pdflatex).
% Pour inclure des .eps, il faut compiler avec latex + dvips + ps2pdf.
\usepackage[francais]{minitoc}

\usepackage{xcolor,colortbl}
\usepackage{color}
\definecolor{newblue}{rgb}{0,0,0.52}
\definecolor{newred}{rgb}{0.5,0,0}
\definecolor{cream}{rgb}{0.9,0.9,0.9}
% Pour avoir des liens hypertexte dans le document compil
\usepackage[citecolor=newblue, colorlinks=true, urlcolor=blue, linkcolor=newred,pagebackref =TRUE]{hyperref}
%\usepackage{nohyperref} %  utiliser pour pouvoir compiler sans gnrer des liens
\renewcommand*{\backref}[1]{(cite page #1)}
%%de starr
%\usepackage[hypertexnames = false, % keeps bookmarks in correct order 
%when counter is reset within document
             %colorlinks=true, urlcolor=black, linkcolor=black, 
%citecolor=black,
 %            colorlinks=true, urlcolor=black, linkcolor=newblue2, 
%citecolor=newblue,
 %            plainpages=false,
 %            pdftex,
 %            pdftitle={Metabolic programming of zebrafish uncovered},
 %            pdfauthor={Starrlight Augustine},
 %            pdfsubject={zebrafish},
 %            pdfkeywords={Dynamic Energy Budget Theory}]{hyperref}


% Pour mettre la bibliographie dans la table des matires avec le bon numro de page (voir plus loin)
\usepackage[nottoc]{tocbibind}

% Pour l'index des notations
\usepackage{makeidx}
\captionsetup{format=plain,labelformat=parens,labelsep=quad,font=footnotesize, margin=20pt
}
\makeindex


% =========================== Commandes diverses ===============================

% Macros-commandes : appel  des fichiers extrieurs
%\input{macros} % macros diverses personnelles 
% (en-ttes et pieds de page, environnements de thormes) - voir macros.tex
%\input{macrosmath} % macros mathmatiques - voir le fichier macrosmath.tex

\usepackage[section]{placeins}
\usepackage{setspace}
\onehalfspacing
%\onehalfspacing %interligne 1.5
\usepackage[figuresright]{rotating}
\usepackage{eso-pic}
\newcommand\BackgroundPic{
\put(0,0){
\parbox[b][\paperheight]{paperwidth}{
\vfill
\centering
\includegraphic[width=\paperwidth,height=\paperheight,keepaspectratio]{couverture.pdf}
\vfill
}}}
\usepackage{wallpaper}
\begin{document}
\dominitoc

\printindex % pour afficher l'index

% Pour basculer la numrotation des pages en chiffres romains, jusqu' la
% table des matires. Dconseill par le ministre !

% Page de garde
\frontmatter
%

\thispagestyle{empty}
% pour ne pas avoir de numro de page sur la page de garde -- le compteur de
% page est cependant  1, c'est--dire que la numrotation commence  partir
% de la page de garde

\begin{center}
	%& \large Université d'Aix-Marseille
  %\end{tabularx}
\vspace*{\fill}

La photographie de couverture est la propri�t� de Phil Hart,\\ Bioluminescence en Australie, 11 Janvier 2013\\ www.philhart.com.
\end{center}

\ThisLRCornerWallPaper{1}{couverture1.pdf}

\thispagestyle{empty}
% pour ne pas avoir de numro de page sur la page de garde -- le compteur de
% page est cependant  1, c'est--dire que la numrotation commence  partir
% de la page de garde

\begin{center}
	%& \large Université d'Aix-Marseille
  %\end{tabularx}
\end{center}


% Dedicace et epigraphe


\thispagestyle{empty}
% pour ne pas avoir de numro de page sur la page de garde -- le compteur de
% page est cependant  1, c'est--dire que la numrotation commence  partir
% de la page de garde

\begin{center}
	%& \large Université d'Aix-Marseille
  %\end{tabularx}
\vspace*{\fill}

La photographie de couverture est la propri�t� de Phil Hart,\\ Bioluminescence en Australie, 11 Janvier 2013\\ www.philhart.com.
\end{center}

%\pdfbookmark[0]{Page de garde}{garde}
% La commande pdfbookmark permet  hyperref d'ajouter la page de garde dans le
% menu du fichier compil (dvi, pdf)
%%%%%%\AddToShipoutPicture*{\BackgroundPic}

\thispagestyle{empty}
% pour ne pas avoir de numro de page sur la page de garde -- le compteur de
% page est cependant  1, c'est--dire que la numrotation commence  partir
% de la page de garde

\begin{center}
  %\begin{tabularx}{\textwidth}{m{5.5cm}Xm{3cm}X}
	%\includegraphics[width=2in]{logo_AMU}
	\includegraphics[width=13cm]{logos}

	%& \large Université d'Aix-Marseille
  %\end{tabularx}
\end{center}

\begin{center}
% Permet de crer un espace vertical de longueur variable (\stretch) et de "poids" 1
{\large \textbf{Aix-Marseille Universit�\\
Institut M�diterran�en d'Oc�anologie (MIO)\\}}
%\vspace{\stretch{1}}
{\large \textbf{Ecole Doctorale des sciences de l'environnement (ED-251)}}

\vspace{\stretch{8}}
%Espace vertical variable de poids 2.

{\LARGE \textsc{Th�se de doctorat}}

\vspace{\stretch{1}}

{\Large Discipline : Oc�anographie }

\vspace{\stretch{6}}

\vspace{\stretch{2}}
\hrule
\vspace{\stretch{0.4}}
\hrule
\vspace{\stretch{4}}
{\LARGE \textbf{La bioluminescence un proxy d'activit�\\ biologique en milieu profond ?\\}}
\vspace{\stretch{2}}

{\LARGE Etude au laboratoire et \textit{in situ} de la bioluminescence en relation avec les variables environnementales.}
\vspace{\stretch{4}}
\hrule
\vspace{\stretch{0.4}}
\hrule

\vspace{\stretch{7}}
{\Large Pr�sent�e par:  \LARGE S�verine \textsc{Martini}}\\
\vspace{\stretch{1}}
{\Large Sous la direction de: Christian \textsc{Tamburini} et David \textsc{Nerini}}

\vspace{\stretch{7}}

{\Large Soutenance pr�vue le 6 d�cembre 2013}\\
\vspace{\stretch{3}}

{\Large Jury compos� de :}\\
\vspace{\stretch{2}}
{
\begin{tabular}{llcr}
Dr. Fran�ois \textsc{Schmitt}& DR CNRS& LOG, France & Rapporteur\\
Pr. Richard \textsc{Lampitt}& Professor & NOCS, UK &Rapporteur\\ 
Dr. Patricia  \textsc{Bonin} & DR CNRS AMU & MIO, France & Examinateur \\
Dr. Steven \textsc{Haddock}& adjunct Associate Professor & MBARI, USA & Examinateur\\
Dr. David \textsc{Nerini} &  MCF AMU & MIO, France & co-Directeur\\
Dr. Christian \textsc{Tamburini} & CR CNRS AMU & MIO, France & Directeur\\
\end{tabular}
}

\end{center}

%\cleardoublepage % pour laisser une page blanche au verso de la page de garde

\newpage

\vspace*{\fill}

\noindent\begin{center}
\begin{minipage}[t]{(\textwidth-2cm)/2}
Institut M�diterran�en\\ d'Oc�anologie (MIO),\\
Campus de Luminy ,\\
 162, Avenue de Luminy, \\
 13288 Marseille, Cedex 09  \\
 France
\end{minipage}
%
\hspace{1.5cm}
%
\begin{minipage}[t]{(\textwidth-2cm)/2}
Ecole doctorale des sciences\\ de l'environnement (ED 251),\\ 
Europ�le M�diterran�en de l'Arbois,\\
BP 80,\\
13545 Aix-en-Provence, Cedex 04\\
France
\end{minipage}
\end{center}

\newpage
\vspace*{\fill}

\begin{flushright}
\textit{A la Baleine Blanche,\\ 
� Salam, � Iluna,}\\
\vspace{5mm}
Parce que, finalement, tout a commenc� l�.\\ Des quarts de nuit, pass�s dans l'humide cockpit, accroch�e � la barre,\\ n�gociant les vagues, et regarder des heures durant, alternativement,\\ les Pl�iades dans le ciel et le sillage lumineux...\\
\end{flushright}

\vspace{70mm} 
\begin{flushright}
\textit{"La mer est sans routes, la mer est sans explications..." }\\
A Barrico. 
\end{flushright}


% Remerciements
\pdfbookmark[0]{Remerciements}{remerciements}
% La commande pdfbookmark permet  hyperref d'ajouter les remerciements dans
% le menu du fichier compil (dvi, pdf)
\chapter*{Remerciements}

\noindent Ce travail a �t� financ� par une bourse minist�rielle MNRT (2010-2013). Un financement EC2CO (projet BIOLUX 2011-2013) a permis la r�alisation des exp�rimentations. Je remercie \textbf{Richard Semp�r�} de m'avoir accueillie, tout d'abord au sein du LMGEM, et maintenant au MIO. Merci � \textbf{Bruno Hamelin} de m'avoir permis d'enseigner, en tant que monitrice, au sein de l'OSU Pytheas. Merci � \textbf{Pierre Rochette} de m'avoir accept�e au sein de l'Ecole doctorale "Sciences de l'environnement" (ED251).\\

\noindent First of all, I would like to sincerely thank Pr. Fran�ois G. Schmitt, Pr. Richard Lampitt, Dr. Steven Haddock and Pr. Patricia Bonin for accepting to judge my work and to be present during my PhD defense.\\

\noindent Je tiens � remercier mes deux directeurs de th�ses, \textbf{Christian Tamburini} et \textbf{David Nerini}, pour ce sujet, passionnant. J'ai tenu, avec conviction, � conserver � la fois une approche math�matique et exp�rimentale dans mon travail de recherche. Loin d'�tre �vident... Pourtant, j'ai eu le privil�ge d'�tre entour�e d'une �quipe dont les comp�tences et l'ouverture d'esprit m'ont permis de naviguer et d'apprendre beaucoup, dans chacun de ces domaines. \textbf{Christian}, la liste de mes remerciements est longue. Merci de m'avoir ouvert la porte de la HP-team il y a 6 ans et de l'avoir toujours laiss�e ouverte depuis. Ta confiance m'est pr�cieuse depuis toutes ces ann�es. Tes conseils et nos discussions m'ont permis de construire ma reflexion scientifique, et c'est un plaisir de travailler avec toi autant scientifiquement qu'humainement. Merci aussi d'avoir toujours su conjuguer avec mon ent�tement, et mon impatience ! \textbf{David}, merci pour la confiance que tu m'as accord�e, parfois plus que je ne m'en accordais moi-m�me. Merci �galement de m'avoir transmis un int�r�t grandissant pour l'enseignement. Comme disait un grand philosophe, \textit{"Il y a souvent de l'admiration dans les yeux d'un doctorant pour son directeur..."-Dr Y. Eynaud, 2012}.\\

\noindent \textbf{Charles-Fran�ois Boudouresque}, merci d'avoir accept� de co-encadrer les d�buts de cette th�se, et de m'avoir enseign� (entre autres) la rigueur scientifique qui me manque encore souvent.\\

\noindent \textbf{Marc Garel}. Merci Marc, d'avoir �t� l� � chaque fois que j'avais besoin de donn�es, d'une r�ponse, de discuter du dernier package sous R, ou d'une commande vraiment extraordinaire sous LATEX !!! Merci pour ta pr�sence quand les manips ne marchent pas et quand il y a une super-id�e-�-essayer-absolument. Merci pour ton professionnalisme, tes connaissances et ta disponibilit�.\\

Merci � l'�quipe BIOLUX:\\
\noindent \textbf{Laurie Casalot}, je te remercie d'avoir �t� aussi pr�sente au cours de ces 3 ann�es, un plaisir de travailler avec toi, and thank you to improve my english... L'�quipe fermenteur: \textbf{Sylvain Davidson}, \textbf{Yannick Combet-Blanc}, et \textbf{Richard Auria}, ce f�t riche scientifiquement, m�me si, malheureusement, tout n'a pas abouti comme je le souhaitais ! \textbf{Val�rie Michotey}, et \textbf{Sophie Guasco} merci pour votre implication dans les manips de biomol et pour avoir pris le temps de me donner des explications, inlassablement...\\

Merci �:\\
\noindent \textbf{Bruno Charri�re} (pour ta gentillesse et tes blagues... parfois dr�les), \textbf{Dominique Lef�vre} (tout vient � point � qui sait attendre), \textbf{G�g� Gr�gori} (pour ta bonne humeur et tes conseils pour la suite), \textbf{C�cile Militon} (pour avoir guid� mes premiers pas de monitrice), \textbf{Christos Panagiotopoulos}, \textbf{Jean-Christophe Poggiale}, \textbf{Philippe Cuny}, \textbf{France Van Wambeke, Cathy, Marc, Najib, Anne}, \textbf{Dominique Estival}, \textbf{Dominique Poirot} et \textbf{Brigitte Pantat} merci beaucoup d'avoir �t� l�, dans le labyrinthe administratif, aux capitaines et �quipages des navires oc�anographiques...\\

Je remercie la collaboration ANTARES:\\
\noindent Tout particuli�rement, \textbf{St�phanie Escoffier}, \textbf{Christian Curtil}, et \textbf{Sylvain Henry} du CPPM pour votre aide, souvent indispensable, et toujours appr�ci�e.\\

\noindent \textbf{Agn�s Dominjon} et \textbf{R�mi Barbier} de l'In2p3 et le Centre Commun de Quantim�trie � Lyon pour avoir encourag� mes id�es... parfois peu lumineuses ! \textbf{Vincent Grossi} et \textbf{Muriel Pacton} pour votre r�activit�.\\

\noindent Je tiens � adresser mes remerciements aux �tudiants de licence qui ont suivi mes enseignements ainsi qu'aux stagiaires de passage au laboratoire. L'enseignement f�t une part importante de mes 3 ann�es de doctorat, que j'ai accompli avec un v�ritable int�r�t et que je souhaite avoir l'opportunit� de poursuivre. Un grand merci aux pr�c�dents doctorants de la HP-Team, \textbf{Badr Al Ali}, \textbf{Mehdi Boutrif} et \textbf{Anne Robert}.\\ 

\noindent Merci � mes amis les plus pr�cieux, qu'ils soient parapentistes (Pascal, St�ph, Serge, Lily, Fredo, Pierrick...), tangueros/as (Philippe, C�dric, Michel, Patrick, Julio, G�raldine, Jordi, Andrea...), plongeurs (l'ELM), doctorants (Maxime, Nico, Marie...), grimpeurs (Micka), slackeurs (Mathieu), marcheurs (Mathieu \& Mathieu), navigateurs (Thom), de partout (Erwan) ou de n'importe o� (Bobo, Nourmouhamad)... pour ce qui fut essentiel � chaque instant, au cours de ces 3 ann�es. A mon aile aussi, pour ne m'avoir jamais laiss�e tomber pendant la r�daction ! Un merci particulier � Micka et Saraline, � Morgan, v�ritable co-�quipier de th�se, et � Axel.\\

\noindent Enfin, je remercie, infiniment, mes parents, pour votre soutien, pour votre pr�sence, pour m'apprendre depuis toujours que: \textit{"L'imagination est plus importante que le savoir" - A. Einstein} Et que \textit{"Les gens qui ne rient jamais ne sont pas des gens s�rieux."-A. Allais}.\\ A ma grande soeur.\\
A Fanny, merci pour ces moments partag�s ensemble qui ont fait ces 3 ann�es �poustouflantes ! Pour les embruns, les strato-cumulus, nos al�as de doctorantes et notre amiti�, indispensable ces derni�res ann�es. A toi bient�t...\\

\noindent Mes remerciements s'adressent �galement � tous ceux qui m'ont questionn�, �cout�, parfois beaucoup appris sur la bioluminescence en pleine mer et dans les abysses. Merci � toutes ces personnes gr�ce auxquelles je me suis attach�e avec curiosit�, et passion � ce sujet.



\pdfbookmark[0]{Liste des abr�viations}{Liste des abr�viations}
% La commande pdfbookmark permet  hyperref d'ajouter les remerciements dans
% le menu du fichier compil (dvi, pdf)
\chapter*{Liste des abr�viations}
\begin{sidewaystable}
%\begin{turn}{180}
\begin{tabular}{lll}
\hline
\textbf{Abr�viation}&\textbf{...in english}&\textbf{...en fran�ais}\\
\hline
ANTARES  & Astronomy with a Neutrino Telescope&Astronomie � l'aide d'un t�lescope � neutrinos\\
& and Abyss environmental RESearch&et recherche environnementale des abysses\\
PMT&Photomultiplier Tube&Photomultiplicateur\\
ADCP& Acoustic Doppler Current Profiler& Profileur de courant par mesure acoustique Doppler\\
CTD& Conductivity Temperature Depth& Conductivit� temp�rature profondeur \\
HPB&High Pressure Bottle&Bouteille haute pression\\
$r$&Growth rate&Taux de croissance \\
$K$&Carrying capacity&Capacit� limite\\
HHG&Hilbert-Huang decomposition& D�composition d'Hilbert-Huang\\
IMF&Intrinsic Mode Function& Fonction de mode intrins�que\\
DOC&Dissolved Organic Carbon& Carbone organique dissout\\
POC& Particulate Organic Carbon& Carbone organique particulaire\\
OD&Optical Density& Densit� Optique\\
BR&Bacterial Respiration&Respiration bact�rienne\\
BP& Biomass Production&Production de biomasse\\
BGE&Bacterial Growth Efficiency&Efficacit� de croissance bact�rienne\\
NW&North-Western&Nord-Ouest\\
PPi& Inorganic Pyrophosphate& Pyrophosphate inorganique\\
AMP&Adenosine-5'-MonoPhosphate&Ad�nosine-5'-Monophosphate\\
FMN&Flavine MonoNucleotid oxidized &Flavine mononucl�otide oxyd�e\\
FMNH$_2$&Flavine MonoNucleotid reduced &Flavine mononucl�otide r�duite\\
ATP&Adenosine-5'-TriPhosphate&Ad�nosine-5'-triphosphate\\
CFU&Colony Forming Units&Unit� formant des colonies\\
CCCP &Carbonyl Cyanid m-Chlorophenylhydrazone&Carbonyle cyanide m-chlorophenylhydrazone\\
AHL&N-Acylated Homoserine Lactone &N-acylated homoserine Lactone\\
UFA& Unsaturated Fatty Acid& Acides gras insatur�s\\
PUFA& Poly-Unsaturated Fatty Acid& Acides gras poly-insatur�s\\
SFA& Saturated Fatty Acid& Acides gras satur�s\\

\end{tabular}
%\end{turn}
\end{sidewaystable}
%\end{turn}
\printindex % pour afficher l'index

% Rsum

% Table des matires
\setcounter{tocdepth}{3} % pour rgler sa profondeur - par dfaut : 2
\pdfbookmark[0]{Table des mati�res}{tablematieres} % pour ajouter la table des matires dans l'``index'' du fichier compil
% Remerciements
%\pdfbookmark[0]{Remerciements}{remerciements}
% La commande pdfbookmark permet  hyperref d'ajouter les remerciements dans
% le menu du fichier compil (dvi, pdf)
\chapter*{Remerciements}

\noindent Ce travail a �t� financ� par une bourse minist�rielle MNRT (2010-2013). Un financement EC2CO (projet BIOLUX 2011-2013) a permis la r�alisation des exp�rimentations. Je remercie \textbf{Richard Semp�r�} de m'avoir accueillie, tout d'abord au sein du LMGEM, et maintenant au MIO. Merci � \textbf{Bruno Hamelin} de m'avoir permis d'enseigner, en tant que monitrice, au sein de l'OSU Pytheas. Merci � \textbf{Pierre Rochette} de m'avoir accept�e au sein de l'Ecole doctorale "Sciences de l'environnement" (ED251).\\

\noindent First of all, I would like to sincerely thank Pr. Fran�ois G. Schmitt, Pr. Richard Lampitt, Dr. Steven Haddock and Pr. Patricia Bonin for accepting to judge my work and to be present during my PhD defense.\\

\noindent Je tiens � remercier mes deux directeurs de th�ses, \textbf{Christian Tamburini} et \textbf{David Nerini}, pour ce sujet, passionnant. J'ai tenu, avec conviction, � conserver � la fois une approche math�matique et exp�rimentale dans mon travail de recherche. Loin d'�tre �vident... Pourtant, j'ai eu le privil�ge d'�tre entour�e d'une �quipe dont les comp�tences et l'ouverture d'esprit m'ont permis de naviguer et d'apprendre beaucoup, dans chacun de ces domaines. \textbf{Christian}, la liste de mes remerciements est longue. Merci de m'avoir ouvert la porte de la HP-team il y a 6 ans et de l'avoir toujours laiss�e ouverte depuis. Ta confiance m'est pr�cieuse depuis toutes ces ann�es. Tes conseils et nos discussions m'ont permis de construire ma reflexion scientifique, et c'est un plaisir de travailler avec toi autant scientifiquement qu'humainement. Merci aussi d'avoir toujours su conjuguer avec mon ent�tement, et mon impatience ! \textbf{David}, merci pour la confiance que tu m'as accord�e, parfois plus que je ne m'en accordais moi-m�me. Merci �galement de m'avoir transmis un int�r�t grandissant pour l'enseignement. Comme disait un grand philosophe, \textit{"Il y a souvent de l'admiration dans les yeux d'un doctorant pour son directeur..."-Dr Y. Eynaud, 2012}.\\

\noindent \textbf{Charles-Fran�ois Boudouresque}, merci d'avoir accept� de co-encadrer les d�buts de cette th�se, et de m'avoir enseign� (entre autres) la rigueur scientifique qui me manque encore souvent.\\

\noindent \textbf{Marc Garel}. Merci Marc, d'avoir �t� l� � chaque fois que j'avais besoin de donn�es, d'une r�ponse, de discuter du dernier package sous R, ou d'une commande vraiment extraordinaire sous LATEX !!! Merci pour ta pr�sence quand les manips ne marchent pas et quand il y a une super-id�e-�-essayer-absolument. Merci pour ton professionnalisme, tes connaissances et ta disponibilit�.\\

Merci � l'�quipe BIOLUX:\\
\noindent \textbf{Laurie Casalot}, je te remercie d'avoir �t� aussi pr�sente au cours de ces 3 ann�es, un plaisir de travailler avec toi, and thank you to improve my english... L'�quipe fermenteur: \textbf{Sylvain Davidson}, \textbf{Yannick Combet-Blanc}, et \textbf{Richard Auria}, ce f�t riche scientifiquement, m�me si, malheureusement, tout n'a pas abouti comme je le souhaitais ! \textbf{Val�rie Michotey}, et \textbf{Sophie Guasco} merci pour votre implication dans les manips de biomol et pour avoir pris le temps de me donner des explications, inlassablement...\\

Merci �:\\
\noindent \textbf{Bruno Charri�re} (pour ta gentillesse et tes blagues... parfois dr�les), \textbf{Dominique Lef�vre} (tout vient � point � qui sait attendre), \textbf{G�g� Gr�gori} (pour ta bonne humeur et tes conseils pour la suite), \textbf{C�cile Militon} (pour avoir guid� mes premiers pas de monitrice), \textbf{Christos Panagiotopoulos}, \textbf{Jean-Christophe Poggiale}, \textbf{Philippe Cuny}, \textbf{France Van Wambeke, Cathy, Marc, Najib, Anne}, \textbf{Dominique Estival}, \textbf{Dominique Poirot} et \textbf{Brigitte Pantat} merci beaucoup d'avoir �t� l�, dans le labyrinthe administratif, aux capitaines et �quipages des navires oc�anographiques...\\

Je remercie la collaboration ANTARES:\\
\noindent Tout particuli�rement, \textbf{St�phanie Escoffier}, \textbf{Christian Curtil}, et \textbf{Sylvain Henry} du CPPM pour votre aide, souvent indispensable, et toujours appr�ci�e.\\

\noindent \textbf{Agn�s Dominjon} et \textbf{R�mi Barbier} de l'In2p3 et le Centre Commun de Quantim�trie � Lyon pour avoir encourag� mes id�es... parfois peu lumineuses ! \textbf{Vincent Grossi} et \textbf{Muriel Pacton} pour votre r�activit�.\\

\noindent Je tiens � adresser mes remerciements aux �tudiants de licence qui ont suivi mes enseignements ainsi qu'aux stagiaires de passage au laboratoire. L'enseignement f�t une part importante de mes 3 ann�es de doctorat, que j'ai accompli avec un v�ritable int�r�t et que je souhaite avoir l'opportunit� de poursuivre. Un grand merci aux pr�c�dents doctorants de la HP-Team, \textbf{Badr Al Ali}, \textbf{Mehdi Boutrif} et \textbf{Anne Robert}.\\ 

\noindent Merci � mes amis les plus pr�cieux, qu'ils soient parapentistes (Pascal, St�ph, Serge, Lily, Fredo, Pierrick...), tangueros/as (Philippe, C�dric, Michel, Patrick, Julio, G�raldine, Jordi, Andrea...), plongeurs (l'ELM), doctorants (Maxime, Nico, Marie...), grimpeurs (Micka), slackeurs (Mathieu), marcheurs (Mathieu \& Mathieu), navigateurs (Thom), de partout (Erwan) ou de n'importe o� (Bobo, Nourmouhamad)... pour ce qui fut essentiel � chaque instant, au cours de ces 3 ann�es. A mon aile aussi, pour ne m'avoir jamais laiss�e tomber pendant la r�daction ! Un merci particulier � Micka et Saraline, � Morgan, v�ritable co-�quipier de th�se, et � Axel.\\

\noindent Enfin, je remercie, infiniment, mes parents, pour votre soutien, pour votre pr�sence, pour m'apprendre depuis toujours que: \textit{"L'imagination est plus importante que le savoir" - A. Einstein} Et que \textit{"Les gens qui ne rient jamais ne sont pas des gens s�rieux."-A. Allais}.\\ A ma grande soeur.\\
A Fanny, merci pour ces moments partag�s ensemble qui ont fait ces 3 ann�es �poustouflantes ! Pour les embruns, les strato-cumulus, nos al�as de doctorantes et notre amiti�, indispensable ces derni�res ann�es. A toi bient�t...\\

\noindent Mes remerciements s'adressent �galement � tous ceux qui m'ont questionn�, �cout�, parfois beaucoup appris sur la bioluminescence en pleine mer et dans les abysses. Merci � toutes ces personnes gr�ce auxquelles je me suis attach�e avec curiosit�, et passion � ce sujet.


\tableofcontents  % pour afficher la table des matires

\mainmatter
% On bascule la numrotation des pages en chiffres arabes jusqu' la fin du
% document - la numrotation reprend  1 (toujours dconseill par le ministre)
\baselineskip=20pt 


% Introduction
\chapter[Introduction et objectifs de l'�tude]{Introduction et objectifs de l'�tude
\chaptermark{Introduction et objectifs de l'�tude}}
\chaptermark{Introduction et objectifs de l'�tude}
\minitoc

\newpage

\section{La bioluminescence marine}

La luminescence d�crit la production d'une lumi�re froide, par opposition � l'incandescence. Ce terme comprend la phosphorescence, la fluorescence ainsi que la bioluminescence. Les deux premiers ph�nom�nes sont li�s � une �mission de lumi�re faisant suite � l'absorption de photons alors que la bioluminescence est le produit d'une r�action photochimique par des organismes vivants. \\

La bioluminescence est un ph�nom�ne connu et observ� depuis l'antiquit�. Aristote (348-322 avt. J.C.) signale, d�j�, une �mission de lumi�re par des poissons morts. Par la suite, dans l'\textit{Historia naturalis}, Pline l'Ancien (23-79 ap. J.C.) d�crit �galement divers organismes bioluminescents tels que les vers luisant, les "champignons", ou les Cnidaires. Dans l'histoire, la bioluminescence marine joua parfois un r�le inattendu. En effet, au cours de la premi�re Guerre mondiale, le sous-marin U-34, fut le dernier submersible allemand coul� par les alli�s, rep�r� par son sillage luminescent. Ce sont ces m�mes sillages qui, durant la seconde Guerre, permettaient parfois aux pilotes de retrouver leur porte-avions la nuit. Enfin, lors de la guerre du Golfe, les soldats am�ricains ont d� modifier le trajet suivi par les navires afin d'�viter les bancs de Dinoflagell�s bioluminescents qui risquaient de trahir leur pr�sence.\\

\subsection{Diversit� phylog�n�tique des organismes bioluminescents}
\label{groupe}

Dans les environnements terrestres peu d'organismes sont d�crits comme bioluminescents alors qu'� l'inverse, en milieu marin, pr�s de 90\% des organismes sont estim�s capables de bioluminescence \citep{widder2010} et sont abondamment d�crits dans la litt�rature. Certains groupes non-planctoniques (se d�pla�ant ind�pendamment des mouvements de masses d'eaux) tels que les c�phalopodes (210 esp�ces bioluminescentes) ou les t�l�ost�ens (273 esp�ces bioluminescentes) sont largement repr�sent�s \citep{poupin1999,herring1987,herring1985}. Les groupes phylog�n�tiques planctoniques  comprenant le plus grand nombre d'esp�ces bioluminescentes sont\string: les Bacteria, Dinoflagell�s, Radiolaires, Cnidaires et Ct�naires, Crustac�s (Cop�podes, Amphipodes, Euphausiac�s, D�capodes), Chaetognathes, et Tuniciers. Des esp�ces bioluminescentes sont �galement d�crites parmi les Mollusques (Gast�ropodes), Ann�lides (Polych�tes), et Echinodermes holothurides (voir Figure \ref{taxa}). Parmi les procaryotes \footnote{procaryote est un terme non-phylog�n�tique regroupant Bacteria et Archaea}, seules des Bacteria bioluminescentes ont �t� d�crites comme capables d'�mettre de la lumi�re. Les Archaea ne sont, � notre connaissance, pas bioluminescentes.\\

\begin{figure}[p]
\linespread{1} 
\centering
\includegraphics[width=14.5cm]{taxa.pdf}
\caption{Phylogenetic representation of the diversity of bioluminescent organisms, from \cite{Haddock2010}.}
\label{taxa}
\end{figure} 

Les premi�res souches bact�riennes bioluminescentes ont �t� d�crites, d�s les pr�mices de la microbiologie, avec les travaux de B. Fischer en 1887, M. Beijerinck en 1889 puis ZoBell en 1946. Les bact�ries\footnote{Dans la suite de ce manuscrit, le terme "bact�rie" sera utilis� pour d�signer les procaryotes} luminescentes sont des Gammaproteobacteria, Gram-n�gatives, non-sporulantes, chemoorganotrophes h�t�rotrophes, dont la majorit� sont a�robies. Les esp�ces connues pour �tre bioluminescentes appartiennent aux genres \textit{Vibrio}, \textit{Photobacterium}, \textit{Shewanella} et \textit{Photorhabdus}. Parmi ces esp�ces, \textit{Photorhabdus} est le seul genre terrestre, les autres �tant pr�sents en milieu marin. \\

Pour illustrer cette diversit�, \cite{poupin1999} proposent, un inventaire non-exhaustif, bas� sur la litt�rature, en seule mer d'Iroise, d'environ 500 esp�ces bioluminescentes planctoniques, parmi lesquelles 6 esp�ces bact�riennes sont d�crites. Phylog�n�tiquement, la facult� de bioluminescence est donc distribu�e tr�s al�atoirement dans le monde du vivant \citep{herring1987}. Les analyses phylog�n�tiques ont sugg�r� que la bioluminescence est un caract�re g�n�tique ayant �volu� ind�pendamment entre 30 et 50 fois \citep{Haddock2010}. Cependant, les auteurs de ces synth�ses phylog�n�tiques, rassemblant les observations d'organismes bioluminescents, mettent en avant les difficult�s d'�tablir un tel inventaire. D'une part, les techniques d'observation de ce ph�nom�ne \textit{in situ} ont d� �voluer en quelques d�cennies (Table \ref{tab}), contraintes par la difficult� de capturer ces organismes. D'autre part, l'�tat physiologique des organismes conditionne l'observation de l'�mission de lumi�re pour certaines esp�ces. La photo-�mission s'amenuise parfois, suite � la capture, � la stimulation \textit{in situ} ou encore au changement de conditions (pression, temp�rature...) li�es � l'�chantillonnage \citep{herring1976}.\\

Lors de l'�mission de bioluminescence, l'observation et la mesure du spectre d'�mission chez les organismes marins a r�v�l� une large palette de couleurs dans le domaine du visible (\citealp{widder1983,haddock1999}, voir Figure \ref{couleurbiol}). Cependant, dans le milieu marin o� la longueur d'onde bleue est celle qui p�n�tre le plus profond�ment, la majorit� des organismes �mettent dans des valeurs proches de 475 nm. Par exemple, l'ensemble des bact�ries bioluminescentes �mettent � 490 nm. Pour des esp�ces c�ti�res et benthiques, il a �t� remarqu� un d�calage d'�mission dans les longueurs d'ondes du vert. D'autres longueurs d'ondes d'�mission de bioluminescence allant du violet au rouge ont �galement �t� observ�es.\\

\begin{figure}[!h]
\linespread{1} 
\centering
\includegraphics[scale=0.3]{biolum_spectra.jpeg}
\caption[Classification of the number of bioluminescent species depending on their wavelength emission for marine organisms.]{Classification of the number of bioluminescent species depending on their wavelength emission for marine organisms. Main bioluminescent species are emitting for wavelength between 450 and 520 nm, in blue-green emission, from \cite{widder2010}}
\label{couleurbiol}
\end{figure} 

\subsection{Une r�action chimique productrice de lumi�re}
%Les premi�res exp�riences scientifiques sur cette question sont dues � R.Boyle (1627-1691), chimiste anglais, membre du "coll�ge invisible", sorte d'association ax�e sur l'exp�rimentation scientifique � l'origine de la Royal Society britannique. Gr�ce � la construction d'une pompe � air avec R.Hooke (1635-1702), alors �l�ve � Oxford, qu'il avait remarqu� et recrut� comme assistant et qui fut l'un des premiers microscopistes, il d�montra que la bioluminescence ne pouvait avoir lieu en absence d'air. Il pensait que quelque chose dans l'air devait �tre indispensable pour cette r�action mais ne put d�montrer que l'air �tait un m�lange de diff�rents gaz. C'est un autre anglais, J. Priestley (1733-1804) qui d�couvrit "l'air d�phlogistiqu�" en 1774, c'est � dire l'oxyg�ne. 
\subsubsection*{R�action eucaryotique}

Les premi�res exp�riences men�es sur la bioluminescence sont associ�es � R. Boyle (1627-1691), un chimiste anglais, membre du "coll�ge invisible", et R. Hooke (1635-1702). Ces scientifiques observent alors que la bioluminescence ne peut avoir lieu en absence d'air, la composition des diff�rents gaz de l'air, et notamment la pr�sence d'oxyg�ne, n'�tant pas encore connu � cette �poque. Par la suite, en 1885, le biologiste fran�ais Raphael Dubois utilise pour l'�tude de la bioluminescence des col�opt�res de la famille des \textit{Elateridae} et du genre \textit{Pyrophorus} (Illiger, 1809), en provenance d'Am�rique centrale. Il pr�l�ve alors un des deux organes lumineux thoraciques d'un individu et le broie. Au bout d'un certain temps, sa lumi�re s'�teint. Le second organe est immerg� dans l'eau bouillante et s'�teint subitement. Lorsque R. Dubois broie ensemble les deux organes, la masse redevient lumineuse. Le ph�nom�ne est alors expliqu� par la pr�sence dans les organes pr�lev�s d'une substance (qu'il nomme lucif�rine) �mettant de la lumi�re, jusqu'� son oxydation compl�te, lorsque la r�action est activ�e par une enzyme diastase (la lucif�rase). La r�action lucif�rine-lucif�rase est une r�action de type substrat-enzyme. En pr�sence d'oxyg�ne, la lucif�rine va r�agir avec l'enzyme lucif�rase produisant une mol�cule d'oxylucif�rine et de la lumi�re (voir Figure \ref{oxyluc}; \citealp{shimomura2012}). La lucif�rine et la lucif�rase sont des termes g�n�riques, dont la composition des mol�cules peut diff�rer en fonction des esp�ces. Au sein de la diversit� du vivant 5 syst�mes lucif�rine-lucif�rase ont pu �tre diff�renci�s \citep{Haddock2010,wiles2005}. \\

\begin{figure}[!h]
\linespread{1} 
\centering
\includegraphics[scale=0.25]{reaclux.pdf}
\caption[Schematic reaction inducing bioluminescence.]{Schematic reaction inducing bioluminescence. The luciferine substrate interacts with ATP being modified into luciferyl-adenylate. The second step of the reaction is the oxydation (using molecular O$_2$) of luciferyl-adenylate into oxyluciferine by the luciferase enzymatic action. This excited molecule get back to a stable state with photon emission (bioluminescence).}
\label{oxyluc}
\end{figure} 

%\begin{figure}[!h]
%\linespread{1} 
%\centering
%\includegraphics[scale=0.3]{biolu_reaction.pdf}
%\caption[Chemical reaction inducing bioluminescence.]{Chemical reaction inducing bioluminescence. The luciferine substrate interacts with ATP being modified into luciferyl-adenylate with the production of an inorganic pyrophosphate (PPi). The second step of the reaction is the oxydation (using molecular O$_2$) of luciferyl-adenylate into oxyluciferine by the luciferase enzymatic action. This excited molecule get back to a stable state with photon emission (bioluminescence) and the production of both adenosine-5'-monophosphate (AMP) and CO$_2$.}
%\label{oxyluc}
%\end{figure} 

\subsubsection*{R�action bact�rienne}

Chez les Bacteria, les g�nes impliqu�s dans la r�action de bioluminescence sont organis�s en cluster dans lequel les g�nes \textit{lux} I, C, D, A, B, F, E ou G sont organis�s en op�ron (appel� op�ron \textit{lux}, voir Figure \ref{complexlux}).\\

\begin{figure}[!h]
\linespread{1} 
\centering
\includegraphics[scale=0.45]{lux-gene.pdf}
\caption[\textit{Lux}-gene organization for the bioluminescent \textit{Photobacterium phosphoreum} bacterial strain.]{\textit{Lux} genes organization for the bioluminescent \textit{Photobacterium phosphoreum} bacterial strains. Arrows represent the direction of transcription. Modified from \cite{dunlap2006}.}
\label{complexlux}
\end{figure} 

Les g�nes \textit{lux}A et \textit{lux}B codent pour deux sous-unit�s de la lucif�rase bact�rienne. Ces deux sous-unit�s alpha et beta composent la lucif�rase bact�rienne, une enzyme h�t�rodim�re de 77 kDa. Les g�nes \textit{lux}C, \textit{lux}D et \textit{lux}E de l'op�ron \textit{lux} codent pour le complexe de la r�ductase permettant la r�g�n�ration de l'aldh�hyde oxyd� par la lucif�rase au cours de la r�action substrat-enzyme (\citealp{ruby1976,meighen1988,meighen1991}). Le g�ne \textit{lux}F code pour une flavoprot�ine de 23 kDa. La s�quence de cette prot�ine poss�de environ 40\% d'homologie avec la partie carboxy-terminale de la sous-unit� \textit{lux}B, il semblerait donc que ce g�ne proviennent de la duplication du g�ne \textit{lux}B \citep{meighen1993,soly1988}. Le g�ne \textit{lux}F a �t� identifi� chez toutes les esp�ces bact�riennes marines en milieux m�so- et bathyp�lagiques. Bien que sa fonction ne soit pas encore d�montr�e, le g�ne \textit{lux}F ne serait cependant pas indispensable dans la r�action �mettrice de bioluminescence \citep{sung2004}. Enfin, l'op�ron \textit{lux} est r�gul� par le g�ne \textit{lux}I ainsi que par le g�ne \textit{lux}R, situ� en amont de l'op�ron. Le g�ne \textit{lux}I synth�tise un auto-inducteur, l'ac�tyl-homoserine lactone (AHL). La quantification de ces autoinducteurs va contr�ler et induire l'expression des g�nes de la bioluminescence (voir Figure \ref{quorum sensing} B et Figure \ref{qs2}, \citealp{meighen1993}). Le syst�me \textit{lux}I / \textit{lux}R va activer l'expression des g�nes \textit{lux} \citep{miller2001}.\\

La r�action bact�rienne (voir Figure \ref{luxbact}) induit l'oxydation d'une mol�cule de flavine mononucl�otide r�duite (FMNH$_2$) en tant que lucif�rine ainsi que d'une longue chaine d'aldh�hyde (RCHO). Cette r�action va produire une mol�cule de flavine mononucl�otide oxyd�e (FMN) ainsi qu'une longue chaine d'acides gras (RCOOH) avec la production de lumi�re \citep{hastings1986,meighen1988}. La lucif�rase contient deux prot�ines LuxA et LuxB (voir \citealp{stabb2005}, et Figure \ref{luxbact}). D'autres prot�ines telles que LuxC LuxD et LuxE sont responsables de la r�g�n�ration de l'ald�hyde alors que la prot�ine LuxG reduit NAD(P)H en FMN pour (r�)g�n�rer le FMNH$_2$.\\

\begin{figure}[!h]
\linespread{1} 
\centering
\includegraphics[scale=0.5]{stabb2005.pdf}
\caption[Biochemistry and physiology of reaction inducing \textit{Vibrio fischeri} bioluminescence and \textit{lux} genes involved into reactions are also ]{Chemical reaction inducing \textit{Vibrio fischeri} bioluminescence. Lux AB sequentially binds FMNH$_2$, O$_2$ and an aldehyde (RCHO) that are converted into an acid, FMN and water. Energy stored as ATP is consumed in regenerating the aldehyde substrate. Then, they are released from the enzyme with the concomitant production of light. From \cite{stabb2005}. }
\label{luxbact}
\end{figure} 

Une avanc�e dans l'�tude de ce ph�nom�ne fut la d�couverte par \cite{shimomura1962} d'une prot�ine (l'aequorine) chez une esp�ce de Cnidaire \textit{Aequorea}. Cette prot�ine avec l'addition de Ca$^{2+}$ entra�ne �galement une r�action intramol�culaire avec �mission de bioluminescence mais sans la pr�sence n�cessaire d'oxyg�ne, la quantit� de photons �mis �tant proportionnelle � la concentration en prot�ine. Une r�action similaire a pu �tre observ� chez \textit{Cheatopterus}, avec l'addition de Fe$^{2+}$. Ces nouveaux types de prot�ines ne correspondant pas aux lucif�rines et lucif�rases d�crites auparavant, le terme de photoprot�ine a �t� propos� (\citealp{Shimomura1969,Prasher1992}, voir Table\ref{avanc�es}). \\

\captionof{table}{Progress in research on bioluminescence, modified from \cite{shimomura2012} } \label{avanc�es} 
\begin{tabular}{ll}
\textbf{Date} & \textbf{Avanc�e}\\
\hline
1885 & D�couverte de la r�action lucif�rine-lucif�rase \\
1947 & N�cessit� de l'ATP pour la bioluminescence de la luciole\\
1954 & N�cessit� de la FMNH$_2$ dans la bioluminescence bact�rienne\\
1962 & D�couverte de l'aequorine\\
1966 & Concept de photoprot�ine\\
1974 & Identification d'une longue chaine d'ald�hyde chez les bact�ries luminescentes\\
1975 & D�couverte de la coelenterazine\\
1981 & D�couverte de la structure des autoinducteurs de la luminescence bact�rienne\\
1984-1985 & Clonage de la lucif�rase des lucioles\\
1985-1986 & Clonage de la lucif�rase bact�rienne\\
1996 & Structure de la lucif�rase bact�rienne\\
2005 & Structure de la lucif�rase des lucioles\\
\end{tabular}
\vspace{5mm}

\subsection{Observation de bioluminescence \textit{in situ}}
 \label{introbiolu}
 
\cite{bradner1987} classent les organismes bioluminescents selon deux groupes principaux. Le premier groupe est constitu� des bact�ries, produisant une �mission de lumi�re constante sans r�ponse � une stimulation ext�rieure. Les bact�ries bioluminescentes �mettent de la lumi�re lorsque les conditions de croissance sont favorables, en pr�sence d'oxyg�ne et lorsque le ph�nom�ne de quorum sensing active la r�action chimique de bioluminescence (voir paragraphe \ref{quorsens}). Cette bioluminescence bact�rienne n'est pas enregistrable par les d�tecteurs de bioluminescence d�velopp�s jusqu'alors. \\

Le second groupe d'organismes bioluminescents d�fini par \cite{bradner1987} est d'une grande diversit� phylog�n�tique et comprend des individus capables d'�mettre des flashes de lumi�re. Ces organismes sont luminescents uniquement apr�s stimulation m�canique ou "naturellement" suite � une stimulation biologique. Au sein de ce groupe, pour la majorit� des esp�ces pluricellulaires, la luminescence est contr�l�e par voie nerveuse. Au contraire, chez les organismes unicellulaires, tels que les Dinoflagell�s ou les Radiolaires, la bioluminescence est d�clench�e par une diff�rence de pression entrainant une d�formation de la surface cellulaire. Le processus m�cano-transducteur n'est pas enti�rement connu mais il est probable que le stimulus m�canique active des m�cano-r�cepteurs engendrant par la suite, un potentiel d'action du tonoplaste, conduisant � une acidification du cytoplasme (due � un flux de protons de la vacuole). Ces processus activeraient enfin la r�action chimique de bioluminescence \citep{Fritz1990}. L'instrumentation permettant la mesure de la bioluminescence marine, qui s'est largement d�velopp�e des ann�es 60 � nos jours (voir Table \ref{tab}), utilisent cette propri�t� de stimulation m�canique pour d�tecter ces organismes.\\

\begin{sidewaystable}
%\begin{table}
\center
\caption{Bathyphotometers developed for \textit{in situ} bioluminescence measurements using mechanical stimulation. Modified from \cite{herren2005}. D\string: diameter, V\string: volume. NA\string: Non Available value.} \label{tab} 
%\captionof{table}{Bathyphotom�tres d�velopp�s pour la mesure de la bioluminescence par excitation des organismes, modifi� de \citep{herren2005} } \label{tab\string: title} 
%\rotatebox{90}{
\begin{tabular}{lllllll}
\textbf{Source} & \textbf{D�ploiement} & \textbf{Excitation} & \textbf{flux} & \textbf{D (cm)} & \textbf{V (L)}\\
\hline
\cite{clark1965} & profil($\rightarrow$ 2,000 m) & turbine & 0.37 L s$^{-1}$ & 2.5 & NA\\
\cite{soli1966} & profils peu profonds & turbine et d�tecteur & variable & 2.54 & 0.1\\
\cite{seliger1970} & tract� & turbine & 0.2 L s$^{-1}$ & 1.3 & NA\\
\cite{hall1978} & profil ($\rightarrow$ 200 m) & turbulence, pompe & NA & NA & 0.025\\
\cite{aiken1984} & profil ($\rightarrow$ 1,000 m) & turbulence & 1-5 dm$^{3}$ s$^{-1}$ & 2.8 & 0.02\\
&&& � 5 m s$^{-1}$&\\
\cite{greenblatt1984} & profil & turbulence & 1.1 L s$^{-1}$ & 1.6 & 0.025\\
\cite{nealson1985} & profils ($\rightarrow$ 300 m) & 1 L s$^{-1}$ & 2.5 & 0.1\\
\cite{swift1985} & profil & turbine & 0.25 L $s^{-1}$ & 1.4 & NA\\
\cite{buskey1992} & profil & grille & 6.3 L s$^{-1}$ & NA & 4.7\\
\cite{widder1993} & profil & grille & 16-44 L s$^{-1}$  & 12 & 11.3\\
\cite{neilson1995} & ligne de mouillage & h�lice/flux naturel & 1-12 L s$^{-1}$ & 12.7 & 5\\

\cite{fucile1996} & profil (2 m s$^{-1}$) & grille & 15.7 L s$^{-1}$ & 10 & 2\\
\cite{geistdoerfer1999} & profil ($\rightarrow$ 600 m) & grille & 0.5 L s$^{-1}$ & 1.7 & 0.19\\
\cite{mcduffey2002} & � bord & turbulence & 1 L s$^{-1}$ & 1.3 & 0.049\\
\cite{bivens2002} & profil & turbulence & 1 L s$^{-1}$ & 1.5 & 0.025\\
\cite{herren2005} & multiplateforme & turbine & 0.5 L s$^{-1}$ & 3.2 & 0.5\\
\end{tabular}
\end{sidewaystable}
%\end{table}

\subsubsection*{Quantification de la bioluminescence eucaryotique}

Le terme de bioluminescence "non-stimul�e" ou "spontan�e" \citep{widder1989} a �t� remplac� par "naturelle", se r�f�rant a une r�action de bioluminescence engendr�e par un stimulus biologique entre organismes \citep{craig2011}. D'autres types de d�tecteurs visuels (cam�ras vid�o), sont utilis�s pour la d�tection automatis�e de cette bioluminescence dite "naturelle". La majorit� des observations ainsi que les hypoth�ses les plus courantes de la litt�rature estiment la fr�quence de ces �v�nements de bioluminescence tr�s faible \citep{priede2006,widder2002}. Cependant, ces fr�quences d'observations semblent controvers�es et d�pendantes de l'instrumentation d�velopp�e. Par exemple, \cite{gillibrand2007} mesurent la fr�quence de cette bioluminescence naturelle de l'ordre de 1 �v�nement h$^{-1}$ entre 2000 et 3000 m de profondeur. D'autres �tudes ont estim� cette bioluminescence de l'ordre de 0,12 �v�nement h$^{-1}$ � 2400 m de profondeur. Cependant, r�cemment, \cite{vacquie2012} ont d�termin� une fr�quence de ces �v�nements nettement sup�rieure, allant de 13 � 25 �v�nements min$^{-1}$ entre 600 m et 1000 m de profondeur. Cette �tude se base sur l'utilisation inattendue de photomultiplicateurs, conjointement � l'utilisation de balises ARGOS, enregistrant la bioluminescence au cours des plong�es d'�l�phants de mer (\textit{Mirounga leonina}). Par ailleurs, \cite{craig2011} proposent une relation lin�aire entre la profondeur et le nombre d'�v�nements par minute. Cette relation estim�e entre 1500 et 2750 m de profondeur montre toutefois une extr�me variabilit�, pouvant �tre li�e aux m�thodes d'�chantillonnage, dans un environnement tr�s h�t�rog�ne spatialement. Finalement, la litt�rature reste pauvre dans l'estimation de cette bioluminescence naturelle en milieu benthique et p�lagique, son importance �tant relativement peu connue. Ce manque d'information semble directement li� � l'instrumentation utilis�e pour estimer ce type de bioluminescence.\\

La bioluminescence marine est largement observ�e de la c�te vers le large et de la surface vers les profondeurs, o� des �missions de bioluminescence ont �t� observ�es jusqu'� 7500 m \citep{priede2006}. Dans le milieu bathyp�lagique (sup�rieur � 1000 m) la bioluminescence est la seule source de lumi�re visible, lui attribuant un r�le majeur dans la d�tection des organismes m�so- et bathyp�lagiques. Sur un profil vertical de la colonne d'eau, en Atlantique, la bioluminescence observ�e diminuerait lin�airement jusqu'� 2500 m puis se stabiliserait jusqu'� 4000 m \citep{geistdoerfer2001}. D'apr�s \cite{rudyakov1989}, au del� de 1000 m de profondeur, la majeure partie de la bioluminescence serait �mise par le m�soplancton (organismes de  0,2 � 20 mm de diam�tre).\\
 
\subsubsection*{Quantification de la bioluminescence bact�rienne potentielle}

D'apr�s une �tude de \cite{yetinson1979} en M�diterran�e orientale, la quantit� de bact�ries bioluminescentes cultivables (Colonie Formant des Unit�s CFU) le long de la c�te et � diff�rentes saisons a �t� estim� constante. A l'inverse, la diversit� bact�rienne bioluminescente dans la colonne d'eau varie. Les esp�ces bact�riennes marines potentiellement bioluminescentes sont\string:  \textit{Vibrio}, \textit{Photobacterium}, et \textit{Shewanella}, toutes appartenant � la sous-classe des Gammaproteobact�ries. Parmi ces esp�ces, \textit{Photobacterium phosphoreum} (Cohn 1878) serait la plus repr�sent�e en M�diterran�e \citep{gentile2009}. Elle est class�e parmi les Bacteria gram-n�gative, en forme de b�tonnet, chemoorganotrophe, non-sporulante, h�t�rotrophe et mobile avec 1 � 3 flagelles \citep{dunlap2006}. D'apr�s \cite{hastings1977}, \cite{dunlap1984} et \cite{makemson1986}, l'�mission de bioluminescence par cette souche bact�rienne est estim�e entre 10$^{3}$ et 10$^{4}$ photons s$^{-1}$ cellule$^{-1}$. Pour l'ensemble des esp�ces bact�riennes bioluminescentes, cette �mission peut varier de 1 � 10$^{5}$ photons s$^{-1}$cellule$^{-1}$ d'apr�s ces m�mes auteurs ou encore de 10$^{-2}$ � 10$^{4}$ photons s$^{-1}$ d'apr�s \cite{bose2008}, en fonction de la souche, de l'environnement ou de l'activation des g�nes \textit{lux}.\\

\cite{ruby1980} estiment entre 0,4 et 30 CFU dans 100 mL de la souche \textit{P. phosphoreum}, entre 100 et 1000 m en Atlantique, sans variations saisonni�res et avec peu d'autres esp�ces bact�riennes associ�es. De 4000 � 7000 m de profondeur, peu de cellules bact�riennes sont observables (< 0,1 CFU). En M�diterran�e, dans le d�troit de Sicile et en mer Ionienne, \textit{P. phosphoreum} repr�sente pr�s de 87\% des bact�ries bioluminescentes alors que \textit{Vibrio} et \textit{Shewanella} spp. ne sont rencontr�es qu'occasionnellement. \cite{gentile2009} ont estim� qu'en mer Tyrrh�nienne, les bact�ries isol�es � 500 m de profondeur sont principalement \textit{P. phosphoreum} alors qu'� 2750 m de profondeur, les bact�ries isol�es appartiennent uniquement � \textit{Photobacterium kishitanii} (taxonomiquement proche de \textit{P. phosphoreum}).\\

Cependant, \textit{P. phosphoreum} est d�crite comme une souche bact�rienne retrouv�e presque uniquement en association avec des poissons, principalement dans les intestins, et non dans les organes d�di�s � la luminescence \citep{herring1982}. Hastings et Marechal (non-publi�, voir \citealp{nealson1979}) cultivent sur bo�te de Petri les bact�ries issues de l'intestin de poissons d'eaux profondes de Sicile (Messine). De 40 � 100\% des CFU sont luminescentes et repr�sent�es par la seule esp�ce \textit{P. phosphoreum}. Une hypoth�se expliquant l'abondance de \textit{P. phosphoreum} dans la colonne d'eau est que les organismes, contenant dans leur tube digestif des bact�ries bioluminescentes, entraineraient un rejet constant de \textit{P. phosphoreum} dans le milieu par les f�ces et donc une augmentation des bact�ries libres ou attach�es � des particules dans la colonne d'eau.\\
 
 
\subsubsection*{Bioluminescence bact�rienne effective, communication "cell-to-cell"}
\label{baithyp}
La pr�sence de ces souches bact�riennes potentiellement bioluminescentes en milieu marin n'indique cependant pas directement une activit� lumineuse. En effet, la synth�se des produits de la r�action aboutissant � l'�mission de photons est contr�l�e par une autoinduction des g�nes responsables de la bioluminescence. Ce ph�nom�ne appel� "quorum sensing" a �t� d�crit historiquement chez \textit{Vibrio fischeri}, dans les ann�es 1970. Chaque cellule produit une quantit� de mol�cules, appel�es autoinducteurs (N-acylated homoserine lactone ou AHL pour \textit{Vibrio fischeri}) traversant les membranes cellulaires. L'accumulation d'autoinducteurs va permettre la transcription du g�ne \textit{lux}, la synth�se de lucif�rase et, par la suite, l'�mission de bioluminescence (\citealp{nealson1970,eberhard1972}, voir Figures \ref{qs2}et \ref{quorum sensing}). Cette forme de communication "cell-to-cell" est largement d�crite et associe ce fonctionnement entre cellules bact�riennes � celui d'un organisme multi-cellulaire.\\
 
 \begin{figure}[!h]
\linespread{1} 
\centering
\includegraphics[scale=0.4]{qs2.pdf}
\caption[Genetics and quorum sensing, from \cite{stabb2005}. ]{Genetics and quorum sensing, from \cite{stabb2005}. \textit{lux}I and \textit{lux}R are involved into the quorum sensing regulation. \textit{Lux}I  generates an autoinducer (AI) that interact with LuxR and stimulate the transcription of \textit{lux} genes.}
\label{qs2}
\end{figure} 

En milieu marin, ces autoinducteurs sont rapidement dilu�s et la concentration cellulaire n�cessaire pour l'�mission de lumi�re, estim�e au minimum entre 10$^8$ et 10$^9$ cells mL$^{-1}$, est rarement atteinte. Cependant, en symbiose dans les organes lumineux de certaines esp�ces, les bact�ries bioluminescentes peuvent atteindre une concentration de 10$^{11}$ cellules mL$^{-1}$. Une telle concentration peut �galement �tre atteinte par la colonisation bact�rienne de particules organiques \citep{hmelo2011}, ou encore de neige marine \citep{azam1998,alldredge1987} permettant une concentration suffisante d'autoinducteurs et par cons�quent l'�mission de bioluminescence (voir Figure \ref{quorum sensing}).\\
\begin{figure}[!h]
\linespread{1} 
\centering
\includegraphics[scale=0.35]{hmelo.pdf}
\caption[Quorum sensing representation modified from \cite{hmelo2011}.]{Quorum sensing representation. 1\string: Bacteria are attached to sinking particles. 2\string: Population grows and  the quorum sensing signal increases. 3\string: Quorum sensing signal reaches a threshold concentration. 4\string: Bacteria initiate a coordinated expression (bioluminescence for example). Modified from \cite{hmelo2011}. }
\label{quorum sensing}
\end{figure} 

Cependant, pour les Eukaryotes comme pour les bact�ries bioluminescentes, ces informations restent d�pendantes de l'instrumentation ou de la m�thodologie utilis�e. En effet, jusqu'alors, seule la bioluminescence stimul�e (second groupe d'organisme d�crit dans le paragraphe \ref{groupe}, \cite{bradner1987}) �tait quantifiable de fa�on instrument�e et automatis�e \textit{in situ}, alors que la quantification de bact�ries bioluminescentes �tait principalement effectu�e par pr�l�vements d'eau de mer, discrets dans le temps et par l'estimation des seules bact�ries cultivables (repr�sentant environ 1\% des bact�ries totales).\\

\setlength{\fboxsep}{3mm}
\fbox{\begin{minipage}{15cm }
\textbf{La mise en place de d�tecteurs automatis�s de bioluminescence "naturelle", le suivi de cette activit� biologique en milieu profond ainsi que la description de sa variabilit� au cours du temps restent � explorer.}
\end{minipage}}


%\begin{figure}[!ht]
%\begin{small}
%\begin{center}
%\includegraphics[scale=0.6]{luxoperon.pdf}
%\caption{\citep{craig2011} fig 3 }
%\label{complexlux}
%\end{center}
%\end{small}
%\end{figure} 
%
%\begin{figure}[!ht]
%\begin{small}
%\begin{center}
%\includegraphics[scale=0.6]{luxoperon.pdf}
%\caption{\citep{Haddock2010} fig 1 }
%\label{complexlux}
%\end{center}
%\end{small}
%\end{figure} 

\subsection{Sensibilit� de la bioluminescence aux variables environnementales}

\textbf{Turbulence et courant}\\
De nombreuses �tudes se sont tout d'abord int�ress�es � l'action de stimulations m�caniques sur les organismes bioluminescents planctoniques, et principalement sur les Dinoflagell�s, consid�r�s comme le groupe phylog�n�tique bioluminescent le plus abondant dans les eaux c�ti�res. \cite{cussatlegras2005} ont d�crit l'effet des stimulations m�caniques de l'�coulement de l'eau, \cite{latz1994} l'effet d'un flux laminaire, et \cite{rohr1998} les turbulences cr��es par la nage des dauphins sur la luminescence �mise par ces organismes. Malheureusement, peu d'�tudes ont quantifi� la stimulation m�canique n�cessaire pour activer la bioluminescence d'organismes du milieu profond. \cite{hartline1999} �tant un des rares auteurs a mesurer la force minimum requise provoquant la r�action de bioluminescence chez \textit{Pleuromamma xiphias}, une esp�ce m�sop�lagique de cop�pode.\\

L'�mission de bioluminescence chez les Dinoflagell�s a �t� largement decrite par \cite{cussatlegras2006}. Ces organismes sont m�caniquement stimul�s par des acc�l�rations ou une pression du fluide. L'auteur a montr� qu'un �coulement turbulent est efficace pour stimuler les organismes bioluminescents et de nombreux bathyphotom�tres utilisent cette propri�t� pour induire la bioluminescence \citep{losee1985, widder1993}. Un flux laminaire peut �galement stimuler les organismes. Pour les bathyphotom�tres, la plupart du temps, l'utilisation d'une grille permet la stimulation des organismes bioluminescents g�n�rant une turbulence uniforme et isotrope \cite{cussatlegras2006}. Suite � une stimulation m�canique, les Dinoflagell�s ont pu �tre observ�s avec l'�mission de lumi�re en moins de 20 ms et pendant une dur�e de 100 � 250 ms. Les Cop�podes, quant � eux, �mettent des flash de lumi�res mesur�s pendant 50 � 150 ms.\\ 

Il est a noter que les bact�ries bioluminescentes ne sont pas influenc�es par des stimulations m�caniques, le courant ou la turbulence \citep{bradner1987}.\\
\newpage
\textbf{Temp�rature}\\
L'action conjointe de la temp�rature associ�e � une stimulation m�canique a �t� appr�hend�e par une �tude de \cite{han2012}. Dans un large volume d'eau de mer, ces auteurs n'ont pu d�terminer de relation entre la temp�rature variant de 15,8 � 19,2\degres C et la bioluminescence �mise par \textit{Noctiluca sp.}. \cite{olga2012} teste l'effet de la temp�rature sur deux esp�ces de Ct�nophores. Cet auteur observe que l'amplitude et la dur�e d'�mission de bioluminescence, stimul�e chimiquement ou m�caniquement, est �galement influenc�e par la temp�rature du milieu avec un optimum � 22\degres C et 26\degres C pour \textit{Beroe ovata} et \textit{Mnemiopsis leidyi}, respectivement.\\

Pour les bact�ries bioluminescentes, l'effet seul de la temp�rature agit sur l'intensit� d'�mission de bioluminescence. Cependant, ce lien est parfois indirect, en effet, la temp�rature va agir sur la croissance des microorganismes et par cons�quent la bioluminescence mesur�e sera �galement modifi�e. Les temp�ratures limitant l'�mission de lumi�re s'av�rent variables en fonction des souches �tudi�es \citep{harvey1952}. Toutefois, la r�action chimique de bioluminescence serait limit�e par la temp�rature d'inactivation de la lucif�rase entre 30 et 35\degres C \citep{dorn2003}.\\

\textbf{pH}\\
Pour la bioluminescence bact�rienne, l'�tude de \cite{dorn2003} montre que la temp�rature et le pH du milieu peuvent justifier 98,1\% de la variation d'intensit� de bioluminescence sur un substrat de salicylate. Ces deux variables poss�dent donc une importance majeure d�montr�e en laboratoire. Les optima de pH sont variables en fonction des souches bact�riennes, alors que l'activation de la lucif�rase est effective pour des valeurs de 6,0 � 8,5 unit�s de pH. \cite{dorn2003} montrent qu'une faible variation, de l'ordre de 0,2 unit� de pH, peut impacter la bioluminescence.\\

\textbf{Salinit�}\\
Pour les souches bact�riennes marines, la salinit� des milieux de culture est g�n�ralement ajust�e avec une concentration en NaCl proche des conditions environnementales \citealp{lee2001,eley1972}). L'intensit� de bioluminescence est augment�e � cette concentration (30 g L$^{-1}$) en sel par rapport � une concentration de 10 g L$^{-1}$. En effet, avec une concentration ionique trop faible, la pression osmotique ne peut �tre maintenue entrainant une rupture de la membrane cellulaire \citep{vitukhnovskaya2001,nunes2003}. Par ailleurs, l'augmentation en poids atomique des anions halog�n�s tels que KCl, KBr et KI provoque une diminution de la bioluminescence bact�rienne \citep{gerasimova2002,kirillova2007}.\\

\textbf{Pression hydrostatique}\\
En milieu marin, la pression hydrostatique joue un r�le primordial avec une augmentation de 0,1 MPa tous les 10 m. Cependant, tr�s peu d'�tudes se sont int�ress�es � l'effet de la pression hydrostatique sur la bioluminescence. Ce param�tre pourrait pourtant fortement influencer des organismes bioluminescents planctoniques au cours de variations nycth�m�rales, ou lors de mouvements de masses d'eau.\\

Parmi les rares �tudes r�alis�es, \cite{strehler1954} montrent que la bioluminescence �mise par un extrait de \textit{Achromobacter fischeri} ou sur des cellules vivantes de \textit{Photobacterium phosphoreum} est li�e � l'effet conjoint de la temp�rature et de la pression. Une augmentation de pression variant de 0,1 � 55 MPa va augmenter l'activit� de bioluminescence pour une temp�rature sup�rieure � son optimum de croissance. A l'inverse cette m�me variation de pression va inhiber l'activit� de bioluminescence pour des temp�ratures plus faibles que cet optimum (Figure \ref{ueda}).\\

\begin{figure}[!h]
\linespread{1} 
\centering
\includegraphics[scale=0.4]{HP_biolu.pdf}
\caption[Effects of temperature and pressure conjointly on A) \textit{Achromobacter fischeri} extract and \textit{Photobacterium phosphoreum} living cells, from \cite{strehler1954} and B) firefly luciferase, modified from \cite{ueda1994}.]{Effects of temperature and pressure conjointly on A) \textit{Achromobacter fischeri} extract and \textit{Photobacterium phosphoreum} living cells, from \cite{strehler1954} and B) firefly luciferase, modified from \cite{ueda1994}.}
\label{ueda}
\end{figure} 

Plus r�cemment, \cite{ueda1994} se sont principalement int�ress�s � l'effet conjoint de la temp�rature et la pression sur la lucif�rase de la luciole (Figure \ref{ueda} B). Au cours de ces exp�rimentations, un m�lange de lucif�rase, de lucif�rine et d'ATP est utilis�. Ces auteurs ont d�montr� qu'une augmentation de pression, jusqu'� 40 MPa, augmente la bioluminescence � une temp�rature sup�rieure � la temp�rature optimale (22,5\degres C) et diminue cette bioluminescence � une temp�rature inf�rieure � la valeur optimale.\\

Dans une approche compl�tement diff�rente, \cite{watanabe2011} estiment l'effet de la pression hydrostatique sur l'esp�ce de Dinoflagell� \textit{Pyrocystis lunula} afin de d�terminer l'effet d'une projection des organismes dans les cassures de vagues (breaking waves), sur la bioluminescence. Ces auteurs appliquent un jet d'eau sur ces organismes pour simuler cet accroissement de pression. Ils d�montrent ainsi que la pression maximale impos�e implique une bioluminescence maximale.\\

\textbf{Bact�ries bioluminescentes et oxyg�ne}\\
\label{inhib}
Dans la r�action chimique d'�mission de bioluminescence, l'oxyg�ne joue un r�le primordial de donneur d'�lectron (voir Figure \ref{oxyluc}). La concentration en oxyg�ne dissous dans le milieu va donc intervenir dans la production de lumi�re de ces organismes. L'oxyg�ne serait attribu� d'abord � la cha�ne respiratoire, puis � la production de lumi�re. Par exemple, \cite{grogan1984} montre que lorsque le syst�me est limit� en oxyg�ne, il y a bien une chute de bioluminescence et que cette diminution est att�nu�e lorsque le syst�me respiratoire est bloqu� � l'aide d'inhibiteurs. Pour les esp�ces en symbiose, �tant limit�es en accroissement de biomasse, l'�nergie provenant des r�actions cataboliques est convertie en lumi�re \citep{bourgois2001}. Au final, le syst�me utilisant la lucif�rase pour �mettre de la bioluminescence est consid�r� comme une alternative dans le transport d'�lectrons. \cite{lloyd1985} ont pu observer qu'� la suite d'une longue p�riode en ana�robie, un retour � l'a�ration du milieu va entra�ner un pic de bioluminescence pendant quelques secondes. Ce pic est d� � l'accumulation du complexe lucif�rase-FMNH$_2$ et n'est pas observ� lorsque l'a�ration du milieu est progressive. D'apr�s \cite{nealson1979}, la quantit� de lucif�rase produite serait identique entre un milieu ana�robie et a�robie.\\

\cite{makemson1986} estime la consommation d'oxyg�ne par les souches bact�riennes bioluminescentes � 50-120 nmol d'O$_2$ min$^{-1}$ 10$^{9}$ cellules, pour \textit{Vibrio fischeri}, et 80-120 nmol d'O$_2$ min$^{-1}$ 10$^{9}$ cellules, pour \textit{Vibrio harveyi}. Ces valeurs varient de 10 � 15 nmol d'O$_2$ min$^{-1}$ 10$^{9}$ cellules pour \cite{karl1980} et peuvent atteindre 120 � 300 nmol d'O$_2$ min$^{-1}$ 10$^{9}$ cellules dans d'autres publications \citep{watanabe1975}.\\

L'utilisation de certains compos�s chimiques a permis de dissocier la part de consommation d'oxyg�ne attribu�e � la croissance bact�rienne et la part d�di�e � l'�mission de lumi�re. Parmi ces compos�s, le CCCP (carbonyl cyanide m-chlorophenylhydrazone), le KCN ou encore le cyanide ont �t� couramment utilis�s \citep{grogan1984,karl1980,makemson1986}. A partir de ces exp�rimentations, la part d'oxyg�ne consomm�e li�e � la bioluminescence est estim�e de 11 � 17\% sans diff�rence entre les diff�rentes esp�ces test�es. Ce pourcentage tomberait � 0,007\% pour les esp�ces non symbiotes. \cite{makemson1986} et \cite{nealson1979} estiment ce pourcentage � 12\%, pour \textit{Vibrio harveyi} et � 20\%, pour \textit{Vibrio fischeri}. Cependant, \cite{dunlap1984} estime cette part � seulement 3.4\%. \cite{hastings1975} estiment la production de 0,0001 � 0,1 photon par mol�cule d'oxyg�ne consomm�e. Cette production est �galement appel�e "quantum yield" \footnote{\textbf{Quantum yield:} le quantum yield \textit{in vivo} de la bioluminescence est le nombre de photons �mis par mol�cule d'O$_2$ utilis�e par la lucif�rase.}.\\
% dont le calcul est d�taill� en Annexe \ref{quantumyield}.\\

\textbf{Variables environnementales \textit{in situ}}\\
\textit{In situ}, peu d'�tudes ont tent� de corr�ler l'intensit� de la bioluminescence mesur�e avec les variables environnementales. \cite{lapota1989}, \cite{cussatlegras2001} et \cite{craig2010} mettent en �vidence une corr�lation entre la bioluminescence \textit{in situ} et la concentration en chlorophylle, rapprochant cette variable mesur�e � la biomasse de Dinoflagell�s photosynth�tiques. La quantification de bact�ries bioluminescentes a �t� corr�l�e � certaines variables environnementales telles que la temp�rature, la profondeur, la salinit�, les limitations en nutriments ou encore la sensibilit� � la photo-oxydation \citep{dunlap2006}. La caract�risation des relations entre variables environnementales et intensit� de bioluminescence mesur�e semble cependant tr�s diff�rente d'un site d'�tude � l'autre, avec une forte d�pendance � la profondeur d'�tude. De m�me, l'�chelle spatiale d'observation de ces corr�lations est �galement tr�s variable.\\

\setlength{\fboxsep}{3mm}
\fbox{\begin{minipage}{15cm }
\textbf{Les liens entre l'activit� de bioluminescence et les variables environnementales doivent continuer � �tre �tudi�s � la fois en milieu contr�l�, en laboratoire ainsi qu'\textit{in situ}. La mise en place, \textit{in situ}, de d�tecteurs de bioluminescence adapt�s � l'�chantillonnage de la bioluminescence "naturelle", associ�e � des capteurs d�di�s aux variables environnementales, permettra de d�terminer la dynamique de ce type d'organismes et leur sensibilit� aux fluctuations de l'�cosyst�me.}
\end{minipage}}

\subsection{R�les �cologiques en milieu marin}
\subsubsection*{Communication chez les Eucaryotes}
Divers organes lumineux ont �t� identifi�s chez les Eucaryotes bioluminescents\string: les scintillons (Dinoflagell�s), les photocytes (Cnidaires, Ct�nophores et Appendiculaires), ou encore les cellules s�cr�trices (Ostracodes) \citep{fogel1972,desa1968}. Les photocytes peuvent �tre r�partis dans l'organisme, ou bien regroup�s dans des organes lumineux appel�s photophores. Chez les Dinoflagell�s, les scintillons sont des v�sicules corticales. Ces organelles se d�tachent du cytoplasme vers la vacuole dans laquelle ils d�versent leur lucif�rine et la lucif�rase, entrainant la r�action d'�mission de bioluminescence. Certaines esp�ces ne poss�dent pas d'organes directement bioluminescents mais sont en symbiose avec des bact�ries bioluminescentes dans des organes sp�cialis�s \citep{ruby1976,dunlap2009,rader2012}. \\

L'�mission de bioluminescence est donc une r�action biologique consommatrice d'�nergie pour les organismes. La dur�e d'�mission de lumi�re sous forme de flashes peut aller de quelques millisecondes � plusieurs secondes et cette �mission est �mise de fa�on continue par les bact�ries bioluminescentes. Une question majeure dans l'�tude de ce ph�nom�ne est de comprendre son r�le et les b�n�fices obtenus par les organismes bioluminescents \citep{stabb2005}. La production de lumi�re a �t� propos�e comme �tant d�terminante dans des fonctions de communication, de pr�dation, de protection, ou de d�tection (\citealp{Haddock2010,widder2010,rivers2012}, Figure \ref{roles}). La bioluminescence peut �tre utilis�e alternativement pour plusieurs ou l'ensemble de ces fonctions selon les circonstances \citep{mesinger1992,fleisher1995,roithmayr1970}.\\

\begin{figure}[!h]
\linespread{1} 
\centering
\includegraphics[width=11.8cm]{roles.pdf}
\caption{Ecological roles of bioluminescence activity, from \cite{Haddock2010}}
\label{roles}
\end{figure} 

Un tout autre r�le de la bioluminescence est directement li� � la d�tection de cette lumi�re en milieu marin. En effet, il est surprenant de voir la port�e de la lumi�re �mise par ces organismes, dans un milieu ou la p�nombre est une caract�ristique majeure \citep{gillibrand2007}. D'apr�s \cite{priede2006} la bioluminescence associ�e aux "food falls" pourrait �tre observ�e jusqu'� une dizaine de m�tres de distance. Si cela semble relativement faible � l'�chelle de l'oc�an, dans un environnement oligotrophe, il s'agit tout de m�me d'une forte augmentation de la probabilit� de trouver une source nutritive, visible pour un grand nombre d'organismes "charognards" \citep{warrant2004,turner2009}. Une �tude r�cente d�montre �galement ce r�le de la bioluminescence. En effet, \cite{vacquie2012} observent que les plong�es des �l�phants de mer � des fins d'alimentation sont positivement corr�l�es aux enregistrements de lumi�re �mise par les organismes. Ces r�sultats ont montr� \textit{in situ} et jusqu'� 1050 m de profondeur, que la pr�sence de bioluminescence est effectivement un indicateur de proies potentielles pour les pr�dateurs, en milieu profond. \\

\subsubsection*{Communication chez les procaryotes}
\label{quorsens}
Concernant le r�le de la bioluminescence chez les bact�ries en symbiose avec certains organismes (poissons, calamars...) \citep{ruby1976}, le gain li� � l'�mission de lumi�re semble clair pour chacun des symbiotes. Les bact�ries fournissent � leur h�te la lumi�re n�cessaire � l'attraction de proies ou de partenaires ou encore � la fuite de pr�dateurs \citep{dunlap2009}. A l'inverse l'h�te offre aux bact�ries un environnement plus favorable � leur croissance (sources nutritives, temp�rature...). Dans le cas de bact�ries bioluminescentes non-symbiotiques, le r�le de la bioluminescence bact�rienne semble moins �vident. \\

L'hypoth�se majeure concernant le r�le des bact�ries bioluminescentes est directement li�e au cycle du carbone et appel�e "hypoth�se de l'app�t" ("bait hypothesis") \citep{hastings1977,robison1977,ruby1979,andrews1984}. En effet, les bact�ries bioluminescentes colonisent les particules chutant au travers de la colonne d'eau ainsi que les pelotes f�cales. La bioluminescence �mise entrainerait une plus forte probabilit� de d�tection visuelle des sources nutritives et donc d'ingestion de ces particules par des organismes zooplanctoniques. Les pelotes f�cales sont, en effet, plus concentr�es en min�raux essentiels n�cessaires � la croissance des organismes avec un rapport C\string: N\string: P de 22\string: 2,8\string: 1, d�montrant une forte concentration en �l�ments phosphor�s \citep{geesey1984}.\\

Cette hypoth�se a �t� observ�e par \cite{andrews1984} et r�cemment d�montr�e par \cite{zarubin2012}. \cite{andrews1984} observent que les pelotes f�cales des cop�podes ainsi que du mat�riel particulaire sont luminescent. Dans \cite{zarubin2012}, les auteurs d�crivent et d�montrent chacune des �tapes de cette th�orie. La bioluminescence est, tout d'abord, test�e comme un attracteur du zooplancton vers les agr�gats et la neige marine, puis la luminescence du zooplancton induite, suite � l'ingestion de ces bact�ries est d�montr�e. Enfin, la consommation du zooplancton devenu bioluminescent est quantifi�. Une fois dans le tractus digestif du zooplancton ou de poisson, les microorganismes trouvent un environnement favorable � leur d�veloppement ainsi qu'un moyen de dispersion efficace.\\ 

\begin{figure}[!h]
\linespread{1} 
\centering
\includegraphics[scale=0.4]{zarubin.pdf}
\caption[A) Fecal pellets produced after feeding on small colony fragments of the bioluminescent bacterium. B) Glow of zooplankton (\textit{Artemia salina}) after contacting and ingesting small particles broken off colonies of the bioluminescent bacterium \textit{Photobacterium leiognathi}.]{A) Fecal pellets of \textit{Artemia salina} produced after swimming and feeding on small colony fragments of the bioluminescent bacterium \textit{Photobacterium leiognathi} (visible in the background) photographed in room light (Left) and in darkness (Right) B) Glow of zooplankton (\textit{Artemia salina}) after contacting and ingesting small particles broken off colonies of the bioluminescent bacterium \textit{Photobacterium leiognathi}. The photograph on the left was taken in room light, and the photograph on the right was taken in darkness using long exposure (30 s). Scale bar\string: 1 cm. From \cite{zarubin2012}.}
\label{zarubin}
\end{figure} 

Une seconde hypoth�se sugg�re que le r�le de la bioluminescence bact�rienne est li� � l'intervention de la lucif�rase dans le transport d'oxyg�ne. D'une part, pour les bact�ries symbiotes pr�sentes dans les organes, la fonction m�tabolique de la bioluminescence serait une alternative au transport d'�lectron  dans ce milieu � faible teneur en oxyg�ne \citep{hastings1985,meighen1993}. D'autre part, \cite{baltar2013} indiquent qu'un stress oxidatif caus� par H$_2$O$_2$ affecterait la croissance des procaryotes ainsi que l'hydrolyse de la mati�re organique. Afin de contrer ce processus, la bioluminescence permettrait de detoxifier la mol�cule d'oxyg�ne par r�duction \cite{timmins2001}.\\

\section{R�flexion de l'�tude et objectifs}
\subsection{Des objectifs multidisciplinaires}  

Les objectifs de ce travail de th�se peuvent �tre d�finis selon deux questions scientifiques\string: \\

\textbf{\textit{(i)} La bioluminescence est d�crite comme "faible" en milieu profond, en comparaison � l'oc�an de surface, mais existe-t-il des variations d'intensit� lumineuse au cours du temps et comment les expliquer ?}\\

Le t�lescope � neutrinos ANTARES a �t� utilis� comme une sentinelle en grande profondeur de la bioluminescence marine. Sur ce site d'observation, pr�s de 850 photomultiplicateurs, dispos�s sur une surface de pr�s de 0.1 km$^2$, permettent la d�tection de bioluminescence, dite non-stimul�e, ou "naturelle", � haute fr�quence et en temps r�el depuis 2007. A proximit� imm�diate de ce r�seau, une ligne de mouillage permet l'�chantillonnage de variables environnementales. Le Chapitre 2 d�crit ces donn�es multivari�es. L'utilisation d'un indicateur biologique (la bioluminescence) en milieu profond, �chantillonn�e de fa�on automatis�e et � haute fr�quence sera valid� � la fin de ce Chapitre. Le Chapitre 3 propose diff�rentes m�thodes statistiques adapt�es � l'analyse des variations en temps-fr�quence, de s�ries temporelles environnementales, d�finies comme non-lin�aires et non-stationnaires. Dans cette partie, l'ensemble des organismes bioluminescents sera pris en compte.\\

\textbf{\textit{(ii)} En milieu profond, quelle est la part de la bioluminescence bact�rienne dans l'�mission de lumi�re \textit{in situ} ?}\\

La ph�nom�ne de bioluminescence chez les Eucaryotes est largement �tudi�. Cependant, en raison de la difficult� d'�chantillonnage et de l'absence d'instrumentation existant, jusqu'� pr�sent, pour d�tecter la bioluminescence bact�rienne, la part de ces organismes procaryotiques n'est que minoritairement pris en compte. En effet, la litt�rature propose peu d'�tudes caract�risant ou quantifiant \textit{in situ} les communaut�s bact�riennes capables de bioluminescence. De plus, ces �tudes sont souvent anciennes avec l'utilisation de techniques d'isolation ou de culture bact�rienne ne permettant qu'une d�tection partielle des souches potentiellement bioluminescentes (environ 1\% des souches bact�riennes sont d�finies comme cultivables). Dans d'autres �tudes, \cite{yetinson1979}, \cite{malave2010} et \cite{gentile2009} effectuent un �chantillonnage ponctuel dans le temps et dans l'espace, alors que \cite{asplund2011} s'int�ressent uniquement � l'oc�an de surface. La bioluminescence bact�rienne interviendrait pourtant dans le taux de remin�ralisation de la mati�re organique en milieu profond d'apr�s l'hypoth�se de l'app�t ("bait hypothesis", voir partie \ref{baithyp}), �voqu�e depuis une trentaine d'ann�es et r�cemment d�montr�e. Sa description ainsi que sa quantification semblent donc une approche int�ressante et encore trop peu d�velopp�e.\\
\newpage
Afin de r�pondre � la d�termination de la part de bioluminescence bact�rienne dans l'�mission de lumi�re, deux axes sont propos�s. Tout d'abord, des for�ages environnementaux peuvent ils vraisemblablement influencer la bioluminescence bact�rienne ? Le Chapitre 4 appr�hende l'effet de variables environnementales (pression hydrostatique, temp�rature, et concentration en carbone) sur la bioluminescence bact�rienne, � l'�chelle populationnelle. Ce travail en laboratoire, en milieu contr�l�, et avec l'utilisation d'une souche mod�le isol�e sur le site d'�tude (\textit{Photobacterium phosphoreum} ANT-2200), est une �tape interm�diaire permettant de r�pondre par la suite aux objectifs scientifiques \textit{in situ}. Enfin, le Chapitre 5 permettra de r�pondre partiellement � la part bact�rienne dans l'�mission de bioluminescence, � l'aide des r�sultats pr�liminaires d'un �chantillonnage effectu� au cours de l'ann�e 2011, � proximit� du site ANTARES, et � 2000 m de profondeur. Les communaut�s procaryotiques totales et bioluminescentes seront d�crites et quantifi�es � l'aide de m�thodes de biologie mol�culaire. La part de bact�ries bioluminescentes dans le signal d�tect� par les photomultiplicateurs sera �galement estim�e et discut�e.\\

La partie "Conclusions and perspectives" permettra une synth�se critique des objectifs et apports de ces travaux ainsi que les perspectives qui devront �tre d�velopp�es � la suite de ce travail.\\

\subsection{Une �tude � plusieurs �chelles}

L'�tude d'un �cosyst�me se confronte au choix d'une �chelle caract�ristique d'observation. L'observation de la dynamique des populations, des communaut�s ainsi que les processus les influen�ant n�cessite la prise en compte des variations dans le temps et dans l'espace afin de comprendre le ph�nom�ne observ� instantan�ment mais �galement sa dynamique spatio-temporelle. L'�chelle d'�tude choisie va d�finir la variabilit� associ�e aux processus observ�s \citep{hewitt2007}. L'�cosyst�me �tudi� �tant tri-dimensionnel (temps, variations spatiales: verticales et horizontales),  chacune de ces dimensions poss�de une variabilit� propre. Des �tudes �cologiques �voquent le concept de th�orie multi-�chelle, prenant en compte un ensemble de processus reli�s � diff�rentes �chelles \citep{legendre1997,anderson1998}.\\

Dans ce travail, nous tenterons de conserver un lien entre les diff�rentes �chelles des processus �tudi�s. Dans les Chapitre 2 et 3, les variations spatio-temporelles seront prises en compte au niveau �cosyst�mique (�chelle d'observation r�gionale). Le Chapitre 4 sera consacr� � une approche au niveau populationelle avec l'�tude d'une souche bact�rienne (micro�chelle d'observation). Enfin, le Chapitre 5 permettra une approche des communaut�s en interaction dans cet �cosyst�me (�chelle d'observation locale).\\
%\chapter[Introduction and objectives]{Introduction and objectives
\chaptermark{Introduction and objectives}}
\chaptermark{Introduction and objectives}
\minitoc

\newpage
\section{Marine bioluminescence}

The luminescence describes the production of cold light, by opposition to the incandescence. This term includes the phosphorescence, the fluorescence as well as the bioluminescence. The first two phenomena are an emission of light following the absorption of photons while the bioluminescence is the production of photons after a chemical reaction by living organisms.\\

The bioluminescence phenomenon is known since Antiquity. Aristote (348-322 BC) already reports light emission by dead fish. Thereafter, in the \textit{Naturalis Historia}, Pline (23-79 AD. JC) also describes various bioluminescent organisms such as glowing worms, "mushroom", or cnidarians. Moreover, in the History, marine bioluminescence, sometimes, played an unexpected role. Indeed, during the First World War, the submarine U-34, was the last German submarine sunk by the allies, identified by its luminescent wake. During the Second War, these wakes sometimes allowed the pilots to return to their aircraft carrier at night. Finally, during the Gulf War, american soldiers had to change the path followed by vessels to avoid the shoals of bioluminescent Dinoflagellates that might betray their presence.\\ 

\subsection{Phylogenetic diversity of bioluminescent organisms}
\label{groupe}
In terrestrial environments, few organisms have been described as bioluminescent. Conversely, in marine environments, nearly 90\% of organisms are estimated able of bioluminescence emission \citep{widder2010} and are extensively described in the literature (see Figure \ref{taxa}). Some non-planktonic groups (moving independently of water-mass movements), such as cephalopods (210 bioluminescent species) or telosteens (273 bioluminescent species), are well represented among bioluminescent organisms \citep{poupin1999,herring1987,herring1985}.\\

\begin{figure}[p]
\linespread{1} 
\centering
\includegraphics[width=14.5cm]{taxa.pdf}
\caption{Phylogenetic representation of the diversity of bioluminescent organisms, from \cite{Haddock2010}.}
\label{taxa}
\end{figure} 

\newpage
Planktonic phylogenetic groups, for which the largest number of bioluminescent species have been described, are\string: Bacteria, Dinoflagellates, Radiolarians, Cnidarians and Ctenophora, Crustaceans (Copepods, Amphipods, Euphausiids, Decapods), Chaetognathes and Tunicates. Bioluminescent species are also described within the Molluscs (Gastropods), Annelids (Polychaetes) and Echinodermata (Holothuroidea) (see Figure \ref{taxa}). Amongst prokaryotes\footnote{procaryote est un terme non-phylog�n�tique regroupant Bacteria et Archaea}, only Bacteria have been described as able to emit light. Archaea are, to our knowledge, not described as bioluminescent.\\

Bioluminescent bacterial strains have been described, from the beginnings of microbiology, with the work of B. Fischer in 1887, Mr. Beijerinck in 1889 and ZoBell in 1946. Luminescent bacteria\footnote{In the following manuscript, the term "bacteria" will be used for procaryotes} are Gammaproteobacteria, Gram-negative, non-spore forming, chemoorganotrophe, heterotrophic, and mostly aerobic. The known bioluminescent species belong to the genera \textit {Vibrio}, \textit{Photobacterium}, \textit{Shewanella} and \textit{Photorhabdus}. Amongst these genera, \textit{Photorhabdus} is the only terrestrial one, the others being present in marine environments.\\

To illustrate this diversity, \citep{poupin1999} proposed a non-exhaustive list in the Iroise sea, based on the literature. About 500 bioluminescent planktonic species, including six bacterial species, are described. Phylogenetically, the faculty of bioluminescence is randomly distributed in the living world \citep{herring1987}. Phylogenetic analyzes suggested that bioluminescence is a genetic trait that evolved independently between 30 and 50 times \citep{Haddock2010}. However, the authors of these phylogenetic syntheses, bringing together observations of bioluminescent organisms, highlight the difficulties of establishing such inventory. On the one hand, the techniques of \textit{in situ} observations of this phenomenon have changed in a few decades. These observations are constraint by the difficulty of capturing these bioluminescent organisms. On the other hand, the physiological state of the organisms determines the observation of light emission for some species. The photon emission sometimes decreases, due to the capture, to \textit{in situ} stimulation or to changing conditions (pressure, temperature ... ) after sampling \citep{herring1976}.\\

Upon bioluminescence emission, the observation and measurement of emission spectrum in marine organisms revealed a wide range of colors in the visible wavelength (\citealp{widder1983,haddock1999}, see Figure \ref{couleurbiol}). However, in the marine environment, where the blue wavelength penetrates more deeply, the majority of organisms emits at about 475 nm. For example, all bioluminescent bacteria emit at 490 nm. For coastal and benthic species, a shift was noticed in the emission wavelength to the green one. Other wavelengths of bioluminescence emission, from violet to red, have also been observed but less frequently.\\

\begin{figure}[!h]
\linespread{1} 
\centering
\includegraphics[scale=0.6]{spectre_biolu.pdf}
\caption[Classification of the number of bioluminescent species depending on their wavelength emission for marine organisms.]{Classification of the number of bioluminescent species depending on their wavelength emission for marine organisms. Main bioluminescent species are emitting for wavelength between 450 and 520 nm, in blue-green emission, from \cite{widder2010}}
\label{couleurbiol}
\end{figure}

\subsection{Chemical reaction producing light}
\subsubsection*{Eukaryotic reaction}

The first experiments on bioluminescence are associated to R. Boyle (1627-1691), an English chemist, member of the "invisible college", and R. Hooke (1635-1702). These scientists observed that bioluminescence can not take place in the absence of air (the gaseous composition of the air, and in particular the presence of oxygen, was not known at this time). Subsequently, in 1885, the French biologist Raphael Dubois used, for the study of bioluminescence, beetles of the family \textit{Elateridae} and gender \textit{Pyrophorus} (Illiger, 1809), from Central America. He took two bright thoracic organs of an individual and grounded them. After some time, the light turned off. The second body was immersed in boiling water and the light turned off suddenly. When R. Dubois grounded together the two bodies, the mass became luminous. The phenomenon was then explained by the presence, in the organs, of a substance, that he called luciferin, emitting light until its complete oxidation, when the reaction is activated by a diastase enzyme (luciferase). The luciferin-luciferase reaction is an enzyme-substrate type. In the presence of oxygen, luciferin will react with the enzyme luciferase producing a molecule of oxyluciferin and light (see Figure \ref{oxyluc}; \citealp{shimomura2012}). Luciferin and luciferase are generic terms. The composition of these molecules may differ depending on the species. Within the diversity of living systems, 5 luciferin-luciferase couples have been differentiated \citep{Haddock2010,wiles2005}.\\

\begin{figure}[!h]
\linespread{1} 
\centering
\includegraphics[scale=0.25]{reaclux.pdf}
\caption[Schematic reaction inducing bioluminescence.]{Schematic reaction inducing bioluminescence. The luciferin substrate interacts with ATP and is modified into luciferyl-adenylate. The second step of the reaction is the oxydation (using molecular O$_2$) of luciferyl-adenylate into oxyluciferin by the luciferase enzymatic action. This excited molecule get back to a stable state with photon emission (bioluminescence).}
\label{oxyluc}
\end{figure}

\subsubsection*{Bacterial reaction}
In Bacteria, the genes involved in the bioluminescence reaction are organized in cluster in which the \textit{lux} ICDABFEG genes are organized in operon (called \textit{lux} operon, see Figure \ref{complexlux}).\\

\begin{figure}[!h]
\linespread{1} 
\centering
\includegraphics[scale=0.45]{lux-gene.pdf}
\caption[\textit{Lux}-gene organization for the bioluminescent \textit{Photobacterium phosphoreum} bacterial strain.]{\textit{Lux} genes organization for the bioluminescent \textit{Photobacterium phosphoreum} bacterial strains. Arrows represent the direction of transcription. Modified from \cite{dunlap2006}.}
\label{complexlux}
\end{figure} 

\textit{lux}A and \textit{lux}B genes encode the two subunits of the bacterial luciferase, an heterodimer of 77 kDa (see \citealp{stabb2005}, Figure \ref{luxbact}). The \textit{lux}C, \textit{lux}D and \textit{lux}E genes encode the reductase complex allowing the regeneration of the aldehyde, oxidized during the enzyme-substrate reaction (\citealp{ruby1976,meighen1988,meighen1991}). The \textit{lux}F gene encodes a 23-kDa flavoprotein. The sequence of this protein has about 40\% homology to the carboxy terminus of the subunit \textit{lux}B. So it seems that this gene comes from gene duplication of \textit{lux}B \citep{meighen1993,soly1988}. The \textit{lux}F gene was only identified in marine bioluminescent bacterial species from meso- to bathypelagic environments. Although its function is not yet established, the \textit{lux}F gene appears not to be essential in the bioluminescence reaction \citep{sung2004}. Finally, the \textit{lux} operon is regulated by the \textit{lux}I and \textit{lux}R genes. The \textit{lux}I gene synthesizes an autoinducer, the acetyl-homoserine lactone (AHL) which will bind to the product of the \textit{lux}R gene. This \textit{lux}I / \textit{lux}R system will then activate the operon expression \citep{miller2001}. These genes are involved in the quorum-sensing response, for a low cell density, LuxI is produced at a basal level. When the population increases and the concentration of LuxI is high enough, it will bind to LuxR to activate the operon transcription and therefore, the production of bioluminescence (see Figure \ref{qs2} and Figure \ref{quorum sensing} B  \citealp{meighen1993}).\\

Bacterial reaction (see Figure \ref{luxbact}) induces the oxidation of one molecule of flavin mononucleotide reduced (FMNH$_2$) and the reduction of a long chain of aldehyde (RCHO). This reaction will produce a molecule of flavin mononucleotide oxidized (FMN) and a long fatty-acid chain (RCOOH) with the production of light \citep{hastings1986,meighen1988}. Other proteins such as LuxC and LuxD are responsible for the regeneration of the aldehyde. LuxG protein transfer the electron from NAD(P)H to FMN in order to regenerate FMNH$_ 2$.\\

\begin{figure}[!h]
\linespread{1} 
\centering
\includegraphics[scale=0.5]{stabb2005.pdf}
\caption[Biochemistry and physiology of reaction inducing \textit{Vibrio fischeri} bioluminescence and \textit{lux} genes involved into reactions are also ]{Chemical reaction inducing \textit{Vibrio fischeri} bioluminescence. Lux AB sequentially binds FMNH$_2$, O$_2$ and an aldehyde (RCHO) that are converted into an acid, FMN and water. Energy stored as ATP is consumed in regenerating the aldehyde substrate. Then, they are released from the enzyme with the concomitant production of light. From \cite{stabb2005}. }
\label{luxbact}
\end{figure} 
\vspace{10mm}
A breakthrough in the study of this phenomenon was discovered by \cite{shimomura1962} with the purification of a protein (aequorin) from a species of Cnidarian (\textit{Aequorea}). This protein, with the addition of Ca$^{2 +}$, also causes an intramolecular reaction with bioluminescence emission and without the necessary presence of oxygen. The amount of emitted photons is proportional to the protein concentration. A similar reaction has been observed in \textit{Cheatopterus} with the addition of Fe$^{2+}$. These new types of proteins do not correspond to the luciferin and luciferase previously described. Therefore, the generic term of photoprotein has been proposed (\citealp{Shimomura1969,Prasher1992}, see Table \ref{avanc�es}).\\

\captionof{table}{Progress in research on bioluminescence, modified from \cite{shimomura2012} } 
\label{avanc�es} 
\begin{tabular}{ll}
\textbf{Date} & \textbf{Advanced}\\
\hline
1885 & Discovery of the luciferin-luciferase system\\
1947 & ATP requirement for the firefly bioluminescence\\
1954 & FMNH$_2$ requirement for bacterial bioluminescence\\
1962 & Aequorine discovery\\
1966 & Photoprotein concept\\
1974 & Identification of a long chain of aldehyde in bioluminescent bacteria\\
1975 & Coelenterazine discovery\\
1981 & Discovery of the autoinducer structure in bacterial luminescence\\
1984-1985 & Cloning firefly luciferase\\
1985-1986 & Cloning bacterial luciferase\\
1996 & Bacterial luciferase structure\\
2005 & Firefly luciferase structure\\
\end{tabular}
\vspace{5mm}

\subsection{Bioluminescence \textit{in situ} observation}
\label{introbiolu}

\cite{bradner1987} classify bioluminescent organisms into two main groups. The first group is composed by bacteria capable of producing a constant light emission without response to external stimulation. Bioluminescent bacteria emit light when the growing conditions are favorable leading to activation of quorum-sensing phenomenon and in presence of oxygen (see paragraph \ref{quorsens}). This bacterial bioluminescence is not detectable using the bioluminescence sensors developed so far.\\

The second group of bioluminescent organisms defined by \citealp{bradner1987} includes a large phylogenetic diversity of individuals capable of emitting flashes. These organisms are luminescent "naturally" as a result of biological stimulation or only after mechanical stimulation. Within this group, for most multicellular species, luminescence is controlled by nerve. On the opposite, in unicellular organisms, such as Dinoflagellates or Radiolarians, the bioluminescence is triggered by a pressure differential, resulting in a  cell-surface deformation. Mechanical-transducer processes are not fully known. However, it is likely that the mechanical stimulus activates mechano-receptors causing, thereafter, a potential action to the tonoplast, leading to the acidification of the cytoplasm (due to the proton-flux vacuole). This pH reduction directly activates the chemical reaction of bioluminescence \citep{Fritz1990}. The instrumentation for measuring marine bioluminescence, which has greatly expanded from the 60s to the present day (see Table \ref{tab}), uses this mechanical-stimulation property to detect bioluminescent organisms.\\

\begin{sidewaystable}
%\begin{table}
\center
\caption{Bathyphotometers developed for \textit{in situ} bioluminescence measurements using mechanical stimulation. Modified from \cite{herren2005}. D\string: diameter, V\string: volume. NA\string: Non Available value.} \label{tab} 
%\captionof{table}{Bathyphotom�tres d�velopp�s pour la mesure de la bioluminescence par excitation des organismes, modifi� de \citep{herren2005} } \label{tab\string: title} 
%\rotatebox{90}{
\begin{tabular}{lllllll}
\textbf{Source} & \textbf{Deployment} & \textbf{Excitation} & \textbf{flux} & \textbf{D (cm)} & \textbf{V (L)}\\
\hline
\cite{clark1965} & profiler($\rightarrow$ 2,000 m) & impeller & 0.37 L s$^{-1}$ & 2.5 & NA\\
\cite{soli1966} & shallow profiler & impeller, detector & variable & 2.54 & 0.1\\
\cite{seliger1970} & towed & impeller & 0.2 L s$^{-1}$ & 1.3 & NA\\
\cite{hall1978} & profiler ($\rightarrow$ 200 m) & turbulence, pump & NA & NA & 0.025\\
\cite{aiken1984} & profiler ($\rightarrow$ 1,000 m) & turbulence & 1-5 dm$^{3}$ s$^{-1}$ & 2.8 & 0.02\\
&&& � 5 m s$^{-1}$&\\
\cite{greenblatt1984} & profiler & turbulence & 1.1 L s$^{-1}$ & 1.6 & 0.025\\
\cite{nealson1985} & profiler ($\rightarrow$ 300 m) & 1 L s$^{-1}$ & 2.5 & 0.1\\
\cite{swift1985} & profiler & impeller & 0.25 L $s^{-1}$ & 1.4 & NA\\
\cite{buskey1992} & profiler & inlet grid & 6.3 L s$^{-1}$ & NA & 4.7\\
\cite{widder1993} & profiler & inlet grid & 16-44 L s$^{-1}$ & 12 & 11.3\\
\cite{neilson1995} & Sea mooring & inlet propeller/surge & 1-12 L s$^{-1}$ & 12.7 & 5\\
\cite{fucile1996} & profiler (2 m s$^{-1}$) & inlet grid & 15.7 L s$^{-1}$ & 10 & 2\\
\cite{geistdoerfer1999} & profiler ($\rightarrow$ 600 m) & inlet grid & 0.5 L s$^{-1}$ & 1.7 & 0.19\\
\cite{mcduffey2002} & shipboard & turbulence & 1 L s$^{-1}$ & 1.3 & 0.049\\
\cite{bivens2002} & profiler & turbulence & 1 L s$^{-1}$ & 1.5 & 0.025\\
\cite{herren2005} & multiplateform & impeller & 0.5 L s$^{-1}$ & 3.2 & 0.5\\
\end{tabular}
\end{sidewaystable}
%\end{table}

\subsubsection*{Quantification of eukaryotic bioluminescence}

The term 'non-stimulated' or 'spontaneous' bioluminescence \citep{widder1989} was replaced by 'natural', referring to a bioluminescence reaction generated by stimulus of biological organisms \citep{craig2011}. Visual sensors (cameras) are used for the automatic detection of this bioluminescence called 'natural'. The majority of observations and the most common assumptions in the literature estimate very low frequency of these bioluminescence events \citep{priede2006,widder2002}. However, the frequency of observations seems controversial and dependent on the instrumentation developed. For example, \cite{gillibrand2007} measure the frequency of this spontaneous bioluminescence at about 1 event h$^{-1}$ between 2,000 and 3,000 m depth. Other studies have estimated the natural bioluminescence to the order of 0.12 event h$^{-1}$  at 2,400 m depth. However, recently, \cite{vacquie2012} determined a significantly higher frequency at about 13-25 events min$^{-1}$, between 600 and 1,000 m depth. This frequency was unexpectedly measured by photomultipliers, together with the ARGOS system, that were used in this study during dives of elephant seals (\textit{Mirounga leonina}). From their observations, \cite{craig2011} provide a linear relationship between the depth and the number of events per minute. This relationship estimated between 1,500 and 2,750 m depth, shows, however, extreme variability maybe due to sampling in spatially heterogeneous environment. Finally, the literature remains poor in the estimation of the non-stimulated bioluminescence in benthic and pelagic environments. Therefore, its importance is relatively unknown. This lack of information seems to be directly connected to the instrumentation used to estimate such bioluminescence.\\

Marine bioluminescence is widely observed from the coast to the open sea and from the surface to the deep, where bioluminescence emission was observed up to 7,500 m \citep{priede2006}. In the bathypelagic environment (deeper than 1,000 m), bioluminescence is the only source of visible light, giving it a major role in the detection of meso- and bathypelagic organisms. On a vertical profile of the water column in the Atlantic, the observed bioluminescence decreases linearly up to 2,500 m then, staying at a stable intensity up to 4,000 m \citep{geistdoerfer2001}. According to \cite{rudyakov1989}, beyond 1,000 m depth, most of the bioluminescence would be emitted by mesoplankton (0.2 to 20 mm of diameter).\\

\subsubsection*{Quantification of potentially bioluminescent bacteria}

According to \cite{yetinson1979}, in the eastern Mediterranean, the amount of cultivable bioluminescent bacteria (Unit Forming Colony UFC) along the coast and at different seasons is estimated constant. In contrast, the bioluminescent bacterial diversity in the water column varies. Potentially bioluminescent marine bacterial species are\string: \textit{Vibrio}, \textit{Photobacterium}, and \textit{Shewanella}, all belonging to the subclass of Gammaproteobacteria. Amongst these species, \textit{Photobacterium phosphoreum} (Cohn 1878) would  be the most represented in the Mediterranean \citep{gentile2009}. It is ranked among the gram-negative Bacteria, rod-shaped, chemoorganotrophe, non-sporulating, heterotrophic and mobile with 1-3 flagella \citep{dunlap2006}. According to \cite{hastings1977,dunlap1984} and \cite{makemson1986}, the emission of bioluminescence by this bacterial strain is estimated between 10$^3$ and 10$^4$ photons s$^{-1}$ cell$^{-1}$. For all bioluminescent bacterial species, these values may vary from 1 to 10$^5$ photons s$^{-1}$ cell$^{-1}$ according to the same authors, or 10$^{2}$ to 10$^4$ photons s$^{-1}$ according to \cite{bose2008}. It depends on the strain, the environment or the \textit{lux}-gene activation.\\

\cite{ruby1980} estimated that, from 100 to 1,000 m in the Atlantic, between 0.4 and 30 UFC, belonging to \textit{P. phosphoreum}, were found in 100 mL. They described little seasonal variation and few other bacterial species associated. From 4,000 to 7,000 m depth, few bacterial cells are observable (<0.1 UFC). In the Mediterranean, in the Strait of Sicily (Ionian Sea), \textit{P. phosphoreum} represents nearly 87\% of bioluminescent bacteria while \textit{Vibrio} and \textit{Shewanella} spp. are occasionally encountered. \cite{gentile2009} estimated that in the Tyrrhenian Sea, the isolated bacteria up to 500 m depth are mainly associated to \textit{P. phosphoreum}, whereas, up to 2,750 m depth, isolated bacteria belong only to \textit{Photobacterium kishitanii} (taxonomically close to \textit{P. Phosphoreum}).\\

\textit{P. phosphoreum} is described as a bacterial strain found almost only in combination with fish, mainly in the intestines, and not in the organs devoted to luminescence \citep{herring1982}. Hastings and Marechal (unpublished, see \citealp{nealson1979}) grew on Petri dish bacteria from the gut of Sicily-deep-water fish (Messina). Amongst these, 40 to 100 \% of UFC are luminescent and represented by the single species \textit{P. phosphoreum}. An hypothesis to explain the abundance of \textit{P. phosphoreum} in the water column is that this bacterium is the major organism in the fish digestive tract. The constant rejection of feces, in the environment, will lead to an increase, in the water column, in free-living or attached to the particles bacteria.\\

\subsubsection*{Effective bacterial bioluminescence, communication "cell-to-cell"}
\label{baithyp}
The presence of potentially bioluminescent bacterial strains in the sea does not indicate a bioluminescent activity. Indeed, the bioluminescence reaction is controlled by the autoinduction of genes. Each cell produces a quantity of molecules, called autoinducers (N-Acylated homoserine lactone or AHL for \textit{Vibrio fischeri}) and cross cell membranes. The accumulation of autoinducers will allow transcription of \textit{lux} gene, the synthesis of
luciferase and, thereafter, the emission of bioluminescence (\citealp{nealson1970,eberhard1972}, see Figures \ref{quorum sensing} and \ref{qs2}). This phenomenon, called "quorum sensing" has been described historically in \textit{Vibrio fischeri}, in the 1970s. This form of communication "cell-to-cell" is widely described and assimilated to the one of a multi-cellular organism.\\

\begin{figure}[!h]
\linespread{1} 
\centering
\includegraphics[scale=0.5]{qs2.pdf}
\caption[Genetics and quorum sensing, from \cite{stabb2005}. ]{Genetics and quorum sensing, from \cite{stabb2005}. \textit{lux}I and \textit{lux}R are involved into the quorum sensing regulation. The autoinducer (AI) LuxI interacts with LuxR and leading to the stimulation of \textit{lux} genes.}
\label{qs2}
\end{figure} 

In the marine environment, these autoinducers are quickly diluted and the cell concentration necessary to the light emission (estimated between 10$^8$ and 10$^9$ cells mL$^{-1}$) is rarely achieved. However, in the light organs of some species, bioluminescent bacteria can reach a concentration of 10$^{11}$ cells mL$^{-1}$. Such concentration can also be reached by
bacterial colonization of organic particles \citep{hmelo2011}, or marine snow \citep{azam1998,alldredge1987} leading to a sufficient concentration of autoinducer and consequently the emission of bioluminescence (see Figure \ref{quorum sensing}).\\

\begin{figure}[!h]
\linespread{1} 
\centering
\includegraphics[scale=0.4]{hmelo.pdf}
\caption[Quorum sensing representation modified from \cite{hmelo2011}. ]{Quorum-sensing representation. 1\string: Bacteria are attached to sinking particles. 2\string: Population grows and the quorum-sensing signal increases. 3\string: Quorum-sensing signal reaches a threshold concentration. 4\string: Bacteria initiate a coordinated expression (bioluminescence for example). Modified from \cite{hmelo2011}.}
\label{quorum sensing}
\end{figure} 

Finally, all the informations in these studies, for both Eukaryotes and bioluminescent bacteria, remain dependent on instrumentation or methodology. Indeed, so far, only the stimulated bioluminescence (second group organization described in paragraph \ref{groupe}, \citealp{bradner1987}) was quantifiable automatically using \textit{in situ} instrumentation.  On the contrary, the quantification of bioluminescent bacteria has to be carried out through the discrete sampling of sea water, allowing only the quantification of cultivable bacteria (estimated to be about 1\% of total bacteria).\\

\setlength{\fboxsep}{5mm}
\fbox{\begin{minipage}{15cm }
\textbf{The introduction of automated detectors for non-stimulated bioluminescence, the monitoring of biological activity into the deep sea and the description of the variability over time remain to be explored.}
\end{minipage}}

\subsection{Sensitivity of bioluminescence to environmental variables}

\textbf{Turbulence and current}\\
Many studies focused on the action of mechanical stimulation on the bioluminescent planktonic organisms (mainly Dinoflagellates, considered as the most abundant phylogenetic group in bioluminescent coastal waters). \cite{cussatlegras2005} describe the effect of mechanical flow stimulation of water. \cite{latz1994} describe the effect of a laminar flow. Then \cite{rohr1998} describe effects of the turbulence created by the swimming dolphins on the luminescence emitted by these organisms. Unfortunately, few studies have quantified mechanical stimulation that is necessary to stimulate bioluminescence in deep-sea animals. However, \cite{hartline1999} measure the minimum force required to elicit bioluminescence reaction from \textit{Pleuromamma xiphias}, a mesopelagic copepod species.\\

Bioluminescence emission for Dinoflagellates has been described in \cite{cussatlegras2006}. These organisms are mechanically stimulated using accelerations or pressure from fluids. A turbulent flow is efficient to stimulate bioluminescent organisms and various bathyphotometers use this principle \citep{losee1985, widder1993} for the light measurement. Laminar flux can also stimulate organisms. For bathyphotometers, most of the time, a grid stimulates organisms by generating a uniform and isotropic turbulence \cite{cussatlegras2006}. Under mechanical stimulation, Dinoflagellates have been observed to emit light within less than 20 ms and with the light emission of a flash lasting 100 ms to about 250 ms. Copepods flash emission was measured during 50 to 150 ms.\\

It is worth noting that the bioluminescence in bacteria is not influenced by mechanical stimulation, current or turbulence \citep{bradner1987}.\\

\textbf{Temperature}\\
The joint action of temperature associated with mechanical stimulation was apprehended by \cite{han2012}. In a large volume of seawater, these authors were unable to determine the relationship between temperature, ranging from 15.8 to 19.2\degre C, and bioluminescence emitted by \textit{Noctiluca} sp.. \cite{olga2012} tested the effect of temperature on two species of Ctenophora. The author observes that the amplitude and duration of bioluminescence, chemically or mechanically stimulated, is also affected by the medium temperature with an optimum at 22 \degres C and 26 \degres C, for \textit{Beroe ovata} and \textit{Mnemiopsis leidyi}, respectively.\\

For bioluminescent bacteria, temperature has, by itself, an influence on the intensity of bioluminescence emission. However, this link is sometimes indirect since temperature will act on the growth of microorganisms and therefore impact the measure of bioluminescence. Temperatures limiting the emission of light are variable, depending on the studied bacterial strains \citep{harvey1952}. However, the chemical bioluminescence reaction is limited by the inactivation temperature of luciferase, between 30 and 35\degres C \citep{dorn2003}.\\

\textbf{pH}\\
For bacterial bioluminescence, the study of \cite{dorn2003} shows that temperature and pH of the medium can justify 98.1\% of the intensity variation for bacteria grown on a salicylate substrate. consequently, these two variables have a major importance. The pH optima are variable depending on the bacterial strains. It is worth noting that the activation of luciferase is effective only between 6.0 and 8.5 pH units, outside this range, this system will not be activated. Moreover, \cite{dorn2003} show that a small variation, of about 0.2 pH units, can impact bioluminescence.\\

\textbf{Salinity}\\
For marine bacterial strains, culture-medium salinity is generally based on NaCl concentration similar to environmental conditions \citep{lee2001,eley1972}. The intensity of bioluminescence increases at this concentration (30 g L$^{-1}$) compared to a concentration of about 10 g L$^{-1}$. Indeed, with the ionic concentration too low, the osmotic pressure can not be maintained, leading to a disruption of the cell membrane \citep{vitukhnovskaya2001,nunes2003}. Furthermore, the increase in atomic weight of halogen anions, such as KCl , KI and KBr, causes lowering of the bacterial bioluminescence \citep{gerasimova2002,kirillova2007}.\\

\textbf{Hydrostatic pressure}\\
In the marine environment, the hydrostatic pressure plays an important role with an increase of 0.1 MPa every 10 m. However, very few studies have examined the effect of hydrostatic pressure on bioluminescence. This parameter could strongly influence planktonic bioluminescent organisms in nychtemeral variations, or when water-masses movements occur.\\

Amongst the few studies, \cite{strehler1954} show that the bioluminescence, emitted by an extract of \textit{Achromobacter fischeri} or on living cells of \textit{Photobacterium phosphoreum}, is related to the combined effect of temperature and pressure. An increase in pressure, ranging from 0.1 to 55 MPa, will increase bioluminescence activity for a temperature higher than the optimum. In contrast the same variation in pressure will inhibit the bioluminescence activity at lower temperatures (Figure \ref{ueda}).\\

More recently, \cite{ueda1994} were interested in the joint effect of the temperature and pressure of the firefly luciferase (Figure \ref{ueda} B). In these experiments, a mixture of luciferase, luciferin and ATP is used. These authors have shown that an increase in pressure up to 40 MPa, increases bioluminescence at a temperature above the optimal temperature (22.5\degres C) and decreases it at temperatures below the optimum value.\\ 

\begin{figure}[!h]
\linespread{1} 
\centering
\includegraphics[scale=0.45]{HP_biolu.pdf}
\caption[Effects of temperature and pressure conjointly on A) \textit{Achromobacter fischeri} extract and \textit{Photobacterium phosphoreum} living cells, from \cite{strehler1954} and B) firefly luciferase, modified from \cite{ueda1994}.]{Effects of temperature and pressure conjointly on A) \textit{Achromobacter fischeri} extract and \textit{Photobacterium phosphoreum} living cells, from \cite{strehler1954} and B) firefly luciferase, modified from \cite{ueda1994}.}
\label{ueda}
\end{figure} 

In a completely different approach, \cite{watanabe2011} estimate the effect of hydrostatic pressure on the Dinoflagellate \textit{Pyrocystis lunula} to determine the effect of breaking waves on bioluminescent organisms. The authors apply, to simulate the effect of increasing pressure, a water jet on a wall. They demonstrate that the imposed maximum pressure leads to a maximum bioluminescence.\\

\textbf{Bioluminescent bacteria and oxygen}\\
\label{inhib}
In the chemical reaction of bioluminescence, oxygen plays a major role of electron donor (see Figure \ref{oxyluc}). The concentration of dissolved oxygen in the medium interacts with the light emission of these organisms. Oxygen would be assigned, at first, to the respiratory chain and then to the light emission. As an example, \cite{grogan1984} shows that when the oxygen system is inhibited, there was a drop in bioluminescence and that this decrease is attenuated when the respiratory system is blocked with inhibitors. For species in symbiosis, with limited possibility to increase in biomass, energy from catabolic reactions is converted into light \citep{bourgois2001}. Finally, the system, using luciferase to emit bioluminescence, is considered as an alternative electron transport. \cite{lloyd1985} observed that after a long period in anaerobiosis, the aeration of the medium will result in a bioluminescence peak for a few seconds. This peak is due to the accumulation of luciferase-FMNH$_2$ complex and is not observed when aeration medium is progressive. According to \cite{nealson1979} the amount of luciferase produced is identical between anaerobic and aerobic conditions.\\

\cite{makemson1986} estimates the oxygen consumption by the bioluminescent bacterial strains to be 50-120 nmol O$_2$ min$^{-1}$ 10$^{9}$ \textit{Vibrio fischeri} cells and 80-120 nmol O$_2$ min$^{-1}$ for 10$^{9}$ \textit{Vibrio harveyi} cells. These values range from 10 to 15 nmol O$_2$ min$^{-1}$ for 10$^{9}$ cells \citep{karl1980} and can reach 120 to 300 nmol O$_2$ min$^{-1}$ for 10$^{9}$  cells in other publications \citep{watanabe1975}.\\

The use of chemicals permit to discriminate the oxygen consumption attributed to the formation of bacterial growth and oxygen consumption dedicated to the issue of light. Amongst these compounds, the CCCP (Carbonyl Cyanide m-Chlorophenylhydrazone), the KCN or the cyanide were commonly used \citep{grogan1984,karl1980,makemson1986}. From these experiments, the proportion of oxygen consumed for bioluminescence is estimated to be about 11 to 17\% with no difference between the various bacterial species tested. This percentage falls to 0.007\% for non-symbiotic species. \cite{makemson1986} and \cite{nealson1979} estimate this percentage to be 12\% for \textit{Vibrio harveyi} and 20\% for \textit{Vibrio fischeri}. However, \cite{dunlap1984} estimates this percentage as low as 3.4 \%. \cite{hastings1975} estimate the production of 0.0001 to 0.1 photon per molecule of oxygen consumed. This production is also called 'quantum yield' \footnote{\textbf{Quantum yield:} the quantum yield \textit{in vivo} of bioluminescence is the number of photons emitted by O$_2$ molecule and used by the luciferase.}.\\
%, the calculation of which is detailed in Appendix \ref{quantumyield}.\\

\textbf{\textit{In situ} environmental variables}\\
\textit{In situ}, few studies have attempted to correlate the intensity of measured bioluminescence with environmental variables. \cite{lapota1989,cussatlegras2001} and \cite{craig2010} show a correlation between the \textit{in situ} bioluminescence and the chlorophyll concentration, using the later as a proxy for photosynthetic-Dinoflagellates biomass. Quantification of bioluminescent bacteria was correlated with certain environmental variables such as temperature, depth, salinity, nutrient limitations or sensitivity to photo-oxidation \citep{dunlap2006}. However, the characterization of the relationship between environmental variables and bioluminescence intensity, seems very different from one study to another, with a strong dependence on the depth of the studied site. Similarly, the spatial scale of observation of these correlations is also very variable.\\

\setlength{\fboxsep}{5mm}
\fbox{\begin{minipage}{15cm }
\textbf{The links between the bioluminescence activity and environmental variables still need to be studied in a controlled environment laboratory or \textit{in situ}. The establishment of \textit{in situ} bioluminescence sensors, suitable for sampling the 'natural' bioluminescence, combined with sensors, dedicated to environmental variables, will determine the dynamics of these organisms and their sensitivity to the ecosystem changes.}
\end{minipage}}

\subsection{Ecological roles in the marine environment}
\subsubsection*{Communication between Eukaryotes}
\label{quorsens}
Various light organs were identified in the bioluminescent eukaryotes\string: the scintillons (Dinoflagellates), the photocytes (Cnidaria, Ctenophora and Appendicularia) or the secretory cells (Ostracods) \citep{fogel1972,desa1968}. The photocytes can be distributed all over the body, or regrouped in light organs called photophores. In Dinoflagellates, the scintillons are cortical vesicles. These organelles migrate from the cytoplasm to the vacuole in which they discharge their luciferin and luciferase, responsible for the bioluminescence reaction. Some species are not bioluminescent by themselves but are symbiotes with bioluminescent bacteria, that they accumulate in specialized organs \citep{ruby1976,dunlap2009,rader2012}.\\

The emission of bioluminescence is an energy consuming biological reaction for organisms. Indeed, the light emission as flashes have duration time from few milliseconds to several seconds. This light is even emitted continuously for bioluminescent bacteria. A major question in the study of this phenomenon is to understand its role and the benefits obtained by bioluminescent organisms \citep{stabb2005}. The light production has been proposed as being necessary to communication, predation, protection, and detection \citep{Haddock2010,widder2010,rivers2012}, Figure \ref{roles}). Bioluminescence can alternatively be used for several or all of these functions depending on the circumstances \citep{mesinger1992,fleisher1995,roithmayr1970}. \\

\begin{figure}[!h]
\linespread{1} 
\centering
\includegraphics[width=11.8cm]{roles.pdf}
\caption{Ecological roles of bioluminescence activity, from \cite{Haddock2010}}
\label{roles}
\end{figure} 

A different role in bioluminescence is directly related to the detection of this light in marine environment. Indeed, it is surprising to see the extent of the light emitted by these organisms, in an environment where the darkness is a major feature \citep{gillibrand2007}. According to \cite{priede2006}, bioluminescence associated with 'food falls' could be observed up to ten meters away. If this sounds relatively low across the ocean, in an oligotrophic environment, it is still a significant increase in the probability of finding a source of nutrition, visible to a large number of 'scavenger' organisms \citep{warrant2004,turner2009}. A recent study also demonstrates the role of bioluminescence. Indeed, \cite{vacquie2012} found that the dives of elephant seals for food are positively correlated with recordings of the light emitted by organisms. These results have been demonstrated \textit{in situ} and up to 1,050 m depth, showing that the presence of bioluminescence is actually an indicator of potential prey for predators in the deep sea.\\

\subsubsection*{Communication in Bacteria}
\label{quorsens}
Regarding the role of bioluminescence for symbiotic bacteria with certain organisms (fish, squid ...) \citep{ruby1976}, the gain related to the light emission seems clear for each symbiont. Bacteria provide the host the necessary light for attracting preys or partners or to escape predators \citep{dunlap2009}. The bacterial host provides an environment more favorable to their growth (nutrient sources, temperature ...). In the case of non-symbiotic bioluminescent bacteria, the role of bacterial bioluminescence is less obvious.\\

The major hypothesis concerning the role of bioluminescent bacteria is directly related to the carbon cycle and called "bait hypothesis" \citep{hastings1977,robison1977,ruby1979,andrews1984}. Indeed, bioluminescent bacteria colonize particles falling through the water column and the fecal pellets. The bioluminescence emitted would lead to a higher probability of visual detection of nutrient sources and therefore to the ingestion of these particles by zooplankton. The fecal pellets are more concentrated in essential minerals needed for the growth of organisms with a C\string: N\string: P ratio of 22\string: 2.8\string: 1, showing a high concentration of phosphorous components \citep{geesey1984}.\\

This hypothesis was observed by \cite{andrews1984} and recently demonstrated by \cite{zarubin2012}. \cite{andrews1984} found that fecal pellets from copepods and particulate materials are luminescent. In \cite{zarubin2012}, the authors described and demonstrated each step of this theory. First of all, bioluminescence was tested as an attractor to the zooplankton. Then, the luminescence of zooplankton following ingestion of these bacteria was demonstrated. Finally, the consumption of the newly bioluminescent zooplankton was quantified. Once in the digestive tract of zooplankton and fish, microorganisms found favorable conditions to their development, as well as a means of effective dispersion.\\

\begin{figure}[!h]
\linespread{1} 
\centering
\includegraphics[scale=0.45]{zarubin.pdf}
\caption[A) Fecal pellets produced after feeding on small colony fragments of the bioluminescent bacterium. B) Glow of zooplankton (\textit{Artemia salina}) after contacting and ingesting small particles broken off colonies of the bioluminescent bacterium \textit{Photobacterium leiognathi}.]{A) Fecal pellets of \textit{Artemia salina} produced after swimming and feeding on small colony fragments of the bioluminescent bacterium \textit{Photobacterium leiognathi} (visible in the background) photographed in room light (Left) and in darkness (Right) B) Glow of zooplankton (\textit{Artemia salina}) after contacting and ingesting small particles broken off colonies of the bioluminescent bacterium \textit{Photobacterium leiognathi}. The photograph on the left was taken in room light, and the photograph on the right was taken in darkness using long exposure (30 s). Scale bar\string: 1 cm. From \cite{zarubin2012}.}
\label{zarubin}
\end{figure} 

A second hypothesis suggests that the role of bacterial bioluminescence is related to the response of the luciferase in the transport of oxygen. On the one hand, \citep{hastings1985,meighen1993} show that, for bacterial symbionts in organs, metabolic function of bioluminescence is an alternative to the electron transport in this low oxygen medium. On the other hand, \cite{baltar2013} indicate that an oxidative stress caused by H$_2$O$_2$ might affects prokaryotic growth and hydrolysis of specific components of the organic matter pool. To counteract this process, the bioluminescence reaction could detoxifies molecular oxygen by its reduction \cite{timmins2001}.\\
\vspace{0.5mm}
\section{Afterthought and objectives of the study}
\subsection{Multidisciplinary objectives}

The objectives of this thesis can be defined following two scientific questions\string: \\

\textbf{\textit{(i)} Bioluminescence is described as "weak" in the deep sea, compared to the ocean surface, but are there variations in light intensity over time and how to explain them?}\\

The ANTARES neutrino telescope was used as a sentinel, in the deep sea, for marine bioluminescence. This observation site, with about 850 photomultipliers, on a surface of nearly 0.1 km$^2$, allows the continuous detection of non-stimulated bioluminescence, at high frequency and in real time, since 2007. At immediate vicinity of this network, a mooring line is dedicated to the sampling of  environmental variables. Chapter 2 describes the multivariate dataset. The use of a biological indicator (bioluminescence) in the deep sea, sampled automatically and at high frequency, will be validated at the end of this Chapter. Chapter 3 provides two different statistical methods appropriated to the analysis of changes in time and frequency of environmental time series, defined as non-linear and non-stationary. The work proposed in these two chapters allows understanding for oceanographic data at the ecosystem scale. In this section, all bioluminescent organisms will be considered.\\

\textbf{\textit{(ii)} In the deep sea, what is the part of bacterial bioluminescence in the emission of light \textit{in situ}?}\\

The phenomenon of bioluminescence in Eukaryotes is widely studied. However, because of sampling difficulties and the lack of existing instrumentation, bacterial bioluminescence has only been poorly considered, so far. Indeed, the literature offers few studies characterizing or quantifying \textit{in situ} bioluminescent bacterial communities. In addition, these studies are often old with the use of isolation techniques or bacterial culture allowing only partial detection of potentially bioluminescent strains (about 1\% of the bacterial strains is defined as cultivable). In other studies, \cite{yetinson1979}, \cite{malave2010} and \cite{gentile2009} perform a one-time sampling in time and space while \cite{asplund2011} focus only on the ocean surface. Bacterial bioluminescence would be involved in the remineralization rate of organic matter in the deep sea according to the 'bait hypothesis' (see section \ref{baithyp}), existing for thirty years but only recently demonstrated. Its description and its quantification seem an interesting approach not enough developed.\\

To reach the understanding of the part of bacterial bioluminescence into light emission, two axes are proposed. First, environmental forcings, are they likely to influence the bacterial bioluminescence ? Chapter 4 develops the effects of environmental variables (hydrostatic pressure, temperature, and carbon concentration) on bacterial bioluminescence, at the population scale. This laboratory work has been performed in a controlled environment, and with the use of a model strain isolated at the study site (\textit{Photobacterium phosphoreum} ANT-2200). This is an intermediate step to answer the scientific objectives \textit{in situ}. Finally, Chapter 5 will approach the question of the bacterial part in the emission of bioluminescence, through the use of the preliminary results of a survey over the year 2011, near the site ANTARES, and at 2,000 m depth. Total prokaryotic communities are described and bioluminescent ones are quantified using molecular biology methods. The part of  bioluminescent bacteria within the bioluminescence signal detected by photomultipliers is also estimated and discussed.\\

The " Conclusions and Perspectives " part will give a critical summary of the objectives and contributions of this work and the prospects that will have to be developed after this work.\\

\subsection{A study at several scales}

The study of an ecosystem confronts with the choice of a characteristic observation scale. The observation of the community population dynamics and of the processes that are influencing them, requires to take into account variations in time and space. This will to the understanding of the observed phenomenon instantaneously, but also its spatio-temporal dynamics. The chosen scale will define variability associated with observed processes \citep{hewitt2007}. The studied ecosystem varying into three dimensions (time, vertical and horizontal spatial variations), each of these dimensions has an inherent variability. Ecological studies suggest the concept of multi-scale theory to take into account a set of connections at different scales \citep{legendre1997,anderson1998}. If the importance of the characteristic scales of studied processes in ecology is recognized, few studies take it into account in the measurements or in the result interpretation.\\

In this work, we try to keep a link between the different scales of studied processes. In Chapter 2 and 3, the spatial and temporal variations are taken into account at the ecosystemic level (annual observation and regional scale). Chapter 4 is devoted to a population level approach with the study of a bacterial strain (hourly observations and microscale). Finally, Chapter 5 provides an approach of the community level interacting in this ecosystem (daily observations and local scale).\\


% Chapitre 1 Introduction
\chapter[\textit{In situ} bioluminescence in the deep Mediterranean Sea \\
(ANTARES site)]{\textit{In situ} bioluminescence in the deep Mediterranean Sea \\
(ANTARES site)
\chaptermark{\textit{In situ} bioluminescence}}
\chaptermark{\textit{In situ} bioluminescence}
\minitoc
%\label{chap1}
%\begin{center}
%\textit{If you are faced with a mountain, you have several options.You can climb it and cross to the other side. You can go around it. You can dig under it. You can fly over it. You can blow it up. You can ignore it and pretend it's not there. You can turn around and go back the way you came. Or you can stay on the mountain and make it your home. }\end{center}
%\begin{}
%\textit{Vera Nazarian}
%\end{flushright}\\
%\newpage
%\vspace{400mm}
%\begin{flushright}
%\textit{"L'homme qui d�place une montagne\\ commence par d�placer\\ les petites pierres"}\\
%\vspace{10mm}
%Confucius
%\end{flushright}
%\newpage

\newpage
\section[Introduction\string: interest of recording long time series in environmental science]{Introduction\string: interest of recording long time series in environmental science
\sectionmark{Introduction to time series records}}
\sectionmark{Introduction to time series records}
Time series are set of data obtained by consecutive measurements with time dependence. Historically, astronomers were first scientists to study such chronological data. However, in the XVIII$^{th}$ century, even if predictions were considered as competing with God, Halley was the first scientist to predict the next comet event, based on its periodicity. Nowadays, one admits that time series analysis is essential to understand, forecast, monitor or control systems.\\

\cite{edwards2010} described, over the last 100 years, some of the longest international environmental and biological time series recorded in oceanography (Figure \ref{TSbiblio}). Records of such time series are useful to detect episodic events in real time. They can be used to coordinate and provide a long term context for shorter duration scientific expeditions. At a different time-scale, long-term survey often lead to scientific strategies for marine-ecosystem management in order to face anthropogenic changes. These various objectives have led to the development of autonomous and remote instrumentation, as well as new sensors (biological, physical or chemical). Moreover, these infrastructures are often dedicated to multidisciplinary research in order to reduce costs and efforts. The increasing maturity of autonomous or mobile platforms enables adaptive observing systems and increases the access to  more extreme environments such as open and deep sea.\\

The ANTARES observatory (Astronomy with a Neutrino Telescope and Abyss environmental RESearch) is a european project belonging to several observatory programs such as the European Seas Observatory NETwork (ESONET), EUROSITES (FP7, UE) or the international Global Ocean Observing System (GOOS). The ANTARES collaboration is composed of about 150 engineers, technicians and physicists and mainly financially supported by 6 countries (France, Netherlands, Spain, Italy, United Kingdom, Russia). The first aim of this program is the detection of high-energy particles with the immersion of a neutrino telescope in the deep Mediterranean Sea. Indeed, neutrinos are transformed, with the emission of a single photon, into muon particles when crossing the Earth crust. This light emission is more easily detected in the sea due to water-transparency properties, using photons detectors, named photomultipliers. Astrophysicists are able to detect the trajectory of these single-photon emissions across the detector on a global volume of about 0.1 km$^3$. The use of the ANTARES telescope has been extended to a multidisciplinary objective by including oceanographers to the project. An instrumented line has been added to the observatory in order to record oceanographic long time series at this station.\\

\begin{figure}[!h]
\linespread{1} 
\centering
\includegraphics[scale=0.5]{TS_biblio.pdf}
\caption[Representation of some of the international open-ocean biological records and time-series since 1900. From \cite{edwards2010}]{Representation of some of the international open-ocean biological records and time series since 1900. Station P (North Pacific), RYOFU line (west Pacific transect), VICM (Vancouver Island Continental
Margin time series), NMFS (National Marine Fisheries Service collection), BATS (Bermuda Atlantic Time series Study), HOT (Hawaii Ocean Time series program), BD Zoo (North Coast of Spain), SO CPR survey (Southern Ocean CPR survey), AMT (Atlantic Meridional Transect), and AZMP (Atlantic Zone Monitoring Program). From \cite{edwards2010}.}
\label{TSbiblio}
\end{figure}

\section[Instrumentation and mathematical concepts for the observation of oceanographic variables]{Instrumentation and mathematical concepts for the observation of oceanographic variables
\sectionmark{Instrumentation and methods}}
\sectionmark{Instrumentation and methods}
\subsection{ANTARES site and environmental context}

The ANTARES site is located in the North-Western Mediterranean Sea, 40 km off the french Provencal coast (42 \degres 48'N,6 \degres 10'E), at 2,475 m depth, down to the steep continental slope. The position and depth for the ANTARES station were optimized for various reasons (technical, geophysical...). The observatory had to be immersed at depth greater than the euphotic zone for light interference and below 1,000 m depth for space and volume reasons. The vicinity to the shore was also of major importance to connect the site to the coast. Moreover, the site location was optimized for the optical and transparency properties of the water as well as the low current speed to avoid sedimentation and biofilms on the structure \citep{amram1999,amram2002} and \citep{aguilar2004}.\\ 

\begin{figure}[!h]
\linespread{1} 
\centering
\includegraphics[scale=0.5]{representation.pdf}
\caption[A) Artistic representation of the ANTARES neutrino telescope in the deep Mediterranean Sea. B) Schematic representation of the IL07 instrumented line immersed close to the ANTARES telescope.]{A) Artistic representation of the ANTARES neutrino telescope in the deep Mediterranean Sea. Each of the spherical globe protects a  photomultiplier (PMT). All those 885 PMTs detect photon emission crossing the telescope and coming from various sources such as neutrino decomposition into muons, $^{40}$K decay of the water and from bioluminescent organisms. Image provided by the ANTARES collaboration. B) Schematic representation of the IL07 instrumented line immersed close to the ANTARES telescope. The mooring line is anchored to the seafloor by a dead weight and maintained vertically using a floating buoy, the line is connected to the junction box by a cable (not shown).}
\label{representationANT}
\end{figure}

Twelve mooring lines dedicated to photon detection are immersed at the ANTARES site (see Figure \ref{representationANT} A) close to an instrumented line, namely IL07 line, mainly dedicated to the sampling of oceanographic and environmental data (Figure \ref{representationANT} B). The IL07 is composed of 2 optical modules (containing photomultipliers tubes), 2 Acoustic Doppler Current Profilers (ADCP), 2 video cameras, 1 Conductivity Temperature Depth probe (CTD) and 1 Aanderaa$^{\mbox{\scriptsize{\textregistered}}}$ optode oxygen sensor. The telescope mooring lines are linked to the shore by a cable providing mechanical strength, electrical contact and optic-fiber readout. This cable is stored to the seabed by a dead-weight anchor and kept vertical by a buoy at the top of each mooring lines. This infrastructure allows the access to real-time data from the deep sea. The ANTARES telescope has been immersed, in its last configuration, in December 2007, with a break in the data acquisition from July to September 2008 due to maintenance operations. In October 2010, the instrumented line IL07 has been disconnected and taken off involving a stop in the survey until its re-immersion in March 2013.\\

\subsection{Description of the IL07 instrumented line}

\subsubsection*{Photomultiplier tubes}
The ANTARES telescope is a net of 12 lines of 450 m length and horizontally separated by 70 m. These lines are vertically divided into 25 storeys of optical detectors separated by about 14.5 m and starting 100 m above the seabed. The lines are linked to an optical cable connected to a junction box \citep{ageron2009}. Each storey is composed of three optical modules oriented downward, 45� from the vertical with a total number of about 885 optical modules (see Figure \ref{pmt}) over the telescope. This orientation prevents the particulate deposit \citep{amram2002}. The optical-module position is controlled using a compass and a tilt meter. The optical modules are spherical glass pressure vessels of 17 cm of diameter, 15 mm of thickness and containing a photomultiplier tube (PMT) 10'' Hamamatsu PMT R7081-20. PMTs are sensitive to light emission ranging from 446 to 500 nm. The photon-counting rates of PMTs, expressed in Hertz (number of photons per second), are integrated over 13 ms for the IL07 instrumented line. The signal recorded by PMTs is a combination of several sources of photon emission. Firstly, the neutrino transformation into muon in the water emits a single photon. Then, a baseline with very low variations, between 40 and 50 kHz, is attributed to the $^{40}$K decay in the seawater. Finally, the main part of the signal is due to light emission from marine bioluminescent organisms. Hence, the photon-counting rate, higher than 50 kHz, will be associated to bioluminescence activity in the rest of this work.\\ 

\begin{figure}[!h]
\linespread{1} 
\centering
\includegraphics[scale=0.3]{pmt.jpg}
\caption[Optical module from the ANTARES neutrino telescope.]{Optical module from the ANTARES neutrino telescope. One photomultiplier is protected from high hydrostatic pressure inside the thick glass sphere. }
\label{pmt}
\end{figure}

\subsubsection*{Video cameras}
Two video cameras have been integrated into the IL07 between December 2007 and October 2010. The first objectives were to use simple, and easily available, video cameras as a primary step to record bioluminescent organisms crossing the ANTARES telescope. The detection of bioluminescent organisms, flowing through the telescope over several years, would permit to access to the ecological information, in order to understand and describe bioluminescent populations and communities. When such organisms cross the video-camera-detection area, they automatically trigger the camera and then, the data are transferred in real time through the cable connected from the telescope to the shore. The camera AXIS221 (Figure \ref{camera} A) used on the IL07 line is optimized for light levels lower than $10^{-5}$ lux. This camera has a large detection angle (90 \degres) covering several cubic meters around. Time exposure is defined at 0.1 s. Cameras are located into spherical glass pressure vessels similar to the ones containing the photomultipliers. There are several differences between the two cameras. One camera is located on floor 1 at 2,400 m depth, 80 m above the seafloor and the second one is located on floor 5 at 2,210 m depth, 270 m above the seafloor (see Figure \ref{camera} B). The floor-1 camera has a detection field half the one of the camera on floor 5. The cameras are looking down to the deep-sea floor with a 90� angle for the camera on floor 1 and 45� for floor-5 camera. Moreover, it has been observed that floor-5 camera possibly detects cosmic-ray-photon emissions that are not detected by floor-1 camera due to their orientation differences.\\

\begin{figure}[!h]
\linespread{1} 
\centering
\includegraphics[scale=0.35]{camera.pdf}
\caption[A) The AXIS 221 camera used on the instrumented line IL07. B) Floor 5 on the instrumented line IL07 of the ANTARES neutrino telescope.]{A) The AXIS 221 camera used on the instrumented line IL07. Two of those are placed on floor 1 (2,400 m depth) and on floor 5 (2,210 m depth). B) Floor 5 on the instrumented line IL07 of the ANTARES neutrino telescope. The video camera is located into the upper optical module. An ADCP looking downward (in yellow) and an optical module with photomultiplier (PMT) are	also located on the same floor.}
\label{camera}
\end{figure}

\subsubsection*{Acoustic Doppler Current Profiler}
Two Acoustic Doppler Current Profiler (ADCP), 300 kHz RDInstruments are located on the IL07 line. The first ADCP is located, on floor 1, at 2,397 m depth, and is looking upward. The second one is located, on floor 5, at 2,207 m depth, and is looking downward. The ADCP detects particle sizes above 3$\times 10^{-3}$ m. For each of those instruments, 125 m of the water column are sampled into 50 cells of 2.5 m length each. The detection of the two ADCP overlaps causing cross sampling on about 63 m of the water column. Turbidity values are determined by 4 beams of the ADCP and are averaged to smooth extreme or artifactual values. From the ADCP measurements, turbidity (dB), vertical and horizontal current speed (cm s$^{-1}$) and current direction (\degres) are sampled. The vertical direction for current speed is not taken into account in this work due to the low signal-noise ratio of the dataset.\\

\subsubsection*{Conductivity-Temperature-Depth sensors}
The conductivity, temperature and depth are recorded using a Conductivity-Temperature Depth Microcat 37 SMP Sea-Bird$^{\mbox{\scriptsize{\textregistered}}}$. Salinity has been computed based on conductivity values and EPS78 formula from \cite{fofonoff1983}. Potential temperature is used and will be named "temperature" in further dataset and representations.\\

\subsection{Data description\string: the regression trees approach}

\subsubsection*{Threshold detection using regression trees}

As first representation of the oceanographic time series recorded between January 2009 and July 2010, regression and prediction trees have been performed to describe variability of the data over time. Regression trees are statistical models concerned with the prediction of a real response variable $Y$ given a set of explanatory variables $\mathbf{X}$. Starting from a set of n $i.i.d.$ observations $\left\{ \left(Y,\mathbf{X}\right),\, i=1,\cdots,n\right\} $ of $(Y,\mathbf{X})\in\mathbb{R}\times\Theta$, they are constructed by partitioning the $\mathbf{X}$ space into a set of hypercubes and fitting a simple model (a constant) for $Y$ in each of these regions.The model can be displayed in the form of a binary tree containing $q$ terminal nodes (the regions) where a predicted value of $Y$ is given.\\

The construction of a tree-based model, the way to select the splits and the measure of the accuracy are achieved using the following criterion called deviance defined for each node $r$ of the tree as: \[R\left(r\right)=\sum_{\mathbf{x}_{i}\in r}\left(y_{i}-\bar{y}_{r}\right)^{2}\]where $\bar{y}_{r}$ is the average of the observations of $Y$ belonging to the region $r$. These models have been widely studied in machine learning and applied statistics and present many advantages like their representation in a form of a binary tree, working in high dimension and variable ranking. Moreover, it can be interesting to construct a regression tree when the predictor variable $X$ is univariate or when observations are not $i.i.d$. In that case, the tree structure provides a nonlinear fitting of the time-series $Y\left(X\right)$ by finding thresholds on the variable $X$ and giving constant predicted values of $Y$ in regions identified by these thresholds. The model can then be displayed as a piece wise constant function and allows to identify particular regions of the time series where values of $Y$ change significantly.

\section[Multivariate survey at high frequency and real-time sampling ]{Multivariate survey at high frequency and real-time sampling
\sectionmark{Multivariate oceanographic dataset}} 
\sectionmark{Multivariate oceanographic dataset}

In the following paragraphs (\ref{currento} to \ref{images}), descriptive analyzes of bioluminescence activity linked to other oceanographic variables will be presented. More detailed description and analysis of all time series will be done in article 1 (\ref{manuscrit1}) and article 2 (\ref{manuscrit2}) for all variables independently and linked to bioluminescence activity.\\

\subsection{Water masses proxies}

Time series recorded at the ANTARES station are represented for bioluminescence, salinity, temperature and horizontal current speed from December 2007 to October 2010 (Figure \ref{TScomp}).\\

\begin{sidewaysfigure}[p]
\linespread{1} 
\centering
\includegraphics[scale=0.7]{TS2.pdf}
\caption[Time series dataset sampled at the ANTARES station from December 2007 to July 2010. ]{Time series dataset sampled at the ANTARES station from December 2007 to July 2010. A break appears in July 2008 due to the maintenance of the instrumented line. A) The median rates of photon-counting rate (assimilated to bioluminescence in kHz), B) salinity C) temperature (\degres C) D) current speed (cm s$^{-1}$) E) oxygen ($\mu$mol kg$^{-1}$).}
\label{TScomp}
\end{sidewaysfigure}

The scientific questions of this work focus on bioluminescence activity in the deep sea as an input to understanding the deep-ecosystem dynamics. Looking at the bioluminescence activity in Figure \ref{TScomp} A, a wide range of values, not expected in this deep environment (see \ref{introbiolu}), is observed over the considered time-period ranging from 40 to about 8,000 kHz. If this photon emission can not be described as high intensity, in absolute term, these values are altogether referring to unexpected level of bioluminescence in the deep sea and to intense variability. The bioluminescence activity shows an intermittent pattern with peaks above a baseline at about 40 kHz, known as the $^{40}$K decay in seawater. Two periods of very intense bioluminescence activity are distinct between March and July in 2009 and 2010. Moreover, in Figure \ref{TScomp} B and C, temperature and salinity have similar variations. In Figure \ref{TScomp} D, current speed  shows moderate variations over time without clear distinct period of high intensity. In Figure \ref{TScomp} E, oxygen shows a clear linear decreasing trend with an important variation in March 2009, conjointly to salinity and bioluminescence time series.\\ 

\subsection{Currentology}
\label{currento}
Figure \ref{courant} represents the horizontal current speed at the deep ANTARES station with direction, intensity and frequency information. Looking at the annual scale, in 2008, 2009 and 2010, a global East-West axis is dominant. Moreover, the current direction is more frequently, and with higher intensity coming from the West than from the East (current speeds up to 20 cm s$^{-1}$ for Western currents and up to 10 cm s$^{-1}$ for Eastern currents). Faster current speeds are observed in 2009 from the Western direction and in 2010 from the South-eastern direction (values above 20 cm s$^{-1}$). No high current-speed values are observed in 2008.\\

Similar representation are drawn with a focus on March 2009 and March 2010 (Figure \ref{courantevents}) referring to intense bioluminescence-activity events already described in Figure \ref{TScomp} A. Those two periods show similar current-direction pattern than the observation on annual scale (Figure \ref{courant}), dominated by East-western direction. However, during those two periods, South-eastern currents are more frequent than during the whole year. March 2009 and 2010 are also periods of more intense current speed compared to the whole year (more than 15\% above 20 cm s$^{-1}$). In March 2009, the highest current-speed intensities are essentially coming from the western direction whereas,
in 2010, this direction is mainly South-East.\\

\begin{figure}[!h]
\linespread{1} 
\centering
\includegraphics[scale=0.42]{courant08-10.pdf}
\caption[Current speed horizontal intensity, incoming direction and frequency representations for 2008, 2009 and 2010. ]{Current speed horizontal intensity, incoming direction and frequency representations for 2008, 2009 and 2010. Due to the stop of the instrumented line IL07 between July and September 2008, no data have been taken into account for that period in 2008. The total circle is subdivided into 10� angular sections. Color scale (from dark blue to dark red) represents the current speed (cm s$^{-1}$) and dotted circles represent the frequency (\% of occurrence).}
\label{courant}
\end{figure} 


\begin{figure}[!h]
\linespread{1} 
\centering
\includegraphics[scale=0.43]{courant_events.pdf}
\caption[Current speed horizontal intensity, direction and frequency representations focused on March 2009 and 2010.]{Current speed, horizontal intensity, direction and frequency representations focused on March 2009 and 2010. The total circle is subdivided into 10� angular sections. Color scale (from dark blue to dark red) represents the current speed (cm s$^{-1}$) and dotted circles represent the frequency (\% of occurrence). }
\label{courantevents}
\end{figure} 

\subsection{Bioluminescence interaction with current}

Such representation can be easily adapted in order to illustrate links between bioluminescence activity and current direction. In Figure \ref{bio}, the median rate of bioluminescence (kHz) and frequency (\% of occurrence) are plotted over current direction (\degres) recorded at the same time. An East-western axis is predominant with similar pattern as current speed in Figure \ref{courant}. In 2008, 2009 and 2010, most of the bioluminescence activity is lower than 500 kHz. However, higher bioluminescence values occurring in 2009 and 2010 (above 1,500 kHz), are mainly related to South-eastern currents.\\

\begin{figure}[!h]
\linespread{1} 
\centering
\includegraphics[scale=0.52]{biol08-10.pdf}
\caption[Bioluminescence representation over the years 2008, 2009 and 2010 depending on direction of current speed, intensity and frequency. ]{Bioluminescence representation over the years 2008, 2009 and 2010 depending on direction of current speed, intensity and frequency. Due to the stop of the instrumented line IL07 between July and September 2008, no data have been taken into account for that period in 2008. The total circle is subdivided each 10\degres . Color scale (from dark blue to dark red) represents the bioluminescence intensity (kHz) and dotted circles represent the frequency (\% of occurrence).}
\label{bio}
\end{figure} 

 A focus in March 2009 and March 2010 is represented (Figure \ref{bioluevents}). If March 2009 shows moderated bioluminescence activity, in March 2010, very high bioluminescence intensity (from 2,000 to 8,000 kHz) is recorded conjointly to a South-eastern current direction. Those two graphs (Figure \ref{bio} and \ref{bioluevents}) show similar pattern to current speed, frequency and direction (Figure \ref{courant} and Figure \ref{courantevents}). The similarity between those representations highlights close links between current speed, current direction and bioluminescence activity. However, no discrimination can be made between the independent effect of the current speed and that of the current direction on bioluminescence activity.\\

\begin{figure}[!h]
\linespread{1} 
\centering
\includegraphics[scale=0.4]{biolu_events.pdf}
\caption[Bioluminescence representation in March 2009 and 2010 depending on direction of current speed, intensity and frequency .]{Bioluminescence representation in March 2009 and 2010 depending on direction of current speed, intensity and frequency. The total circle is subdivided each 10\degres . Color scale (from dark blue to dark red) represents the bioluminescence intensity (kHz) and dotted circles represent the frequency (\% of occurrence).}
\label{bioluevents}
\end{figure} 

\subsection{Univariate and multivariate approaches using regression-tree method}

For each variable independently, regression trees have been performed on dataset recorded  between January 2009 and December 2010. This representation defines changes in intensity, taking into account time dependence. In Figure \ref{trees}, on the left, the regression trees show time series partition into classes with mean value defined for the final leaves. On the right, the previously defined classes, performed by the regression trees, are represented on time series.\\

For bioluminescence (Figure \ref{trees} A), 5 classes are described, whereas for salinity (Figure \ref{trees} B) and temperature (Figure \ref{trees} C), 4 classes are discriminated over time.\\

\begin{figure}[!h]
\linespread{1} 
\centering
\includegraphics[scale=0.6]{trees.pdf}
\caption[Regression trees for A) bioluminescence, B) salinity, C) temperature and D) current speed. Time series with the class mean values (black lines) defined by the leaves of regression trees. ]{On the left, regression trees for A) bioluminescence, B) salinity, C) temperature and D) current speed. On the right, time series with the class mean values (black lines) defined by the leaves of regression trees. The regression trees give an overview of the occurrence of events inside the dataset by dividing the data into classes. The time series representation of these classes shows the variability over time.}
\label{trees}
\end{figure}

Interestingly, the same period of time, between the 18$^{th}$ and 31$^{st}$ of March 2010 (the two first nodes), is highlighted for bioluminescence and salinity (Figures \ref{trees} A and B). These changes in variability over time are clearly observed with long branches and high-mean values (1,901 kHz for bioluminescence and 38.480 for salinity). The threshold corresponding to the 18$^{th}$ March 2010 also appears for temperature (last node in Figure \ref{trees} C) but not for current speed. This first observation highlights potentially common dynamics between bioluminescence activity, salinity and temperature on a global time scale.\\

From these results, a prevision tree, based on data acquisition from December 2007 to July 2010, has been performed (Figure \ref{S4}). This prediction tree computes information from temperature, salinity, current speed in order to predict bioluminescence values.\\

\begin{figure}[!h]
\linespread{1} 
\centering
\includegraphics[width=13cm]{S4.pdf}
\caption[Regression tree for the prediction of bioluminescence activity using oceanographic variables (salinity, temperature and current speed) from December 2007 to July 2010. Modified from \cite{tamburini2013}, supplementary data.]{Regression tree for the prediction of bioluminescence activity using oceanographic variables (salinity, temperature and current speed) from December 2007 to July 2010. The top and bottom of each box-plot represent 75\% (upper quartile) and 25\% (lower quartile) of all values, respectively. The horizontal line is the median. The ends of the whiskers represent the 10$^{th}$ and 90$^{th}$ percentiles. Outliers are represented by empty dots. Modified from \cite{tamburini2013}, supplementary data.}
\label{S4}
\end{figure} 

The tree defines 4 nodes and 5 classes with an average bioluminescence intensity for each. Class 1 (mean 121.1 kHz), 2 (mean 435.0 kHz) and 3 (mean 552.4 kHz) gather low bioluminescence intensity mainly due to low sea-current speed (below 19.0 cm s$^{-1}$). The branch length shows the strong dependence of bioluminescence to the current-speed threshold of 19.0 cm s$^{-1}$. Indeed, class 3 and 4 are firstly described by high current-speed intensity (>19.0 cm s$^{-1}$) but, as a second environmental condition, the temperature threshold of 12.92\degres C divides these two classes between high (class mean of 1,393.0 kHz) and highest (class mean of 5,108.0 kHz) bioluminescence values. Moreover, the boxplot representation of data within each class (Figure \ref{S4}) revels a straight range of values within the class of strongest bioluminescence activity (1$^{st}$ and 3$^{rd}$ quantiles between 4,500 and 5,200 kHz).\\

Similar variations are observed for bioluminescence activity and salinity (period between the 18$^{th}$ and the 31$^{st}$ of March 2010, in Figure \ref{trees} A and B). Time dependence seems to be of major importance for highlighting links between bioluminescence activity and environmental variables. Moreover, a current speed above 19.0 cm s$^{-1}$ strongly discriminates the bioluminescence activity. However, intense bioluminescence activity also relies on temperature (threshold of 12.92\degres C in Figure \ref{S4}).\\

This classification method is dependent on the variability threshold and on the final number of classes, both defined by the user. If some clues and hypotheses can be proposed using both time series observation and regression trees, more robust statistical-analysis methods have to confirm, or not, this first investigation. Such improvement will be proposed in article 1 (\ref{manuscrit1}) and article 2 (\ref{manuscrit2}).\\

%\subsection{Turbidity}

%\begin{figure}[]
%\linespread{1} 
%\centering
%\includegraphics[scale=0.35]{2008_echo_F1.jpg}
%\caption{}
%\label{2008turb}
%\end{figure} 

%\begin{figure}[]
%\linespread{1} 
%\centering
%\includegraphics[scale=0.35]{2009_echo_F1.jpg}
%\caption{}
%\label{2009turb}
%\end{figure} 

\subsection{Analysis of \textit{in situ} images from bioluminescent organisms}
\label{images}

%During the first immersion of the IL07 instrumented line on the ANTARES site, video-camera systems have been . 

\subsubsection*{Qualitative interpretations}

The use of video camera for automatic detection is nowadays under development, with numerous recent studies. Such developments are done in various fields as for example the zooplankton detection \citep{stemmann2008}, or other \textit{in situ} organisms detection  (Eye-in-the-sea, MARS observatory, USA or NEPTUNE, Canada, \citealp{aguzzi2011,aguzzi2009}). To determine bioluminescent organisms crossing the ANTARES observatory, the ANTARES collaboration decided to immerse video cameras. In a first attempt, easily available and simple to use AXIS video-monitoring cameras were chosen to test the feasibility (1) to connect them to the ANTARES telescope, (2) to trigger them in the presence of luminescent organisms detected by the ANTARES PMTs, (3) to real-time visualize and monitor the instrumentation, (4) to automatically transfer and stock images in the ANTARES data base and (5) to visualize luminescent organisms using only infrared light.\\

%The immersion of two video-camera on the ANTARES IL07 line was dedicated to the determination of bioluminescent organisms crossing the ANTARES observatory. The implementation of camera validate the ability of such technology to be used on the IL07 instrumented line and to detect bioluminescent organisms in the deep Mediterranean Sea (see Figure \ref{imagevid}). 

\begin{figure}[!h]
\linespread{1} 
\centering
\includegraphics[width=8cm]{bioluimage.jpg}
%\caption[Example of images recorded by video-camera at the ANTARES site, on floor 5. ]{Example of images recorded by video-camera at the ANTARES site, on floor 5. The second line correspond to successive images from one shot video.}
\caption[Example of images recorded by video-camera at the ANTARES site. ]{Example of colored images recording bioluminescent organism crossing the video-camera detection area, at the ANTARES site, on 10$^{th}$ December 2007.}
\label{imagevid}
\end{figure}

While the first objectives were successfully achieved, the image quality obtained, using the 2 AXIS cameras placed on the IL07, does not permit an identification of bioluminescent objects (Figure \ref{imagevid}). So, only the events detected by automatic triggers have been investigated in a quantitative approach and can be interpreted for now.\\

\subsubsection*{Quantitative analysis\string: number of events}

Between December 2007 and July 2010, a total number of 874 events has been recorded by the two video cameras on the IL07 line. Bioluminescence time series recorded by PMTs from the IL07 (Figure \ref{tsvideo} A) and the number of detected events by the video cameras for both floor 1 and 5 (Figure \ref{tsvideo} B) are represented. During July and August 2008, the video cameras were canceled due to the IL07 maintenance involving a gap into the dataset. A peak in the number of events is recorded from March to June 2010, with a maximum value occurring in March 2010 with 27 and 201 events detected on floor 1 and 5, respectively.\\ 

\begin{figure}[!h]
\linespread{1} 
\centering
\includegraphics[scale=0.58]{c.pdf}
\caption[Bioluminescence emission and video records. A) The bioluminescence emission recorded from the ANTARES-IL07 photomultipliers. B) Bioluminescence events recorded from the two video-cameras using automatic detection on the ANTARES-IL07.]{Bioluminescence emission and video records. A) The bioluminescence emission recorded from the ANTARES-IL07 photomultipliers. B) Bioluminescence events recorded from the two video-cameras using automatic detection on the ANTARES-IL07.}
\label{tsvideo}
\end{figure}

Video cameras detect less events at floor 1 than floor 5 (281 and 593 total events with median values of 7.0 and 9.0, respectively, see Figure \ref{boxplot}). However, there is no significant difference between these two boxplot distributions corresponding to floor 1 and floor 5.\\

\begin{figure}[!h]
\linespread{1} 
\centering
\includegraphics[scale=0.35]{boxplot.jpeg}
\caption[Representation of the video-event distribution between floor 1 and floor 5. ]{Representation of the video-event distribution between floor 1 and floor 5. The video camera at floor 1 is close to the seafloor and the one at floor 5 is assumed to be in the water column. Data recorded in March 2010, with 201 events detected, is not shown in this graph, since this data was considered as an extreme value out of the boxplot distribution. Total number of events are 281 and 593 for floor 1 and floor 5, respectively.}
\label{boxplot}
\end{figure}

In order to quantify the bioluminescence activity detected by video cameras, the light emission recorded by the PMTs in Figure \ref{tsvideo} A, was monthly integrated from January 2009 to July 2010. Correlation between the number of events and bioluminescence rate per month is represented (Figure \ref{correl}). These results show a correlation (correlation coefficient of about 0.7) between the number of events detected and the bioluminescence rate recorded. On a monthly scale, the number of bioluminescent events from video cameras reflects the variations in bioluminescence activity recorded at the ANTARES station. Exceptions occur in April 2009, and March 2010 (red crosses in Figure \ref{correl}) that are outside the 95\% confidence intervals. In April 2009, bioluminescence rate per month is high with a number of events detected by the video camera relatively low referring to the regression. However, in March 2010, a high number of events is recorded compared to the monthly bioluminescence rate also referring to the regression. This correlation is based on 19 months, with  about 76\% of the data in the lower part of the regression line (number of events lower than 35 and monthly bioluminescence lower than 1$\times 10^6$ kHz, red dotted lines in Figure \ref{correl}). According to this remark, the few events of high bioluminescence activity support heavy weight to determine the regression slope.\\

\begin{figure}[]
\centering
\linespread{1} 
\includegraphics[width=12cm]{correlationb2.pdf}
\caption[Correlation between the bioluminescence emission monthly integrated and the number of events recorded by video cameras at the same time scale. ]{Correlation between the bioluminescence emission monthly integrated and the number of events recorded by video cameras at the same time scale. The data are studied from January 2009 to July 2010.  The dotted lines represent the 95 and 75\% of confidence intervals. In red, the data referring to April 2009 and March 2010 are considered as extreme couples of data. Red dotted lines represent 76\% of the data.\\}
\label{correl}
\end{figure}

%\begin{figure}[!h]
%\linespread{1} 
%\centering
%\includegraphics[width=13cm]{video_courant_intensity.pdf}
%\caption[Current speed time series (black line) and cumulative sum of events detected from video-camera (red line). ]{Current speed time series (black line) and cumulative sum of events detected from video-camera (red line). The occurrence of data detected is also represented over time (red crosses).}
%\label{event_cour}
%\end{figure}

Organisms detected by the video-cameras can be mechanically stimulated by current speed, by turbulence or emit light when crashing to the optical module. Consequently, the bioluminescence activity is linked to the number of events detected by the video cameras. However, at some specific dates this relation is not confirmed (April 2009 and March 2010 in Figure \ref{correl}). Other organisms, such as bacteria, are probably poorly detected in free living or when attached to particles and can probably modify the video-camera triggering.\\

\section[Article 1]{Article 1
\sectionmark{ARTICLE 1}}
\sectionmark{ARTICLE 1}
\label{manuscrit1}

\subsection{Foreword}

Time series records at the ANTARES station lead to the observation of high variability in bioluminescence activity, especially in March 2009 and March 2010, that were unexpected in the deep sea.
In the following article, these variations are described and explained at the regional scale of oceanographic processes observed in the Gulf of Lion (NW Mediterranean Sea). Indeed, the conjoint study of complementary time series, from monitored stations in the Gulf of Lion gave the opportunity to highlight the formation of new-water masses at the surface that are then exported to the deep sea. Such events explain changes observed on the ANTARES time series involving modifications of water-masses characteristics (temperature, salinity) but also of possible population changes in this deep ecosystem. These populations can be modified either by the input of organisms among them being bioluminescent (coming from the surface layers), or by the enrichment in carbon and energy of the deep sea due to the incoming of newly-formed water masses. This carbon, oxygen and nutrients input to the deep sea might involve an increase in the deep-sea biological activity and so higher bioluminescence activity.\\

%L'enregistrement des s�ries temporelles enregistr�es sur le site ANTARES a permis d'observer des variations d'activit� de bioluminescence inattendues en milieu profond, principalement en mars 2009 et mars 2010. Dans le manuscrit ci-apr�s, ces variations de bioluminescence sont d�crites et expliqu�es � l'�chelle de processus oc�anographiques observ�s dans le Golfe du Lion (M�diterran�e NW). En effet, la mise en commun de s�ries temporelles compl�mentaires, provenant de sites instrument�s dans le Golfe du Lion, ont permis de mettre en avant des ph�nom�nes de masses d'eau nouvelles cr��es en surface et export�es vers le milieu profond. Ces �v�nements expliquent les changements observ�s sur les s�ries temporelles d'ANTARES entrainant des modifications des caract�ristiques des masses d'eau (temp�rature, salinit�) mais �galement de probables changements de populations dans cet �cosyst�me profond. Ces populations peuvent �tre modifi�es soit par l'apport d'organismes, potentiellement bioluminescents, provenant des couches sup�rieures de l'oc�an, soit, par l'enrichissement du milieu profond avec l'arriv�e de ces masses d'eau nouvellement form�es. Cet apport de carbone, d'oxyg�ne ou de nutriments vers le milieu profond induirait une augmentation de biomasse et par cons�quent une hausse de l'activit� de bioluminescence.\\

\textit{The ANTARES collaboration developed the underwater telescope and as members of the ANTARES collaboration, we have access to the data. C. Tamburini and S. Escoffier are the main authors for this publication. M. Canals, X. Durieu de Madron, L. Houpert, F. D'Ortenzio gave clues and interpretation for oceanographic data as well as data from the Gulf of Lion. D. Lefevre and A. Robert calibrated temperature and salinity data, thanks to the MOOSE monthly sea campains, as well as the supplementary data in Figure {S3}. I contributed to the time series representation, the statistical analyzes and the article understanding and writing.}


%\textit{La collaboration ANTARES a mis en place le t�lescope et permis l'acc�s aux donn�es, C. Tamburini et S. Escoffier sont les auteurs principaux de cet article,  M. Canals, X. Durieu de Madron, L. Houpert, F. D'Ortenzio ont permis le traitement et la compr�hension des donn�es d'oc�anographie physique ainsi que l'acc�s aux donn�es du site d'�chantillonnage du Golfe du Lion. D. Lefevre at A. Robert ont permis la calibration des donn�es de temp�rature et de salinit� par les sorties mensuelles des campagnes MOOSE, ainsi que la r�alisation des donn�es suppl�mentaires et de la Figure \ref{S3}. J'ai contribu� � la repr�sentation des s�ries temporelles, aux analyses statistiques pr�sent�es ainsi qu'� la r�fl�xion et l'�criture de ce manuscrit.}\\

\newpage
\subsection[Deep-sea bioluminescence blooms after dense water formation at the ocean surface]{Deep-sea bioluminescence blooms after dense water formation at the ocean surface
\sectionmark{ARTICLE 1}}
\sectionmark{ARTICLE 1}
\label{manuscrit1}

\begin{center}
Christian Tamburini$^{1,2,\ast}$, St�phanie Escoffier$^{3,\ast}$, Miquel Canals$^3$, Xavier Durrieu de Madron$^4$, Loic Houpert$^4$,
Dominique Lef�vre$^{1,2}$, S�verine Martini$^{1,2}$, Fabrizio D'Ortenzio$^5$, Anne Robert$^{1,2}$, Pierre Testor$^6$,  and the ANTARES collaboration $^{\ast \ast}$\\
\end{center}
\vspace{1mm}

$^1$ Aix Marseille Universit� , CNRS/INSU, IRD, Mediterranean Institute of Oceanography (MIO), UM 110, Marseille, France \\
$^2$ Universit� de Toulon, CNRS/INSU, IRD,
Mediterranean Institute of Oceanography (MIO), UM 110, La Garde, France\\
$^3$ CPPM-Aix-Marseille Universit� , CNRS/IN2P3, Marseille, France\\
$^4$ GRC Geoci�ncies Marines, Departament d'Estratigrafia, Paleontologia i Geoci�ncies Marines,
Facultat de Geologia, Universitat de Barcelona, Campus de Pedralbes, Barcelona, Spain\\
$^5$ Universit� de Perpignan, CNRS-INSU, CEFREM UMR5110, Perpignan, France\\
$^6$ Universit� Pierre et Marie Curie, CNRS-INSU, LOV UMR7093, Villefranche-sur-mer, France\\
$^7$ Universit� Pierre et Marie Curie, CNRS-INSU, Institut Pierre Simon Laplace, LOCEAN UMR 7159, Paris, France\\

\vspace{1mm}
$\ast$ Corresponding authors\string: \underline{christian.tamburini@univ-amu.fr}; \underline{escoffier@cppm.in2p3.fr}\\
\noindent $\ast \ast$ For easier reading, the complete list of authors belonging to the ANTARES collaboration has not been developed in this article. For a complete list, please refer to the PLoS ONE published version.\\

\noindent \textbf{Tamburini C, Canals M, Durrieu de Madron X, Houpert L, Lef�vre D, et al. (2013) Deep-Sea Bioluminescence Blooms after Dense Water Formation at the Ocean Surface. PLoS ONE 8(7)\string: e67523. doi\string: 10.1371/journal.pone.0067523}\\

\newpage
\textbf{Abstract}\\
The deep ocean is the largest and least known ecosystem on Earth. It hosts numerous pelagic organisms, most of which are
able to emit light. Here we present a unique data set consisting of a 2.5-year long record of light emission by deep-sea
pelagic organisms, measured from December 2007 to June 2010 at the ANTARES underwater neutrino telescope in the
deep NW Mediterranean Sea, jointly with synchronous hydrological records. This is the longest continuous time-series of
deep-sea bioluminescence ever recorded. Our record reveals several weeks long, seasonal bioluminescence blooms with
light intensity up to two orders of magnitude higher than background values, which correlate to changes in the properties
of deep waters. Such changes are triggered by the winter cooling and evaporation experienced by the upper ocean layer in
the Gulf of Lion that leads to the formation and subsequent sinking of dense water through a process known as 'open-sea
convection'. It episodically renews the deep water of the study area and conveys fresh organic matter that fuels the deep
ecosystems. Luminous bacteria most likely are the main contributors to the observed deep-sea bioluminescence blooms.
Our observations demonstrate a consistent and rapid connection between deep open-sea convection and bathypelagic
biological activity, as expressed by bioluminescence. In a setting where dense water formation events are likely to decline
under global warming scenarios enhancing ocean stratification, \textit{in situ} observatories become essential as environmental
sentinels for the monitoring and understanding of deep-sea ecosystem shifts.\\

\newpage
\textbf{Introduction}\\
The deep-sea ecosystem is unique because of its permanent
darkness, coldness, high pressure and scarcity of carbon and
energy to sustain life. Most of its biological activity relies on the
arrival of carbon in the form of organic matter from surface
waters. Ninety percent of the numerous pelagic organisms that
inhabit the deep ocean are capable of emitting light \citep{robison2004} through the chemical process of bioluminescence, which appears to be the most common form of communication in this remote realm \citep{robison2004,herring1987,Haddock2010}. Deep-sea bioluminescence is also viewed as an expression
of abundance and adaptation of organisms to their environment
\citep{widder2010}. Marine bioluminescent organisms include a variety of distinct taxa \citep{widder2010}. When stimulated mechanically or electrically, eukaryotic bioluminescent organisms emit erratic luminous flashes, and also spontaneous flashes to attract prey and mates for recognition of congeners or for defence purposes \citep{robison2004,Haddock2010,widder2010}. In contrast, luminescent bacteria are unaffected by mechanical stimulation and can glow continuously for many days under specific growth conditions \citep{nealson1979,miller2005}. Bioluminescent bacteria occur in marine waters as free-living forms, symbionts in luminous organs of fishes and crustaceans and attached to marine snow aggregates sinking through the water column \citep{nealson1979,andrews1984}. During micro-algae blooms, strong bioluminescence produced by colonies of bacteria could even lead to spectacular marine phenomena such as 'milky seas' in surface waters \citep{miller2005}.\\

Bioluminescence sources have been observed and quantified over the last three decades using a variety of observational platforms and instruments such as manned submersibles \citep{robison2004}and autonomous underwater vehicles \citep{shulman2005}, \textit{in situ} high sensitivity cameras \citep{widder1989}, \citep{priede2006}, underwater photometers \citep{andrews1984}, \citep{swift1985}, \citep{geistdoerfer1999}, and remote satellite imagery \citep{miller2005}. In most cases, deep-sea bioluminescence is triggered and observed after external mechanical stimulation using, for instance, pumped flows through turbulence-generating grids \citep{widder1999} or downward moving grids that collide with the organisms \citep{priede2006}. While these procedures provide crucial information on the nature and distribution of deep-sea bioluminescent organisms in the water column \citep{Haddock2010} and references therein, they are not suited to investigate the temporal variability of naturally occurring light production (i.e. non artificially triggered) at specific sites over long periods of time, which requires sustained high frequency \textit{in situ} measurements.\\

An unanticipated application of underwater neutrino telescopes
is to provide direct measurements of bioluminescence in the deep sea \citep {bradner1987,aoki1986,amram2000}. A neutrino telescope aims at detecting the faint Cherenkov light emission radiated by elementary charged particles
called muons that are produced by neutrino interactions.
Darkness, transparency and water shielding against cosmic ray
muons make the deep sea an ideal setting for a neutrino telescope.
Here we make use of both the high frequency bioluminescence
and hydrological time-series of the cabled ANTARES neutrino
telescope \citep{ageron2011} located 40 km off the French coast (42'489N, 6'109E) at 2,475 m in the NW Mediterranean Sea (Fig. \ref{F1} a).\\

\begin{figure}[!h]
\linespread{1} 
\centering
\includegraphics[width=13cm]{Fig1.png}
\caption[Map of the NW Mediterranean Sea showing the location of the ANTARES, LION and Lacaze-Duthiers Canyon (LDC) sites (a) as well as the extension of open-sea convection area in the Gulf of Lion and beyond from 2008 to 2010 (b-d).]{Map of the NW Mediterranean Sea showing the location of the ANTARES, LION and Lacaze-Duthiers Canyon (LDC) sites (a) as well as the extension of open-sea convection area in the Gulf of Lion and beyond from 2008 to 2010 (b-d). The boundaries of the convection area in winter 2008 (red in b), 2009 (blue in c) and 2010 (green in d) are derived from MODIS-Aqua satellite-based surface Chlorophyll-a concentration images. The limits of the convection area for each of the three successive winters correspond to their maximum extents during periods of deep water formation measured at the LION site (see Text S1 and Fig. \ref{S5}). Black arrows indicate the direction of the two main continental winds leading to the cooling and subsequent sinking of surface waters\string: Mistral (M) and Tramontane (T). The grey arrow indicates the path of the cyclonic surface mesoscale Northern Current bordering the open-sea convection region.}
\label{F1}
\end{figure} 

The NW Mediterranean Sea is one of the few regions in the
world's ocean where both dense shelf water cascading and open-sea
convection take place \citep{mertens1998,marshall1999,canals2006,stabholz2013} (Fig. \ref{F1} a). This results in
the formation of deep water owing to the combination of
atmospheric forcing and regional circulation that lead the water
column to overturn \citep{marshall1999,canals2006,bethoux2002}. Dense deep water formation occurs during late winter and early spring due to cold, strong and
persistent northern winds (Mistral and Tramontane) causing
surface cooling of the Modified Atlantic Water (MAW) both on
the shelf and over the deep basin. When the cooled shallow waters
on the shelf become denser than the ambient waters, they start
sinking, overflow the shelf edge, and cascade downslope until they
reach their density equilibrium depth, which may vary from
150 m to more than 2,000 m \citep{canals2006,palanques2012}. At the same time,
convection in the adjacent deep basin involves a progressive
deepening of the upper ocean mixed layer, which first reaches the
warmer and saltier underlying Levantine Intermediate Water
(LIW) and eventually extends all the way down to the basin floor,
should the atmospheric forcing be intense enough \citep{marshall1999}. Both
processes and the subsequent renewal of the Western Mediterranean
Deep Water (WMDW) show a high interannual variability
because of their sensitivity to atmospheric conditions \citep{mertens1998,durrieu2013}. The
newly-formed deep water (nWMDW) resulting from both dense
shelf water cascading and open-sea convection has been observed
to spread over the deep basin floor within months \citep{bethoux2002,durrieu2013,schroder2006,testor2006}.
Studies about the response of deep ecosystems to such processes
are scarce and focus on the impact of dense shelf water cascading
on benthic and epi-benthic organisms \citep{pusceddu2013,company2008}. Other recent works
highlight how deep water formation triggers the resuspension of
deep sea sediments, including organic matter \citep{stabholz2013}, and the development and spreading of a thick bottom layer loaded with
resuspended particulate matter across the NW Mediterranean
Basin as a result of dense shelf water cascading \citep{puig2013}.\\

\begin{figure}[p]
\linespread{1} 
\centering
\includegraphics[width=10cm]{Fig2.png}
\caption[Time series measured at the ANTARES IL07 mooring line. ]{Time series measured at the ANTARES IL07 mooring line. (a) Median PMT counting rates (log scale), salinity, potential temperature and current speed from December 2007 to June 2010. Shading indicates periods (b) from January to June 2009 and (c) from January to June 2010, in which bioluminescence blooms were recorded. The lack of data from June 24 to September 6, 2008 is due to a cable technical failure.}
\label{F2}
\end{figure} 

Here we present compelling evidence of the quick response of
the deep-sea pelagic ecosystem to seasonal atmospheric forcing
leading to dense water formation and sinking, expressed by
particularly intense bioluminescence events captured by neutrino
telescope photomultiplier tubes. Observations on bioluminescence
are supported by a two and a half years long unique and consistent
record of hydrological and hydrodynamical variables obtained at
the ANTARES deep-sea neutrino telescope itself but also at two
independent mooring arrays equally located in the deep NW
Mediterranean Sea.\\

\textbf{Results and Discussion}\\
\textbf{Bioluminescence blooms at the ANTARES site}\\

We report time-series measurements of light intensity expressed
in median counting rates on photomultiplier tubes as well as
temperature, salinity and current speed from December 2007 to
June 2010 (Fig. \ref{F2} a), collected between 2,190 and 2,375 m depth in
the ANTARES IL07 mooring line (see Methods and Fig. \ref{S1}).
While the light intensity background rate is predominantly
between 40 and 100 kHz, which mainly includes the $^{40}$K rate
(see Methods and Fig. \ref{S2}), two remarkable bioluminescence
events reaching up to 9,000 kHz were recorded between March and
July in 2009 and 2010 (Fig. \ref{F2}). Because of their high intensity and
duration we call these events "bioluminescence blooms", defined
here as periods with PMT median rates higher than 600 kHz, i.e.
higher than the 96$^{th}$ percentile of the entire PMTs record.\\

Our records show that bioluminescence primarily increases with
current speed, which is due to mechanical stimulation either by
impacts of small-sized organisms and particles on the PMTs
\citep{amram2000,priede2008} or by the reaction of organisms to enhanced turbulent
motion in the wakes of the PMTs \citep{bradner1987,aoki1986,amram2000}. However, current
speed alone fails to explain the complete record of bioluminescent
activity since, for moderate current speeds, differences in the
median rates of up to one order of magnitude are observed in 2009
(Fig. \ref{F2} b) and 2010 (Fig. \ref{F2} c). For instance, on March 8, March 11
and April 8-12, 2010, bioluminescence peaks at 800 to 1300 kHz while
current speeds are rather low, from 10 to 15 cm s$^{-1}$ (Fig. \ref{F2} c) a
speed range usually associated to median rates of around 100 kHz.
These bioluminescence bursts clearly correspond to significant
increases in both potential temperature ($\Delta \theta = 0.03-0.05\degres C$) and salinity ($\Delta S = 0.005-0.015$). As the deep water mass at the ANTARES site is the WMDW, characterized in 2008 by a narrow range of temperature and salinity ($\theta = 12.89-12.92\degres C, S = 38.474-38.479$), the increases above the normal range of variation observed in 2009 and 2010 are indicative of the intrusion of a distinct water mass (Fig. \ref{F2}, see Text S1 and Fig. \ref{S3}). It is noteworthy that neither deep-water thermohaline modification nor bioluminescence blooms were recorded in 2008 (Fig. \ref{F2} a).\\

\begin{figure}[]
\linespread{1} 
\centering
\includegraphics[width=11cm]{Fig3.png}
\caption[Links between bioluminescence, current speed and the modification of the properties of the Western Mediterranean
Deep Water (WMDW). ]{Links between bioluminescence, current speed and the modification of the properties of the Western Mediterranean
Deep Water (WMDW). Box-and-whisker plot of median PMT counting rates (log scale) versus current speed classes for salinities higher (red) or lower (grey) than 38.479 for data recorded in (a) 2008, (b) 2009 and (c) between January and June 2010. The salinity threshold of 38.479 is used as a marker of the intrusion of newly formed deep water at the ANTARES site. While bioluminescence increases with current speed, it is also enhanced by the modification of WMDW (red box-plots). The top and bottom of each box-plot represent 75\% (upper quartile) and 25\% (lower quartile) of all
values, respectively. The horizontal line is the median. The ends of the whiskers represent the 10th and 90th percentiles. Outliers are not represented. The statistical comparison between the two box-plots (red and grey) in each current class is given by the Kruskal-Wallis test\string: the observed difference between the two samples is significant beyond the 0.05 (*), the 0.01 (**) and the 0.001 (***) levels. The absence of an asterisk in some current classes indicates that the difference between the two box-plots is not significant. The number of measurements for salinity lower or higher than 38.479 is given in black or in red, respectively. Note the different scales of figures a, b and c.}
\label{F3}
\end{figure} 

To illustrate the link between the intrusion of newly formed
deep water and high bioluminescence, we use a salinity threshold
of 38.479 as marker of such intrusions at the ANTARES site. This
value has been defined using a statistical decision tree (Fig. \ref{S4}) and
also corresponds to the 96$^{th}$ percentile of the entire salinity record. Bioluminescence data, divided into two groups above and below this salinity threshold, are presented as box-and-whisker plots
versus current speed classes (Fig. \ref{F3}). Close examination of Figure \ref{F3}
shows that bioluminescent activity is enhanced by both increasing
current speed and the renewal of the deep water. Indeed, on the
one hand, the bioluminescence rates increase with current speed
for each of the two bioluminescence data groups (grey and red box-and-whisker plots) and on the other hand, bioluminescence
rates are always higher for new deep water (red boxes, S>38.479)
than for pre-existing deep water (grey boxes, S<38.479). The
Kruskal-Wallis test performed on the box-and-whisker plots attests
that the red and grey boxes are significantly different (p<0.001) for
current speeds up to 18 and 24 cm s$^{-1}$ in 2009 and 2010,
respectively (Fig. \ref{F3} b-\ref{F3} c), which means that bioluminescence rates
are dependent on water mass properties too. This is illustrated, for
instance, by the 2010 record (Fig. \ref{F3}c), which shows that the
median bioluminescence rate for the 0-3 cm s$^{-1}$ current range is
about 60 kHz for the existing deep water (grey box-plots), while it
reaches 400 kHz within the new deep water (red box-plots).
Bioluminescent bacteria, which are not affected by mechanical
stimulation \citep{nealson1979,bradner1987} and are able to glow continuously under specific conditions \citep{nealson1979,miller2005}, are excellent candidates as main contributors to these bioluminescence blooms.\\

\textbf{Deep-water convection in the NW Mediterranean Sea}\\

To determine the origin of the newly formed deep water
observed at the ANTARES site in 2009 and 2010, we investigated
whether dense shelf water cascading and/or open-sea convection
occurred in winter months.\\

Instrumented mooring lines located at the center of the deep
convection region (LION site at 42'029N, 04'419E; Fig. \ref{F1}a) and in
Lacaze-Duthiers Canyon (LDC site at 42'269N, 03'339E; Fig. \ref{F1} a)
provided temperature, salinity and current speed time-series from
different water depths (Fig. \ref{F4}) synchronous to the ANTARES
record. While no deep (>1,000 m) dense shelf water cascading
took place during the study period (Fig. \ref {F4} a), bottom-reaching
open-sea convection was observed in the basin down to 2,300 m
depth during wintertime in 2009 and 2010, which led to the
homogenization of the water column (Fig. \ref{F4} b). Increases in deep water temperature (Fig. \ref{F4} b) and salinity (Fig. \ref{F4} c) are due to the
mixture of sinking cold surface water with warmer and saltier
LIW. In winter 2008, open-sea convection only affected the upper
1,000 m of the water column and did not alter the deep water
mass. Current measurements showed the strong barotropic
character of horizontal velocities (Fig. \ref{F4} d) and high vertical
velocities (Fig. \ref{F4} e) during intense mixing periods. Once the surface
forcing abates, convection ceases and intense sub-mesoscale eddies
carry discrete volumes of the newly formed deep water away from
the convection area \citep{testor2006}. The delay between the appearance of
the thermohaline anomalies at the LION site in late winter and
their arrival at the ANTARES site in spring is compatible with the
spreading of the newly-formed deep water in the Gulf of Lion and
subsequent mixing with pre-existing deep water \citep{bethoux2002,durrieu2013,schroder2006,testor2006}.
Further mixing could take place at the ANTARES site due to
enhancement of vertical motion by the interaction of instabilities
in the surface cyclonic Northern Current with the topography of
the continental slope \citep{vanharen2011}. The area of open-sea convection, as obtained from satellite imagery (see Fig. \ref{S5}), was much smaller
during winter 2008 than in 2009 and 2010 when it covered most
of the deep Gulf of Lion (Figs. \ref{F1} b-d). Furthermore, it was larger
and closer to the ANTARES site in 2010 than in 2009 (Fig. \ref{F1} c-d),
which may explain why the signature of new deep-water recorded
at the ANTARES site is stronger in 2010 than in 2009 (Fig. \ref{F2}).\\

\begin{figure}[]
\linespread{1} 
\centering
\includegraphics[width=13cm]{Fig4.png}
\caption[Time series of oceanographic parameters measured at the Lacaze-Duthiers Canyon (LDC) and the open-sea convection
region in the Gulf of Lion (LION) from January 2008 to June 2010.]{Time series of oceanographic parameters measured at the Lacaze-Duthiers Canyon (LDC) and the open-sea convection
region in the Gulf of Lion (LION) from January 2008 to June 2010. (a) Potential temperature at 500 and 1,000 m depth at the LDC mooring site and (b) from various water depths at the LION site, jointly with (c) salinity at 2,300 m depth, (d) horizontal current speed and (e) vertical current speed from various water depths at the LION site. The four levels of temperature measurements at LION presented here are a sub-set of measurement depths (see Fig. \ref{S1}). Essentially stable temperatures in the deepest layers in 2008 show that open-sea convection reached only 700 m and did not modify the deep water in the study area. In contrast, strong convection events, reaching 2,300 m depth, occurred during February-March 2009 and 2010 with an abrupt cooling of the upper water column and an increase in temperature and salinity in the deep layers. A concurrent
increase in current speed was also noticed in winter 2009 and 2010. The 5-month long data gap in 2009 is due to a damaging of the mooring line during the April 2009 recovery, which induced a postponement of its redeployment to September 2009.}
\label{F4}
\end{figure} 

\textbf{Link between bioluminescence blooms and deep-water
convection}\\

All evidence points to deep-water formation by open-sea
convection in the Gulf of Lion as the cause of the renewal of
deep water at the ANTARES site that triggered the bioluminescence
blooms observed in 2009 and 2010.\\

During and in the aftermath of the convection period large
amounts of organic matter, both in particulate (POC) and
dissolved (DOC) form, are exported from the productive upper ocean layer down to the deep \citep{stabholz2013,martin2010,santinelli2010}. The resuspension of soft sediments covering the deep seafloor by bottom currents during the reported period could also inject organic matter into the deep-water mass \citep{stabholz2013,martin2010}. Changes in DOC concentration at the
ANTARES site are shown by discrete measurements carried out at
2,000 m depth during oceanographic cruises from December 2009
to July 2010 (Fig. \ref{S6}). DOC concentration significantly increased
from $42 \pm 1$ mM in December 2009, prior to the convection period, to $63 \pm 1$ mM in March and May 2010 when the new deep water mass occupied the ANTARES site, concurrently with higher
oxygen contents in bottom waters between March and mid-June
2010. Subsequently, DOC concentration decreased to 45 in mid-
June and mid-July 2010 (Fig. \ref{S6}). Such an injection of organic
matter into the deep water mass has the potential to fuel the deep-sea
biological activity, thus stimulating bioluminescence activity.
The increase in DOC concentration matches with observations
reported by \citep{santinelli2010} for different regions of the
Mediterranean Sea where deep convection occurs. These authors
showed a high mineralization rate of DOC in recently ventilated
deep waters, which is mainly attributed to bacteria. Bioluminescent
bacteria were isolated at the ANTARES site during a
previous period of high bioluminescent activity in 2005 \citep{alali2010}.
Amongst them, we identified a piezophilic strain, \textit{Photobacterium phosphoreum} ANT-2200 \citep{alali2010,martini2013}, \textit{P. phosphoreum} being the dominant bioluminescent species in the Mediterranean Sea \citep{gentile2009}. These luminous bacteria likely represent the main organisms responsible for the higher level of bioluminescence detected at the ANTARES
site. Such contribution is especially noticed when the current speed
is low within the convection season (Fig. \ref{F3} b-\ref{F3} c). Finally, the flow
associated with deep convection events might likely carries
significant amounts of bioluminescent organisms too, which can
also contribute to the bioluminescence blooms observed in 2009
and 2010 due to their collision with PMTs and/or their
stimulation by turbulent motion in the wakes of PMTs when
current speed is high.\\

\textbf{Conclusions}\\

We present evidence for seasonal episodes of dense water
formation driven by atmospheric forcing being a major vector in
fuelling the deep-sea pelagic ecosystem and inducing bioluminescence blooms after a fast transfer of the ocean surface signal. Since dense water formation occurs in other ocean regions worldwide \citep{marshall1999}, we anticipate that an enhancement of the deep pelagic ecosystem activity similar to that observed in the NW Mediterranean Sea occurs there too, challenging our understanding of the carbon dynamics in the ocean. Dense water formation is likely to be altered by the on-going global warming. Recent models \citep{somot2006,herrmann2008} based on the A2 IPCC scenario indicate a strong reduction in the convection intensity in the Mediterranean Sea for the end of the 21$^{st}$ century, which will induce a massive reduction in organic matter supply and ventilation of the deep basin. Hence changes in the deep
Mediterranean ecosystem more intense than those already
observed in both the Eastern \citep{roether1996,weikert2001} and the Western Mediterranean \citep{pusceddu2013} basins are forecasted for the near future, a situation that could also occur but remain unnoticed in other sensitive areas
of the world ocean. Our results illustrate the potentially far reaching
multidisciplinary scientific and societal benefits of the
installation of cabled deep-sea observatories in critical ocean areas.\\

\textbf{Methods}\\

The ANTARES neutrino telescope comprises a three-dimensional
array of 885 Hamamatsu R7081-20 photomultiplier tubes
(PMTs) distributed on 12 mooring lines \citep{ageron2011,aguilar2005}. These PMTs are
sensitive to the wavelength range of 400-700 nm, which matches
the main bioluminescence emission spectrum (440-540 nm) as
reported in \cite{widder2010}. An extra mooring line (named IL07)
equipped with RDI 300 kHz acoustic Doppler current profilers, a
conductivity-temperature-depth (SBE 37 SMP CTD) probe and
PMTs was added to monitor environmental variables (Fig. \ref{S1}). All
moorings are connected to a shore-station via an electro-optical
cable that provides real-time data transmission \citep{aguilar2007}. A dedicated
program of bioluminescence monitoring was implemented to
measure the total number of single photons detected every 13 ms
for each PMT. To consistently compare PMT counting rates
(bioluminescence) with oceanographic data (temperature, salinity,
current speeds) considering the acquisition interval of the later
(15 minutes), we calculated the median rates as a mathematical
estimator of PMT counting rates. The median was selected instead
of the arithmetic mean because of its higher robustness and least
disturbance by extreme values. Median rates were expressed in
thousands of photons per second or kHz (see Text S1 and
Fig. \ref{S2} a). The main light contributions recorded by PMTs result
from dark noise, from Cherenkov radiation induced by the beta
decay of $^{40}$K in seawater and from bioluminescence. The dark
noise is about $3 \pm 1$ kHz and remains constant with time \citep{aguilar2005}. The Cherenkov radiation induced by the beta decay of $^{40}$K in seawater produces a background of about $37 \pm 3$ kHz \citep{amram2002}, found to be constant within the statistical errors over a period of a few years \citep{aguilar2006,aguilar2010}. Therefore, all light increases over this constant background ($40 \pm 3$ kHz) can only be due to bioluminescence. The records of light intensity at IL07 are representative of those collected by the whole array of ANTARES PMTs (see Text S1
and Fig. \ref{S2} b).\\

Potential temperature, salinity, horizontal and vertical current
speeds (Fig. \ref{S1}) at the LION mooring line were measured with
SBE 37 SMP CTD probes and Nortek Aquadopp Doppler
current-meters regularly spaced between the subsurface (150 m)
and the seabed (2,350 m). Potential temperatures and vertical
velocities were corrected for the current-induced tilting and
deepening of the line. Hourly potential temperatures at the LDC
mooring line were measured with the temperature sensor of
Nortek Aquadopp Doppler current meters at 500 and 1,000 m
depth.\\

Proper calibrations of the CTD probes were performed using
the pre- and post-deployment calibrations made by the manufacturer.
The inter-comparison of instruments complied with quality control
procedures.\\

\textbf{Acknowledgments}
The authors thank the captains and crews of R.V. Tethys II and L'Europe for cruise assistance. Technical support by Ifremer, Assistance Ing�nierie Management (AIM) and Foselev Marine during sea operation and CCN$_2$P$3$ for providing computing facilities is greatly appreciated.\\

\textbf{Funding}\string:\\
This work was partially funded by the ANR-POTES program (ANR-05-BLAN-0161-01), ANTARES-Bioluminescence project (INSU-IN2P3), AAMIS project (Univ. M�diterran�e), EC2CO Biolux project (CNRS INSU), Excellence Research Groups (2009-SGR-1305, Generalitat de Catalunya), EuroSITES (FP7-ENV-2007-1-202955), MARINERA-REDECO (CTM2008-04973-E/MAR), HERMIONE (FP7-ENV-2008-1-226354), KM3NeT-PP (212525), ESONET NoE (FP6- GOCE-036851), DOS MARES (CTM2010-21810-C03-01) and CONSOLIDER-INGENIO GRACCIE (CSD2007-00067) projects. SM was granted a MERNT fellowship (Ministry of Education, Research and Technology, France). LH acknowledges the support of the Direction G�n�rale de l'Armement (supervisor\string: Elisabeth Gibert-Brunet). The authors also acknowledge the financial support of the funding agencies\string: CNRS, CEA, ANR, FEDER fund and Marie Curie Program, R�gions Alsace and Provence-Alpes-C�te d'Azur, D�partement du Var and Ville de La Seyne-sur-Mer, France; BMBF, Germany; INFN, Italy; FOM and NWO, the Netherlands; Council of the President of the Russian Federation for young scientists and leading scientific schools supporting grants, Russia; ANCS, Romania; MICINN (FPA2009-13983-C02-01), PROMETEO (2009/026) and MultiDark (CSD2009-00064). The funders had no role in study design, data collection and analysis, decision to publish, or preparation of the article.\\

\begin{sidewaysfigure}[!h]
\linespread{1} 
\centering
\includegraphics[width=16cm]{S1.jpg}
\caption[Configuration of the mooring lines from which the data presented in this study were obtained.]{Supporting information. Configuration of the mooring lines from which the data presented in this study were obtained. They include the cabled IL07 ANTARES as well as the autonomous LION and Lacaze-Duthiers Canyon (LDC) mooring lines. Location is shown in Fig. \ref{F1}.}
\label{S1}
\end{sidewaysfigure} 

\begin{figure}[!h]
\linespread{1} 
\centering
\includegraphics[width=13cm]{S2.jpeg}
\caption[(a) Raw counting rates from one photomultiplier (PMT) on the IL07 line (ANTARES site). (b) Median rates from the IL07 PMT (red) and mean of all median rates of the 885 ANTARES PMTs (blue) from January to April 2009.]{Supporting information. (a) Raw counting rates from one photomultiplier (PMT) on the IL07 line (ANTARES site). Counts are expressed in thousands of photons per second (kHz). The median rate is computed for each 15-minute data sample (red horizontal line). The dataset shown in the figure was recorded on March 28$^{th}$, 2010 with a median rate of 68 kHz and a current speed of 13 cm s$^{-1}$. (b) Median rates from the IL07 PMT (red) and mean of all median rates of the 885 ANTARES PMTs (blue) from January to April 2009.}
\label{S2}
\end{figure} 

\begin{figure}[!h]
\linespread{1} 
\centering
\includegraphics[width=10cm]{S3.jpeg}
\caption[Potential temperature versus salinity diagram of near-bottom CTD time-series at the ANTARES site from the IL07 line (red dots) and CTD profiles (lines) collected close to the ANTARES site.]{Supporting information. Potential temperature versus salinity diagram of near-bottom CTD time-series at the ANTARES site from the IL07 line (red dots) and CTD profiles (lines) collected close to the ANTARES site. (a) May 2007 to January 2009; (b) January to December 2009; and (c) December 2009 to January 2011. The data shown are from depths in excess of 1,000 m. Dotted lines correspond to potential density anomaly isolines in kg m$^{-3}$.}
\label{S3}
\end{figure} 

\begin{figure}[!h]
\linespread{1} 
\centering
\includegraphics[width=13cm]{S5.jpeg}
\caption[Illustrative ocean colour satellite images used to outline the limits of winter open-sea convection areas in the Gulf of Lion.]{Supporting information. Illustrative ocean colour satellite images used to outline the limits of winter open-sea convection areas in the Gulf of Lion. (a) Images plotted with a classical, full range, linear palette. (b) Images plotted with a simplified four level palette. The images shown correspond to days 1, 2, 7 and 18 February 2010, which are also transferred into Fig. \ref{F1} b-d. White pixels are indicative of lack of data due to cloud coverage.}
\label{S5}
\end{figure} 

\begin{figure}[!h]
\linespread{1} 
\centering
\includegraphics[width=14cm]{S6.jpg}
\caption[Dissolved organic carbon and oxygen concentrations at the ANTARES site in 2010. ]{Supporting information. Dissolved organic carbon and oxygen concentrations at the ANTARES site in 2010. Dissolved Organic Carbon (DOC) was measured by high temperature combustion on a Shimadzu TOC 5000 analyzer \citep{sohrin2005}. A four point-calibration curve was performed daily with standards prepared by diluting a stock solution of potassium hydrogen phthalate in Milli-Q water. Procedural blanks run with acidified and sparged Milli-Q water ranged from 1 to 2$\mu$M C and were subtracted from the values presented here. Deep seawater reference samples (provided by D. Hansell; Univ. Miami) were run daily (43.5$\mu$M C, n = 4) to check the accuracy of the DOC analysis. Oxygen concentration time-series was obtained using an oxygen optode Anderaa fitted on the IL07}
\label{S6}
\end{figure}  

\section{Conclusions}
Through this chapter, the first analysis of multivariate time series at the ANTARES site, leads to the detection of several observations, despite a descriptive approach. The deep ANTARES site is characterized by current direction East-West with speed generally lower than 20 cm s$^{-1}$. This current direction is unexpectedly following the general path of the cyclonic surface mesoscale Northern Current (NC) \citep{millot2005,millot1999}. However, phenomena such as deep circulation, topography, and mesoscale eddies could possibly locally modify the expected current direction at this deep station \citep{testor2006}. To successfully explain such result, a detailed study is needed explaining the respective weight of local (NC instability forced by topographic effects) vs. regional circulations (downward propagating mesoscale variability of the NC, interactions with the spreading Western Mediterranean Deep Water...). The bioluminescence recorded on the instrumented line IL07 is representative of the whole telescope and generally lower than 1,500 kHz. Moreover, we found that the bioluminescence intensity is correlated with the event detection using cameras placed on the instrumented line.\\

However, high bioluminescence-activity events have been distinguished in March 2009 and March 2010. During these events, bioluminescence intensity is linked to current-speed values above a threshold of 19 cm s$^{-1}$, and determined using regression trees. Moreover, in 2009, the highest current-speed intensity is essentially coming from the western direction. On the contrary, in 2010, this direction is mainly South-East. Furthermore, a high bioluminescence activity is linked to similar current directions, West in 2009 and South-East in 2010. These similar informations do not permit to discriminate if the high bioluminescence activity observed is mainly due to current direction or current speed. Regression trees permit to detect changes in variability over time that occurred at the same time for bioluminescence, salinity and temperature. Noticeably, a threshold defined for the second event detected has been highlighted in March 18$^{th}$ 2010. These changes in variability have not been noticed for current speed. The highest values for bioluminescence are dependent on both current speed above 19 cm s$^{-1}$ and temperature above 12.92\degres C (see regression trees Figure \ref{S4}) or with a salinity above � 38.479 (see \citealp{tamburini2013}), depending on mathematical methods used. During these strong events, the study of video-camera images recorded using automatic detection demonstrates that the number of detected events is not anymore correlated to the bioluminescence activity.\\

\begin{figure}[!h]
\linespread{1} 
\centering
\includegraphics[width=12cm]{illustration.jpg}
\caption[Artistic representation of newly-formed water masses in the Gulf of Lion and their impact on biological activity (bioluminescence) at the ANTARES station.]{Artistic representation of newly-formed water mass in the Gulf of Lion and their impact on biological activity (bioluminescence) at the ANTARES station. Graphical design\string: \underline{www.mathildedestelle.com}.}
\label{illus}
\end{figure} 

The analysis of surrounding oceanographic data, at a regional scale, gave clues to explain such sudden changes in bioluminescence activity by new water-mass formation at the surface inducing a deep-sea convection (Figure \ref{illus}). This phenomenon reaches the ANTARES observatory in March 2009 and 2010, modifying both the activity and the presence of potentially bioluminescent populations. Possible links between (1) new water-mass input, enrichment of the environment, (2) duration of the emission of  bioluminescence, (3) increase of the signal baseline as well as (4) photomultiplier ability to detect bioluminescent bacteria, led this work to focus on the characterization of bacterial bioluminescence. These first investigations highlighted the interest of developing methods for time series analysis, in order to detect such high bioluminescence-intensity events as well as discriminating possible links between environmental variables over time.\\
% Chapitre 2
\chapter[\textit{In situ} dataset analyzes using statistical methods adapted to time series]{\textit{In situ} dataset analyzes using statistical methods adapted to time series
\chaptermark{Statistical methods for time series analysis}}
\chaptermark{Statistical methods for time series analysis}

\minitoc
%\label{chap2}
\newpage
%\section{Introduction: time series analysis}
%One major hypothesis for time series analysis is to assume the ergodicity theory. Ergodicity is defined as a long term behavior system, evolving in time, and time average of one sequence of events is the same as the ensemble average. As \textit{in situ} time series are real time-dependent, there is only one set of data supposed to be statistically representative of the system involving that ergodicity theory is assumed. This lack of increase difficulties for statistical methods. To reach informations from these datasets, several time series analyses methods have been developed as model-based approach (ARIMA, GAM...), or data-analysis methods in time domain or frequency domain.  \\
\section{Article 2}
\subsection{Foreword}

Following the observation and description of two events of high bioluminescence intensity in March 2009 and March 2010, the use of appropriate statistical methods was needed for time series analysis. In signal processing, several methods exist and can be based on model approach (ARMA or ARIMA models for example), time-dependent methods (Markov chains, autocorrelation) or frequency-dependent methods (spectral analyzes, seasonality). In this work, a time-frequency analysis has been performed and developed on chronological data characterized as non-stationary and non-linear. A second objective of time series analysis from the ANTARES station  is to discriminate bioluminescence events related to the current speed and those related to water-mass modifications. The necessity to cross informations of this multivariate dataset implies the comparison and the use of two conjoint methods such as the wavelet and the Hilbert-Huang decomposition methods.\\

%Suite � l'observation et � la description de deux �v�nements de forte activit� de bioluminescence en Mars 2009 et 2010, l'analyse de ces donn�es a n�cessit� l'utilisation de m�thodes statistiques appropri�es au traitement de s�ries temporelles. En analyse du signal, plusieurs m�thodes existent et peuvent �tre bas�es sur une approche de mod�lisation (mod�les ARMA, ARIMA), d'analyses dans le domaine temporel (chaines de Markov, autocorr�lation) ou dans le domaine des fr�quences (analyse spectrale, saisonnalit�). Dans ce travail, une analyse en temps-fr�quence a �t� d�velopp�e sur les s�ries chronologiques caract�ris�es comme non-stationnaires et non-lin�aires. Un second objectif de l'analyse des s�ries temporelles issues du site ANTARES �tait de discriminer les �v�nements d'activit� de bioluminescence li�es au courant, et ceux li�s aux modifications de masses d'eau. La n�cessit� de croiser les informations de ce jeu de donn�es multivari� a impliqu� la comparaison et l'utilisation conjointe de deux m�thodes: la d�composition en ondelettes et la d�composition d'Hilbert-Huang.\\

\textit{This work has been initiated during my master 2 and finalized during my PhD work. I performed the statistical analyzes supervised by D. Nerini. C. Tamburini took part in the result interpretation within an ecological context.}\\
 %\textit{Ce travail a �t� initi� au cours de mon stage de master 2 et finalis� pendant mon doctorat. David Nerini et moi-m�me avons effectu� l'ensemble des analyses statistiques. Christian Tamburini a apport� un avis critique sur l'interpr�tation �cologique des r�sultats.}\\

\newpage
\subsection[Relation between deep bioluminescence and oceanographic variables: a statistical analysis using time-frequency decompositions.]{Relation between deep bioluminescence and oceanographic variables: a statistical analysis using time-frequency decompositions.
\sectionmark{ARTICLE 2}}
\sectionmark{ARTICLE 2}
\label{manuscrit2}
\begin{center}
S�verine Martini$^{1,2,\ast}$, David Nerini$^{1,2}$, Christian Tamburini$^{1,2}$
\end{center}
\vspace{10mm}

$^1$Aix-Marseille Universit\'{e},  Mediterranean Institute of Oceanography (MIO), 13288, Marseille, Cedex 09, France\\
$^2$Universit\'{e} du Sud Toulon-Var (MIO), 83957, La Garde cedex, France ; CNRS/INSU, MIO  UMR 7294; IRD, MIO UMR 235\\
\vspace{2mm}
\begin{center}
$\ast$ Corresponding author : \underline{severine.martini@univ-amu.fr}\\
\end{center}
\vspace{18mm}
\textbf{Submitted to Progress in Oceanography, under revision}\\

\newpage
\textbf{Abstract}\\
We consider the statistical analysis of a two-years high-frequency-sampled time series, between 2009 and 2010, recorded at the ANTARES observatory in the deep NW Mediterranean Sea (2,475 m depth). The objective is to analyze relationships between bioluminescence and environmental data (temperature, salinity and current speed). As this entire dataset is characterized by non-linearity and non-stationarity, two time-frequency-decomposition methods (wavelet and Hilbert-Huang) have been tested. These mathematical methods are dedicated to the analysis of oscillations within a signal at various time and frequency scales. Many applications of these methods dealing with only one variable may be found in the literature. The methodological contribution of this work is to propose some statistical tools dedicated to the study of relationships between two time series. Our study highlights three events of high bioluminescence activity in March 2009, December 2009 and March 2010. We show that the two events occurring in March 2009 and 2010 are correlated to the arrival of newly formed deep water masses at frequencies about $4.8 \times 10^{-7}$ (period of 24.1 days). In contrast, the event in December 2009 is only correlated with current speed at frequencies about $1.9 \times10^{-6}$ (period of 6.0 days). The jointed use of wavelet and Hilbert-Huang decompositions has proved to be successful for the analysis of multivariate time series. These methods are well-suited in a context of increasing number of long time series recorded in oceanography.\\

\newpage
\section*{Introduction}
The sampling and understanding of complex environmental systems aim at the detection of potential disturbances as a shift from the intrinsic variability of these systems. Marine systems are variable at all time and space scales \citep{hewitt2007} and their variability is still poorly understood due to sampling strategy, instrumentation and spatio-temporal-heterogeneity challenges. In response to this lack of knowledge, there is an international effort to capitalize oceanographic data and costs of autonomous and for mobile infrastructures that enable the detection of long term environmental context as well as episodic events or perturbations \citep{favali2006}. As an example, OceanSITES, and more generally the Global Ocean Observing System (GOOS), are the major scientific teams integrating a global network of more than 60 \textit{in situ} observatories and acquiring long-term and high-frequency time series \citep{send2011} over the world.\\

The ANTARES project (Astronomy with a Neutrino Telescope and Abyss environmental RESearch) developed one of those deep-sea-cabled observatory, in the North-Western Mediterranean Sea (Figure \ref{map}), since the end of 2007. This infrastructure is part of a global data network such as EMSO, KM3NeT, ESONET and EuroSITES for example. At first, the ANTARES observatory is dedicated to the search of high-energy particles such as neutrino \citep{amram2000, aguilar2007, ageron2011}. About 885 photomultiplier tubes (PMTs) are installed between 2,000 and 2,400 m depth for the purpose of particle Physics. All the 12 ANTARES mooring lines are connected, via an electro-optical cable to a shore station providing real-time acquisition. With the installation of a specific line, namely IL07 \citep{tamburini2013}, this deep observatory gives the opportunity to record simultaneously high frequency oceanographic data such as current speed, salinity and temperature (Figure \ref{TS}). The IL07 is also equipped with PMTs devoted to the recording of bioluminescence activity. These datasets provide an extraordinary way to study the dynamics of the deep ecosystem in real time and at high frequency \citep{craig2009}.\\

Recently, \citet{tamburini2013} proposed an introductory and  descriptive analysis of ANTARES oceanographic time series gathering both physical and biological variables. This has been jointly performed with synchronous hydrological records from a surrounding station located in the Gulf of Lion. They highlighted a link between high bioluminescence activity and changes in the properties of deep waters (temperature and salinity as proxies). Such changes attributed to an open-sea-convection event have renewed the deep waters (so called newly formed deep water) by the fall of upper ocean layer through the water column \citep{tamburini2013, marshall1999, stabholz2013, bethoux2002}. Then, open-sea convections represent a major vector in fueling the deep-sea ecosystem and inducing bioluminescence blooms.\\

The present study attempts to understand the mechanisms inducing deep-sea bioluminescence activity using temperature, salinity, current speed and bioluminescence time series on the same ANTARES dataset of \citet{tamburini2013}. The exploration of characteristic scales in time and frequency, in this high-frequency-sampled dataset, is achieved using statistical methods for signal processing. While most environmental data are non-stationary \citep{rao2008, cazelles2008, ghorbani2013} most methods used in time-series analyzes are based on stationarity assumptions, as the Fourier decomposition does \citep{frazier1999}. Firstly, we propose the use of two complementary mathematical methods (Wavelet and Hilbert-Huang decomposition) to deal with non-stationarity and to decompose each time series within time and frequency space. Secondly, we suggest original tools to quantify and illustrate links between bioluminescence time-frequency decomposition and other environmental variable time-frequency decompositions.\\

\begin{figure}[!h]
\linespread{1} 
\centering
\includegraphics[width=13cm]{Fig1_2.jpg}
\caption[Map of the North-Western Mediterranean Sea and the ANTARES site (black dot) where the underwater neutrino telescope is immersed at 2,475 m depth. ]{Map of the North-Western Mediterranean Sea and the ANTARES site (black dot) where the underwater neutrino telescope is immersed at 2,475 m depth. High-frequency-sampled time series of bioluminescence, temperature, salinity and current speed have been sampled between 2009 and 2010 at this station.}
\label{map}
\end{figure} 

\section*{Dataset for the deep bioluminescence study}
\label{nonstat}
The ANTARES site is located 40 km off the French Mediterranean coast (42\degres 48'N, 6\degres 10'E) at 2,475 m depth (see Figure \ref{map}). This work focuses on multivariate time series sampled from the beginning of 2009 until October 2010, for oceanographic variables such as salinity, potential temperature ($\degres C$) and current speed ($cm$ $s^{-1}$) (Figure \ref{TS}). Moreover, bioluminescence emission ($kHz$) has been recorded in 2009 and 2010, from the IL07 instrumented line of the ANTARES telescope. The unit of bioluminescence activity (kHz) refers to the photon-counting rate per second. We consider the bioluminescence time series as a variable response submitted to changes in environmental characteristics.\\

\begin{figure}[!h]
\linespread{1} 
\centering
\includegraphics[width=14cm]{Fig2_2.jpeg}
\caption[Time series recorded in the Mediterranean Sea, at the ANTARES station, between January 2009 and July 2010, for bioluminescence, current speed ($cm$ $s^{-1}$), salinity, and temperature ($\degres C$). ]{Time series recorded in the Mediterranean Sea, at the ANTARES station, between January 2009 and July 2010, for bioluminescence, current speed ($cm$ $s^{-1}$), salinity, and temperature ($\degres C$). Those high-frequency data, recorded each 1 to 15 minutes, are sampled at depth between 2,478 m and 2,169 m and transferred in real time to the database.}
\label{TS}
\end{figure} 

In Figure \ref{TS}, each environmental variable is recorded with a high-frequency-sampling rate, involving the presence of a high level of noise. The base line of each signal is close to a constant background signal and does not show a clear trend, looking at low-frequency variations. However, some breaking sequences appear on each time series giving their non-linear and intermittent appearance. There is no clear time connection between those sequences from one time series to another. However, they are all characterized by a sudden increase in variability with two main events observed simultaneously occurring in April 2009 and from April to June 2010. If the connection between variables is relatively clear for the event in 2010, it is more tricky to conclude for the event in 2009 without adapted mathematical methods.\\

As shown previously the time series are not stationary. A time series $X(t), t\in\mathbb{R}^+$ is said to be stationary if  $X(t)$ owns the same statistical properties (mean, variance,..) as the shifted time series $X_{t+h}$ for $h>0$. A Kwiatkowski-Phillips-Schmidt-Shin (KPSS) test, from \cite{Kwiatkowski1992}, shows that all time series are non-stationary (Table \ref{statest}).\\% The non-stationarity is confirmed when displaying the distribution of the bioluminescence data. Indeed, the distribution is bimodal with a heavy tail for high values of bioluminescence (data not shown). Data are alternatively concentrated around two modes (4.36 and 5.34 referring to 78.2 and 208.5 kHz respectively) and, more rarely, it can appear some strong values in bioluminescence, referring to the extreme events above the value of $\log (7)$.\\

\begin{table}
\center
\caption{KPSS-test for testing stationarity of each time series.}
\begin{tabular}{cccc}
\textbf{Variable} & \textbf{KPPS value} & \textbf{p-value} & \textbf{Interpretation}\\
\hline
Bioluminescence (kHz) & 8.24 & 0.01 & non-stationary\\
Current speed (cm $s^{-1}$) & 5.63 & 0.01 & non-stationary\\
Temperature (\degre C) & 8.50 & 0.01 & non-stationary\\
Salinity & 2.29 & 0.01 & non-stationary\\
\end{tabular}
 \label{statest} 
\end{table} 

%\begin{figure}[!h]
%\linespread{1} 
%\centering
%\includegraphics[width=10cm]{Fig3_2.pdf}
%\caption[Density of sampled values for the bioluminescence intensity (into logarithmic scale) with superimposed highest density regions.]{Density of sampled values for the bioluminescence intensity (into logarithmic scale) with superimposed highest density regions. Green, red and blue bars represent quantile intervals centered around local maxima of the density that respectively contains 50\%, 95\% and 99\% of observed data. The Hyndman's density quantile algorithm is used for computation of the quantiles \citep{Hyndman1996}. The black line is a kernel density estimate. The bimodal distribution with a heavy tail on the highest bioluminescence intensity values show the non-stationarity of dataset.}
%\label{hdr}
%\end{figure} 

\section*{Methods dealing with non-stationary time series}
Most methods in time-series analysis consist in an expansion of the studied signal $X(t)$ into a linear combination of known basis functions. For the Fourier decomposition, the signal is expressed as a linear combination of trigonometric functions:
\begin{eqnarray}
X(t)\ =\  \sum_{j=0}^\infty a_j e^{i\omega_jt}. 
\end{eqnarray} 
This decomposition provides an analytical expression of the function $X(t)$ with amplitude coefficients $a_j$ giving weight to frequencies $\omega_j$. Both coefficients $a_j$ and  $\omega_j$ are independent of time providing a global decomposition of the signal in the space of frequencies.\\

The Hilbert-Huang (HHG) decomposition proposes a generalization of the Fourier decomposition by expanding $X(t)$ such that:
\begin{eqnarray}
\label{fourier}
X(t)\ =\ \sum_{j=1}^{n} a_j(t) e ^{i2\pi\int \omega_j (\tau)d\tau} + R_{n+1}(t) 
\end{eqnarray} 
where both amplitude coefficients $a_j(t)$ and frequency coefficients $\omega_j(t)$ are functions of time. As further developed, this decomposition relies on a preliminary step in order to decompose the time series into $n+1$ modes where $R_{n+1}(t)$ is the trend of the time series.\\

Another decomposition with time-dependent parameters can be constructed using the continuous wavelet decomposition 
\begin{eqnarray}
X(t)\ =\ \frac{1}{W_\psi} \int_{-\infty}^{\infty} \int_{0}^{\infty} W_X(a,b) \psi_{a,b}(t) \frac{1}{a^2}da db
\end{eqnarray}
In that case, the shape of the basis function, $\psi_{a,b}(t)$, is controlled by a scale parameter $a$ which can be interpreted as a reciprocal of frequency and a shifting parameter $b$, both parameters varying continuously over $\mathbb R^+$ and $\mathbb R$ respectively. The analysis is conducted through the estimation of the coefficient $W_X(a,b)$, the wavelet transform, whose values are scale and time dependent and on the wavelet function $\psi_{a,b}(t)$. The constant $W_\psi$ is an admissible value depending on the chosen wavelet function.\\

Once the basis expansion is achieved, each method gives the ability to analyze the distribution of energy over each frequency through the coefficient values of the decomposition. However, the Fourier analysis relies on a global expansion of the signal which can induce spurious harmonic components that cause energy spreading. This misinterpretation especially occurs when dealing with non-linear and non-stationary datasets as shown in Figure \ref{freq} B (see introduction of \citet{Huang1998} for more details) compared to stationary signal decomposed in Figure \ref{freq} A. Unlike Fourier decomposition, the continuous wavelet transforms and the HHG decomposition possesses the ability to construct a time-frequency representation as the coefficients of the decomposition are locally time dependent, which is well adapted to the analysis of non-stationary signals.\\

\begin{figure}[!h]
\linespread{1} 
\centering
\includegraphics[width=12cm]{Fig4_2.pdf}
\caption[Two examples of signal decomposition. On the left, the signal represented over time, on the right, the periodogram over frequencies A) non-stationary and B) stationary signal.]{Two examples of signal decomposition. On the left, the signal represented over time, on the right, the periodogram over frequencies A) non-stationary and B) stationary signal. The non-stationarity spreads energy over frequencies in the periodogram representation.}
\label{freq}
\end{figure} 

\subsection*{Wavelet decomposition}

\subsubsection*{Decomposition into wavelet basis}
The continuous wavelet decomposition is a consecutive pass-band filter through time series \citep{Torrence1998,addison2010}. The decomposition acts as a linear filter which extracts special features inside the signal through a projection over a wavelet basis $\psi_{a,b}$ giving the coefficient
\begin{eqnarray} 
W_X (a,b) =\int_{-\infty}^{+\infty} X(t) \psi^{\ast} _{a,b} (t;a,b) dt
\end{eqnarray}
where $\psi^{\ast}_{a,b}$ denotes the complex conjugate of the basis wavelet. The peculiarity of that decomposition is the continuous dependency of the basis functions 
\begin{eqnarray} 
\psi _{a,b} (t)= \frac{1}{\sqrt{a}} \psi (\frac{t-b}{a})
\end{eqnarray}
regarding scale parameter $a$ and location parameter $b$. The scale parameter $a$ is inversely proportional to the frequency. The decomposition mainly relies on the choice of a mother-basis function. \cite{Torrence1998} define some elements to choose the mother wavelet: orthogonality, real or complex function, shape of the mother wavelet related to shape of analyzed time series. In this study, we choose the Morlet wavelet function, commonly used in wavelet analysis. However, other choices can be relevant as well \citep{ahuja2005,cazelles2008}. The Morlet wavelet is defined as:
\begin{eqnarray}
\psi (t)= \frac{1}{\pi ^{1/4}} e^{i2 \pi f_0 t} e^{-t^2/2}
\end{eqnarray}
With $\frac{1}{\pi ^{1/4}}$ a normalization factor, $e^{i2 \pi f_0 t}$ the complex sinusoidal (referring to $cos (2\pi f_0 t + i sin (2 \pi f_0 t)$) and $e^{-t^2/2}$ the Gaussian envelope. This periodic wavelet owns symmetrical properties ($ \int \psi(t) dt = 0$) and a unit norm ($ \int \vert \psi(t)\vert ^2 dt=1$). The decomposition of time series consists in coefficients for the real part of wavelet transform $W_X(a,b)$, at the position $b$. These coefficients are calculated by continuously moving the wavelet along the signal (modifying the position parameter $b$) related to a specific scale parameter $a$. Then, a bi-dimensional surface of $W_X(a,b)$ is constructed (Figure \ref{wlt}).\\

\subsubsection*{Wavelet spectral representation}
Using the wavelet decomposition, the time-frequency spectrum is plotted for bioluminescence time series in Figure \ref{wlt} A, current speed in Figure \ref{wlt} B, and salinity in Figure \ref{wlt} C. For the bioluminescence spectrum (Figure \ref{wlt} A), three distinct periods appear with high $W_X (a,b)$ coefficients (in red). The two main periods in April 2009 and April 2010, appear with low excited frequencies, between $4.8 \times 10^{-7} Hz$ (period of 24.1 days) and $3.8 \times 10^{-6} Hz$ (period of 3.0 days). A third one in December 2009 is observed at about $1.9 \times 10^{-6} Hz$ (period of 6.0 days). Moreover, frequencies at about $2.0 \times 10^{-5} Hz$ (period of 0.6 days or 14 h) are also excited with high amplitude coefficients.\\

\begin{figure}[p]
\linespread{1} 
\centering
\includegraphics[width=14cm]{Fig5_2.pdf}
\caption[Time-frequency representation using the wavelet decomposition for A) bioluminescence B) current speed and C) salinity time series.]{Time-frequency representation using the wavelet decomposition for A) bioluminescence B) current speed and C) salinity time series. The color scale for $W_X(a,b)$ is from blue (low coefficient) to red (high coefficient). For the bioluminescence decomposition, three periods with high coefficients are distinct close to the dates 04.2009, 12.2009 and 04.2010. Black-contour lines represent the 5\% significant level. The shaded part represents the cone of influence where edge effects appear.}
\label{wlt}
\end{figure} 

For the current speed spectrum (Figure \ref{wlt} B), lower frequencies at about $4.8 \times 10^{-7} Hz$ (period of 24.1 days) appear with high coefficients for the whole period. Moreover, a very distinct frequency band is highlighted at $2.0 \times 10^{-5} Hz$ (period of 0.6 days or 14 h). This last range of excited frequencies is known to be linked to the "internal waves" \citep{huthnance1995}. These periodic oscillations generated from the surface and transmitted to the deep sea have already been observed at periods about 17.6 h at the ANTARES station \citep{vanharen2011}. The spectrum shows that if these waves are present during the whole sampled period, their frequency and amplitude coefficients vary over time. As an example, there are higher coefficients in January 2010 than in September 2009.\\

For salinity (Figure \ref{wlt} C), two main events are distinct in April 2009 and April 2010 with very high coefficients, and excited frequencies at about $9.5 \times 10^{-7}$ (period of 12.2 days) and $4.8 \times 10^{-7}$ (period of 24.1 days), respectively.\\

\subsection*{Hilbert-Huang decomposition}

\citet{Huang1998} proposed a second method for non-stationary and non-linear signal analysis \citep{Huang1998,Huang2008,Huang2009}. The Hilbert-Huang decomposition (HHG) is based on the assumption that a signal $X(t)$ is multi-component meaning that it can be expanded into a sum of $n$ signals such that  
\begin{eqnarray}
X(t)=\sum_{j=1}^{n}C_j(t)+R_{n+1}(t),
\end{eqnarray} 
where the $C_j$ are called intrinsic mode functions (IMFs) and $R_{n+1}$ is the trend of $X(t)$. Each IMF is supposed to be almost mono-component \textit{i.e.} composed with only one instantaneous frequency, evolving with time, in a bounded range. The HHG decomposition is then realized in two main steps : the extraction of IMFs and their decomposition into instantaneous frequencies.\\
 
\subsubsection*{Empirical mode decomposition}
\label{HHGmeth}
To constitute mono-component signals, the $n$ IMFs must fulfill two conditions: \\$ (i) $ the number of times passing through the origin is identical to the number of extreme or differs by one for the most \\$ (ii) $ in any time, the mean between the local maximum and the local minimum is zero.\\

Following these conditions, the decomposition of $X(t)$ into IMFs is based on the following steps : \\ 
\textbf {Step 1} - Identification of the extremes (minimum and maximum) of the signal $X(t)$. \\
 \textbf {Step 2} - Connection of these extremes by natural cubic spline interpolation for the construction of the maximal envelope $ e_ {max} (t) $. Same process used for the lower envelope $ e_ {min} (t ) $. \\
 \textbf {Step 3} - Computation of the average of the two envelopes: $ m (t) = [e_ {min} (t) + e_ {max} (t)] /2 $, as shown in Figure \ref{consIMF}. \\
 \textbf {Step 4} - Computation of the locally centered time series with $ d (t) = X (t)-m (t) $. \\
 \textbf {Step 5} - Steps 1-4 are repeated on $d(t)$ until convergence towards a time series with zero mean $m(t)$ is sufficient. The convergence is controlled by a sifting criterion. \\
 \textbf {Step 6} - If $ d (t) $ is zero mean then properties (i) and (ii) are checked and the function constitutes an IMF: $ C(t) $. \\
 \textbf {Step 7} - Construction of a new $X(t)$ by  subtracting $C(t)$ to previous $X(t)$. Repeat steps 1 to step 6. Extracting IMFs is proceeded until rule $(i)$ is broken.\\ 
 \textbf {Step 8} - The last remainder constitutes  the trend $R_{n+1}(t)$ which is, by construction, a monotonic or a constant function.\\
 
\begin{figure}[!h]
\linespread{1} 
\centering
\includegraphics[width=13cm]{Fig6_2.pdf}
\caption[Schematic representation of the signal decomposition using the Hilbert-Huang method.]{Schematic representation of the signal decomposition using the Hilbert-Huang method. The upper, and lower envelopes are represented by solid lines enclosing the signal. The mean is the centered line and the empty circles represent the extremes.}
\label{consIMF}
\end{figure} 

Decomposing the data into IMFs, an important question is to compare the signal decomposition with a white-noise decomposition. This would differentiate the decomposition using the Hilbert-Huang method from an artefactual decomposition due to random noise. The interpretation of physical processes embedded in the data will then be reinforced. Figure \ref{signifjak} represents the Hilbert-Huang decomposition of a hundred simulated white-noise data (filled circles) and a partial set of data from the bioluminescence time series decomposition (empty circles). The log  of energy (or squared amplitude coefficients) is represented  over the log of mean frequency for each IMF. The empty circles outside the dark gray zones show that IMF decomposition is different from Gaussian noise.\\ 

\begin{figure}[!h]
\linespread{1} 
\centering
\includegraphics[width=14cm]{Fig7_2.pdf}
\caption[Example of significance for the IMFs decomposition from a partial set of data.]{Example of significance for the IMFs decomposition from a partial set of data. The gray zones, from dark to light gray, represent high density regions containing respectively 0.01, 0.25, 0.75, 0.99 $\%$ of the data centered around local maximum. Empty circles, out of the area are the IMFs, significantly different from Gaussian noise. E is the energy into IMFs and mean frequency is the average frequency of each IMF band. This graph shows that HHG method removes the interpretation of IMF from a randomly artefactual decomposition.}
\label{signifjak}
\end{figure} 

\subsubsection*{IMFs on ANTARES time series}
All oceanographic variables recorded at the ANTARES site have been decomposed into IMFs using the HHG method. The empirical mode decomposition (EMD) extracts 10 IMFs and a trend for each time series. Figure \ref{imfbiolu} represents the IMFs for the bioluminescence time series with associated variability intervals. Variability intervals are calculated by generating one hundred decompositions of each time series varying the sifting criterion, more or less 10\%, to stop the process.\\

\begin{figure}[p]
\linespread{1} 
\centering
\includegraphics[width=15cm]{Fig8_2.pdf}
\caption[Hilbert-Huang decomposition for the bioluminescence signal into 10 intrinsic mode functions, from C1 to C10.]{Hilbert-Huang decomposition for the bioluminescence signal into 10 intrinsic mode functions, from C1 to C10. These intrinsic mode functions are stationary for the first order. In gray, the variability intervals are plotted by moving the threshold defined to stop the sifting process. The trend of the data is not plotted.}
\label{imfbiolu}
\end{figure}

For the first IMFs (C1 to C5), the variability intervals are very low and can not be differentiated from the main IMF. They become more important for the last IMFs with, however, similar oscillations over time. By construction, the IMFs are first order stationary. The first extracted IMFs are high-pass filter and last IMFs are low-pass filter. The two bioluminescence events in 2009 and 2010 appeared on functions of higher frequencies (Figure \ref{imfbiolu} from C1 to C7). Once the linear decomposition achieved, the next step of the Hilbert-Huang method is to extract time-dependent frequencies (instantaneous frequencies) from each IMF.\\

\subsubsection*{Instantaneous frequency using the Hilbert transform}
The original signal $X(t)$ is now expressed as a linear combination of IMFs which are supposed to be mono component. Considering an IMF $C(t)$, the function
\begin{eqnarray}
Z(t) = C(t)+iY(t),
\end{eqnarray}
is constructed, where $Y(t)$ is the Hilbert transform of the IMF $C(t)$ with:
\begin{eqnarray}
Y(t)=\frac{1}{\pi}P\int_{-\infty}^{\infty} \frac{C(t')}{t-t'}dt',
\end{eqnarray}
with $P$ the Cauchy principal value \citep{paget1972}. The construction of $Z(t)$ makes the decomposition of the IMF possible such that:
\begin{eqnarray}
Z(t) = a(t) e ^{i \theta(t)},
\end{eqnarray}
where $a(t)$ is the  instantaneous amplitude and $\theta(t)$ the instantaneous phase. The parameter $a(t)$ reflects how the energy of the signal varies with time and is given with:\begin{eqnarray}
a(t)=[C(t)^2+Y(t)^2]^{1/2}.
\end{eqnarray}
The instantaneous phase $\theta(t)$ computed with: 
\begin{eqnarray} 
\theta(t) = \arctan \left( \frac{Y(t)}{C(t)} \right)
\end{eqnarray}
yields the instantaneous frequency $w(t)$ such that:
\begin{eqnarray}
w (t) = \frac{1}{2\pi}  \frac{d\theta(t)}{dt}. \end{eqnarray}

Applying the instantaneous frequency decomposition for each IMF $C_j$ forms the HHG decomposition of the signal $X(t)$. It can be expressed as a linear combination of trigonometric functions with amplitude coefficients $a_j(t)$ and frequency coefficients $w_j(t)$ depending on time:
\begin{eqnarray}
 X(t)\ =\ \sum_{j=1}^{n} a_j(t) e ^{i2\pi\int \omega_j (\tau)d\tau} + R_{n+1}(t) 
\end{eqnarray}

\subsubsection*{Dyadic frequency decomposition}
In the literature, \citet{Huang2009} give an empirical maximal number of IMFs between 6 and 12, depending on the complexity of time series. The IMF decomposition acts as a pass-band filter by separating frequency bands \citep{Flandrin2003}. A way to understand how the empirical decomposition takes place for the HHG method is to study the frequency narrow band for each extracted IMF. A linear relationship (using least-square method) between the logarithms of the IMF average frequencies and over the number of IMF, is defined empirically following an exponential law such as
\begin{eqnarray}\label{eq:dyadic}
log(\omega _m (n)) \approx -n\ log(\alpha)
\end{eqnarray}

with $w_m(n)$ the mean frequency for the IMF $n$ \citep{huang2009river}. On this log-representation, the relation is linear with a slope coefficient $\alpha$ and the $R^2$ coefficient determined to characterized the quality of the fit.\\

\begin{figure}[!h]
\linespread{1} 
\centering
\includegraphics[width=14cm]{Fig9_2.pdf}
\caption[Representation of the empirical dyadic decomposition of the HHG method.]{Representation of the empirical dyadic decomposition of the HHG method. The linear relation is fitted for the logarithm of the mean frequency of each intrinsic mode function, for each variable. Error bars are added for each intrinsic mode function. The slope coefficient close to 2 define a dyadic decomposition for all time series.}
\label{dyadic}
\end{figure}

Figure \ref{dyadic} represents the mean frequency and the standard deviation for the instantaneous frequencies calculated for each IMF from the time series of salinity, bioluminescence, temperature and current speed. The $\alpha$ coefficients from equation (\ref{eq:dyadic}) estimated for each of the variables are between 2.27 and 2.30 except for the current speed for which the coefficient is lower (1.92). These relations define a dyadic decomposition. Each mean frequency from an IMF is about a half of the one extracted before. This is observed empirically in our case as well as in the literature \citep{Flandrin2003,Huang2008}, but no mathematical assumptions have been done at the beginning of the decomposition method \citep{Huang2009,Massei2012}. This observation differs from the wavelet decomposition for which a dyadic assumption is done using the set of scales $b$ selected as a functional power of 2:
\begin{eqnarray}
b_i = b_0 \times 2^{i \delta},  i=0,...,M
\end{eqnarray}

with $ M = \frac{1}{\delta} log_2 \frac{(N \Delta t)}{b_0}$, N the number of values in the time series, $\Delta t$ the time sampling and $\delta$ a scale factor (small values for $\delta$ will give a finer definition). $b_0$ is the initial parameter $b$ and $b_i$ the parameter at time $i$.\\

\subsubsection*{Thin-plate spline smoothing}
The evolution of instantaneous frequencies and amplitude over time can be represented as a time-frequency-amplitude plot. However, a HHG spectrum is discrete and cannot be cross-correlated between two variables. Indeed, the IMFs do not form a common basis where time-series could be expanded. The number of extracted IMFs is different from one time series to another, depending on the properties of the time series themselves. One way to cross time-frequency decompositions can be achieved by smoothing spectra over both time and frequency in order to obtain a continuous representation. This is done with thin plate regression splines \citep{bookstein1989,mardia1996,wood2003}.\\

We dispose of  $n$ observations $\left( y_{i},\mathbf{x}_{i}\right) $ where $\mathbf{x}_{i}=\left( \omega_i,t_i\right) \in \mathbb{R}^{2}$ is a position in the time-frequency space and $y_i$, the computed value of the discrete Hilbert spectrum from the IMFs decomposition. We want to estimate the regression surface $H\left( \mathbf{x}\right)$ as a continuous spectrum such that:
\begin{eqnarray}y_{i}=H\left( \mathbf{x}_{i}\right) +\varepsilon _{i},\end{eqnarray}
where $\varepsilon _{i}$ is a random error term. Thin plate splines can be used to estimate $H$ by finding the function $g$ minimizing
\begin{eqnarray}\left\Vert \mathbf{y-g}\right\Vert ^{2}+\lambda J\left( g\right),\end{eqnarray}
where $\mathbf{y}$ is the vector of data with entries $y_{i}$, $\mathbf{g=}\left( g\left( \mathbf{x}_{1}\right) ,...,g\left( \mathbf{x}_{n}\right)\right) ^{\prime }$, $J\left( g\right) $ is a penalty function 
\begin{eqnarray}J\left( g\right) = \iint\nolimits_{\mathbb{R}^{2}}\left( \frac{\partial ^{2}g}{\partial \omega^{2}}\right)^2 +2\left( \frac{\partial ^{2}g}{\partial\omega\partial t}\right)^2+\left(\frac{\partial ^{2}g}{\partial t^2}\right)^2d\omega dt,\end{eqnarray}
giving a measure of roughness for $g$ and $\lambda $, a parameter which controls the trade-off between the fit to data and the smoothness of $g$.\\

The value of $\lambda$ is determined by cross-validation. The solution of the spline regression problem provides a continuous Hilbert spectrum $H$ for a given variable. The Hilbert spectrum gives a measure of amplitude contributing to each frequency and one can calculate the associated marginal spectrum $h (\omega)$ such that:
\begin{eqnarray}h (\omega)= \int_{0}^{T} H(\omega ,t) dt
\end{eqnarray}

\subsubsection*{HHG spectral representation} 
Figures \ref{spectres} (from A to C) are the continuous spectral representations after smoothing the HHG spectra using thin plate splines for the bioluminescence (Figure \ref{spectres} A), the current speed (Figure \ref{spectres} B), and the salinity (Figure \ref{spectres} C). \\

For bioluminescence, in Figure \ref{spectres} A, three events are distinct. The two main occur from April to May 2009, and from March to June 2010 with high amplitude coefficients and at frequencies between $4.8 \times 10^{-7}$ (period of 24.1 days) and $1.0 \times 10^{-5} Hz$ (period of 1.2 days). The third one in December 2009  is defined with lower amplitude coefficients and at $1.9 \times 10^{-6} Hz$ (period of 6.0 days). Frequency of $2.0 \times 10^{-5} Hz$ (period of 0.6 days or 14 h) is not excited contrary to the Figure \ref{wlt} A.\\
 
For current speed, in Figure \ref{spectres} B, a large band of frequencies is excited between $4.8 \times 10^{-7}$ (period of 24.1 days) and $1.0 \times 10^{-5} Hz$ (period of 1.2 days). Another band of excited frequencies is defined at  $2.0 \times 10^{-5} Hz$ (period of 0.6 days or 14 h) (dotted box). These frequencies are linked to internal waves previously observed and described in Figure \ref{wlt} B. The internal waves are mainly embedded in the two first IMF (Figure \ref{dyadic}). \\

For salinity, in Figure \ref{spectres} C, the two main events clearly appear in April 2009 and April 2010 at about $9.5 \times 10^{-7}$ (period of 12.2 days) and at about $3.8 \times 10^{-6}$ (period of 3.0 days).  \\

These spectra are similar to those obtained using the wavelet decomposition Figure \ref{wlt}. Both wavelet and Hilbert-Huang decomposition methods result in a tri-dimensional continuous representation in the time-frequency domain. To quantify links in multivariate time series, the following step is to cross these spectra in pairs.\\

\begin{figure}[!h]
\linespread{1} 
\centering
\includegraphics[width=11cm]{Fig10_2.pdf}
\caption[Spectral representation using HHG decomposition and thin-plate spline smoothing for A) bioluminescence, B) current speed and C) salinity from January 2009 to July 2010.]{Spectral representation using HHG decomposition and thin-plate spline smoothing for A) bioluminescence, B) current speed and C) salinity from January 2009 to July 2010. The x-axis is the time, the y-axis the frequency (Hz) and the z-colored-scale is the amplitude coefficient value. In B) the dotted box highlights frequencies excited by internal waves.} 
\label{spectres}
\end{figure}

\section*{Bivariate time-frequency cross-analysis}
\subsection*{Wavelet coherence}

One of the key issue for the analysis of the ANTARES time series is to compare decomposition spectra between variables. This point is well developed for the wavelet method using the cross-wavelet representation including both coherence-coefficient values and phase delays \citep{Torrence1998,addison2010}. We consider two time series X(t) and Y(t) and their wavelet transforms $ W_X(a,b)$ and $W_Y(a,b)$, respectively. The coherence is a measure of the squared correlation between wavelet functions of time series for time and frequency instantaneously. This value allows to distinguish the higher coefficients for both crossed variables, $W_X(a,b)$ and $W_Y(a,b)$ \citep{grinsted2004,torrence1999}. The coherency corresponds to the square of the crossed and normalized spectra:

\begin{eqnarray}
C^2(a,b)=\frac {S (W_{XY}(a,b))}{S(\mid W_X (a,b) \mid ^2)  S(\mid W_Y (a,b)) \mid ^2}, \forall 0 \leq C^2(a,b) \leq 1.
\end{eqnarray}
S is a smoothing operator on both time and scale of the wavelets, and $W_{XY}(a,b)$ the cross-wavelet defined from :
\begin{eqnarray}
W_{XY}(a,b) = W_X(a,b) W^*_Y(a,b).
\end{eqnarray}

\subsection*{HHG cross analysis}
In order to cross two continuous spectrograms using HHG, only few solutions have been proposed compared to the wavelet method \citep{chen2010}. Considering smoothed time-frequency spectrum $H_X(\mathbf{x})$ and $H_Y(\mathbf{x})$ for variables $X$ and $Y$ (Figure \ref{spectres}), one way to analyze the links between variables is to summarize the dependence of the time-frequency spectrum across the only frequency argument. The computation of the correlation between both images is carried out by transformation of each time-frequency spectrum in a collection of curves.\\

Let $\left\lbrace t_1,...,t_n \right\rbrace $ be an arbitrary regular fine grid over time such that $t_1<t_2<...<t_n$. The construction of a collection of pairwise curves $\left\lbrace (X_k(f),Y_k(g)),k=1,...,n\right\rbrace$ with $X_k(f)=X(t_k,f)$ and $Y_k(g)=Y(t_k,g)$ is done taking slices of time-frequency spectra for fixed times with $f$ and $g$ specific frequencies. Starting with these pairs of functions considered as a sample, the marginal spectra is defined :\begin{eqnarray}
\overline{X}(f)=\frac{1}{n}\sum_{k=1}^{n}X_k(f) \end{eqnarray}
and \begin{eqnarray}
\overline{Y}(g)=\frac{1}{n}\sum_{k=1}^{n}Y_k(g) \end{eqnarray}
and associated variance functions are \begin{eqnarray}V_X(f)=\frac{1}{n}\sum_{k=1}^{n}\left(X_k(f)-\overline{X}(f)\right)^2 \end{eqnarray} and \begin{eqnarray} V_Y(g)=\frac{1}{n}\sum_{k=1}^{n}\left(Y_k(g)-\overline{Y}(g)\right)^2 \end{eqnarray}\\
The cross-correlation between $X_k(t)$ and $Y_k(t)$ is given by:
\begin{eqnarray}
C_{XY}(f,g)=\frac{1}{n}\sum_{k=1}^{n}\frac{(X_k(f)-\overline{X}(f))}{\sqrt{V_X(f)}}\frac{(Y_k(g)-\overline{Y}(g))}{\sqrt{V_Y(g)}}\end{eqnarray}\\
This bivariate function has the property that: $-1 \leq C_{XY}(f,g) \leq 1, \forall f,g \in [0,f_{max}]$. It gives a measure of links between variables $X$ and $Y$ in the space of frequency. This cross-correlation on frequencies gives clues on major frequencies in a different way than the wavelet-coherence spectrogram.\\

\section*{Discussion}
\subsection*{Bioluminescence mechanically stimulated by current speed}

The effects of current speed on bioluminescence activity have been well investigated in the literature \citep{cussatlegras2007,cussatlegras2005,Fritz1990}. This mechanical stimulation triggers more bioluminescent response enhancing cell membrane excitation in fluid motions \citep{cussatlegras2007,Blaser2002}. Based on ANTARES time series, \citet{tamburini2013} and \citet{vanharen2011} show that generally bioluminescence activity increases with current speed. This is thought to be due to either impacts of small-sized bioluminescent organisms and particles on the PMTs or the reaction of such organisms to enhance turbulent motion in the wakes of the PMTs.\\

In Figure \ref{coherence} A, when crossing the bioluminescence and current speed spectra, from the wavelet decomposition method, coherence coefficients are between 0.6 and 0.95, most of the time, over a wide range of frequencies. Moreover, there is a high cross coherence coefficient of about 0.85 at frequency about $2.0 \times 10^{-5} Hz$ (period of 0.6 days or 14 h), referring to the internal wave frequencies \citep{huthnance1995,vanharen2011}. These observations reinforce the well-understood link between current speed and bioluminescence already known as a mechanical stimulation of bioluminescent organisms by current speed. Since no phase delay is noticed (represented by the right direction of arrows in Figure \ref{coherence} A) the light emission appears instantaneously after mechanical stimulation.\\

\subsection*{Links between bioluminescence and new water masses}

\cite{tamburini2013} interpret salinity and temperature time series as proxies of newly formed deep water, spreading at the ANTARES site. The authors propose the use of a salinity threshold value to highlight links between water masses and bioluminescence activity. As shown by these authors, from an ecological point of view, deep-water formations impact bioluminescent organisms in two different ways. On the one hand, these water inputs carry significant amount of bioluminescent planktonic organisms down from the surface. On the other hand, dissolved organic matter exports as well as re-suspension of particulate organic matter from sediment to the water column fuel deep-sea bioluminescent activity \citep{tamburini2013}. In the present work, we explore the ANTARES time series using statistical analyzes taking into account non-stationarity and non-linearity properties of this multivariate dataset.\\

Figure \ref{coherence} B displays results from the wavelet decomposition, crossing bioluminescence and salinity time series. Similar results are obtained crossing bioluminescence and temperature (Figure not shown). Both salinity and temperature variables represent proxies of newly formed deep water. Figure \ref{coherence} B shows two spots of high-coherence values ranging from 0.8 to 0.9, from March to May 2009 and March to June 2010. In 2009, high-coherence coefficients are highlighted for frequencies of $4.8 \times 10^{-7} Hz$ (period of 24.1 days). In 2010, high correlations are observed for frequencies between $2.4$ and $9.5 \times 10 ^{-7} Hz$ (periods of 48.2 and 12.2 days). This cross-spectra emphasized a similar range of excited frequencies at $4.8 \times 10 ^{-7} Hz$ in both 2009 and 2010 with, in addition, a very high correlation between water-mass proxies and bioluminescence activity, for both time. Nevertheless, no hypothesis for the low frequency of $4.8 \times 10 ^{-7} Hz$ can be validated.\\

\begin{figure}[!h]
\linespread{1} 
\centering
\includegraphics[width=12cm]{Fig11_2.pdf}
\caption[Cross spectrogram between bioluminescence and A) current speed and B) salinity, using the wavelet decomposition and coherence measurement.]{Cross spectrogram between bioluminescence and A) current speed and B) salinity, using the wavelet decomposition and coherence measurement. Black arrows represent the phase delay (right arrows mean no phase delay). Black line represents the edge effect for no significant results. The color-scale is the value for the coherence coefficient from 0 to 1 (blue to red).}
\label{coherence}
\end{figure}

Interestingly, a third bioluminescence event, occurring in December 2009, is observed on time series (Figure \ref{TS}) and on spectrogram (Figure \ref{wlt} A) at frequencies between $9.5$ and $1.9 \times 10^{-6} Hz$ (period of 6.1 days). While this event is correlated with current speed in Figure \ref{coherence} A (coherence coefficient between 0.7 and 0.8), there is no changes in water masses (Figure \ref{coherence} B, coherence values below 0.2).  Consequently, in December 2009, current speed generates a mechanical stimulation of bioluminescent organisms and, by its only effect, increases the bioluminescence activity. The Figure \ref{coherence} A and B clearly helps to discriminate the events of high bioluminescence intensity affected by newly formed deep water and current speed (April 2009 and 2010) and the events only affected by current speed (December 2009).\\

\begin{figure}[!h]
\linespread{1} 
\centering
\includegraphics[width=10cm]{Fig12_2.pdf}
\caption[Cross spectrogram between bioluminescence and oceanographic variables. ]{Cross spectrogram between bioluminescence and oceanographic variables. The cross-correlation coefficients (represented between -0.2 and 0.8) are plotted as isolines and color scale for bioluminescence with A) current speed and B) salinity. x and y axes represent frequencies for each variable. The dotted lines represent the range of frequencies of the most correlated intrinsic mode functions. These intrinsic mode functions are shown in Figure \ref{crossIMF}. }
\label{tenso}
\end{figure}

Figure \ref{tenso} A and B, resulting from the HHG decomposition, illustrates cross-correlations between frequency spectrum of bioluminescence and those of environmental variables (current speed and salinity). This representation is an original way to measure cross variations between two HHG spectra but induces a loss in time dependency. The correlation coefficients vary from -0.2 to 0.79 with higher coefficients on the diagonal (black line on Figure \ref{tenso} for both A and B panels). This 1:1 relation shows that highest excited frequencies are the same for both variables (see Figure \ref{dyadic}). Nevertheless, in Figure \ref{tenso} B, crossing bioluminescence and salinity, some patterns of high cross coefficients occur out of the 1:1 line. It shows that salinity (consequently newly formed deep water) and bioluminescence may be correlated at different frequencies. Then, in order to get back to a time-dependent representation, we isolate the IMFs referring to the higher correlation coefficients (black dotted lines in Figure \ref{tenso}). In Figure \ref{tenso} A, the 3$^{rd}$ IMF for both current speed and bioluminescence are isolated and plotted simultaneously in Figure \ref{crossIMF} A. The mean frequency of IMF 3 for bioluminescence is about $6.1 \times 10^{-6}$ (period of 1.9 days) and for current speed $5.7 \times 10^{-6}$ (period of 2.0 days). In Figure \ref{tenso} B, the 5$^{th}$ IMF for salinity and the 2$^{nd}$ one for bioluminescence are represented together in Figure \ref{crossIMF} B. This Figure shows that the high-correlation coefficient is mainly due to the two events of high bioluminescence activity with simultaneous oscillations induces. The mean frequency of IMF 5 for salinity is  $1.9 \times 10^{-6}$ (period of 6.1 days) and for IMF 2 for bioluminescence is $1.1 \times 10^{-5}$ (period of 1.1 days). These results from the HHG decomposition are not easily detected in Figure \ref{coherence}.\\

\begin{figure}[!h]
\linespread{1} 
\centering
\includegraphics[width=13cm]{Fig13_2.pdf}
\caption[Representation of the most correlated intrinsic mode function for A) bioluminescence and current speed, B) bioluminescence and salinity.]{Representation of the most correlated intrinsic mode function for A) bioluminescence and current speed, B) bioluminescence and salinity. In A), mean frequency for bioluminescence IMF 3 is $6.1 \times 10^{-6}$ (period of 1.9 days) and for current speed IMF 3 is  $5.7 \times 10^{-6}$ (period of 2.0 days). In B), mean frequency for bioluminescence IMF 2 is $1.1 \times 10^{-5}$ (period of 1.1 days) and for salinity IMF 5 is  $1.9 \times 10^{-6}$ (period of 6.1 days).  }
\label{crossIMF}
\end{figure}

\subsection*{Two different mathematical methods}

From a technical point of view, HHG and the wavelet decomposition provide complementary results. The HHG method is based on empirical decomposition which provides a basis expansion composed with IMFs. This decomposition only relies on the data itself without any assumption on time series shape. This is not the case for wavelet decomposition which relies on an arbitrary choice of the mother wavelet to decompose the signal. However, the bivariate analysis is more efficient using wavelet decomposition. Indeed, due to the different number of IMFs between variables, time-frequency cross spectra cannot be defined using HHG method. We propose in this study, a bivariate cross analysis which allowed to get the cross information between variables in only the frequency space.\\

\section*{Conclusion}
There is a clear interest in recording long time series to understand intrinsic variations in an ecosystem. High frequency samplings help to detect characteristic scales within a signal. Few time-frequency-decomposition methods are adapted to the analysis of non-stationary and non-linear signals and consequently, to the detection of frequencies excited during unusual events. Using Huang-Hilbert and wavelet decomposition, we show that current speed involves mechanical stimulation of bioluminescent organisms and increases light emission. The use of time and frequency-decomposition methods demonstrates that current speed stimulates bioluminescence at almost all frequencies and without delay. Their undeniable advantage is the possibility to highlight correlation between bioluminescence and the arrival of newly-formed deep waters at specific time (beginning in March 2009 and March 2010) for frequencies at about $4.8 \times 10^{-7} Hz$ (period of 24.1 days). This excited frequency has no clear interpretation yet. Moreover, these statistical methods discriminate events due to newly-formed-deep-water spreading and events only due to variations in current speed. Finally, the use of those two methods applied to signal processing in oceanography for coherent dataset is definitely a way to better interpret variability, special events and links between variables.\\

\noindent \textbf{Acknowledgements}\\
SM was granted a MERNT fellowship (Ministry of Education, Research and Technology, France). The authors thank the Collaboration of the ANTARES deep-sea observatory for providing time series data. Authors especially thank S. Escoffier, M. Garel and C. Curtil for their comments and their inputs to this work. Cross wavelet and wavelet coherence software were provided by A. Grinsted.\\
\newpage

\section{Complementary data}
%\subsection{Example with tsunami dataset}
Preliminary tests have been performed for Fourier, wavelets and Hilbert-Huang decomposition using well-defined time series. Dataset is the sea level record during a tsunami arrival in 2004 at the Jackson Bay, in New zealand (data from NOAA \underline{http ://wcatwc.arh.noaa.gov/}). The signal is divided into two processes. The first one is the tidal periodic oscillation at low frequency and the second one is the tsunami arrival at higher frequencies. The second signal is starting at day 362 and with a modification of its amplitude over time (Figure \ref{tsun1} A).\\

\begin{figure}[!h]
\linespread{1} 
\centering
\includegraphics[width=12cm]{tsun1.pdf}
\caption[Fourier decomposition of the sea level dataset during a tsunami event.]{Fourier decomposition of the sea level dataset during a tsunami event. A) tidal periodic signal (grey line) extracted using Fourier decomposition, B) Periodogram with tidal frequency highlighted and C) Residuals from the signal.}
\label{tsun1}
\end{figure} 

In Figure \ref{tsun1} the Fourier decomposition is performed with 7 coefficients (see \ref{fourier}). This method easily define the periodic tidal wave in Figure \ref{tsun1} B, however, it does not permit to access to the tsunami arrival signal represented by residuals in Figure \ref{tsun1} C. This is mainly due to the non-stationarity problem highlighted in \ref{nonstat}.\\ 

The decomposition of the dataset using the Hilbert-Huang method well-describe the two processes in Figure \ref{tsun2}. A high frequency function and a low frequency one are extracted as the sum of Intrinsic Mode functions (IMFs). The time frequency spectra represented from this decomposition is shown with instantaneous frequency and amplitude coefficient evolving over time in Figure \ref{tsun3} a. Thin plate splines are applied on this dataset and represented after smoothing in Figure \ref{tsun3} b. \\

\begin{figure}[!h]
\linespread{1} 
\centering
\includegraphics[width=9cm]{tsun2.pdf}
\caption[Time series decomposition within pass-band filter using the Hilbert-Huang decomposition method (see \citep{Huang1998} and \ref{HHGmeth}).]{Time series decomposition within pass-band filter using the Hilbert-Huang decomposition method (see \citep{Huang1998} and \ref{HHGmeth}). On top time series of sea level during tsunami arrival is represented. The second plot represents the first extracted functions at high frequencies representing the tsunami arrival. The third plot represents the last functions extracted from data with low frequency and can be interpreted as the tidal periodic signal.}
\label{tsun2}
\end{figure} 

\begin{figure}[!h]
\linespread{1} 
\centering
\includegraphics[width=14cm]{hhg_tsun.pdf}
\caption[Time-frequency spectra using the Hilbert-Huang decomposition method (see \citep{Huang1998} and \ref{HHGmeth}).]{Time-frequency spectra representation using the Hilbert-Huang decomposition. a) Discrete representation before smoothing B) after thin plate spline smoothing method. }
\label{tsun3}
\end{figure} 

Similar pass-band filter should be represented using the wavelet decomposition method. Time series and time-frequency spectra are represented in Figure \ref{tsun3} A and B respectively.\\

\begin{figure}[!h]
\linespread{1} 
\centering
\includegraphics[width=11cm]{tsun3.pdf}
\caption[A) Tsunami arrival time series and B) time-frequency spectra (frequency is $\frac{1}{Period}$).]{A) Tsunami arrival time series and B) time-frequency spectra (frequency is $\frac{1}{Period}$). This graphical representation clearly define both signals with a low frequency periodic over the whole time considered (dark red band) and the tsunami event occurring at day 362 (blue patterns at high frequencies). }
\label{tsun3}
\end{figure} 

This well-defined and already described example clearly illustrate problems of non-stationarity and advantages of time-frequency decomposition methods.\\




% Chapitre 3
\chapter[In the laboratory, effects of forcing variables using a bioluminescent bacterial strain as a model.]{In the laboratory, effects of forcing variables using a bioluminescent bacterial strain as a model.
\chaptermark{Effects of forcing variables}}
\chaptermark{Effects of forcing variables}

\minitoc
%\label{chap3}
\newpage
\section[Introduction: working with controlled variables and models.]{Introduction: working with controlled variables and models.
\sectionmark{Introduction: laboratory experiments}}
\sectionmark{Introduction: laboratory experiments}
In previous chapters, high bioluminescence events, observed in March 2009 and 2010, have been explained to be linked to newly-formed-water masses sinking to the deep ocean. Bioluminescent organisms might be involved in this hydrological phenomenon into two different ways. First, bioluminescent organisms could be brought from the surface by physical transport and spread to the deep sea before reaching the ANTARES site. Secondly, newly-formed-deep-water inputs into the deep sea involve modifications in dissolved oxygen or carbon availability. Small variations, such as those observed in temperature and salinity, can not rationally influence growth and physiology of organisms or involve modifications in their bioluminescence activity. However, pressure effect on sinking organisms or carbon content are more foreseeable to affect growth, physiology and then bioluminescence activity.\\  

Amongst the diversity of bioluminescent organisms, and due to a global lack of knowledge on \textit{in situ} bacterial bioluminescence, we investigated bioluminescent bacteria as a part of the light signal. To validate or not this assumption, an intermediate step is to evaluate the potential bioluminescence activity of such bacteria under various environmental conditions. The bacterial strain \textit{Photobacterium phosphoreum} ANT-2200 has been isolated in 2005, at the ANTARES station during a high bioluminescence event \citep{alali2010}. Moreover, \textit{Photobacterium phosphoreum} has been described as the main bioluminescent bacterial species in the deep Mediterranean Sea \citep{gentile2009}. For those two major reasons, this strain is used as model for experiments in the laboratory. In this Chapter, we will apprehend the effects of high pressure, temperature, and carbon availability on both growth and bioluminescence activity. The results from these experiments will permit to criticized our hypotheses under controlled conditions and extrapolate such results at \textit{in situ} scale. To reach such an aim, we divide laboratory experiments into two intermediate steps.\\

\begin{itemize}
\item In \ref{part1}, \textit{Photobacterium phosphoreum} ANT-2200 is cultivated into a carbon rich growth medium (Sea Water Complete medium) under various temperature and pressure conditions to determine optimal growth conditions for this strain and potential effects on bioluminescence activity. Moreover, logistic-growth model has been applied to those experiments.\\
  
\item Finally, in \ref{part3}, \textit{Photobacterium phosphoreum} ANT-2200 is cultivated into a growth medium closer to environmental composition (ONR7a) and as close as possible to \textit{in situ} conditions under high pressure (22 MPa) and \textit{in situ} temperature (13\degres C). The instrumentation has been developed in the laboratory and these experiments are validating the ability to measure both bioluminescence and oxygen consumption over time.\\
\end{itemize}
 
Part \ref{part1} has been investigated and major results published into an article \citep{martini2013}. Then, \ref{part3} validates the instrumentation, methods and the ability to record bioluminescence and bacterial growth as well as physiological parameters close to environmental conditions. The part \ref{part3} is preliminary and still need to be improved and replicated in order to robustly interpret the results. However, these experiments are presented given the originality of the methods and the ability to record the data at high frequency and in an automatic way.\\
 
\section{Description of \textit{Photobacterium phosphoreum} ANT-2200 strain}
The bioluminescent bacterial strain \textit{Photobacterium phosphoreum} ANT-2200 has been isolated in the laboratory in 2005. At sea, 500 mL of water have been sampled at 2,200 m depth, at the ANTARES station and filtered. After cultivation on a solid medium, one luminous UFC (Unit Forming Colony) has been isolated. A 16S-rRNA-gene sequence defines the strain close to \textit{Photobacterium phosphoreum} (100\%) strain IFO 13896. Based on phylogeny, the isolate has been named \textit{Photobacterium phosphoreum} ANT-2200 (Figure \ref{ant2200}). For more details, see \cite{alali2010}.\\

\begin{figure}[!h]
\linespread{1} 
\centering
\includegraphics[scale=0.4]{photobact.pdf}
\caption[\textit{Photobacterium phosphoreum} ANT-2200 strain emitting bioluminescence in liquid growth medium and solid medium.]{\textit{Photobacterium phosphoreum} ANT-2200 strain emitting bioluminescence in liquid growth medium and solid medium. Bioluminescent bacteria emit light at wavelength about 490 nm (blue-green emission).}
\label{ant2200}
\end{figure}

\begin{figure}[!h]
\linespread{1} 
\centering
\includegraphics[scale=0.6]{phyloANT.pdf}
\caption[Neighbor-joining tree showing the phylogenetic position of \textit{Photobacterium phosphoreum} ANT-2200 strain.]{Neighbor-joining tree showing the phylogenetic position of \textit{Photobacterium phosphoreum} ANT-2200 strain and related taxa based on 16S rDNA sequences. Numbers at nodes are levels of bootstrap support (\%) based on neighbor-joining analyzes of 1,000 re-sampled datasets. Only values above 70\% are given. Bar 0.01 substitutions per nucleotide position. From \cite{alali2010}.}
\label{phyloant2200}
\end{figure}

\newpage
\section{Article 3}
\label{part1}
\subsection{Foreword}
In this article, the effects of environmental variables (hydrostatic pressure and temperature) on the activity of \textit{Photobacterium phosphoreum} ANT-2200 strain have been studied. This bacterial strain, isolated at the ANTARES site, has been used as a model. This work has been started during B. Al Ali PhD (2010) with a part of the data acquisition for growth and bioluminescence, and has been finalized during my PhD work.\\

%Dans ce manuscrit, les effets de variables environnementales (pression hydrostatique et temp�rature) sur l'activit� de la souche (croissance et  bioluminescence) ont �t� �tudi�s. \textit{Photobacterium phosphoreum} ANT-2200, isol�e sur le site ANTARES, a �t� utilis�e comme mod�le. Ce travail, d�but� au cours de la th�se de Badr Al Ali (2010) avec l'acquisition d'une partie des donn�es de croissance et de bioluminescence, a �t� finalis� dans le cadre de mon travail de th�se.\\

%\subsection{Description du syst�me hyperbare de bioluminescence}

%Le syst�me hyperbare de bioluminescence sous pression a �t� r�alis� en collaboration avec Metro-mesures (Menecy, France). Ce systeme est constitu� d'un reservoir hyperbare ainsi que d'une chambre de culture. Adapt� des bouteilles haute pression (High Pressure Bottles HPBs, Figure \ref{HP} A), le reservoir est d�fini pour maintenir une pression hydrostatique maximale de 40 MPa. \\

%\begin{figure}[!h]
%\begin{small}
%\begin{center}
%\includegraphics[scale=0.1]{HP.jpg}
%\caption{A) High pressure bottle. On the left, the high pressure tank in stainless steel is opened, on the right, the top end cap with manometer for high pressure control. B) Hyperbaric bioluminescence tank.}
%\label{HP}
%\end{center}
%\end{small}
%\end{figure} 

%Il est constitu� d'un cylindre principal en acier APX4 (215-mm OD, 150-mm ID et 200-mm de longueur totale), ainsi qu'un conteneur cylindrique pourvu d'un couvercle d'extremit� en acier APX4 (205-mm OD, 69.7-mm de hauteur) ainsi que d'une couronne en bronze permettant de sceller le syst�me (160-mm OD, 98-mm de hauteur). Le conteneur dispose de deux ports de remplissage, une valve de s�curit� (920.86.00 H, Top Industrie S.A., France) ainsi qu'un manom�tre. A l'int�rieur du conteneur rempli d'eau MilliQ st�rile, la chambre de culture principale est constitu�e d'un tube cylindrique en PEEK (polyetheretherketones) 1/8''. La chambre de culture en ertalyte (mat�riau chimiquement et biologiquement inerte, de couleur blanche refl�tant la lumi�re) est constitu�e d'un cylindre (80-mm ID, 100-mm de hauteur) ainsi que d'un piston flotant surmont� d'un anneau. Afin de mesurer la bioluminescence dans la chambre de culture, un cone de plexiglass (13.3-mm de diam�tre inf�rieur,8.8-mm de diametre sup�rieu, 21.4-mm de hauteur) est positionn� � l'int�rieur d'un support en acier APX4 ins�r� � l'int�rieur de la chambre de culture en ertalyte. L'ensemble du syst�me est autoclavable afin de pr�server les conditions de st�rilit� de la culture. Le cone de plexiglas transmet les photons emis par les bact�ries bioluminescentes � l'int�rieur de la chambre de culture par le biais d'une fibre optique conn�ct�e � un photomultiplicateur (H7155, Hammamatsu) reli� � un boitier de comptage (C8855). Le comptage des photons est effectu� sur une int�gration au cours du temps d'�chantillonnage. Le bruit de fond est estim� avant chaque exp�rimentation et contr�l� comme inf�rieur � 10 photons sec$^{-1}$. Le volume maximal de la chambre de culture est de 500 mL. La temp�rature est r�gul�e � l'aide d'un tube flexible enroul� autour du syst�me hyperbare et r�gul� par un dispositif de refroidissement exerne. La pression hydrostatique est transmise au moyen d'un g�n�rateur de pression pilot�, repr�sent� en Figure \ref{GPP}, vers la chambre de culture par le piston flottant. \\

%\begin{figure}[!h]
%\begin{small}
%\begin{center}
%\includegraphics[scale=0.1]{GPP.jpg}
%\caption{Generator for Piloted Pressure dedicated to the high hydrostatic pressure }
%\label{GPP}
%\end{center}
%\end{small}
%\end{figure} 

\textit{This work granted by EC2CO-BIOLUX. C. Tamburini is the PI for the project. B. Al Ali, M. Garel, C. Tamburini and me performed the experimental part on the high pressure platform. I joined the bioreactor platform with S. Davidson, Y. Combet-Blanc, and R. Auria, for an experimental part using bioreactors (see Perspective part \ref{bioreactor}). L. Casalot, S. Isart and me worked using the pure culture of the bacterial strain. I did the modelisation part supervised by D. Nerini and M. Garel also join us for data analysis. V. Grossi, P. Cuny and M. Pacton have performed the physiological analyzes. M. Garel and C. Tamburini developed the hyperbaric system as well as the bioluminescence and oxygen measurement systems under high pressure.}

%\textit{Ce travail, financ� par le projet EC2CO-BIOLUX, est le r�sultat d'une collaboration au sein du laboratoire MIO. Christian Tamburini est porteur du projet. Badr Al Ali, Marc Garel, Christian Tamburini et moi-m�me avons r�alis� les exp�rimentations sur la plateforme haute pression. J'ai int�gr� l'�quipe de la plateforme fermenteur avec Sylvain Davidson, Yannick Combet-Blanc, et Richard Auria, pour les exp�rimentations en fermenteur. Laurie Casalot, St�phie Isart (stagiaire �cole d'ing�nieur) et moi-m�me avons travaill� sur la culture pure de la souche. La mod�lisation des donn�es a �t� effectu�e par David N�rini et moi-m�me, Marc Garel est �galement intervenu dans le traitement des r�sultats. Vincent Grossi, Philippe Cuny et Muriel Pacton ont r�alis� les analyses physiologiques compl�mentaires. Marc Garel et Christian Tamburini ont d�velopp� l'instrumentation hyperbare, ainsi que les mesures de bioluminescence et d'oxyg�ne sous haute pression hydrostatique.}\\

\newpage
\subsection[Effects of hydrostatic pressure on growth and luminescence of a moderately-piezophilic luminous bacteria \textit{Photobacterium phosphoreum} ANT-2200]{Effects of hydrostatic pressure on growth and luminescence of a moderately-piezophilic luminous bacteria \textit{Photobacterium phosphoreum} ANT-2200
\sectionmark{ARTICLE 3}}
\sectionmark{ARTICLE 3}

\begin{center}
Martini S.$^{1,2}$, Al Ali B.$^{1,2,\diamondsuit}$, Garel M.$^{1,2}$, Nerini D.$^{1,2}$, Grossi V.$^3$, Pacton M.$^3$, Casalot L.$^{1,2}$, Cuny P.$^{1,2}$ and Tamburini C.$^{1,2,\ast}$\\
\end{center}
\vspace*{10mm}

$^1$ Aix-Marseille Universit\'{e},  Mediterranean Institute of Oceanography (MIO), 13288, Marseille, Cedex 09, France ; \\
$^3$ Universit\'{e} du sud Toulon-Var (MIO), 83957, La Garde cedex, France ; CNRS/INSU, MIO  UMR 7294; IRD, MIO UMR 235.\\
$^4$ Universit\'{e} Lyon1, CNRS, Laboratoire de G\'{e}ologie de Lyon, UMR5276 Villeurbanne, France\\
$^5$ ETH Zurich, Geological Institute, CH-8092 Zurich, Switzerland\\

$^\diamondsuit $Present address: Tishreen University, High Institute of Marine Research, P.O. Box 2242, Ministry of Higher Education, Syrian Arab Republic\\

\begin{center}
$\ast$ Corresponding author : \underline{christian.tamburini@univ-amu.fr}\\
\end{center}
\vspace{10mm}
\textbf{Martini S, Al Ali B, Garel M, Nerini D, Grossi V, et al. (2013) Effects of Hydrostatic Pressure on Growth and Luminescence of a Moderately-Piezophilic
Luminous Bacteria Photobacterium phosphoreum ANT-2200. PLoS ONE 8(6): e66580. doi:10.1371/journal.pone.0066580}\\

\newpage
\textbf{Abstract}\\
Bacterial bioluminescence is commonly found in the deep sea and depends on environmental conditions. \textit{Photobacterium phosphoreum} ANT-2200 has been isolated from the NW Mediterranean Sea at 2200-m depth (\textit{in situ} temperature of 13\degres C) close to the ANTARES neutrino telescope. The effects of hydrostatic pressure on its growth and luminescence have been investigated under controlled laboratory conditions, using a specifically developed high-pressure bioluminescence system. The growth rate and the maximum population density of the strain were determined at different temperatures (from 4 to 37\degres C) and pressures (from 0.1 to 40 MPa), using the logistic model to define these two growth parameters. Indeed, using the growth rate only, no optimal temperature and pressure could be determined. However, when both growth rate and maximum population density were jointly taken into account, a cross coefficient was calculated. By this way, the optimum growth conditions for \textit{P. phosphoreum} ANT-2200 were found to be 30\degres C and, 10 MPa defining this strain as mesophile and moderately piezophile. Moreover, the ratio of unsaturated vs. saturated cellular fatty acids was found higher at 22 MPa, in agreement with previously described piezophile strains. \textit{P. phosphoreum} ANT-2200 also appeared to respond to high pressure by forming cell aggregates. Its maximum population density was 1.2 times higher, with a similar growth rate, than at 0.1 MPa. Strain ANT-2200 grown at 22 MPa produced 3 times more bioluminescence. The proposed approach, mimicking, as close as possible, the \textit{in situ} conditions, could help studying deep-sea bacterial bioluminescence and validating hypotheses concerning its role into the carbon cycle in the deep ocean.\\

\newpage
\subsection*{Introduction}

The deep sea, under 1,000 m depth, is characterized by a high hydrostatic pressure ($\geq$ 10 MPa), with, generally, a low temperature and a low organic-matter concentration. Laboratory experiments using pure cultures of piezophilic Bacteria have highlighted microbial adaptations to high hydrostatic pressure. The adaptive traits include those related to growth \citep{zobell1949,abe1999}, membrane \citep{yayanos1995}, and storage lipids \citep{grossi2010}, membrane and soluble proteins \citep{bartlett1989,kato2008}, the respiratory-chain complexes \citep{abe1999,yamada2000}, replication, transcription and translation \citep{bartlett1995,lauro2008}. Most isolated piezophilic Bacteria belong to the genera: \textit{Carnobacterium, Desulfovibrio, Marinitoga, Shewanella, Photobacterium, Colwellia, Moritella,} and \textit{Psychromonas} within the Gammaproteobacteria subclass reviewed by \cite{bartlett2007}.\\

Darkness is another major characteristic of this deep-sea environment that can be disturbed by a biological phenomenon named bioluminescence. Bioluminescence is the process by which living micro- or macro-organisms emit light. Among the bioluminescent organisms, marine luminous bacteria are ecologically versatile and can be found as free-living forms, epiphytes, saprophytes, parasites, symbionts in the light organs of fishes and squids, and commensals in the gut of various marine organisms \citep{ruby1980,hastings1983,herring1987}. Metagenomic analysis from deep eastern-Mediterranean water samples shows a surprising high number of \textit{lux} genes directly involved in bioluminescence \citep{martin2007}. As far as we know, all-known marine bioluminescent bacteria are phylogenetically affiliated to the \textit{Vibrio}, \textit{Photobacterium} and \textit{Shewanella} genera within the Gammaproteobacteria subclass \citep{kita2006}. Among them, \textit{Photobacterium phosphoreum} is the predominant species found in the Mediterranean Sea \citep{gentile2009}.\\

Those of the most studied micro-organisms are, for piezophily, \textit{Photobacterium profundum} SS9 (e.g. \citealp{vezzi2005}), not known as luminous, and for bioluminescence, \textit{P. phosphoreum} (e.g. \citealp{dunlap2006}). Up to date, little information is available concerning potential physiological-adaptation mechanisms of luminous bacteria to hydrostatic pressure, especially for both piezophily and bioluminescence. In this study, we used a bioluminescent strain isolated from Mediterranean deep-sea waters (sampled at 2,200-m depth) and identified as \textit{Photobacterium phosphoreum} ANT-2200 \citep{alali2010}. At this depth, the \textit{in situ} conditions of pressure and temperature are about 22 MPa and 13\degres C, respectively. The purpose of this study is (1) to define temperature and pressure optima for growth and (2) to study pressure effect (0.1 versus 22 MPa, 13\degres C) on growth and bioluminescence activities of \textit{P. phosphoreum} ANT-2200 using a new laboratory controlled hyperbaric system dedicated to high-pressure and bioluminescence studies.\\

\subsection*{Material and methods}

\subsubsection*{Growth parameters of \textit{P. phosphoreum} ANT-2200 under various temperature and hydrostatic-pressure conditions}
\textit{P. phosphoreum} ANT-2200 (GenBank accession number EU881910) was isolated from sea water collected in the Northwestern Mediterranean Sea at the ANTARES neutrino-telescope site (42\degres 54'N, 06\degres 06'E) at 2200-m depth (13\degres C) (see \citealp{alali2010} for details). Phenotypic and enzymatic characterizations are available in Supporting Information (Table S1). Procedures for pre-culturing were performed as described by \cite{alali2010}. For the determination of growth rate and maximum population density as a function of pressure and temperature, mid-log cultures were inoculated 1\string:10 into 5 mL sterilized syringes supplied with 3/4 of seawater complete medium (SCW medium) and with 1/4 oxygen-saturated Fluorinert$^{TM}$ FC-72 (3M$^{TM}$). The impoverished SCW liquid medium contained per liter (pH 7.5): 3 mg  of yeast extract, 5 mg of bio-peptone, 3 mL of glycerol, 250 mL of distilled water, and 750 mL of old sea water \citep{nealson1978}. Fluorinert$^{TM}$ FC-72 was used as oxygen supplier to ensure the growth and the luminescence of the bacterial strain in closed conditions \citep{kato1994,kato1995,yanagibayashi1999}.\\

Triplicates cultures were incubated at pressures of 0.1, 10, 22, 30 and 40 MPa and for temperatures of 4, 13, 20, 30 and 37\degres C. Syringes were placed into high-pressure bottles (HPBs). In order to avoid decompression-recompression of the samples, each HPB corresponded to one incubation time. Bacterial growth was estimated by measuring the optical density (OD$_{600nm}$) using a spectrophotometer (Perkin Elmer, Lamda EZ201 UV/Vis spectrophotometer).\\

Direct counting was also performed using epifluorescence microscopy with DAPI-staining procedure, according to \citep{porter1980}. This counting method was used to define the link between total-cell counts (DAPI counts) and optical density (OD$_{600nm}$) according to the equation (1). For DAPI-cell counts, to avoid possible artefact due to the aggregates, the samples were sonicated (3 min), vortexed (1 min), diluted with milliQ-water, then, sonicated (2 min), vortexed (1 min) and finally filtered on 0.2-$\mu$m-pore-size polycarbonate filters. The data have been, firstly, treated separately for atmospheric-pressure (0.1 MPa) and high-pressure (22 MPa) conditions. Since no significant difference has been observed between the two sets of data, a common relation has been defined as following:
\begin{center}
Number of DAPI-stained cells mL$^{-1}$ $= 6.7 \times 10^8 \times$ OD$_{600nm} - 2.3 \times 10^7 (1)$\\
$(R^2 = 0.79, N = 14)$
\end{center}
Traditionally, a linear regression is used to determine the growth rate of a strain during the logarithmic phase. The logistic (or Verhulst) model \citep{verhulst1838} was used in this study to determine both the growth rate ($r$) and the maximum population density ($K$). This model gives a continuous function of optical density, fitting discrete experimental data measured during the bacterial growth. Its hypotheses take into account limited resources in the medium and are defined as:\\
The birth rate:
\begin{eqnarray}
n (x)=\alpha - \beta x
\end{eqnarray}
The mortality rate:
\begin{eqnarray}
m (x)=\gamma - \delta x
\end{eqnarray}
$n(x)$ and $m(x)$ are linear functions with $\alpha$, $\beta$, $\gamma$, $\delta$ four real numbers and $x$ is the population density. The birth and mortality rates are supposed to be constant during time:
\begin{eqnarray}
\frac{dx}{dt} = nx - mx
\end{eqnarray}
Meaning that the logistic model is written as:
\begin{eqnarray}
\frac{dx}{dt} = rx 1- \frac{x}{K}
\end{eqnarray}
Where:
\begin{eqnarray}
K=\frac{\alpha - \gamma}{\beta - \delta}
\end{eqnarray}

$K$, being the maximum population density (expressed in optical density, OD$_{600nm}$) that \textit{P. phosphoreum} ANT-2200 can reach according to the growth conditions (temperature, pressure...) and $r$, being the growth rate, defined as:
\begin{eqnarray}
r = \alpha - \gamma
\end{eqnarray}
Biologically, the intrinsic growth rate ($r$, expressed in $h^{-1}$) is supposed to be positive (meaning that $\alpha > \gamma$).\\

A cross coefficient ($C_{r,K}$) has been calculated for both temperature and pressure effects on growth. If $R$ and $K$ are two  matrices with $n$ the number of temperature and $m$ the number of pressure conditions tested, the $C_{r,K}$ is defined as:
\begin{eqnarray}
C_{r,K}=\left( \frac{R}{max(R)} \right) \left( \frac{K}{max(K)}\right)
\end{eqnarray}
With  $ 0 < C_{r,K} < 1$\\

The values for growth rate ($r$, h$^{-1}$), maximum population density ($K$, OD$_{600nm}$) and cross coefficient ($C_{r,K}$) were used to construct extrapolated-contour plots for the pressure-temperature dependency using R software \citep{Rsoftware}.

\subsubsection*{Scanning electron and transmission electron microscopes}

Cultures of \textit{P. phosphoreum} ANT-2200 were performed at 0.1 and 22 MPa at 13\degres C. Cells were harvested at the end of the logarithmic phase and prepared for electron microscopy in order to observe cellular morphology and structure according to the pressure conditions.\\

Scanning Electron Microscopy (SEM) was performed according to two different procedures. On the one hand, cells were fixed with 0.2 \% glutaraldehyde, filtered on 0.2-$\mu m$-pore-size nucleopore membranes, washed with filtered and sterilized seawater with 2 \% osmic acid, and then with Milli-Q water. Washed cells were dehydrated and observed using SEM (FEI Quanta 250 FEG, Centre Technologique des Microstructures, University Claude-Bernard, Lyon 1). On the other hand, cells were rapidly frozen in liquid nitrogen and lyophilized for 48 h using a CHRIST beta 2-4 LT+LD lyophilizator, operated at a temperature of -50\degres C and a pressure of 4 Pa. After complete dehydration, samples were attached onto stubs with double-sided adhesive (carbon type) and sputter coated, in a Baltec MED020 Sputter Coater, with a thin film of platinum to improve electrical conductivity of the sample surface. Samples were subsequently observed using SEM (FEI Quanta 250 FEG, Centre Technologique des Microstructures, University Claude-Bernard, Lyon 1).\\

Transmission Electron Microscopy (TEM) was carried out using cells fixed with 2\% glutaraldehyde, buffered with PBS and embedded in 2\% agar. Cells were post-fixed in 1\% osmium tetroxide, dehydrated in a graded series of ethanol and embedded in Epon. Sections of 70 nm were realized using an ultramicrotome (Leica ultracut S), contrasted with uranyl acetate and lead citrate, and observed under a Philips CM 120 Transmission Electron Microscope at 80 kV.\\
\newpage

\subsubsection*{Cellular fatty-acid composition of \textit{P. phosphoreum} ANT-2200 grown at 0.1 and 22 MPa (13\degres C)}
Cultures of \textit{P. phosphoreum} ANT-2200 were grown in 300-mL completely-filled polyethylene bottles (188 mL of culture + 62 mL of oxygenated Fluorinert$^{TM}$ FC-72), at 13\degres C and at 0.1 or 22 MPa (into high-pressure bottles). The bottle stoppers were equipped with a septum through which the pressure was applied. Cells in the late logarithmic stage of growth were harvested by centrifugation (20 min, 5,500 rpm at 0\degres C). Bacterial pellets were immediately frozen at -20\degres C and lyophilized. Lipids were extracted using the modified method of \cite{bligh1959} with dichloromethane/methanol/water (DCM/MeOH/H2O, 1\string:2\string:0.8, v/v/v) under sonication. Following the addition of DCM and water to allow phase separation (final DCM/MeOH/H2O ratio of 1/1/0.9), the lower DCM layer was collected and the upper aqueous phase was further extracted with DCM ($\times$2). The combined lipid extracts were concentrated, dried over anhydrous sulfate and evaporated to dryness (N$_2$ flux) before being trans-esterified (50\degres C, 2h) with 2\% sulfuric acid in MeOH in the presence of toluene \citep{christie1989}. 
Individual fatty acids were identified and quantified by gas-chromatography-mass spectrometry (GC-MS), using an Agilent 6890N gas chromatograph interfaced to an Agilent 5975 mass spectrometer (electronic impact at 70 eV). 
The GC was equipped with a splitless injector and a HP5-MS capillary column (30 m $\times$ 0.25 mm $\times$ 0.25 $\mu$ m). Helium was used as the carrier gas (constant flow of 1 mL min$^{-1}$) and the oven temperature was programmed from 70 to 130\degres C at 20\degres C min$^{-1}$, and then at 4\degres C min$^{-1}$ from 130 to 300\degres C at which it was hold for 20 min.\\

\subsubsection*{Bioluminescence of \textit{P. phosphoreum} ANT-2200 at 0.1 and 22 MPa (13\degres C)}
Bioluminescence (photons sec$^{-1}$) was monitored with a high-pressure bioluminescence system shown in Figure  \ref{1_3} A. Luminous bacteria were cultivated within a culture chamber placed inside a high-pressure tank (Fig. \ref{1_3} B). The hydrostatic pressure is transmitted (via the HP-chamber valve) from the high-pressure tank to the culture chamber via a floating piston (Fig. \ref{1_3} B). Sub-sampling is done by opening the culture-chamber valve, while the pressure is monitored, using a piloted pressure generator \citep{tamburini2009} connected to the HP-chamber valve. The culture chamber is made in ertalyte (chemically and biologically inert material, white for light reflection) and sustains a plexiglass cone which transmits photons emitted by luminous bacteria by the way of an optical fiber (Fig. \ref{1_3} B). Photon counting was obtained by integrating signals during 10 seconds using a photomultiplier (H7155, Hammamatsu) linked to its counting unit (C8855, Hammamatsu). Temperature was regulated using an external housing of tubing around the high-pressure bioluminescence tank. Temperature was controlled with a thermo chiller and monitored using a K-type thermocouple directly fitted within the high-pressure tank. More details of the high-pressure bioluminescence tank can be found in \citep{alali2010}. Experiments were performed three times for 0.1 and 22 MPa.\\

\begin{figure}[!h]
\linespread{1} 
\centering
\includegraphics[scale=0.6]{fig0.jpg}
\caption[High-pressure bioluminescence system.]{High-pressure bioluminescence system. (A) Photography of the high-pressure bioluminescence system and (B) schematic section diagram of the high-pressure bioluminescence tank. PMT: photomultiplier tube; OF: optical fiber; CU: photomultiplier counting unit; T: high-pressure temperature sensor; Tc: Tubing around tank for temperature control connected to a thermo chiller (not shown); Tl: Data logger for temperature sensor; Da: PC for data acquisition of bioluminescence and temperature}
\label{1_3}
\end{figure}

\subsection*{Results and discussion}

\subsubsection*{Growth temperature and pressure optima of \textit{P. phosphoreum} ANT-2200}

Using the logistic model, growth-rate ($r$ expressed as $h^{-1}$) and maximum-population-density ($K$ expressed as OD$_{600nm}$) parameters were defined for each temperature (4, 13, 20, 30, 37\degres C) and pressure (0.1, 10, 22, 30, 40 MPa) conditions (Fig. \ref{S1_3}). Figure \ref{2_3} presents the model curves fitting with empirical data obtained at 13\degres C and 30\degres C, for all tested pressures. The model parameters have been estimated qualitatively using the confidence interval of the logistic growth curves.

\begin{figure}[!h]
\linespread{1} 
\centering
\includegraphics[scale=0.6]{fig1.jpg}
\caption[Example of logistic model fitting empirical growth data of \textit{P. phosphoreum} ANT-2200.]{Example of logistic model fitting empirical growth data of \textit{P. phosphoreum} ANT-2200. Experiments were done at pressures of 0.1, 10, 22, 30 and 40 MPa and at temperatures of 13\degres C and 30\degres C. The logistic model (line) improves the r and K parameter estimation on empirical growth data (dots). Dashed lines are levels of confidence for the 0.05 and 0.95 quantile curves and the 0.25 and 0.75 quantile curves. Mean $\pm$ standard deviation for growth rate (r, $h^{-1}$) and maximum population density (K, OD$_{600 nm}$) parameters are indicated. The dotted frame is the growth curve under optimum pressure and temperature conditions using both r and K parameters. The solid line frame is the growth curve under \textit{in situ} conditions, at 22 MPa and 13\degres C. N is the number of replicates done for the same pressure and temperature conditions.}
\label{2_3}
\end{figure} 


\begin{sidewaysfigure}
%\begin{figure}[p]
%\begin{turn}{180}
\linespread{1} 
\centering
\includegraphics[scale=1]{S1_2013.jpg}
\caption[Representation of growth curves for temperatures of 4, 13, 20, 30 and 37\degres C and for pressure of 0.1, 10, 22, 30, and 40 MPa.]{Representation of growth curves for temperatures of 4, 13, 20, 30 and 37\degres C and for pressure of 0.1, 10, 22, 30, and 40 MPa. The logistic model (line) improves the $r$ and $K$ parameter estimation on empirical growth data (dots). Dashed lines are levels of confidence for the 0.05 and 0.95 quantile curves and the 0.25 and 0.75 quantile curves. The growth rate ($r$, h$^{-1}$) and maximum population density ($K$, OD$_{600 nm}$) parameters are indicated.}
\label{S1_3}
%\end{turn}
\end{sidewaysfigure} 
%\end{figure}

The $r$ and $K$ parameters were used to construct the extrapolated-contour diagram of their temperature-pressure dependence (Fig. \ref{3_3} A and Fig. \ref{3_3} B, respectively). \textit{P. phosphoreum} ANT-2200 was able to grow at hydrostatic pressures ranging from 0.1 to 40 MPa and at temperatures ranging from 4 to 37\degres C.\\

\begin{figure}[!h]
\linespread{1} 
\centering
\includegraphics[scale=0.67]{fig2.jpg}
\caption[Extrapolated-contour diagram of the temperature-pressure dependence of \textit{P. phosphoreum} ANT-2200.]{Extrapolated-contour diagram of the temperature-pressure dependence of \textit{P. phosphoreum} ANT-2200. The diagrams are plotted for (A) the growth rate ($r$, h$^{-1}$) and (B) the maximum population density ($K$, OD$_{600 nm}$). The gray circles indicate parameter values used to extrapolate the contours. Size is proportional to their value. The black circle corresponds to the \textit{in situ} conditions for the strain. Isolines define zones with same level of parameter values.}
\label{3_3}
\end{figure} 

The strain grew very slowly at 4\degres C, with a minimum $r$ value obtained at 0.1 MPa (0.058 $\pm$ 0.014 $h^{-1}$) and a low K of $0.167 \pm 0.032$ (Fig. \ref{3_3} A). The higher $r$ values were observed at 30\degres C / 0.1 MPa and at 37\degres C / 40 MPa (0.624 $\pm$ 0.035 $h^{-1}$ and 0.596 $\pm$ 0.038 $h^{-1}$ respectively, Fig. \ref{3_3} A). While growth rates appeared to depend on temperature, pressure did not clearly affect it, at least in the tested range (Fig. \ref{3_3} A). Furthermore, the stationary phase was very short with fast and strong cell lyses at 37\degres C for all pressures (Supporting Information, Fig. \ref{S1_3}). In opposition to what is generally observed \citep{yayanos1995,eloe2011}, it was not possible to define the pressure affinity of strain ANT-2200 using the growth rate only. Therefore, we also used the same approach to overlook the effect of temperature and pressure conditions on maximum population density reached by strain ANT-2200. Interestingly, both influenced $K$ and for all tested conditions, the highest value ($0.512 \pm 0.011$ OD$_{600nm}$) was observed at 30\degres C and 10 MPa (Fig. \ref{3_3} B). While the growth rate is commonly the main growth parameter used in microbiology, our experiments show that the maximum population density has a strong influence on the definition of the optimal conditions for growth. So, we propose to cross the $r$ and $K$ parameters (using the $C_{r-K}$ coefficient) in order to define these optima (in our case, temperature and pressure).\\
 
 \begin{figure}[!h]
\linespread{1} 
\centering
\includegraphics[scale=0.63]{fig3.jpg}
\caption[Cross diagram of the temperature-pressure dependence of \textit{P. phosphoreum} ANT-2200.]{Cross diagram of the temperature-pressure dependence of \textit{P. phosphoreum} ANT-2200. (A) Extrapolated-contour diagram of the temperature-pressure dependence for both the growth rate ($r$, $h^{-1}$) and the maximum population density ($K$, OD$_{600 nm}$) for \textit{P. phosphoreum} ANT-2200. The cross coefficient $C_{r-K}$ is defined as: $0 < C_{r-K} < 1$. (B) Standard deviation associated to the $C_{r-K}$ coefficient. The gray circles indicate parameter values used to extrapolate the contours. Size is proportional to their values. The black circle corresponds to the \textit{in situ} conditions for the strain. Isolines define zones with same level of values.}
\label{4_3}
\end{figure}

An extrapolated-contour diagram was drawn for the cross coefficient $C_{r-K}$ (Fig. \ref{4_3} A). The standard deviation, associated to the $C_{r-K}$ coefficient, was calculated using the confidence-interval estimation on parameters from the logistic model and illustrated in Figure \ref{4_3} B. The standard-deviation values were one order below the cross-coefficient values, meaning that the cross-coefficient interpretation was robust. The highest $C_{r-K}$ coefficient value (0.78) was found at 30\degres C and 10 MPa (Fig. \ref{4_3} A). These optima allowed characterizing strain ANT-2200 as mesophile and moderately piezophile \citep{fang2010}. As previously observed \citep{yayanos1995}, the optimal pressure for piezomesophiles is often found lower than their habitat pressure, while their optimal temperature is higher than their habitat temperature. Different hypotheses can be evoked to explain the temperature shift \citep{yayanos1986} (1) inheritance from ancestors who lived in a warmer environment or (2) life in warmer temperatures in the gut of deep-sea animals. Even if we identified optimal conditions for growth at 30\degres C and 10 MPa, we decided to perform further experiments at in situ conditions (13\degres C) and to compare atmospheric pressure (0.1 MPa) to high pressure (22 MPa). This allowed to checking the piezophilic character of strain ANT-2200 and studying its morphology and its fatty-acid composition, known to be affected by hydrostatic pressure \citep{lauro2008,bartlett1992,bartlett2002}. The pressure-dependent (0.1 versus 22 MPa, 13\degres C ) bioluminescence activity of \textit{P. phosphoreum} ANT-2200 was also characterized.\\

\subsubsection*{Morphology of \textit{P. phosphoreum} ANT-2200} 
\textit{P. phosphoreum} ANT-2200 was observed by SEM and TEM after cultivation at 0.1 and 22 MPa (Fig. \ref{5_3}). \textit{P. phosphoreum} ANT-2200 is a rod-shaped bacterium. The size of the cells is in average equal to 2.4 $\pm$ 1.4 $\times$ 10$^{-1} \mu$m long and 0.8 $\pm$ 0.7 $\times$ 10$^{-1} \mu$m wide at atmospheric pressure (Fig. \ref{5_3} A1-3). Cells at 22 MPa display a smaller size (\textit{i.e}., less than 2 $\mu$m long) and contain numerous intracellular inclusions (Fig. \ref{5_3} B1-3). The exact nature of these inclusions has not yet been determined. Such inclusions may serve as energy reserve, may contribute directly to the metabolic capabilities of the cell, and/or may be involved in the cell ability to cope with changing environmental conditions \citep{shively2011,campbell2003}. In any case, this confirms an adaptation strategy of \textit{P. phosphoreum} ANT-2200 cells to high hydrostatic pressure. Besides, cells also appear to aggregate more at 22 MPa than at 0.1 MPa (Fig. \ref{5_3} A1-B1).\\

\begin{figure}[!h]
\linespread{1} 
\centering
\includegraphics[scale=0.7]{fig4.jpg}
\caption[Micro-photographs of \textit{P. phosphoreum} ANT-2200 cells by electron microscopy.]{Micro-photographs of \textit{P. phosphoreum} ANT-2200 cells by electron microscopy. Observation at 0.1 MPa (A1 on dehydrated samples, A2 on freeze-dried samples) and 22 MPa (B1 on dehydrated samples, B2 on freeze-dried samples) using SEM and at 0.1 MPa (A3) and 22 MPa (B3) using TEM. Intracellular inclusions are indicated by arrows.}
\label{5_3}
\end{figure}

\subsubsection*{Pressure effects on the cellular fatty-acid composition of \textit{P. phosphoreum} ANT-2200}
The effect of hydrostatic pressure on the cellular fatty-acid composition of \textit{P. phosphoreum} strain ANT-2200 was determined for cultures grown at 0.1 and 22 MPa at 13\degres C  (Fig. \ref{6_3}). The main fatty acid was C16\string:1, representing 40.8 and 43.1\% of the total cellular fatty-acid at 0.1 MPa and 22 MPa, respectively. Growth at 22 MPa also induced an increase in the relative proportions of C16\string:1, C17\string:0, C17\string:1, C18\string:1 and C18\string:2 fatty acids. The ratio of total unsaturated vs. total saturated fatty acids (UFA/SFA) was 1.9 at 0.1 MPa and 2.3 at 22 MPa. These values are similar to those found by \cite{delong1985} for the piezophilic bacterium CNPT-3 grown under similar pressures. The increase in the relative proportions of mono-unsaturated fatty acids at elevated pressure is in good agreement with previous studies \citep{kamimura1993,yano1998}, indicating that many piezophilic bacteria respond to an increase in hydrostatic pressure by modifying their membrane lipid composition. This homeoviscous adaptation allows to tailor the membrane to environmental conditions with suited physical properties \citep{bartlett2002}.\\

\begin{figure}[!h]
\linespread{1} 
\centering
\includegraphics[scale=0.65]{fig5.jpg}
\caption[Relative total fatty-acid composition (\%) of \textit{P. phosphoreum} ANT-2200.]{Relative total fatty-acid composition (\%) of \textit{P. phosphoreum} ANT-2200. Strain ANT-2200 is grown at 0.1 (black bar) and 22 MPa
(gray bar). Others\string: sum of C17\string:0, C18\string:2 and C 19\string:1 fatty acids; UFA\string: unsaturated fatty acids; SFA\string: saturated fatty acids.}
\label{6_3}
\end{figure}

The results obtained with \textit{P. phosphoreum} ANT-2200 further argue for the piezophilic character of this bioluminescent strain. It is noteworthy that the poly-unsaturated C20\string:5 fatty acid (C20\string:5 PUFA or eicosapentaenoic acid) was not detected in this strain. \cite{delong1997} suggested that C20\string:5 PUFA could be used to define strains originating from low temperature and high pressure environments. Nevertheless, the absence of such PUFA in \textit{P. phosphoreum} ANT-2200 may be due to its origin from warmer deep-sea waters (Mediterranean Sea, average temperature about 13\degres C), and to its optimal temperature of growth (30\degres C).\\

\subsubsection*{Pressure effects on bioluminescence of \textit{P. phosphoreum} ANT-2200} 
Three successive experiments using the high-pressure bioluminescent tank were carried out in order to quantify the luminescence produced by \textit{P. phosphoreum} ANT-2200, at 0.1 MPa and 22 MPa, 13\degres C . Our results showed that the higher maximum cell density at 22 MPa than at 0.1 MPa is associated to higher luminescence intensity. Maximum luminescence intensity is reached, in average, at 17.6 h at 22 MPa and at 13.3 h at 0.1 MPa (Fig. \ref{7_3} A). The average value of the maximum luminescence for the three replicates is three times higher at 22 MPa (3.5 $\pm$ 0.1 $\times$ 10$^6$ photons sec$^{-1}$) than at 0.1 MPa (1.2 $\pm$ 0.2 $\times$ 10$^6$ photons sec$^{-1}$). When bioluminescence is maximal, the total cell number is 1.40 $\pm$ 0.01 $\times$ 10$^8$ cells mL$^{-1}$ at 0.1 MPa and 1.90 $\pm$ 0.01 $\times$ 10$^8$ cells mL$^{-1}$ at 22 MPa (Fig. \ref{7_3} B). At this time, the light emission capacity represents 8.4 $\times$ 10$^{-3}$ photons cell$^{-1}$ mL$^{-1}$ at 0.1 MPa and 19.0 $\times$ 10$^{-3}$ photons cell$^{-1}$ mL$^{-1}$ at 22 MPa. Noticeably, the ratio of photons emitted per cell and volume unit is higher at 22 MPa than at 0.1 MPa, clearly indicating the pressure dependence of bioluminescence. Since strain ANT-2200 is characterized as piezophile, its light emission appeared to be an adaptive trait more than a stress response to pressure as suggested by \cite{czyz2000}.\\

\begin{figure}[!h]
\linespread{1} 
\centering
\includegraphics[scale=0.9]{fig7.jpg}
\caption[Bioluminescence and growth of \textit{P. phosphoreum} ANT-2200.]{Bioluminescence and growth of \textit{P. phosphoreum} ANT-2200. (A) Bioluminescence (photons sec$^{-1}$) of \textit{P. phosphoreum} ANT-2200 at 0.1 MPa (blue lines) and 22 MPa (red lines). (B) Fitted logistic growth curves for 0.1 MPa experiments (blue lines) and 22 MPa experiments (red lines). The dashed lines represent levels of confidence for the 0.05, 0.95 and 0.25, 0.75 quantile curves. Cell number is estimated using equation (1). On (A) and (B) blue and red dotted lines represent the mean time of the bioluminescence peak for both pressure conditions.}
\label{7_3}
\end{figure}

Actually, during the growth, the respiration and the bioluminescence emission are two processes competing for the consumption of oxygen \citep{nealson1970,nealson1977,bourgois2001,grogan1984}. To explain the differences in light emission between high pressure and atmospheric conditions, the oxygen availability has been checked. An oxygen optode (PreSens GmbH) permitted to control the remaining presence of enough oxygen at the end of the growth (oxic condition) both under atmospheric and high-pressure conditions (data not shown). The oxygen concentration seems not to explain the differences in bioluminescence emission per cell described in these experiments. A second explanation to these results is based on the ecological aim of the bioluminescence emission. Metabolic processes, such as the increase of bioluminescent-bacterium biomass, will increase the luminescence by an autoinduction phenomenon. Many bacteria use this cell-density-dependent signaling system, also called quorum sensing, to coordinate the expression of the genes involved in biofilm formation and luminescence production \citep{hmelo2011}. In our study, the aggregates formed at 22 MPa (Fig. \ref{5_3} B1) keep the cells close together, miming a higher cell density, and this could possibly induce a quorum-sensing response leading to higher bioluminescence intensity. This is in agreement with previous hypotheses from \cite{pooley2011}.\\

Three different ecological niches with high cell density, enhancing quorum sensing and indirectly bioluminescence, have been described in the literature so far. Firstly, light organs of marine squids or fish contain up to 10$^{11}$ cells mL$^{-1}$ of luminescent bacteria. This symbiosis provides an advantage for the host (prey or partner attraction...) and an ideal growth environment for bacteria \citep{Haddock2010,widder2010}. Secondly, marine snows are millimetre- to centimetre-size aggregates of macroscopic flocculent particles consisting of detritus, inorganic particles and phytoplankton on which micro-organisms grow \citep{azam1998,azam2001,alldredge1987}. Bacteria can develop swimming behavior to colonize this sinking organic material, therefore reaching a cell density 100 to 10,000 times higher than in the water column (up to $10^8$ to $10^9$ cells mL$^{-1}$) \citep{schweitzer2001,ploug2000}. At this density, they are able to emit light in order to attract preys. Then, they might be ingested by macro-organisms to live in a better-growing environment \citep{ruby1979,andrews1984}. Thirdly, luminous bacteria are known to be present in the gastro-intestinal tracts of marine organisms. Their expelled faecal pellets are enriched in micro-organisms, including bioluminescent bacteria, up to 10$^5$ to 10$^6$ times more than the surrounding waters \citep{ruby1979,andrews1984,zarubin2012}. Ingestion rate and cycling of pellet constituents are affected by bioluminescence phenomenon \citep{zarubin2012}, suggesting that bioluminescence bacteria might play an important role in the carbon cycle in the deep ocean.\\

\section*{Conclusion}
The strain \textit{P. phosphoreum} ANT-2200 was isolated from a deep-water sample (2,200 m, 13\degres C, 22 MPa) close to the ANTARES site in the Mediterranean Sea. It has been shown that, using only growth rate, it was not possible to characterize the strain growth optima. However, using both growth rate and maximum population density of strain ANT-2200, optimal temperature and pressure have been estimated at 30\degres C and 10 MPa. As observed in other deep-sea strains, the ratio of total unsaturated vs. total saturated fatty acids is higher at elevated pressure. All these points converge to characterize this strain as mesophile and moderately piezophile. The strain ANT-2200 produces higher luminescence intensity at high pressure (22 MPa) than at atmospheric pressure (0.1 MPa). To our knowledge, this is the first time that such phenomenon is described. Genetic determinism and corresponding ecological benefit of this pressure-controlled bioluminescence still have to be determined.\\

\textbf{Acknowledgements}\\
Authors thank S. Escoffier, and the ANTARES collaboration. Authors are also grateful to Mrs A. Rivoire and C. Boul\'{e} (Centre Technologique des Microstructures, Universit\'{e} Lyon 1) for electron microscopy. Comments from two anonymous reviewers allowed to significantly improving a previous version of the article.\\

\textbf{Fundings}\\
This work was funded by the ANR-POTES program (ANR-05-BLAN-0161-01) supported by the Agence Nationale de la Recherche, by the AAMIS program (Aix-Marseille Universit\'{e}) and by EC2CO BIOLUX program (CNRS). SM was granted a MERNT fellowship (Ministry of Education, Research and Technology, France). BAA was supported by a fellowship funded by the Syria Ministry of Higher Education.\\

\newpage
\section[Effects of hydrostatic pressure and growth medium on growth, luminescence and oxygen consumption]{Effects of hydrostatic pressure and growth medium on bacterial growth, luminescence and oxygen consumption
\sectionmark{Controlled conditions}}
\sectionmark{Controlled conditions}
\label{part3}
\subsection{Growth medium}
\vspace{10mm}
Based on these first experiments and results, growth medium has been modified to get closer to \textit{in situ} environmental water composition. The initial medium compositions (see SWC mediums composition in Table \ref{Zob}) were based on literature contents \citep{hastings1977,karl1980,pooley2004}. These mediums were optimized for fast bacterial growth (about 24 h for \textit{Photobacterium phosphoreum} ANT-2200) and intense bioluminescence activity with high yeast extract, bio-polypeptone and glycerol concentrations.\\

\begin{table}
\center
\caption{Composition of the rich Sea Water Complete medium (SWC) and impoverished SWC medium. SWC is described as rich due to its high carbon content (yeast extract, bio-polypeptone and glycerol concentrations). pH is set at 7.5. \\}
\begin{tabular}{lll}

\textbf{chemical component} & \textbf{SWC medium}& \textbf{SWC medium}\\
\textbf{} & \textbf{rich}& \textbf{impoverished}\\

\hline
Old seawater & 750 mL L$^{-1}$&750 mL L$^{-1}$\\
Distilled water & 250 mL L$^{-1}$&250 mL L$^{-1}$\\
Bio-polypeptone & 5 g L$^{-1}$&5 mg L$^{-1}$\\
Yeast extract & 3 g L$^{-1}$&3 mg L$^{-1}$\\
Glycerol & 3 mL L$^{-1}$&3 mL L$^{-1}$\\
\end{tabular}
 \label{Zob} 
\end{table}    

The growth medium has been modified with reduced yeast extract, no bio-polypeptone and lower glycerol concentration (see ONR7a composition in Table \ref{ONR7a}). These changes lead to fewer carbon sources but easier to detect and measure during the growth. Both NH$_4$Cl and Na$_2$HPO$_4$ 2H$_2$O concentrations have been increased to avoid nutrients limitation. The ONR7a growth medium involves slower bacterial growth (about 70 h in ONR7a compared to 24 h in SWC medium to reach the stationary phase). Cultures performed under similar experimental conditions show higher $K$ and lower $r$ parameters for ONR7a than SWC medium (data not shown). These changes in $K$ and $r$ parameters involving a higher carrying capacity \footnote{\textbf{Carrying capacity} is the maximum population size that the environment or medium can sustain based on food or habitat available in the environment} with a lower growth rate.\\
% in order to determine the ability of \textit{Photobacterium phosphoreum} ANT-2200 to grow under controlled conditions and automated sampling. \\ 

\begin{table}
\center
\caption{Composition of the ONR7a medium with low carbon concentration. Before mixing, solutions 1 and 2 are autoclaved separately to avoid precipitation and solution 3 is filtered on 0.2 $\mu$m. pH is set at 7.5\\}
\begin{tabular}{ll}

\textbf{Chemical component} & \textbf{ONR7a medium}\\
\hline
Solution 1& (in 700 mL)\\
\hline
NaCl&22.79 g L$^{-1}$\\
Na$_2$SO$_4$&3.98 g L$^{-1}$\\
KCl&0.72 g L$^{-1}$\\
NH$_4$Cl&0.8 g L$^{-1}$\\
Na$_2$HPO$_4$ 2H$_2$O&0.2 g L$^{-1}$\\
NaBr&83 mg L$^{-1}$\\
NaF&2.6 mg L$^{-1}$\\
NaHCO$_3$&31 mg L$^{-1}$\\
H$_3$BO$_3$&27 mg L$^{-1}$\\
Glycerol& 2 mL L$^{-1}$\\
Yeast extract&50 mg L$^{-1}$\\
\hline
Solution 2& (in 300 mL)\\
\hline
MgCl$_2$ 6H$_2$O&11.18 g L$^{-1}$\\
CaCl$_2$ 2H$_2$O&1.46 g L$^{-1}$\\
SrCl$_2$ 2H$_2$O&24 mg L$^{-1}$\\
\hline
Solution 3& (prepared in 10 mL, add 1 mL) \\
\hline
FeSO$_4$ 7H$_2$O& 0.025 g L$^{-1}$\\
\end{tabular}
 \label{ONR7a} 
\end{table} 


%\section[Effects of hydrostatic pressure and growth medium on growth, luminescence and oxygen consumption]{Effects of hydrostatic pressure and growth medium on growth, luminescence and oxygen consumption
%\sectionmark{Oxygen consumption}}
%\sectionmark{Oxygen consumption}
\subsection{High-pressure and oxygen consumption system}

The final aim of this experimental Chapter is to cultivate a bioluminescent bacterial strain potentially active at the ANTARES site as close as possible to environmental conditions. In order to sample oxygen during growth and under high hydrostatic pressure, new instrumentation dedicated to the measure of the bacterial oxygen consumption has been developed in the laboratory (Figure \ref{o2}) and is under validation (Figure \ref{oxytest}). This led to the improvement of new technology and adapted protocols. In this experimental part, we present first results of the measure of bacterial oxygen consumption during growth. These experiments take into account hydrostatic pressure (22 and 0.1 MPa), temperature (13\degres C) and carbon availability (ONR7a medium) to describe oxygen consumption and bioluminescence activity of \textit{Photobacterium phosphoreum} ANT-2200.\\

 \begin{figure}[!h]
\linespread{1} 
\centering
\includegraphics[scale=0.55]{HP_schema.pdf}
\caption[A) and B) Representation of the end cap for High Pressure Bottles (HPBs) with top end cap improved for oxygen and temperature sensors. A) Photograph of the external part, from above B) Photograph of the external part, from below. C) Photograph of the internal part. Schematic representation of the end cap for High Pressure Bottles (HPBs) with oxygen and temperature fibers and connections.]{A) and B) Representation of the end cap for High Pressure Bottles (HPBs) with top end cap dedicated to oxygen consumption measurements. The end cap is made with titanium mater and both temperature and oxygen probes can be connected for direct measurements. Probes are connected with oxygen and temperature fiber to a computer, for data acquisition. A) Photograph of the external part, from above B) Photograph of the external part, from below. C) Photograph of the internal part. Schematic representation of the end cap for High Pressure Bottles (HPBs) with oxygen and temperature fibers and connections.}
\label{o2}
\end{figure} 

High-Pressure Bottles (HPBs), described in \cite{bianchi1999} and \cite{tamburini2003}, are composed by a APX4 stainless steel cylinder with a PEEK coating for biological compatibility closed by two end caps. One end-cap in 316 L stainless steel is covered with a  sheet of PEEK. The second end-cap has been modified in order to support the oxygen sensor as described below and in Figure \ref{o2}. A PEEK floating piston permit to transmit the pressure inside the HPB and to perform sub-sampling if necessary. The modified end cap has been built in Titanium (grad 4, biologically and chemically compatible) and modified as following: four pecking, two for fluid pathway, one for Pt100 temperature sensor, and one stopped by sapphire window to measure dissolved oxygen concentration. The sapphire window is 2 mm thickness and 4 mm diameter  and permit to support an optical fiber (OXY4 PreSens$^{\mbox{\scriptsize{\textregistered}}}$). On the end cap three Viton$^{\mbox{\scriptsize{\textregistered}}}$ O'ring allow to insure etancheity and to prevent organic carbon contamination.\\

The measurement  of dissolved oxygen is based on oxygen luminescence quenching of a platinum porphyrine complex caused by the collision between the excited luminophore and the quencher (oxygen) resulting in radiationless deactivation, called dynamic quenching. The decrease in fluorescence intensity and/or change in fluorescence decay lifetime can be used as a measure of oxygen concentration. The luminophore was contained in polymer support coated on sapphire and was sterilized by autoclave before use, in order to avoid contamination during experiments.\\

HPBs are filled with milli-Q water before sterilization. Oxygen and temperature sensors are then screwed to the end cap and a first high pressure input up to 40 MPa is transmitted to the HPBs. This necessary pre-conditioning involves a peak in the oxygen measurement due to high pressure input (Figure \ref{oxytest}). After this pre-conditioning step, similar high pressure level will not modify oxygen measurement above the noise level. Culture medium is oxygen saturated by intense stirring before bacterial inoculation and saturated Fluorinert is used as oxygen supplier in HPBs. Inoculated culture medium and Fluorinert (25\% of total volume) are transferred to the HPB under high pressure.\\ 

 \begin{figure}[!h]
\linespread{1} 
\centering
\includegraphics[scale=0.5]{oxy_test.pdf}
\caption[Test for oxygen measurement before inoculation and conditioning for high pressure experiment.]{Test for oxygen measurement before inoculation and conditioning for high pressure experiment. Two channels (blue and black dots) are connected to the HPB for noise level comparison. Hydrostatic pressure involved to the system is plotted over time (red line). The first high pressure transmission, at 40 MPa involves a burst in oxygen measure. Then after this first conditioning, fairly low modifications of oxygen measure appear with similar high pressure transmission. Thanks to this test, the HPBs are pre-conditioned with higher pressure transmission (40 MPa) before experiments. Data and representation from M. Garel.}
\label{oxytest}
\end{figure} 

\subsection{Data acquisition into High Pressure Bottles}

Following experiments led to the measure of growth, bioluminescence activity and oxygen consumption with temperature and pressure-controlled conditions. In this part, we aimed at understand bioluminescence-bacterial answer as close as possible to the \textit{in situ} conditions. Experiments have been performed under atmospheric conditions (0.1 MPa) (Figure \ref{oxylabo} A 1-3) and similarly under \textit{in situ} pressure conditions (22 MPa) (Figure \ref{oxylabo} B 1-3).\\

 \begin{figure}[!h]
\linespread{1} 
\centering
\includegraphics[scale=0.6]{oxy_labo.pdf}
\caption[\textit{Photobacterium phosphoreum} ANT-2200 is cultivated in ONR7a growth medium.]{\textit{Photobacterium phosphoreum} ANT-2200 is cultivated in ONR7a growth medium and experiments are performed at 13\degres C and A) Atmospheric pressure (0.1 MPa) B) High pressure (22 MPa). 1) Bacterial growth (DO$_{600nm}$) 2) Oxygen consumption ($\mu$mol L$^{-1}$) 3) Bioluminescence emission. Dashed lines are levels of confidence for the 0.05 and 0.95 quantile curves and 0.25 and 0.75 quantile curves. A logistic model is fitted for both growth and oxygen consumption dataset and the model defined the two characteristic parameters $r$ (h$^{-1}$) and $K$ (OD$_{600nm}$) for bacterial growth and $r_{ox}$ (h$^{-1}$) and $K_{ox}$ ($\mu$ mol L$_{-1}$) for oxygen consumption.}
\label{oxylabo}
\end{figure} 

\textbf{Growth measurements}\\

Growth measurements (OD$_{600nm}$) are triplicates for each pressure condition (0.1 vs 22 MPa). A logistic model (see \citealp{martini2013} and \citealp{verhulst1838}) has been applied to this dataset and is plotted (red and blue lines, Figure \ref{oxylabo} A-1 and B-1) with confidence intervals (dashed lines Figure \ref{oxylabo} A-1 and B-1). Higher $r$ and lower $K$ values are observed under atmospheric conditions than under high pressure (0.079/0.43 and 0.066/0.53 respectively). Furthermore, $r$ and $K$ parameters have been measured with higher values in ONR7a than SWC growth medium. However, the observation of higher $r$ and $K$ parameters under atmospheric pressure is similar to the variations observed for SWC medium at 13\degres C and between similar pressure conditions (see Figure \ref{S1_3}). These observations confirm the effects of high pressure on bacterial physiology independently of the culture conditions.\\
\newpage
\textbf{Oxygen consumption}\\

Using the newly-developed setup to record O$_2$ concentration within HPBs, we measured the oxygen consumption for \textit{Photobacterium phosphoreum} ANT-2200 (Figure \ref{oxylabo} A-2 and B-2). For bioluminescent bacteria, oxygen consumption is dedicated to both bacterial respiration and bioluminescence reaction. The data have been represented using a model as follow:\\
\begin{eqnarray}
x_t = \frac{K_{ox}}{1 + exp((xmid - t) /scal)}
\end{eqnarray} 
with $x_t$ the oxygen concentration, $xmid$ the value at the inflection time $t$, $scal$ a negative scale parameter, $K_{ox}$ the asymptotic parameter represented as the initial oxygen concentration, and $t$ the time. The oxygen-consumption rate is defined as $r_{ox} = 1/scal$.\\ 

Oxygen consumption is closely related to the biomass during bacterial growth, this last variable following the logistic model hypotheses (see \cite{martini2013} and \cite{verhulst1838} for details). The oxygen-consumption model performed is close to the logistic one with a negative coefficient for oxygen consumption. Similar $r$ and $r_{ox}$ values were expected between bacterial growth and oxygen consumption (increase in bacterial growth will linearly increases the oxygen consumption). However, higher $r_{ox}$ than $r$values have been measured for both pressure conditions meaning that oxygen consumption is faster than bacterial growth. These differences can possibly be explained by oxygen consumption dedicated to the bioluminescence reaction.\\

For both pressure conditions, a similar $r_{ox}$ negative coefficient is computed (-0.100 and -0.109 for atmospheric and high pressure conditions respectively) meaning that similar oxygen consumption occurs. Then, for similar oxygen consumption between both pressure conditions, there is a higher bioluminescence activity and higher biomass under high pressure condition. This result possibly led to a better oxygen efficiency under high pressure than atmospheric one. Such results would be in accordance with expected ones for this moderately piezophile bacterial strain. However, due to experimental constraints, the initial oxygen values are not similar between those two experiments (416.1 and 581.7 $\mu$mol O$_2$ L$^{-1}$ for atmospheric and high pressure conditions respectively). Such differences in initial-oxygen concentrations could possibly lead to misinterpretation and need to be fixed in further experiments.\\

\textbf{Bioluminescence activity}\\

Bioluminescence activity has been measured for both pressure conditions (Figure \ref{oxylabo} A-3 and B-3). The maximum bioluminescence intensity is lower at atmospheric than at high pressure conditions (1.0 $\times$ 10$^6$ kHz and 1.9 $\times$ 10$^6$ kHz respectively). Under atmospheric pressure (0.1 MPa), the bioluminescence peak is long, occurring between 7 and 18 h whereas under high pressure (22 MPa), the peak is very straight and occurs few hours later, between 22 and 25 h. The photon emission per cell has been estimated for both pressure conditions with 2.0 $\times$ 10$^{-2}$ and 2.1 $\times$ 10$^{-2}$ photons cell$^{-1}$ for 0.1 and 22 MPa respectively. From these results, only weak differences are observed between both pressure conditions. However, we still observe higher photon emission per cell for high-pressure conditions (22 MPa) compared to atmospheric ones (0.1 MPa).\\

\section{Conclusions}

From these successive sets of experiments under various pressure conditions (0.1 and 22 MPa) and growth medium composition with different carbon content (SWC and ONR7a growth mediums), several results have been highlighted.\\

\textbf {Pressure effects}\\
Firstly, \textit{Photobacterium phosphoreum} ANT-2200 strain has been defined as moderately piezophile. This strain isolated from the ANTARES station seems to be adapted to this deep environment. Moreover, higher pressure affects the bioluminescence emission by increasing the number of photons per cell (see Table \ref{summary}). The effects of high pressure on bioluminescence activity have been clearly demonstrated for SWC growth medium \citep{martini2013}, however, this is less obvious for ONR7a medium with low variations between both pressure conditions. These photons emission per cell are very low compared to the data from the literature. \cite{makemson1986} estimates about 1.0 $\times$ 10$^3$ and 3.0 $\times$ 10$^3$ photons cell$^{-1}$ s$^{-1}$ for \textit{Vibrio harveyi} and \textit{Vibrio fischeri} respectively. However, in these experiments, the maximal values for OD$_{600nm}$ are measured between 0.5 and 4. Such units refer to about 10$^7$ to 10$^8$ total cells in the culture medium. These values are fairly low for quorum sensing threshold known to activate the bioluminescence reaction at about 10$^8$ to 10$^9$ bacterial cells. The difference in bacterial strain, in growth medium and the maximal cell concentration can explain these very low values.\\

\begin{table}
\center
\caption{Summary of results measured in High Pressure Bottles. SWC: Sea Water Complete medium.}
\label{summary}
\begin{tabular}{lll}
\textbf{Growth medium}& \textbf{P condition}&\textbf{Photons cell$^{-1}$ s$^{-1}$}\\
\hline
SWC&0.1 MPa&8.4 $\times$ 10$^{-3}$\\
SWC&22 MPa&1.9 $\times$ 10$^{-2}$\\
%bioreactor&ONR7a&0.1 MPa&9.6 $\times$ 10$^{-3}$\\
ONR7a&0.1 MPa&2.0 $\times$ 10$^{-2}$\\
ONR7a&22 MPa&2.1 $\times$ 10$^{-2}$\\
\hline
\end{tabular}
\end{table}

\textbf {Carbon availability}\\

Then, carbon availability for bacterial growth has been investigated between a carbon-rich medium (Sea Water Complete) and a lower carbon-content medium (ONR7a). For the two pressure conditions (0.1 and 22 MPa), the photons per cell per second are higher for lower carbon-content than rich carbon-content mediums. The rich carbon-content medium (SWC medium) is possibly well adapted for faster bacterial growth but probably limited in mineral-content available for growth (NH$_4$Cl and NaHPO$_4$2H$_2$O). This probably happens before the carbon-limitation. The low carbon-content medium (ONR7a) is limited in carbon under constant oxygen saturation. However, in HPBs, without constant input of oxygen, the bacterial growth is limited by oxygen availability (Figure \ref{oxylabo} A-2 and B-2).\\

Further investigations on physiological processes have to be pursued using the low carbon content growth medium (ONR7a). Indeed, physiology is altered for \textit{Photobacterium phosphoreum} ANT-2200 strain as observed in \cite{martini2013}, meaning that oxygen efficiency for respiration or bioluminescence reaction can be altered within high pressure. Therefore, the difference in dissolved O$_2$ saturation at the beginning of the experiment between both conditions ($K$ values, 416.1 and 581.7 for 0.1 and 22 MPa respectively) has to be fixed. Then, the part of oxygen dedicated to the respiration and the one involved into the bioluminescence reaction have to be estimated. Several methods can be used and these estimations in the literature have been achieved using inhibitors for the bioluminescence reaction such as CCCP or cyanide (see \ref{inhib}) components. Similar experiments can be performed with \textit{Photobacterium phosphoreum} ANT-2200 to define this percentage.\\

\textbf{Illustrated conclusion}\\

To conclude, catabolic and anabolic pathways in aquatic bacteria have been summarized (Figure \ref{delgiorgio} from \cite{del1998}) and this schematic representation has been implemented with bioluminescence activity. From this schematic representation (Figure \ref{delgiorgio}) bioluminescence (process in orange) appears as a positive feedback on the biomass/storage and product compartments and a negative action on the oxygen consumption for cell.\\ 

Firstly, the part of oxygen dedicated to the substrate oxidation is modified due to the oxygen taking for bioluminescence reaction ('m' and 'n' rates in Figure \ref{delgiorgio}). Moreover, the bioluminescence reaction involving ATP (rate 'o' on Figure \ref{delgiorgio}) influences the ATP uptake rate for catabolic and anabolic pathway (rates 'a', 'b', 'c', 'd' on Figure \ref{delgiorgio}). Consequently, bioluminescence reaction is costly for the bacterial cell and the benefice back is the major issue to investigate. To explain such benefice, in the ocean, \cite{baltar2013} indicate that oxidative stress caused by H$_2$O$_2$ affects prokaryotic growth and hydrolysis of specific components of the organic matter pool. Thus, the authors suggest that oxidative stress may have important consequences on marine carbon and energy fluxes. Bioluminescence could be, in these cases, an ecological advantage by the detoxification of molecular oxygen due to its reduction during the chemical reaction \citep{timmins2001}.\\

\begin{figure}[!h]
\linespread{1} 
\centering
\includegraphics[scale=0.6]{delg.pdf}
\caption[Bacterial metabolic and anabolic pathways within and without the cell.]{Bacterial metabolic and anabolic pathways within and without the cell. "+" are positive action of light emission on substrate availability and biotic interactions, "-" negative action on oxygen consumption for the cell. Products are cell expenditure for maintenance.\\
a: rate to which oxidation of organic compounds contributes to the energy pools as ATP.\\
b: rate to which energy is required for active transport of substrates into the cell.\\
c: rate to which ATP is utilized for anabolic reaction. \\
d: rate to which ATP is utilized for maintenance expenditure.\\
$\mu$: growth rate\\
$\mu_e$: endogenous metabolism by biomass decomposition in absence of substrate\\ 
e: rate to which ATP is supplied in the absence of exogenous substrates by degradation of biomass.\\
m: rate to which oxygen is attributed to the bioluminescence reaction.\\
n: rate to which oxygen is attributed to the substrate oxidation.\\
o: rate to which ATP is consumed during the bioluminescence reaction.\\
Adapted from \cite{del1998}}
\label{delgiorgio}
\end{figure} 

Then, indirectly, the light emission influences the cellular mechanisms by its ecological role. Indeed, on the one hand, bacterial bioluminescence will act positively on substrate availability and oxygen by improving environmental factors (better growth condition for example in symbiotic association). As demonstrated by \cite{zarubin2012}, bioluminescent bacteria are more attractive for other organisms and would be more easily consumed on fecal pellets, this leading to the symbiosis between such organisms involving a better growing environment considering the carbon, temperature or oxygen conditions. On the other hand without bioluminescence reaction there is no impact on carbon availability and consequently, possibly no positive contribution to the biomass/storage and product compartments.\\

Further work can be investigated, based on these results. On the one hand, using the high pressure equipment, it is possible to adapt experimental protocol to test the pressure effect on bioluminescent bacteria attached to particles sinking from the surface waters to the deep sea. On the other hand, we only investigated bacteria as possible organisms involved in the bioluminescence activity recorded during intense events at the ANTARES station. It could be of interest to test the hypothesis that newly formed water masses bring other bioluminescent organisms from the surface to the deep. Such organisms also undergo pressure effects that probably affect their physiology and bioluminescence.\\

% Chapitre 4
\chapter[\textit{In situ} prokaryotic survey, quantification and bacterial bioluminescence\string: estimation at the ANTARES station]{\textit{In situ} prokaryotic survey, quantification and bacterial bioluminescence\string: estimation at the ANTARES station
\chaptermark{\textit{In situ} prokaryotic survey}}
\chaptermark{\textit{In situ} prokaryotic survey}
\minitoc
%\label{chap4}
\newpage

\section[Introduction\string: from laboratory to \textit{in situ} measurements]{Introduction\string:  from laboratory to \textit{in situ} measurements
\sectionmark{Introduction\string: from laboratory to \textit{in situ} measurements}}
\sectionmark{Introduction\string: from laboratory to \textit{in situ} measurements}

In Chapter 4, experimentations have been performed, in the laboratory, under controlled growth conditions and with a pure culture, using the bioluminescent bacterial strain \textit{Photobacterium phosphoreum} ANT-2200 as a model. This necessary first step gave us some clues, using experimental results, to validate or not the relevance of considering bioluminescent bacteria as inducers of the intense bioluminescence events detected at the ANTARES station, as we proposed in Chapters 2 and 3. Laboratory experiments were efficient to define independent effects of hydrostatic pressure, temperature and growth medium on oxygen consumption, growth and bioluminescence activity. We demonstrate that high hydrostatic pressure (22 MPa) compared to atmospheric pressure (0.1 MPa), increases bioluminescence activity for this bacterial strain, defined as moderately piezophile. Moreover, several physiological adaptations, better yield for both oxygen consumption and photon emission, have been highlighted under \textit{in situ} conditions compared to atmospheric ones. Furthermore, physiological aspects still need to be defined. \\

%\textit{In situ}, the bioluminescence peaks described using times series, in Chapter 1 and 2, at the ANTARES station in March 2009 and March 2010, reach several thousand of photons per second (maximal median rate about $8 \times 10^6$ Hz). At laboratory, in \cite{martini2013}, we described about $1.9 \times 10^{-2}$ photons s$^{-1}$ per cell at 13\degres C and 22 MPa (\textit{in situ} conditions) and in SWC medium. These values give an extrapolated number of bioluminescent cells about 4.2 $\times$ 10$^9$ to reach such \textit{in situ} light emission. This bacterial count can be related to volume unit as photomultipliers have been estimated to detect photon within a volume of about XXX m3. These estimations refer to a concentration about XXX cell mL$^{-1}$.\\
 
As far as we know, until the present study, milky sea were the only large phenomena of high bioluminescence activity detected \textit{in situ}. Such phenomenon has been observed by naked eye and by satellites. \cite{miller2005}, \cite{nealson2006} and \cite{lapota1988} attribute these phenomena to bioluminescent Bacteria and give an estimation for their concentration of about 2.8 $\times $ 10$^8$ cells cm$^{-2}$. However, these phenomena were described from satellite images or seamen opportunist observations and recorded at the surface of the oceans. Using neutrino telescope and, more generally, PMTs as bioluminescence detectors, such surprising events will be more easily observed, quantified, and explained even in extreme environments such as the dark ocean. To validate such an aim, the part of bacterial bioluminescence has to be quantified and differentiated from the total bioluminescence signal.\\

\begin{figure}[!h]
\linespread{1} 
\centering
\includegraphics[width=10cm]{milkysea.jpg}
\caption[Milky sea detection from satellite observation in the Indian Ocean.]{Milky sea detection from satellite observation in the Indian Ocean. The bioluminescent feature (lower right) is to scale, but has been colorized and enhanced, so it appears much brighter in relation to the scene than it would naturally. The glowing area has been estimated about 15,400 km$^2$ and attributed to bioluminescent Bacteria (\textit{Vibrio harveyi}) associated to \textit{Phaeocystis} (Haptophyte, Eukaryotes). From \underline{http://biolum.eemb.ucsb.edu/organism/milkysea.html}.}
\label{esp}
\end{figure}

%\newpage
%\section{Foreword}

In the next Chapter, we aim at answering the question of bioluminescent bacteria abundance, variability and their possible implication in the signal recorded from PMTs. As first objective, we describe the prokaryotic community over one-year survey at the ANTARES station. At the same time, discrete sampling for environmental variables (temperature, salinity, DOC, NO$_3^-$,PO$_4^{3-}$,Si(OH)$_4$, and dissolved oxygen) are performed during MOOSE and CASCADE cruises. Unfortunately, the stop of the IL07 instrumented line, at the end of 2010, does not permit to access to high frequency dataset for temperature and salinity. From these data, the bioluminescent bacterial contribution, its activity and variability over the year are quantified thanks to molecular biology methods and analyzed conjointly with environmental data obtained monthly by discrete profiles and water sampling. This sampling strategy involving instrumentation at sea, meteorology hazards have been one of the major constraint. However, from this preliminary work, presented as an "article in preparation", first results and perspectives will be proposed.\\

Conjointly to this survey, the CASCADE cruise was conducted in 2011, from March 1$^{st}$ to 30$^{th}$, in the Gulf of Lion and at the ANTARES site aiming at the description of deep convection phenomena. During this cruise, hydrological data show a winter convection in March 2011, reaching 1,600 m depth (X. Durieu de Madron pers. com., M. Boutrif PhD thesis 2012). Hence, this phenomenon observed in 2011 is much less intense and less deep than the previous ones described in Chapter 1, \citep{tamburini2013}, \ref{manuscrit1} reaching 2,300 m depth in 2010 (see Figure \ref{F4}) and consequently not directly impacting the ANTARES station.\\ 

%Ce Chapitre 4 a pour objectif de d�finir l'importance de l'�mission de bioluminescence bact�rienne dans le signal enregistr� par les photomultiplicateurs, sur le site ANTARES.\\

%Un premier objectif de ce Chapitre est le suivi, au cours de l'ann�e 2011, de la communaut� procaryotique totale sur le site ANTARES. A partir de ce suivi, la part de Bact�ries bioluminescentes, son activit� ainsi que sa variabilit� au cours du temps sont quantifi�s. L'arr�t de la ligne oc�anographique c�bl�e sur le site ANTARES, en fin d'ann�e 2010, n'a pas permis l'utilisation des donn�es environnementales �chantillonn�es � haute fr�quence et en temps r�el. Un �chantillonnage discret a donc �t� r�alis� au cours des campagnes MOOSE et CASCADE. Ces pr�l�vements, n�cessitant des moyens � la mer, ont �t� contraints par les al�as m�t�orologiques. Cependant, � partir de ce travail pr�liminaire, pr�sent� sous la forme d'un manuscrit en pr�paration, de premiers r�sultats, une am�lioration des protocoles ainsi que des perspectives seront propos�es.\\

%Parall�lement � ce suivi, la campagne CASCADE s'est d�roul�e du 1$^{er}$ au 11 Mars 2011 dans les eaux du Golfe du Lion ainsi que sur le site ANTARES avec pour objectif l'�tude des ph�nom�nes de convections profondes. Au cours de cette campagne, les donn�es hydrologiques collect�es indiquent la formation d'une convection hivernale peu intense en Mars 2011 ne d�passant pas 1,600 m de profondeur (th�se M. Boutrif 2012, comm. pers. X. Durieu de Madron) et par cons�quent, sans impact sur le site ANTARES. Ce ph�nom�ne observ� en 2011 est sans commune mesure avec les donn�es pr�sent�es au Chapitre 1, \citep{tamburini2013}, \ref{manuscrit1} d�crivant une convection jusqu'� 2,300 m de profondeur au cours de l'ann�e 2010 (voir Figure \ref{F4}).\\

\textit{This work has been granted by EC2CO-BIOLUX. C. Tamburini is the PI for the project. D. Lef�vre, M. Garel, C. Tamburini and me performed the sampling at sea at the ANTARES station during both MOOSE and CASCADE campains. L. Casalot and et S. Isart elaborated the beginning of the study with \textit{lux} gene sequence determination and preliminary work. V. Michotey and S. Guasco performed the experimentations and the interpretation of the results for molecular biology. I summarized, interpreted and wrote the results.}\\

%\textit{Ce travail a �t� financ� par le projet EC2CO-BIOLUX, r�sultat d'une collaboration au sein du laboratoire MIO. Christian Tamburini est porteur du projet. Dominique Lef�vre, Marc Garel, Christian Tamburini et moi-m�me avons effectu� les pr�l�vements en mer sur le site ANTARES au cours des campagnes MOOSE et CASCADE. Laurie Casalot et St�phie Isart (stagiaire �cole d'ing�nieur) ont �labor� les pr�mices de l'�tude concernant la s�quence du g�ne \textit{lux} utilis�. Val�rie Michotey et Sophie Guasco ont effectu� les manipulations ainsi que l'interpr�tation des r�sultats de biologie mol�culaire. J'ai regroup�, �crit et trait� les r�sultats pr�sent�s ci-apr�s.}\\

\newpage
\section{Article 4}
\subsection[Temporal survey of prokaryotic communities, presence and activity of bioluminescent bacteria at the deep ANTARES station (North-western Mediterranean Sea)]{Temporal survey of prokaryotic communities, presence and activity of bioluminescent bacteria at the deep ANTARES station (North-western Mediterranean Sea).
\sectionmark{ARTICLE 4}}
\sectionmark{ARTICLE 4}

\begin{center}
%\Large{Temporal survey, presence and activity of bioluminescent bacteria at the deep ANTARES station\\ (North-Western Mediterranean Sea).}\\
Martini S.$^{1,2\ast}$, Michotey V.$^{1,2}$, Guasco S.$^{1,2}$, Casalot L.$^{1,2}$,\\ Garel M.$^{1,2}$, Tamburini C.$^{1,2}$
\end{center}
\vspace{20mm}

\noindent $^1$ Aix-Marseille Universit�,  Mediterranean Institute of Oceanography (MIO), 13288, Marseille, Cedex 09, France ; \\
$^2$ Universit� du sud Toulon-Var (MIO), 83957, La Garde cedex, France ; CNRS/INSU, MIO  UMR 7294; IRD, MIO UMR 235.\\
\null
\vspace{15mm}
\begin{center}
$\ast$ Corresponding author\string:  \underline{severine.martini@univ-amu.fr}\\
\end{center}
\vspace{32mm}
\textbf{Article in preparation} 

\newpage
\textbf{Abstract}\\

The ANTARES station is indirectly used as an observatory for the bioluminescence record in the deep (2,000 m depth) North-western Mediterranean Sea. In this study we describe the prokaryotic community, at the deep ANTARES site, over one-year survey in 2011. Relatively stable prokaryotic abundance and diversity have been recorded during the year with the detection of \textit{Chloroflexi} and \textit{Proteobacteria} as the most active members. Conjointly to these data, environmental conditions are characterized using water samplings (temperature, salinity, nutrients, dissolved organic carbon and oxygen). During this survey, no hydrological particularities affecting the ANTARES station have been observed with minor variability in temperature, salinity, nutrients, dissolved organic carbon and oxygen. Then, the final objective was to focus on bioluminescent-bacterial variability over the year and during a low bioluminescence event, recorded in the deep sea. Using the \textit{lux}F gene, about 0.1 to 1\% of bioluminescent bacteria have been detected in the sampling period. These bacteria were mainly belonging to \textit{Photobacterium} genus. Moreover, bioluminescent bacteria have been found to be mainly active cells even during a low bioluminescence event.

\newpage
\subsection*{Introduction}

Bioluminescence is the production of light by organisms as a result of a chemical reaction. In the ocean, this process is the primary source of light \citep{Haddock2010}. Bioluminescent organisms are spread within 14 phyla \citep{widder2002} covering eukaryotic and prokaryotic domains. Within evolution,  the distribution of bioluminescence ability does not appear to follow any obvious phylogenetic distribution. In the mesopelagic zone, from 500 to 1,000 m, \cite{herring2002} estimate that 90\% of deep-sea fish, 65\% of decapods, 20-30\% of copepods as well as ostracods are bioluminescent. This wide variety and abundance of species using bioluminescence as well as the pattern of light emission, decreasing with depth, have lead to the assumption that bioluminescence-activity records could be used as a relative proxy of the marine-organisms abundance \citep{moline2009,tokarev1999}.\\

In the literature, the \textit{in-situ} bioluminescence emission is mainly attributed to eukaryotic organisms such as dinoflagellates \citep{cussatlegras2005,widder2010}, or zooplankton \citep{craig2010}. Correlations have been shown, above 1,000 m depth, between the concentration of  chlorophyll-\textit{a} \citep{craig2010,lapota1989} or dinoflagellates \citep{piontkovski2003} and the bioluminescence emission \citep{Haddock2010}. Some studies assert that bioluminescence intensity reflects zooplankton abundance. In \cite{craig2010}, this relation was not supported below 1,000 m depth suggesting the possible implication of other organisms. Bioluminescent bacteria could explain part of this signal during special events of deep convection as suggested by \cite{tamburini2013}. However, bioluminescence from bacteria does not respond to mechanical stimulation. In most studies, bioluminescence is estimated by bathyphotometers using mechanical stimulation of bioluminescent eukaryotic organisms \citep{cussatlegras2007}. Consequently, the use of only bathyphotometers does not permit to detect bacterial bioluminescence. This technical approach probably leads to an underestimation of bacterial contribution to the bioluminescence activity.\\ 

Dark ocean represents about 95\% of the oceans and accounts for 55\% of all prokaryotes (Archaea and Bacteria) found in aquatic habitats \citep{whitman1998,aristegui2009,carlson2010}. Within the biological pump process, these organisms contributes to the carbon cycle by being active in the remineralization of organic matter in the deep sea \citep{aristegui2009,nagata2010,cho1988} and by feeding back nutrients and dissolved inorganic carbon by respiration. Amongst these deep-sea prokaryotes, bioluminescent bacteria attached to particles are known to play a role in the degradation of particulate carbon. However their \textit{in situ} importance is still unclear \citep{zarubin2012}. Described bioluminescent bacteria belong, so far, to the Gammaproteobacteria subclass and are affiliated to \textit{Vibrio}, \textit{Photobacterium} and \textit{Shewanella} genera \citep{kita2006}. Bioluminescence activity is driven by the product of the \textit{lux}ICDABFEG genes belonging to one operon.  For a cellular concentration, estimated at about 10$^7$ to 10$^8$ cells in liquid culture, bioluminescent genes are activated by quorum-sensing phenomenon \citep{nealson1970,eberhard1972}. Such high cellular concentration can be reached in symbiotic association, and most probably for bacteria attached to sinking particles or "marine snow" \citep{azam1998,alldredge1987}. At the surface, bioluminescent bacteria  have frequently been recorded. They can be responsible for large and intense emission of bioluminescence referred as "milky sea" \citep{miller2005,miller2006}. Bacterial bioluminescence has been intensively studied from punctual \textit{in-situ} samplings \citep{malave2010} or from experiments at laboratory \citep{hastings1983}. However, only few studies have dealt with \textit{in-situ} temporal dynamic of such bacteria, due to time-consuming and expensive reasons \citep{ruby1978,yetinson1979,orndorff1980}. Within these few studies, seasonal changes in bioluminescent bacteria have been recorded in surface layers and a correlation between \textit{Vibrio} \textit{harveyi} and the surface temperature has been demonstrated \citep{ruby1978}. \cite{yetinson1979} and \cite{orndorff1980} also investigate, over time, links between bioluminescent bacterial strains and environmental variables at the surface, such as salinity or temperature. In the deep layer of the ocean (below 1,000 m depth), less information on the \textit{in-situ} bioluminescence and presence of organisms is available due to technical constraints. \cite{gentile2009} recorded bioluminescent bacteria in  the deep Mediterranean Sea (Thyrrenian Sea), based on the \textit{lux}A-gene detection from environmental DNA. These authors found unexpected high number of \textit{lux}A lineages mainly belonging to the \textit{Photobacterium} cluster. Furthermore, at the deep Mediterranean ANTARES site (2,000 m depth), \cite{tamburini2013} proposed bioluminescent bacteria to be the main contributors to the intense bioluminescence activity recorded during periods of deep convection occurring in the Gulf of Lion. The presence, at this station, of bioluminescent bacterial strain (\textit{Photobacterium phosphoreum} ANT-2200), isolated during a previous event of deep convection and high bioluminescence activity, also supports this hypothesis \citep{alali2010,martini2013}.\\

The recent development of neutrino underwater telescope \citep{amram2000} gives access to long time series luminescent records in real time. Such structures use photomultiplier tubes (PMTs) as photon detector for high-energy particles. This instrumentation is used by oceanographers as a bioluminescence sensor \citep{tamburini2013,craig2010} with no direct mechanical stimulation (excepted from current speed), compared to bathyphotometers. In this study, we use the ANTARES neutrino telescope, in the North-western Mediterranean Sea, as one of those observatory possibly able to detect no only bioluminescence emissions from eukaryotes but also from bacteria. A one-year water sampling conjointly to that continuous luminescence record, at 2,000 m depth, were used (1) to analyze the composition of the global prokaryotic community at this site on qualitative, quantitative, activity aspects based on ribosomal information, (2) to investigate the bioluminescent bacterial counterpart using \textit{lux} genes, (3) and to estimate possible links between \textit{in-situ} bioluminescence activity and bioluminescent bacteria.\\

\subsection*{Methods}

\subsubsection*{Site location and continuous record of deep bioluminescence}
The ANTARES project (Astronomy with a Neutrino Telescope and Abyss environmental RESearch) developed, since the end of 2007, a deep-sea cabled observatory, in the North western Mediterranean Sea. The ANTARES site is located 40 km off the French Mediterranean coast (42\degres 48'N, 6\degres 10'E) at 2,475 m depth. At first, this site was dedicated to the search of high energy particles such as neutrino \citep{amram2000}, \citep{aguilar2007}. About 885 PMTs are located between 2,000 and 2,400 m depth for the purpose of particle Physics. All the 12 ANTARES mooring lines are connected, via an electro-optical cable to a shore station providing real-time acquisition. These PMTs sample photons emission at high frequency (0.013 s), and are connected to the ANTARES-transmission cable, delivering real-time data. The bioluminescence activity analyzed in this study is the mean rate of PMTs recorded at the ANTARES telescope, at 2,000 m depth, and sampled on a 7-days period around the sampling day (period from $n-3$ to $n+3$ with $n$ the water-sampling day). Current speed has been sampled using an Accoustic Doppler Current Profile 300 kHz RDInstruments.\\

\subsubsection*{Sample collection and on-field processing}
The monitoring strategy for the biogeochemical sampling has been established in the framework of the MOOSE and CASCADE programs. The ANTARES site has been visited in January, March, May, June, August and October 2011. A 12-Niskin-bottle rosette of 8 L was used to sample water at 2,000 m depth. These water samplings give access to\string: potential temperature, salinity, Dissolved Organic Carbon (DOC), nitrates (NO$_3^-$), phosphates (PO$_4^{3-}$), silicates (Si(OH)$_4$), and dissolved oxygen (O$_2$).\\

Potential temperature and salinity have been sampled from a CTD probe SBE 9+. The dissolved oxygen has been recorded using an oxygen sensor SBE 43 cross-calibrated by oxygen Winkler measurements \citep{Gaarder1927, Bryan1976,Williams1981}.\\

DOC content was determined  using a Shimadzu model TOC-V Total Carbon Analyzer with a quartz combustion column filled with 1.2\% Pt on silica pillows. Standardization of the instrument was performed daily using Milli-Q water as blank and potassium hydrogen phthalate diluted in Milli-Q water (range 0-125 mMC) prepared just before sample analysis as a standard \citep{sohrin2005}. The running blank injected in triplicate after every 4 samples corresponding to the peak area of the Milli-Q water acidified with H$_3$PO$_4$. The DOC  concentration was determined from 3 to 4 independent analyzes after subtracting the running blank. Low-carbon water (LCW) and deep-seawater reference (DSR) were kindly provided by the Bermuda Biological Station and were measured daily to monitor the accuracy and the stability of the TOC analysis. The nominal analytic precision of the analysis procedure was within 2\%.\\

The water samples, for NO$_3^-$, PO$_4^{3-}$, Si(OH)$_4$ determinations, were collected with Niskin bottles and preserved at -20\degres C until analysis, according to \cite{aminot2004,aminot2007}. Nutrient concentrations were determined colorimetrically with an automated method, using a semi automatic Technicon Autonanalyser II (detection limit=0.05 $\mu$moles L$^{-1}$), according to \cite{treguer1975}.\\

%Values for chlorophylle-\textit{a} are satellite data recorded at the sea surface using weekly average. These data are from MODIS algorithm \citep{letelier1996} and have been analyzed in collaboration with F. d'Ortenzio.\\
%For bacterial count, samples have been performed by cytometry, using the PRECYM platform and a sorting-cytometer Influx\textsuperscript{\textregistered} (Becton Dickinson). Samples are taken into 5 cm$^3$ cryotubes where 0.2 cm$^3$ of PFA are added to 1.8 cm$^3$ of seawater. Homogenized samples are kept at room temperature for 20 min before to be stored into nitrogen. SYBR Green or DAPI stains are used to differentiate cells from particles. DAPI stain is a DNA-fluorochrom  ($\lambda$ excitation = 351 nm, $\lambda$ emission = 461 nm), and SYBR Green, is a more general nucleic acid stain ($\lambda$ excitation = 490 nm, $\lambda$ emission = 520 nm). High level of Nucleic Acids (HNA) and Low level of Nucleic Acids (LNA) are supposed to be linked to active and inactive cells, respectively \citep{Lebaron2002}, \citep{Lebaron2001}, \citep{Servais2003}. Within the HNA class, HNA-low and HNA-high can be distinguished referring to normal and dividing cells. \\
%\subsubsection*{Discrete water sampling for biology}
%The prokaryotic survey has been conducted close to the ANTARES station. The site has been visited in January, March, May, June, August and October 2011 and water has been sampled with a Niskin-bottles at 2,000 m depth. 
For DNA and RNA analyzes, between 6 and 17 L have been gently filtered, using a peristaltic pump, on autoclaved 0.22 $\mu$m GPWP 47 mm filters (Millipore$^\copyright$). For RNA analyzes, filtration time was limited to 15 min resulting in a filtered volume from 0.4 to 1.6 L. All filters were stored with 1 mL of RNAlater (Sigma-aldrich$^\copyright$) into liquid nitrogen directly after filtration.\\

\subsubsection*{DNA and RNA extraction} 
DNA was extracted from filters using ultra clean water DNA kit (MO BIO\textsuperscript{\textregistered}, CA) according to the manufacturer's instructions. The DNA was stored at -20\degres C. Total RNA was extracted from 0.4 to 1.6 L of water collected by gentle filtration on 0.22 $\mu$m filter using RNeasy minikit (Qiagen, Germany). Putative traces of DNA were removed by the Turbo DNase-free digestion (AMBION\textsuperscript{\textregistered}, Austin, Texas). The RNA was stored at -80\degres C. Synthesis of  cDNA was performed from about 50 ng of RNA using   GOscript reverse transcriptase (Promega) and reverse primer of PCR for bacterial (907R) or archaeal (915R or 1100R) SSU RNA and \textit{lux}F gene (\textit{luxF} 572R).\\
 
\subsubsection*{Screening and quantification of communities} 
The quantification of bacterial, archaeal 16S rRNA and 16S rDNA gene and \textit{lux}F gene and mRNA copy numbers were determined by qPCR with Sso Advanced\textsuperscript{\texttrademark} SyberGreen Supermix using a CFX96 Real Time System (C1000 Thermal Cycler, Bio-Rad Laboratories, CA, USA) with a calibration curve using a plasmid harboring corresponding gene fragment. For bacterial 16S rRNA, quantification was performed with primer sets (GML5F-Uni516R) \citep{takai2000,michotey2012}. Archaeal 16S rRNA gene and cDNA were quantified with two sets of primer\string: 1100R-931f \citep{Einen2008} and 300-516 \citep{michotey2012}.\\ 

For bioluminescent bacteria, different candidate genes were tested (\textit{lux}A, \textit{lux}G, \textit{lux}F). From the literature, \textit{lux}A codes for two sub-units of the bacterial luciferase. The \textit{lux}G codes for a protein close to the flavin reductase enzym. As far as we know, the \textit{lux}F function has not been precisely determined; However, it is detected in all mesopelagic and bathypelagic species, known so far, suggesting a major role for deep bioluminescent species \citep{meighen1993}.\\ 

Three sets of degenerated primers for amplifying \textit{lux}A, \textit{lux}F and \textit{lux}G genes were designed by aligning sequences available in the databases. Eight sequences from marine bacteria were used for \textit{lux}F (AY849520, AY849502, AB367391, AY849504, DQ988874, AY849485, AY341064 and DQ790856). Eight sequences were used for \textit{lux}G (AB261992, AY341064, AY849485, DQ988874, AY849502, AB367391, AY849504 and AY849486). Twelve sequences were used for \textit{lux}A (KC332289, X58791, AB058949, AY456753, AY341064, DQ988874, AY849504, AY345887, AY849502, AY849520). The primers were chosen according to various characteristics\string: optimum size for PCR product between 150 and 200 bp (range 75-250 bp), 25 bp primers (range 20-30 bp), optimal $T_m$ at 60\degres C (range 57-63\degres C), GC\% between 30 and 70\% and no more than 3 identical consecutive nucleotides.\\

Classical PCR gave better results on \textit{Photobacterium phosphoreum} DNA and on environmental DNA with primer sets amplifying \textit{lux}F (\textit{lux}F-385F ATGTTACATGTCAATGTYAATGAGG, \textit{lux}F-527R AATTACCAGCAAGATTCGCTACAT) or \textit{lux}G (\textit{lux}G-179F AGTTACATGTTGGGAGTTCGGTAAA, \textit{lux}G-349R ATAACCCTGTACCTCCAGCIATAAG) compared to \textit{lux}A (\textit{lux}A-598F AARAAAGCICARATGGAACTITATAATG, \textit{lux}A-784R TATTGGTNGCATTIACGTAIGAITC), in consequence these primers were kept for qPCR tests. As qPCR parameters were much better for \textit{lux}F than \textit{lux}G (threshold templates number corresponding to 40 versus 4750, PCR efficiency corresponding to 101\% versus 61\%), \textit{lux}F was chosen to detect bioluminescent bacteria. The program consists of a denaturing step of 10 s, an hybridization step of 10 s at 57.3\degres C, with an elongation step of 10 s. At the end of the PCR reaction, the specificity of the amplification was checked from the first derivative of their melting curves and by electrophoresis analysis.\\ 

\subsubsection*{Molecular fingerprinting analysis of microbial community} 
Denaturing Gradient Gel Electrophoresis (DGGE) analysis was used to screen samples prior to barcoding experiment. Ribosomal fragment were amplified using the GML5F 907MR primer set for Bacteria \citep{Bonin2002,Goregues2005} and 344f  915R primer-set for Archaea \citep{Casamayor2002}. To improve resolution of DGGE analyzes, primers GML5F-GC or 344f-GC contained a 40-nucleotide GC-rich sequence at the 5' end. PCR amplifications were carried out in 20 $\mu L$ reaction mixtures containing 1-2 ng of template DNA, 1.5 mM MgCl$_2$, 0.2 mM of dNTP, 1.25 or 0.31 $\mu M$ of each primer for Bacteria and Archaea respectively, and 1 U of Hot start polymerase plus 1 $\mu L$ of solution Q (Qiagen, Germany). The thermal cycling programs were similar to those previously described \citep{Casamayor2002,michotey2012}. DGGE was performed using a D-code Universal Mutation Detection System (Bio-Rad Laboratories Inc.). Equal amounts of PCR products were loaded in each lane ($\sim$300 ng SSU rRNA gene fragment). The 1 mm thick, 6\% (wt. /vol) polyacrylamide gels presented a denaturing gradient of 30\%-50\% for SSU rRNA gene fragment analysis.\\

Pyrosequencing of bacterial cDNA ribosomic gene of May and October samples were performed on fragment generated with the 16S universal Eubacterial primers (bact 343F, bact 806R) by MR DNA (Texas, USA). A single-step PCR using HotStarTaq Plus Master Mix Kit (Qiagen, Valencia, CA) were used under the following conditions\string: 94\degres C for 3 minutes, followed by 28 cycles of 94\degres C for 30 seconds; 53\degres C for 40 seconds and 72\degres C for 1 minute; after which a final elongation step at 72\degres C for 5 minutes was performed.  Following PCR, amplicon products were purified using Agencourt Ampure beads (Agencourt Bioscience Corporation, MA, USA). Samples were sequenced utilizing Roche 454 FLX titanium instruments and reagents according to manufacturer's instructions. Sequence data derived from the sequencing process were processed using a proprietary analysis pipeline (www.mrdnalab.com, MR DNA, Shallowater, TX). Sequences were depleted of barcodes and primers. Short sequences < 200 bp sequences with ambiguous base calls, and sequences with homopolymer runs exceeding 6 bp were removed. Subsequently a denoising step and chimeras check were performed. Operational Taxonomic Units (OTU) were defined after removal of singleton sequences, clustering at 3\% divergence (97\% similarity) \citep{dowdcallaway2008,dowdsun2008,edgar2010,capone2011,dowd2011,eren2011,swanson2010}. OTUs were then taxonomically classified using BLASTn against a curated GreenGenes database \citep{desantis2006} and ambiguous affiliations on species level were checked manually and compared with environmental clones.\\

\section*{Results and discussion}

\subsection*{Environmental conditions at the ANTARES station}

During the year 2011, potential temperature, salinity, Dissolved Organic Carbon (DOC), NO$_3^-$, Si(OH)$_4$, and oxygen concentrations have been sampled over vertical profiles, using discrete water sampling, at the ANTARES station (see Figure \ref{profiles}).  Continuous profiles have been represented, over the water depth, for each variable and smoothed, using a local polynomial regression (loess method \citealt{cleveland1991}) based on dataset (Figure \ref{profiles}). Then, the annual mean values at 2,000 m depth are summarizes (Table \ref{dataenvi}). Only low variations are detected over the water column, as well as at 2,000 m depth suggesting a stability of environmental conditions. In comparison with other bathypelagic regions already studied in the literature \citep{nagata2010}, the deep-sea waters at the ANTARES station (NW Mediterranean Sea) present higher temperature (12,896 / 0.4-5.4), lower nitrate (8.1 / 16-43), phosphate (0.35 / 1-3.2) and silicate concentrations (8.1 / 10-177). The recorded values at the ANTARES station are in the range of already observed ones in the deep Mediterranean Sea \citep{santinelli2010,Hansell2001,Hansell2009}.\\

\begin{figure}[!h]
\linespread{1} 
\centering
\includegraphics[width=14cm]{profils.pdf}
\caption{Vertical profiles for temperature, salinity, DOC, NO$_3^-$, PO$_4^{3-}$, and Si(OH)$_4$ using water sampling. Dots are samples values from the 6 sampling times, continuous lines are the profile using a local polynomial regression (loess method \citealt{cleveland1991}) and dotted lines are the standard error for each profile.}
\label{profiles}
\end{figure}

The Mediterranean Sea has long been described as a low nutrient concentration basin. Measurements of C\string:N\string:P ratios are useful to describe the functioning of the ecosystem and the 105\string:16\string:15\string:1 Redfield ratios proposed in the North-Atlantic for C\string:N\string:Si\string:P are still used as a reference \citep{Redfield1963,sterner2008,arrigo2004} to examine these changes in nutrient limitation. During this survey, at the ANTARES station, the N\string:P ratio varies between 21.6 and 26.6 and the Si\string:N ratio is stable between 0.9 and 1 (see Table \ref{dataenvi}). The deep Western Mediterranean basin (under 1,000 m depth) is characterized by nutrient ratios different from the Redfield ratio and are estimated at about N\string:P$\sim$20 and Si\string:N$\leq$1.0 \citep{santinelli2010,moutin2002}.\\

 \begin{table}
\center
\caption{Average and standard deviation for the main environmental variables over the year 2011, sampled at 2,000 m depth. T = potential temperature. S = salinity. DOC = Dissolved Organic Carbon. }
\begin{tabular}{lllllll}
\textbf{T} &\textbf{S}& \textbf{DOC}&\textbf{NO$_3^-$}&\textbf{PO$_4^{3-}$}& \textbf{Si(OH)$_4$}& \textbf{Oxygen}\\
 \textbf{(\degres C)} &\textbf{-}& \textbf{($\mu$mol L$^{-1}$)}&\textbf{($\mu$mol L$^{-1}$)}&\textbf{($\mu$mol L$^{-1}$)}& \textbf{($\mu$mol L$^{-1}$)}& \textbf{(mol kg$^{-1}$)}\\
 \hline
  \textbf{12.896} &\textbf{38.474}& \textbf{49.2}&\textbf{8.1}&\textbf{0.35}& \textbf{8.1}& \textbf{202.5}\\
  \textbf{(0.003)} &\textbf{(0.003)}& \textbf{(4.8)}&\textbf{(0.7)}&\textbf{(0.03)}& \textbf{(0.7)}& \textbf{(3.1)}\\
\end{tabular}
\label{dataenvi}
\end{table}
\vspace{5mm}

CTD profiles, for both temperature and salinity, give information on water-masses modifications. Using these set of data, a Theta-S diagram (Figure \ref{thetaS}) allows to identify hydrological changes for the different sampling times, below 1,200 m depth (and until 2,400 m depth). This representation highlights relatively stable water masses with however an intense hydrological event in March (pink dots) due to incursions of water masses from the surface and spreading to the ANTARES site (De Madron, pers. com.). However, these modifications of water characteristics returned to the previous stable state in May (orange dots). Water depth sampled for microbiological studies (2,000 m depth) were highlighted using bigger dots in Figure \ref{thetaS}. This shows that the water masses modifications clearly occur above 2,000 m, i.e above our sampling depth and above the ANTARES telescope, with no impact on the ANTARES telescope. \\

\begin{figure}[!h]
\linespread{1} 
\centering
\includegraphics[width=9cm]{theta-S.jpeg}
\caption{Potential temperature versus salinity diagram of CTD time series at the ANTARES site in January, March, May, June, September, and October 2011. Data shown are above 1,200 m. Colored dots represent data point sampled at 2,000 m depth. Changes in water masses occur at depth below 2,000 m with a come back to the stable state in May 2011.}
\label{thetaS}
\end{figure}
 
%During the sampled period, seasonal spring bloom can occur at the surface and involve carbon input into deep layers, to observe such phenomena, chlorophyll-\textit{a} records have been investigated. The Mediterranean Sea is described as an oligotrophic basin with an East-West gradient in chlorophyll-\textit{a} concentrations. In Figure \ref{chlorophyll}, chlorophylle-\textit{a} time series at the ANTARES station are recorded from satellite data and attributed to phytoplankton abundance. In March, a seasonal spring bloom occurred with a clear peak in chlorophyll-\textit{a}. These concentrations vary from 0.4 at the beginning of March to 1.2 $\mu$g dm$^3$ in the middle of March. However, the water sampling in this study has been performed few days before and 1.5 month after this spring phytoplanktonic bloom. Looking at the same time-scale as discrete water sampling for environmental variables, the 1$^{st}$ of March and the 22$^{nd}$ of May, a decrease is observed for chlorophyll-\textit{a} values in Figure \ref{chlorophyll}. After the occurrence of spring bloom at the surface, this organic mater is sinking with a high degradation rate by organisms over the water colum \citep{crispi2001}. This downward flux of suspended mater (MES) and Particular Organic Carbon (POC) is known to enhance higher bacterial production \citep{azam2001} into both the mesopelagic \citep{cho1988} and bathypelagic zones (for supporting data, see supplementary data, Figure \ref{resp_anne}).\\

%\begin{figure}[!h]
%\linespread{1} 
%\centering
%\includegraphics[width=14cm]{chlorophylle.pdf}
%\caption[Chlorophyll-\textit{a} time series recorded at the ANTARES station from satellite data, data acquisition.]{Chlorophyll-\textit{a} time series recorded at the ANTARES station from satellite data, data acquisition in collaboration with F. D'Ortenzio. Grey bars represent the discrete water sampling.}
%\label{chlorophyll}
%\end{figure}

%\begin{table}
%\center
%\caption{Variability of the major environmental variables over the year 2011, months corresponding to the minimal and maximal values are specified in brackets. DOC = Dissolved Organic Carbon.}
%\begin{tabular}{lllll}
%\textbf{Variable} &\textbf{Min value}& \textbf{Mean}&\textbf{Max}&\textbf{SD}\\
%\hline
%\textbf{Temperature (\degres C)}& 12.890 (June) & 12.897& 12.905 (Oct.) & 0.005\\
%\textbf{Salinity}& 38.470 (June) &38.475 & 38.480 (Oct.)& 0.004 \\
%\textbf{DOC ($\mu mol.dm^3$) }& 43.4 (May) & 48.3 & 53.2 (March) & 3.8\\
%\textbf{Nitrate ($\mu mol.dm^3$) }& 5.36 (Jan.)& 7.48 & 8.53 (May) & 1.21\\
%\textbf{Phosphate ($\mu mol.dm^3$)}& 0.300 (Jan.)& 0.325 & 0.355 (Oct.)& 0.02\\
%\textbf{Oxygen ($mol.kg^{-1}$)} & 199.0 (Aug.)& 202.6 & 206.1(Jan.) & 3.2\\
%\end{tabular}
%\label{}
%\end{table}

Consequently, during this survey, environmental conditions at the deep ANTARES station can be defined as relatively stable in nutrients, and without water mass input.\\

\subsection*{Annual dynamic of the prokaryotic community in the deep Mediterranean Sea}

The prokaryotic community was analyzed during this one-year-survey through ribosomal information (Figure \ref{qPCR1} A and B).\\

\begin{figure}[!h]
\linespread{1} 
\centering
\includegraphics[width=11cm]{qPCR1.pdf}
\caption[Quantitative estimation of A) 16S archaeal DNA (black bars) and RNA (grey bars) B) 16S bacterial DNA (black bars) and RNA (grey bars) among the year 2011, at the ANTARES station.]{Quantitative estimation of A) 16S archaeal DNA (black bars) and RNA (grey bars) B) 16S bacterial DNA (black bars) and RNA (grey bars) among the year 2011, at the ANTARES station (2,000 m depth). NA\string: no sampling for RNA. }
\label{qPCR1}
\end{figure} 

Archaeal 16S rDNA appeared not to be abundant ranging from 0.4$\times 10^2$ to 4.7$\times 10^2$ genes mL$^{-1}$ (Figure \ref{qPCR1} A). To investigate the possibility of underestimation due to primers selectivity, another primer set (primer 931F-1100R) was used for qPCR quantification. Results obtained with the second primer set were in good accordance with the previous one (see Table \ref{supqPCR}) and confirmed the low values obtained. These values are extremely low for deep-sea environment. Indeed, it has been previously observed that Archaea were nearly as abundant as Bacteria in meso- and bathypelagic systems \citep{church2003,herndl2005,karner2001,tamburini2009,teira2006} with about 10$^4$ cells mL$^{-1}$. At the ANTARES station, in April 2005, \cite{alali2010} refer to values of about 5 $\times$ 10$^4$ cells mL$^{-1}$ and about 40\% of DAPI counts belonging to Archaea  detected by CARD-FISH method. At this stage, an underestimation due to cell lysis problem could not be excluded. However, methodological differences between CARD-FISH method and qPCR are noticed and can possibly modify the archaeal detection between those methods. Then, the activity of this community was estimated by 16S rRNA abundance. A weak activity with very few archaeal 16S rRNA was detected and rRNA/rDNA ratios ranging from 0.03 to 0.35 corresponding to October and June respectively (Figure \ref{qPCR1} A).\\

\begin{sidewaystable}
\center
\caption{Values of molecular analysis survey in 2011 at the ANTARES station for both Archaea and Bacteria. ND\string:  Not Determined.}
\small
\begin{tabular}{lllllll}
\textbf{} &\textbf{January}& \textbf{March}&\textbf{May}&\textbf{June}&\textbf{August}&\textbf{October}\\
\hline
\textbf{Bacterial 16S rDNA}& 13.35 & 0.69 & 7.53 & 0.68 & 6.24 & 7.06\\
gene mL$^{-1}$ $\times 10^4$ & ($\pm$ 0.31) &($\pm$ 0.05)&($\pm$ 0.24)&($\pm$ 0.03)&($\pm$ 1.34)&($\pm$ 0.08)\\
&&&&&&\\
\textbf{Archaeal 16S rDNA} & 4.25 & 0.76 & 2.81 & 0.38 & 2.34 & 3.78 \\
gene mL$^{-1}$ $\times 10^2$ & ($\pm$ 0.06) &($\pm$ 0.48)&($\pm$ 0.20)&($\pm$ 0.02)&($\pm$ 0.20)&($\pm$ 1.23)\\
(Primer 300F-516R)&&&&&&\\
&&&&&&\\
\textbf{Archaeal 16S rDNA}& 4.72 & 0.77 & 2.42 & 0.28 & 2.90 & 2.73 \\
gene mL$^{-1}$ $\times 10^2$ & ($\pm$ 0.29) &($\pm$ 0.19)&($\pm$ 0.16)&($\pm$ 0.01)&($\pm$ 0.1)&($\pm$ 0.49)\\
(Primer 931F-1100R)&&&&&&\\
&&&&&&\\
\textbf{Bacterial 16S rRNA}& 31.95 & ND & 0.46 & 0.13 & ND & 4.00\\
gene mL$^{-1}$ $\times 10^4$ & ($\pm$ 5.40) &ND&($\pm$ 0.04)&($\pm$ 0.01)&ND&($\pm$ 0.07)\\
&&&&&&\\
\textbf{Archaeal 16S rRNA} & 0.258 &ND & 0.093 & 0.134 & ND & 0.104 \\
gene mL$^{-1}$ $\times 10^2$ & ($\pm$ 0.005) &ND&($\pm$ 0.002)&($\pm$ 0.018)&ND&($\pm$ 0.018)\\
&&&&&&\\
\textbf{\textit{Lux} DNA}& 1.4& 2.6& 1.8 & 5.0 & 7.2 & 72.0\\
gene mL$^{-1}$ $\times 10^2$ & ($\pm$ 0.2) &($\pm$ 2.9)&($\pm$ 1.2)&($\pm$ 2.2)&($\pm$ 7.7)&($\pm$ 5.3)\\
&&&&&&\\
\textbf{\textit{Lux} mRNA}& 5.5& ND& 7.7 & 7.0 & ND & 15.4\\
gene mL$^{-1}$ $\times 10^{-2}$ & ($\pm$ 0.1) &ND&($\pm$ 0.1)&($\pm$ 0.3)&ND&($\pm$ 7.0)\\
&&&&&&\\
%\hline
%\textbf{\textit{lux} mRNA /}&&&&&&\\
%\textbf{Bacterial rRNA }&0.2&ND&16.7&53.8&ND&3.8\\
%\textbf{ratio}&&&&&&\\
%\hline
\end{tabular}
\label{supqPCR}
\end{sidewaystable}


Abundance of bacterial 16S rDNA was higher than archaeal counterpart and varied from 0.7$\times 10^4$ to 13.3$\times 10^4$  genes mL$^{-1}$ (Figure \ref{qPCR1} B and Table \ref{supqPCR}). These results for bacterial 16S rDNA quantification are in good accordance with previous data obtained using FISH method at this station (\citealp{alali2010} and see references above). As for Archaea, the bacterial activity was estimated by 16S rRNA abundance. Bacteria appeared more active than Archaea although overall rRNA/rDNA ratios were not very high ranging from 0.06 to 2.39 in May and January respectively (Figure \ref{qPCR1}).\\ 

\begin{figure}[!h]
\linespread{1} 
\centering
\includegraphics[width=15cm]{phylo_archa.pdf}
\caption[A) Denaturing Gradient Gel Electrophoresis (DGGE) analysis estimation of 16S archaeal rDNA among the year 2011, at the ANTARES station. B) Phylogenetic tree of the archaeal 16S rDNA gene sequence from the clones DGGE bands.]{A) Denaturing Gradient Gel Electrophoresis (DGGE) analysis estimation of 16S archaeal rDNA among the year 2011, at the ANTARES station. Blue dot (\# 14) represents the sampled band in May and red dot (\# 17) the sampled band in October. \# 16, \# 19 and \# 15, \# 18 are similar sequences sampled into two different DGGE bands. B) Phylogenetic tree of the archaeal 16S rDNA gene sequence from the clones from DGGE bands. Values at nodes represent bootstrap value using neighbor joining. Scale bar denotes 0.02 substitution per nucleotide position.}
\label{arbre1}
\end{figure}

The diversity of total and active prokaryotes has been firstly analyzed by DGGE. Archaeal 16s rDNA DGGE profiles present similar patterns over the year 2011 (Figure \ref{arbre1}). Five clear bands are visible on the gel and sequences from the 3 most intense ones were obtained. These organisms are affiliated to the Euryarchaeaota  phylum in marine groups 2 and 3 cluster (Figure \ref{arbre1} B). The low quantity of archaeal 16S rRNA does not allow enough PCR material to analyze active archaeal community using DGGE.\\

 \begin{figure}[!h]
\linespread{1} 
\centering
\includegraphics[width=7cm]{DGGE_Bact.pdf}
\caption[Denaturing Gradient Gel Electrophoresis (DGGE) analysis of A) 16S bacterial rDNA and B) 16S bacterial rRNA during the year 2011, at the ANTARES station.]{Denaturing Gradient Gel Electrophoresis (DGGE) analysis estimation of A) 16S bacterial rDNA and B) 16S bacterial rRNA among the year 2011, at the ANTARES station. No sampling for 16S rRNA have been performed in March and June. }
\label{DGGE1}
\end{figure}

\begin{figure}[!h]
\linespread{1} 
\centering
\includegraphics[width=12cm]{arbre16S.pdf}
\caption[Phylogenetic tree of the bacterial 16S rDNA gene sequence from the clones DGGE bands.]{Phylogenetic tree of the bacterial 16S rDNA gene sequence from the clones DGGE bands. Values at nodes represent bootstrap values using neighbor joining. Scale bar denotes 0.02 substitution per nucleotide position.}
\label{arbre2}
\end{figure}

DGGE profiles of bacterial 16S rDNA show a major band (\# S1) at all sampling periods (Figure \ref{DGGE1} A), surrounded by numerous other fainter bands. The sequence of this intense band fell within \textit{Chloroflexi} phylum (Figure \ref{arbre2}), and is closely related to \textit{Chloroflexi bacterium} SCGC AAA240-O15, belonging to SAR 202 group, retrieved at 770 m depth at the HOT ALOHA station, subtropical ocean  gyre (99\% similarity, HQ975645, \citealp{swan2011}). \textit{Chloroflexi bacterium} was also observed over the water column, between 200 and 4,000 m at the ALOHA station \citep{delong2006} and in other mesopelagic and bathypelagic areas of various oceans \citep{nagata2010}. Moreover, concentration of this phylum has been observed to increase over the water column (mainly for SAR202, \citealp{varela2008}). Since qPCR results indicate that most of the cells are inactive, the diversity of the active members was investigated firstly by DGGE on RT-PCR fragment (see Figure \ref{DGGE1} B). About 8 intense bands are visible with similar patterns for January, May and June samples, and slightly different for October sample presenting less contrasted bands. The most intense band (\# S1) previously observed on DNA profiles also appear on all RT-RNA counterparts. This observation suggests a highly active strain in abundance all year round. Other active members visible although the corresponding bands on DNA DGGE gel show relatively lower intensity, such as \# S7 and \# S8 (identified as Gammaproteobacteria). In order to further investigate the diversity of active member, pyrosequencing was performed on cDNA of 16S rRNA of May and October samples, presenting different DGGE patterns.\\

\begin{figure}[!h]
\linespread{1} 
\centering
\includegraphics[width=14cm]{chap4_diag.pdf}
\caption[A) Venn diagram representation of total number of OTUs in October and May. B) Number of sequences per OTU for May and October samples showing a higher diversity for October than May.]{A) Venn diagram representation of total number of OTUs in October (110 OTUs) and in May (71 OTUs) and OTUs shared in between. About 50 OTUs (67\% of  sequences) are shared between those two samples. The Venn diagram depicts the percentage of present sequences. B) Number of sequences per OTU for May and October samples showing a higher diversity for October than May with 3 major OTU in May sample.}
\label{venn}
\end{figure}

For May and October samples 662 and 735 sequences were obtained respectively, spread in 133 OTUs with 3\% divergence cutoff. Their lengths vary between 200 and 480 nucleotides but are mainly about 400 nucleotides long. October sample appear more diverse with 110 OTUs detected in contrast to 71 for May (Figure \ref{venn} A). In accordance with DGGE analysis, May sample presents structured active community with 3 OTUs of major abundance whereas for October, the abundance within OTU is less contrasted (Figure \ref{venn} B). Amongst both samples, 50 OTUs are shared accounting for  67\% of all sequences (Figure \ref{venn} A). The affiliation of these OTUs showed that most of them belong to \textit{Chloroflexi} and \textit{Proteobacteria} with percentages reaching 58 and 33\% for May and 37 and 46\% for October, respectively (Figure \ref{esp}). The \textit{Chloroflexi} sequence identified as major member on DGGE analysis is also found by pyrosequencing as one of the abundant OTU (data not shown). Beside these phyla, the lower abundant OTUs fell in \textit{Bacteroides}, \textit{Cyanobacteria}, \textit{Synergistetes}, \textit{Nitrospirare}, \textit{Firmicutes}, \textit{Planctomycetes}, \textit{Actinobacteria} phyla, and candidate division ksb1 (Figure \ref{esp}).\\

\begin{figure}[!h]
\linespread{1} 
\centering
\includegraphics[width=9cm]{chap4_esp.pdf}
\caption[Diversity percentages representation among major phyla for October and May samples.]{Diversity percentages representation among major phyla for October and May samples. \textit{Chloroflexi} and \textit{Proteobacteria} are the main contributors to the diversity at the ANTARES station for those two samples.}
\label{esp}
\end{figure}

\subsection*{Quantification and dynamic of bioluminescent bacteria in a deep environment}

In parallel, current speed (Figure \ref{biolu} A) and bioluminescence activity (Figure \ref{biolu} B) were recorded continuously through the ANTARES telescope.\\

\begin{figure}[!h]
\linespread{1} 
\centering
\includegraphics[width=10cm]{TS_biolu3.pdf}
\caption[A) Current speed time series B) Bioluminescence time series recorded by photomultiplier tubes at the ANTARES station over the year 2011.]{A) Current speed time series B) Bioluminescence time series recorded by photomultiplier tubes at the ANTARES station over the year 2011.}
\label{biolu}
\end{figure}

Relatively low bioluminescence activity (Figure \ref{biolu} B) was observed in comparison to data previously monitored at that site \citep{tamburini2013}. It is worth noting that, at sampling times, the recorded current speeds were low with values of about 7 cm s$^{-1}$. In consequence, bioluminescent eukaryotes that emit light under mechanical stimulation are not stimulated and their contribution to the signal is probably not an important part of the signal. This observation might make easier the determination of bacterial bioluminescence within the signal. During these periods of low bioluminescence and without extreme events of water-mass input, the abundance of bioluminescent bacteria was investigated based on \textit{lux} genes detection, to monitor bioluminescent bacteria. Indeed, from ribosomal analysis, no genera are known to harbor bioluminescent bacteria and that it is known from the literature that this character does not fit with phylogenetic affiliation \citep{Haddock2010}.\\ 

%Between March and May 2011, a weak decrease in median bioluminescence activity (from 284 to 101 kHz) is observed in Figure \ref{biolu}. 

%Moreover, it has to be noticed that in the literature, deep-sea bioluminescent bacteria can be found, in the water column, as free living, attached to sinking particles or released by symbionts on fecal pellets. Due to quorum sensing \citep{hmelo2011,nunes2003}, a concentration of about 10$^8$ to 10$^9$ bioluminescent bacteria per mL$^{-1}$ has been estimated to involve the bioluminescence reaction. At this density, autoinductors concentration involves the chemical reaction for light emission. Such concentration can not be reached by free living bacteria, however, symbiotic or attached ones are possibly bioluminescence emitters. The \textit{lux} gene concentration has been detected at concentration up to 15.4 $\times$ 10$^4$ mL$^{-1}$. Consequently, bioluminescent bacteria are probably not directly linked to this low signal.\\

While most of the year, the abundance of \textit{lux} gene attained around 10$^2$ genes mL$^{-1}$ (Table \ref{supqPCR}) it corresponds to 0.1-1\% of the bacterial community (\textit{lux} gene DNA/bacterial 16S rDNA ratio) most of the year. In October sample, \textit{lux} genes concentration increased and reached 7.2$\times$ 10$^3$ genes mL$^{-1}$. The activity of \textit{lux} bacteria has been estimated by \textit{lux} RNA quantification. Their concentrations were in the same range than that of the DNA counterpart (about 10$^2$ \textit{lux} mRNA mL$^{-1}$) showing the bioluminescence activity for some of them. Bioluminescent bacteria seemed more active than the overall bacterial community since \textit{lux} RNA/DNA ratio (range 1.4 to 4.2) was higher than the 16S bacterial counterpart (0.06 to 2.39 range). \textit{Lux} population did not follow similar trend as the overall bacterial community. The highest specific activity estimated by RNA/DNA ratio (ratio of 4.2) was observed in May (Figure \ref{qPCR2}) whereas it corresponds to January for 16S bacterial ratio (ratio value of 2.39), in Figure \ref{qPCR1}.\\

\begin{figure}[!h]
\linespread{1} 
\centering
\includegraphics[width=10cm]{qPCR2.pdf}
\caption[Quantitative estimation of \textit{lux}F genes for bacterial DNA (black bars) and RNA (grey bars) among the year 2011, at the ANTARES station.]{Quantitative estimation of \textit{lux}F genes for bacterial DNA (black bars) and RNA (grey bars) among the year 2011, at the ANTARES station. NA\string: no sampling for RNA. }
\label{qPCR2}
\end{figure} 

Despite the small size of the \textit{lux} RT-qPCR fragment, sequences obtained in May and October, permit to affiliate the harboring organisms to \textit{Photobacterium} genera since all sequences fell in \textit{lux}F cluster including \textit{P.  phosphoreum}, \textit{kishitanii} and \textit{leiognathi} (Figure \ref{arbrelux}). Interestingly, this observation is similar to data reported in \cite{gentile2009} in Tyrrhenian Sea, at 2,750 m depth and for which numerous \textit{lux}A gene fell in \textit{Photobacterium} cluster. Furthermore, a \textit{Photobacterium} strain was also isolated at the ANTARES station \citep{alali2010}.\\

\begin{figure}[!h]
\linespread{1} 
\centering
\includegraphics[width=11cm]{arbreluxf.jpg}
\caption[Phylogenetic tree of the \textit{lux} genes sequence from the clones DGGE bands.]{Phylogenetic tree of the \textit{lux} genes sequence from the clones DGGE bands. Values at nodes represent bootstrap value using neighbor joining. Scale bar denotes 0.01 substitution per nucleotide position. Blue dots are selected band from May sample and blue ones are selected from October sample.}
\label{arbrelux}
\end{figure} 

Finally, we estimate, possible links between bioluminescent bacteria and bioluminescence total activity. In Figure \ref{boxplotbiolu}, bioluminescence activity has been integrated over 7 days and active bioluminescent bacteria are described using a \textit{lux}mRNA / bacterial rRNA ratio. Using data from PMTs, integrated during water sampling periods, bioluminescence activity is relatively low with only few variations over the year. Bioluminescence recorded in January and October are very close to background and higher values were observed for the other months (March, May, June and August). March and August samples correspond to highest values whereas no RNA data was available for these sample. Over the 4 available samples \textit{Lux} mRNA/bacterial 16S rRNA ratios ranged from 0.2 $\times$ 10$^{-2}$ to 53.8 $\times$ 10$^{-2}$. The two highest ratios (16.7 $\times$ 10$^{-2}$ and 53.8 $\times$ 10$^{-2}$) were obtained for May and June samples respectively. The lack of replicate due to technical reason did not allow to perform statistical tests. However, it should be noticed that similar trend is observed between bioluminescence activity and \textit{lux} mRNA/bacterial 16S rRNA ratios.\\ 

%For prokaryotic investigations, there are only low variations in DNA and RNA \textit{lux} genes number over the year and higher values in October whereas the bioluminescence signal recorded by PMTs at the ANTARES station show higher values in March and August (median values about 300 kHz) and extremely low in October (40 kHz). Therefore, it is difficult to relate these two dataset. For the complete bioluminescence time-series recorded in 2011, a high bioluminescence event is detected reaching 3,899 kHz (median 109 kHz for the whole time series) on February 2$^{nd}$ 2011. At the same time, current speed reach up to XX, such high current-speed intensity can mechanically stimulate eukaryotic bioluminescent organisms.\\

\begin{figure}[!h]
\linespread{1} 
\centering
\includegraphics[width=11cm]{bioluvsproka.pdf}
\caption[Box-and-whisker plot of bioluminescence 7 days around water sampling periods and \textit{lux} mRNA/bacterial rRNA ratio.]{Box-and-whisker plot of bioluminescence 7 days around water sampling periods. The top and bottom of each box-plot represent 75\% (upper quartile) and 25\% (lower quartile) of all values, respectively. The horizontal line is the median. The ends of the whiskers represent the 10$^{th}$ and 90$^{th}$ percentiles. Outliers are represented by empty dots. The $^{40}$K baseline at about 40 kHz is represented as a dashed line. \textit{Lux} mRNA/bacterial rRNA ratio are presented as part of active bioluminescent bacteria over the total active bacteria.}
\label{boxplotbiolu}
\end{figure} 

\subsection*{Conclusion}

The aim of this work is to describe total prokaryotic communities in the deep Mediterranean Sea, and to focus on bioluminescence activity conjointly to the detection of bioluminescent bacteria. During this survey, stable hydrological and physico-chemical conditions were observed with no input of water masses from the surface. In this context, prokaryotic community appeared relatively stable over the year. On a quantitative and qualitative aspect, \textit{Chloroflexi} and \textit{Proteobacteria} appeared as the most active members. In consequence, bioluminescence from eukaryotic organisms was limited due to the absence of mechanical stimulation and the identification of the bacterial part in light emission was probably easier to determine. Moreover, \textit{lux}F genes and transcripts were detected all over the year demonstrating the presence and activity of bioluminescent bacteria affiliated to \textit{Photobacterium} genus.\\

This study increases the interest to use similar approach during high bioluminescence activity events already detected in real-time by the ANTARES observatory in 2010 and 2009 \citep{tamburini2013, martini2013}. On the one hand, short sampling cruises should be managed during high bioluminescence events detection to compare both prokaryotic communities and bioluminescent bacteria quantification and activity. On the other hand, as continuous record of bioluminescence emission is already done by real-time data from the ANTARES cabled observatory. Automatic sampler for prokaryotic sampling might be an innovative tool to follow, understand and, at the end, predict extreme bioluminescence events in the deep sea.\\

%A deep Environmental Sampling Profiler (deep-ESP) is developed at the MBARI to reach such aim \citep{ussler2013}. This instrumentation should be adapted for abyssal environment and specific bacterial organisms detection as bioluminescent ones for example.\\

\textbf{Acknowledgements}\\
This work has been founded by the EC2CO-BIOLUX program. SM was granted a MENRT fellowship (Ministry of Education, Research and Technology, France). Authors thank N. Garcia, P. Raimbault for the nutrients analysis, B. Charri�re for the DOC analysis, D. Lef�vre for the environmental data availability, S. Isart for her laboratory work, the crew of the N/O Thetys and the CASCADE-team. Authors thank the Collaboration of the ANTARES deep-sea observatory for providing
time series data.\\

%\newpage
%\section{Supplementary data}

%To support an increase in bacterial production, respiration rate was recorded in the deep sea using a newly developed system to sample prokaryotic oxygen consumption, the IODA $_{6000}$. Figure \ref{resp_anne} is the time series from 2009 to 2012 recorded at the ANTARES station.\\

%\begin{figure}[!h]
%\linespread{1} 
%\centering
%\includegraphics[width=12cm]{resp_anne.pdf}
%\caption[This dataset has been sampled using a new innovative tool, the \textit{in situ} Oxygen Dynamics Auto-sampler (IODA$_{6000}$).]{Supplementary data. This dataset has been sampled using a new innovative tool, the \textit{in situ} Oxygen Dynamics Auto-sampler (IODA$_{6000}$). This apparatus has been developed at laboratory to measure \textit{in situ} dissolved oxygen. The IODA$_{6000}$ is an auto-sampler with an oxygen optode fitted within an incubation chamber to measure biological activity and an external oxygen optode that measures \textit{in situ} O$_2$ concentration. For more information see \cite{robert2012}. These results are time series of prokaryotic respiration ($\mu$mol O$_2$ dm$^3$ d$^{-1}$) measured by IODA$_{6000}$ at 2,000 m depth at the ANTARES station. The prokaryotic respiration shows an increase in March 2011 enhancing the hypothesis of prokaryotes bloom in the deep sea after the spring bloom in chlorophylle-\textit{a} at the surface.}
%\label{resp_anne}
%\end{figure}

\section{Conclusions}

In Chapter 2 and 3, bioluminescence and new water masses events were related (see \citealp{tamburini2013, martini2013b}) in 2009 and 2010 impacting the deep ANTARES station. In this Chapter 5, we focused on year 2011.The CASCADE cruise occurs from 1$^{st}$ to 30$^{th}$ of March 2011 in the Gulf of Lion and sampling data from this expedition, show that no deep convection occurs below 1,600 m depth. This weak convection does not lead to deep water formations influencing bioluminescence activity at the ANTARES observatory (2,000 m) meaning that no links between water masses and bioluminescence activity can be highlighted during this period. This survey was an opportunity to describe the ANTARES site characteristics and the prokaryotic community over the year within stable hydrological conditions.\\

Between October 2010 and March 2013, the IL07 instrumented line has been turned off. The bioluminescence-activity data presented in this work have been extracted from one of the PMTs located in the whole observatory. Contrarily to the ones located on the IL07, during too high bioluminescence-intensity events, these PMTs are turned off. Such technical constraint induces gaps in time series record. However, these periods of high bioluminescence activity are shorter than those observed in 2009 and 2010 and does not occur during the sampled period.\\

The bioluminescence activity continuously recorded at the deep ANTARES station during this survey is relatively low and does not show high and long-time variation in bioluminescence activity in 2011, as previously observed by \cite{tamburini2013} and \cite{martini2013b}. To conclude from this survey, without deep-sea convection or cascading phenomena, bioluminescent bacteria have been estimated at about 0.1 to 1\% with high rate of active cells compared to total prokaryotes. Moreover, bioluminescent bacteria can not be excluded as a potential contributor of high level of bioluminescence activity already described in the deep sea. Sampling during high bioluminescence events are still needed to describe the bacterial role during high bioluminescent activity detected at the ANTARES station. Thanks to the real time data acquisition on the ANTARES observatory, adapted water sampling must be adapted to reach that goal in further investigations.\\

As perspective work, from this Chapter 5, firstly, the \textit{lux} genes analyzes might be improved. \textit{Lux}F gene was the most efficient at laboratory and has been used to detect bioluminescent bacteria. The \textit{lux}F function is identified for all meso and bathypelagic species involving a major role for deep-bioluminescent-species but with no clear identification. In further investigations, \textit{Lux}A primers coding for two sub-units of the bacterial luciferase has been used however the results were not decisive to improve our results robustness.\\

Moreover, during this survey, seasonal spring bloom has been observed in March 2011, at the surface of the ANTARES station, using chlorophylle-\textit{a} observations.\\

\begin{figure}[!h]
\linespread{1} 
\centering
\includegraphics[width=14cm]{chlorophylle.pdf}
\caption[Chlorophyll-\textit{a} time series recorded at the ANTARES station from satellite data, data acquisition.]{Chlorophyll-\textit{a} time series recorded at the ANTARES station from satellite data, data acquisition in collaboration with F. D'Ortenzio. Grey bars represent the discrete water sampling.}
\label{chlorophyll}
\end{figure}

Such bloom is already known to involve an increase in POC and DOC sinking to the deep sea. Indeed, it is known that 1 to 40\% of of the photosynthetically fixed carbon is exported into the dark realm where it is remineralized \citep{ducklow2001} by prokaryotes. Such seasonal spring blooms could involve higher abundance of bioluminescent bacteria fueling the deep sea and, by the end, an increase in bioluminescence activity. Using a suitable sampling survey, this phenomenon has to be distinguished from newly formed deep water events. However, to reach such information, it is necessary to sample after the occurrence of such bloom, and at high frequency, DOC, POC as well as prokaryotic communities and bioluminescence activity. 

%To estimate downward flux reaching the deep sea, the Martin model \citep{martin1987} is commonly used to estimate the organic carbon flux toward the ocean's interior following the form\string: \\
%$F=F_{100} \times (z/100)^{-b}$\\
%with $F_{100}$ the particulate organic flux at 100 m $z$ the depth and $b$ the particles transfer efficiency.\\  

%\begin{figure}[!h]
%\linespread{1} 
%\centering
%\includegraphics[width=13cm]{chap4-summary.pdf}
%\caption{Schematic representation of carbon input (POC and DOC) to the deep sea after a phytoplanktonic bloom occurring at the surface. Yellow boxes represent light emition from bioluminescent bacteria. This situation probably happens in 2011 during our prokaryotic survey with low bioluminescence emission.}
%\label{tambu2003}
%\end{figure}
% Conclusions
\chapter{Conclusions and Perspectives }
\minitoc
\chaptermark{}{}
%\label{concl}
\newpage
\newpage
\section{Conclusions}

The aim of this thesis was to investigate if bioluminescence can be defined as a proxy of biological activity in the deep sea. To reach this goal, we follow two main axes\string: \\ 

\textbf{\textit{(i)} Bioluminescence is described as "weak" in the deep sea, compared to the ocean surface, but are there variations in light intensity over time and how to explain them?}\\

\textbf{\textit{(ii)} In the deep sea, what is the part of bacterial bioluminescence in the emission of light \textit{in situ}?}\\

The first axis has been mainly developed in Chapter 2 and 3, the second axis in Chapter 4 and 5. Moreover, in Chapter 1, (Introduction part) the interest of taking into account the different scales to defined processes has been highlighted. In this "Conclusions and Perspectives" section, results will be summarized, at various scales, into a more global view in order to answer to the general question.\\

\subsection{Is bioluminescence a proxy ?}

A proxy is an intermediate with the ability to mimic the behavior of something else. In this work, the bioluminescence was investigated as a proxy of biogeochemical processes, as well as a proxy of prokaryotic communities.\\

Photomultipliers have been used as detectors of 'natural' bioluminescence. In this work, the ability of such apparatus to record \textit{in situ} duration and intensity of bioluminescence events, with continuous, automated, and high-frequency data, has been validated \citep{tamburini2013}. Only few detectors dedicated to biological-variable recording exist, and such instrumentation is of importance to the global network of ocean observation. Moreover, the instrumented line, used at the ANTARES, station led to the conjoint record of physical variables, at high frequency and continuously. Discrete water samplings have also been performed over one-year survey for both physical and biological descriptions.\\

% Following the ergodicity therory, over time, it exists only one replicate of these processes, thereby, \\

At a regional scale, due to time dependence of those data, only one replicate (time series) is available to study those processes. Well-adapted methods were needed in order to analyze and interpret those multivariate time series. For such an aim, the Hilbert-Huang and wavelet methods have been used to interpret these time series. Our results demonstrated that the bioluminescence is, at all time and frequency, linked to current speed due to mechanical stimulation. Moreover, we highlighted that in peculiar conditions, the bioluminescence is strongly linked to changes in water-mass properties and this is mainly observed for the highest bioluminescence intensity. Such high bioluminescence events are mainly linked to current speed faster than 19 cm s$^{-1}$, temperature above 12.92 $\degres$ C and salinity higher than 38.479 and explained by deep convection flowing through the ANTARES observatory. During these events, water masses could carry to the site significant amounts of bioluminescent organisms and/or could fuel the deep sea with particulate and dissolved organic matters. These hydrological movements could induce, at the end, an increase in the bioluminescence activity. Moreover, those threshold values are of great interest for the prediction of events of high bioluminescence activity in the future. For astrophysicists, these results are also of interest to protect photomultipliers from too high light activity, that could damage these photon detectors. The survey of these water-mass changes requests a more global survey at the Mediterranean Sea level since the convection of newly formed water masses, described in the Gulf of Lion, will spread into the deep-sea and impact the ANTARES station few days later.\\

At a local scale, in Chapter 4, discrete water samplings have been performed every 2 months and over a one-year period in 2011, depending on the meteorological conditions. This sampling protocol was well adapted for the local description of the ANTARES site, as we observed only few variations in hydrological variables and prokaryotic communities. Moreover, the year 2011 was characterized by no deep convection reaching the seafloor (contrary to 2010) and represents a relatively "calm" period in term of bioluminescence activity, at the ANTARES station. Besides, during this survey, the sampled bioluminescence (integrated over 7 days for discrete information) and the bacterial abundance were too limited discrete observations to be correlated and therefore to interpret bioluminescence as a proxy. To reach this information, there is a need of higher frequency or rather longer survey, including a sampling during newly formed deep sea waters, to evaluate the increase in bioluminescent bacteria. Indeed, longer survey describing variations in bioluminescence intensity, as well as in bacterial abundances, will give more reliability on a possible correlation.\\

Finally, bioluminescence has been efficiently used as a proxy at the regional scale for hydrological processes. However, to validate the use of bioluminescence as a proxy of bacterial abundance at a local scale, longer investigations are needed. Consequently to these results, the observation scale (local or regional) was strongly dependent on the process to be described (water masses and bacterial communities). So is the use of bioluminescence as a proxy.\\

The ANTARES station, at first, was dedicated to astrophysics. Therefore, environmental sciences only have a minor weight within the ANTARES collaboration which sometimes leads to go against scientific decisions for oceanographic goals and to the loss of the necessary continuity of time series. This is a major limit for recording long time series. For example, the IL07 instrumented line has been shut down in October 2010 and it was not possible to re-immerse it before March 2013. Another major constraint is the huge amount of stored data to be analyzed. For discrete water samplings, major limits are the meteorology uncertainties as well as the low frequency records. The development of automatic water sampling and filtration would be a great improvement for scientific community. As an example, the deep-ESP (deep Environmental Sample Processor), developed at the MBARI (Monterey Bay Research Aquarium), is dedicated to automatic samplings that permit water filtration at relatively high frequency into bathypelagic environment. Moreover, this automated molecular biology laboratory can detect microorganisms using their genetic material. Such device can transmit informations in real time. For \textit{in situ} observatories, the sampling frequency of oceanic observations might be adapted, using real-time monitoring. This will lead to define oceanographic processes acting at various observation scales and to limit the quantity of data stored.\\

\subsection{Is bioluminescence linked to biological activity ?}

Bioluminescence is a reaction produced by both Eukaryotes and Bacteria. We investigated this phenomenon as a biological activity correlated with the physiological state of these organisms.\\

At first, from time series records at the regional scale, this work describes \textit{in situ} variations of bioluminescence over time. Three main events of high bioluminescence activity (in March 2009, December 2009 and March 2010) were detected. These bioluminescence events have been defined as intense compared to the global fluctuations recorded over time. Such high variations were unexpected. \\

Furthermore, looking at the local scale, in Chapter 5, prokaryotic survey over the year 2011 led to a first link of bioluminescence with the activity of bioluminescent bacteria. This was based on \textit{lux} genes quantification. In particular, this work demonstrates that bioluminescent bacteria (belonging to \textit{Photobacterium} genus) were active. This was observed even during a period of low bioluminescence at the ANTARES site.\\ 

In Chapter 4, at the microscale in the laboratory, bioluminescence has also been demonstrated to be involved in bacterial activity. The bacterial strain used as a model, \textit{Photobacterium phosphoreum} ANT-2200, has been isolated during an event of high bioluminescence activity in 2005. In \cite{martini2013}, we demonstrated that the bioluminescence of \textit{Photobacterium phosphoreum} ANT-2200 strain is influenced by environmental growth conditions (carbon availability, temperature, and pressure). Higher activity of bioluminescence has been recorded at high pressure \textit{vs.} atmospheric pressure, showing that it does not have an inhibitory effect on the bioluminescence activity. On the contrary, bioluminescent bacteria potentially involved in the bioluminescence activity are adapted to extreme conditions encountered in the deep sea. This shows that bioluminescence is an important physiological response for bacterial activity.\\

All these results allowed to validate that bioluminescence is associated to the physiological state of these organisms. However, questions remain concerning the type of organisms emitting light and recorded by photomultipliers. Indeed, photomultipliers are still inefficient to discriminate bacterial and eukaryotic bioluminescence. In this work, we focused on bioluminescent bacteria due to the lack of \textit{in situ} informations in the literature. The wide diversity of bioluminescent organisms involved in the bioluminescence activity detected by the ANTARES telescope, has to be further explored. Bathyphotometers have been developed to enhance turbulence and mechanical stimulation on eukaryotic bioluminescent organisms. Photomultipliers, detect \textit{a priori} all light emission. The combined use of both devices should allow to measure the global and the eukaryotic bioluminescence and, therefore, to estimate the prokaryotic part in the signal. An efficient protocol has to be developed given the different sampling strategies between those instruments. Comparison between records over time will probably lead to the description of organisms, as well as the ability and limits of those complementary methods.\\

Moreover, the use of adapted video cameras nearby a photomultiplier could give interesting information describing each bioluminescent organisms. The successive record of images using a camera with an active lighting, followed by red lighting (invisible for most of organisms) could permit to identify organisms. The photomultiplier will measure the bioluminescence intensity and shape of emission over time. These couples of characteristics could help to discriminate automatically bioluminescent organisms crossing the ANTARES observatory. Detection methods are widely developed and image processing easily automatized as shown by \cite{stemmann2008}, \cite{aguzzi2009} and the Eye-in-the-Sea project \underline{www.mbari.org/mars/science/eits.html}. By the end, using a network of about 885 photomultipliers will be a fascinating way to detect organisms crossing the entire 3D-telescope over space and time in the deep sea. Indeed, spatial analyzes can be performed using all photomultipliers within a total volume of 0.1 km$^3$ and would permit to estimate spatial dynamics of bioluminescent organisms through the observatory. \\ 

\subsection{Does bioluminescence play a major role in the deep sea ?}

About 95\% of the hydrosphere (deeper than 200-300 m) is characterized by its darkness and the deep sea (under 1,000 m depth) is also defined by high hydrostatic pressure. The interest of the question "Does bioluminescence play a major role in the deep sea ?" is to evaluate possible roles of bioluminescence in such environment and to investigate how high pressure would impact organisms in the deep sea.\\

At the regional scale, in Chapter 2, we demonstrated the input of newly formed deep waters from the surface to the deep, increasing, in an unexpected way, the bioluminescence activity (see \cite{tamburini2013} for details). Moreover, its has been observed that seasonal spring blooms of chlorophyll-\textit{a} occur at the surface of the ANTARES station. Such blooms are known to involve an increase in POC and DOC sinking into the deep sea. Moreover, as we described all over this work, bioluminescent bacteria can be attached to particles and therefore be transported to the deep. Such blooms can also involve bioluminescent eukaryotic organisms, sinking into the deep sea. In this work, first investigations do not permit to identify links between bioluminescence and surface seasonal spring blooms. However, in further work, using a suitable sampling survey, this phenomenon and its impact on bioluminescence activity and, by the end on bioluminescent organisms, has to be distinguished from newly formed deep-water-mass effects.\\

At the local scale, in Chapter 4, we have estimated between 0.1 to 1\%, the part of bioluminescent bacteria among the total bacterial community at 2,000 m depth. This percentage proved the presence of such bacteria at the deep ANTARES station and do not exclude them as a part of the bioluminescence signal. But this survey has been done out of intense-bioluminescence-activity periods and further investigation during these events are needed. Sinking particles are known to be hot spot of microbial activity in the dark ocean \citep{cho1988,turley2002,long2001,azam2001}. Indeed, prokaryotic abundance in particles is up to 3 order of magnitude higher than the abundance of free-living procaryotes in the same volume of sea water \citep{turley1995}. Focusing on sinking particles or marine snow for investigating \textit{lux} genes might be of interest for defining the role of bioluminescent bacteria into the carbon cycle, and in the deep sea.\\ 

At the microscale, in Chapter 3, bioluminescent bacterial physiology has been investigated in relation to temperature and pressure, as the two main characteristics of the deep Mediterranean Sea. Their effects have been analyzed on both growth rate and maximum population density for the bioluminescent bacterial strain \textit{Photobacterium phosphoreum} ANT-2200. This study defined an innovative way to describe bacterial optimal conditions for growth and permitted to characterized this strain as moderately piezophile. Moreover, we highlighted the increase in bioluminescence for this bacterial strain under high pressure conditions. Then, the first results also demonstrated that under high pressure conditions, this model strain allocated a better yield for oxygen consumption than under atmospheric pressure.\\ 

To conclude from the results of this thesis work, we investigated the bioluminescence as a proxy of biological activity using two sets of indicators: scales and levels. On the one hand, observation scales were ranging from micro to local and to regional. On the other hand, ecological levels were defined from population to communities and to ecosystem. We demonstrated that bioluminescence can be used as a proxy of biological activity for some levels (population and ecosystem) and some scales (micro and regional). For local scale and community level, we can not yet conclude and several perspective works have been proposed in this conclusion part.\\


\section{Perspectives}

A first perspective developed hereafter proposes to investigate the cultivation of the bacterial strain in a bioreactor, under controlled and monitored conditions. Using this method, physiological parameters described in Chapter 4 could be improved at atmospheric pressure before going back to high pressure system. Finally, a project concerning the implementation of video cameras on the ANTARES site and the concomitant analysis of data from photomultipliers is presented.\\

\subsection[Physiological parameters determination using bioreactor culture]{Physiological parameters determination using bioreactor culture
\sectionmark{Bioreactor culture}}
\sectionmark{Bioreactor culture}
\label{bioreactor}
\subsubsection*{Bioreactor platform}

The \textit{Photobacterium phosphoreum} ANT-2200 pre-culture is inoculated into a bioreactor fairmentec (total volume 2.2 L and working volume of 1.5 L) with an erthalite culture chamber (biologically and chemically neutral mater of white color and reflecting light). The bioreactor is autoclaved within 121\degres C for 20 min into an autoclave system Federagi FVG. The bioreactor is equipped with sensors controlling automatically the temperature (Julabo F25, France), the pH (InPro 3253, Mettler Toledo, Suisse), the redox-potential (InPro3253, Mettler Toledo, Suisse), and the dissolved oxygen (InPro6800, Mettler Toledo, Suisse). The transmitter for both pH and redox electrods is a Mettler Toledo PH2100e. The transmitter for oxygen is a Mettler Toledo O2 4100e. Sensors are calibrated before each experiment and a verification is done at the end, to validate measurements. On the outlet gas stream line, a CO$_2$ probe (Vaisala GMT 221, Finland) was fitted downstream from the glass exhaust water cooler to enable on-line measurement of CO$_2$ concentration. To control the gaz-mix for the oxygenation into the bioreactor, a debimeter (Bronkhrost, el flow, Netherlands) has been used. The debit injected was from 0 to 200 mL min$^{-1}$. An HPLC (shimadzu RID6A refractometer and colonne HPX87H) allow the control of glucose consumption throughout the experiment. The bioluminescence is recorded using an optical fiber (FVP600660710 Photon Lines) plugged to the bioreactor end cap and connected to a photomultiplier (H7155, Hammamatsu) linked to a counting box (C8855). Photons counting rate is recorded automatically on a computer, with an integrating rate of 10 s. In order to simplify the photon acquisition, experiments have been performed into dark room.\\

\begin{figure}[!h]
\linespread{1} 
\centering
\includegraphics[scale=0.5]{fermenteur.pdf}
\caption[The bioreactor platform with pH, O$_2$, temperature and CO$_2$ data acquisition and control in real time and high frequency sub-sampling.]{The bioreactor platform with pH, O$_2$, temperature and CO$_2$ data acquisition and control in real time and high frequency sub-sampling. Glycerol consumption and Optical Density (OD) are measured with culture sub-sampling over time. The culture room was completely dark during bioluminescence measurements.}
\label{fermenteur}
\end{figure} 

Bioreactor liquid volume and NaOH consumption,
regulating the pH, were monitored by two balances (respectively Combics 1 and BP 4100, Sartorius, France). Temperature, pH,
gas flow rates and stirrer speed were regulated through control units (local loops). All this equipment was connected to an automat (Wago, France) via either a serial link (RS232/RS485), a 4-20 mA analog loop, or a digital signal. The automat was connected to a computer and a BatchPro software (Decobecq Automatismes, France)
was used to monitor, acquire data and manage the processes with
good flexibility and accuracy. \\

\subsubsection*{Data acquisition into bioreactor}

The bioreactor is an optimal system for the bacterial cultivation under controlled conditions and to the record of variables simultaneously. Moreover such instrumentation is dedicated to the cultivation of a large volume of bacterial culture. In this experiment, \textit{Photobacterium phosphoreum} ANT-2200 has been cultivated for 70 h. Dissolved oxygen concentration was set at 100 \% of saturation (oxygen dissolved in the medium) to avoid limitation, temperature regulated at 13\degres C and pH at 7.5. During the growth, air input (O$_2$) (by bacterial consumption), glycerol consumption (by bacterial metabolism), bioluminescence emission, CO$_2$ (by bacterial respiration), and Optical Density (OD$_{600nm}$) (linked to bacterial biomass) are recorded (see Figure \ref{fermdata}). Contamination has been tested, from culture sub-sampling dedicated to the OD measurement, by microscopy observation. High OD$_{600nm}$ values have been recorded due to stirring conditions, bioreactor volume and air input.\\

\begin{figure}[!h]
\linespread{1} 
\centering
\includegraphics[scale=0.6]{fermenteur_data.pdf}
\caption[Data acquisition from the bioreactor. \textit{Photobacterium posphoreum} ANT-2200 strain has been cultivated in an ONR7a growth medium  under controlled conditions.]{Data acquisition from the bioreactor. \textit{Photobacterium posphoreum} ANT-2200 strain has been cultivated in an ONR7a growth medium  under controlled conditions. Oxygen concentration has been controlled to be non-limitant at 100\% of saturation (red dots), temperature is 13\degres C and pH is 7.5. CO$_2$ (blue dots), air input to the medium (green dots), bioluminescence activity (yellow dots) are recorded at high frequency and in real time. Culture has been sampled each 4 to 12 hours for both optical density (OD$_{600nm}$, black dots) and glycerol measurements (brown line). Black lines are the data continuous fit for each variable. The OD$_{600nm}$ has been plotted using a logistic model (see \citealp{martini2013} and \citealp{verhulst1838}). Grey area represents the bacterial growth duration.}
\label{fermdata}
\end{figure} 
The chemical equation for glycerol respiration is defined as :
\begin{eqnarray}
C_3H_8O_3 + 3.5 O_2 \longrightarrow 3 CO_2 + 4 H_2O + Biomasse
\end{eqnarray}
The glycerol concentration shows a limitation in carbon content occurring at 65 h of growth (reaching 0 g L$^{-1}$). This limitation involves the end of bacterial population growth with the end of exponential phase (OD$_{600nm}$ value about 5), CO$_2$ release and O$_2$ consumption, as well as a spontaneous stop of the bioluminescence activity.\\

Thanks to this continuous record and variable acquisition, oxygen consumption, bioluminescence emission, growth, and glycerol consumption helped to determine the medium limitations and to a first estimation of physiological parameters. \cite{del1998} defined Bacterial Growth Efficiency (BGE) as the amount of new bacterial biomass produced per unit of organic C substrate assimilated. BGE can be estimated using the bacterial Biomass Production (BP) and the Bacterial Respiration (BR) with CO$_2$ production as BGE=BP/(BP+BR). In natural ecosystems, BGE is estimated between 0.01 and 0.6 for oligotrophic to eutrophic environments \citep{del1998}. These BGE values are higher in batch setup at laboratory with BGE values between 0.3 and 0.9 \citep{del1998}. In this work, the experimental BGE value is 0.86 showing an important biomass production per unit of organic carbon. In this work, the experimental BGE value can be estimated with both glycerol and O$_2$ consumption, the OD$_{600nm}$ / dry weight ratio and the CO$_2$ production. However, this value has to be discussed with caution because the computation is based on only one experiment. There is a need of replicates for robust interpretation. Moreover, further investigations with several replicates will lead to the computation of physiological parameters for this bacterial strain.\\

%several physiological parameters can be estimated.\\

%\begin{table}
%\center
%\caption{Summary of physiological parameters from bio-reactor experiments into ONR7a medium, at 13\degres C and atmospheric pressure. BGE: Bacterial Growth Efficiency.}
%\begin{tabular}{ccc}
%\textbf{Parameter} & \textbf{Formula}& \textbf{Experimental value}\\
%\hline
%\textbf{CO$_2$ production}& (CO$_2$)$_{Tf}$ - (CO$_2$)$_{T0}$&15.7 mM\\
%\textbf{O$_2$ consumption}& (O$_2$)$_{Tf}$ - (O$_2$)$_{T0}$&19.1 mM\\
%\textbf{OD$_{600nm}$/dry weight}& weighted & 3.2 g OD$_{unit}$ $^{-1}$\\
%\textbf{Biomass}& OD$_{max}$ / 3.2 & 1.2 g L$^{-1}$\\
%\textbf{Glycerol consumption}& glycerol$_{Tf}$ - glycerol$_{T0}$&14.4 mM\\
%\hline
%\textbf{Glycerol respiration} & C$_3$H$_8$O$_3$ + 3.5 O$_2$ $\rightarrow$ 3 CO$_2$ + 4 H$_2$O&\\
%\hline
%\textbf{CO$_2$/O$_2$ ratio}& theoretical = 0.86 & 0.82\\
%\textbf{BGE} & BP /(BP+BR) & 0.86\\
%\textbf{Biomass C content} &  & 0.5\\
%\textbf{\% O$_2$ for bioluminescence}& see \ref{oxypercent} & 4.8 \%\\
%\end{tabular}
 %\label{calculs} 
%\end{table}    

%In Table \ref{calculs}, the values of gaz exchanges and physiological parameters have been determined before the end of the growth, between T$_0$=0 h and T$_f$=67 h. Under these experimental conditions, the total CO$_2$ consumption, estimated as the difference between the initial and the final CO$_2$ concentrations, is 15.7 mM. The O$_2$ consumption determined as the cumulative oxygen input to the system is about 19.1 mM. The OD$_{600nm}$ / dry weight ratio of 3.9 has been determined by desiccation of 1.5 L of bacterial culture, the biomass can also be estimated from this ratio. The glycerol consumption is defined as the difference between the initial glycerol input in the growth medium and the final glycerol concentration at the end of the bacterial growth (T$_f$=67 h). During this experiment, the glycerol consumption has been measured about 14.4 mM. Moreover, the Bacterial Growth Efficiency (BGE) is defined by \cite{del1998} as the amount of new bacterial biomass produced per unit of organic C substrate assimilated. BGE can be estimated using the bacterial Biomass Production (BP) and the Bacterial Respiration (BR) with CO$_2$ production as BGE=BP /(BP+BR). In natural ecosystems, BGE is estimated between 0.01 and 0.6 for oligotrophic to eutrophic environments \citep{del1998}. These BGE values are higher in batch setup at laboratory. In this work, the experimental BGE value is 0.86 showing an important biomass production per unit of organic carbon.\\

%In order to determine the part of oxygen consumption involved into the increase of bacterial biomass, the theoretical and experimental CO$_2$ / O$_2$ ratios have been estimated. 
%The chemical equation for glycerol respiration is :
%\begin{eqnarray}
%C_3H_8O_3 + 3.5 O_2 \longrightarrow 3 CO_2 + 4 H_2O
%\end{eqnarray}
%From this stoechiometric relation, the theoretical $\frac{CO_{2resp}}{O_{2resp}}$ ratio is about 0.86 compared to an experimental value of 0.82. %These two values of experimental O$_2$ consumption and theoretical value assumed by the glycerol respiration give a potential of 4.8\% of oxygen dedicated to the bioluminescence reaction (see \ref{oxypercent} for computation details). This value  has to be discussed with caution because indirectly estimated, using theoretical and experimental values. However, this percentage is close to values estimated by \cite{makemson1986,nealson1979} and \cite{dunlap1984}. For improvement, several methods can be developed for this bacterial strain cultivated into bioreactor. Indeed, the estimated values from the literature have been achieved using inhibitors of the bioluminescence reaction such as CCCP or cyanide (see \ref{inhib}) components. Similar experiments can be performed with \textit{Photobacterium phosphoreum} ANT-2200 to confirm this percentage.\\ 

\subsection[Proposal for the detection of bioluminescent organisms]{Proposal for the detection of bioluminescent organisms
\sectionmark{Proposal}}
\sectionmark{Proposal}


\subsubsection*{Importance of bioluminescence into the carbon cycle}

Following this work, we propose to deploy autonomous PMTs through the water column in order to check bioluminescent organisms at the upper and lower level of the mesopelagic zone as well as "bioluminescent" fecal pellets flux.\\

\subsubsection*{PMTs acquisition and calibration in the laboratory.} 

The Centre de physique des Particules de Marseille (CPPM) is one of the main institutes of the ANTARES collaboration, in charge of ANTARES PMTs development. Recently, in the framework of the MEUST project (Mediterranean Eurocentre for underwater sciences and technologies), the CPPM develops autonomous PMTs dedicated to the survey of a new site for the future km$^3$ neutrino telescope. In close collaboration with the CPPM, we could ask the development of specific PMTs dedicated to the bioluminescence study. Indeed, using this first experiment, we are able to define the need of two different data sampling rates, in order to increase memory space acquisition and the efficiency of PMTs data acquisition.\\

\subsubsection*{First essays of PMTs deployments on mooring line.} 

In the framework of the site survey dedicated to the MEUST project, a deep mooring line has already been deployed. We plan to reproduce the schema of this mooring line also in the upper and lower part of the mesopelagic zone, adding UVP and LISST instruments. First essays will be done to choose the best implementation of all instruments, first data acquisition and cross-data calibration. More precisely, two sampling rates have to be improved with first deployments. Firstly, a low frequency sampling rate (about minutes) will be dedicated to the record of bioluminescence intensity linked to environmental variables. Then, at other periods, higher frequency sampling rate will be dedicated to the organisms' recognition spectra. Modulation in frequency sampling rate is a good compromise between data acquisition interest and data storage. \\

\subsubsection*{Time-frequency analyzes of bioluminescence time-series and oceanographic variables (temperature, salinity, currents, oxygen, POC fluxes)}

One of the approaches proposed for further investigations, is to understand links between bioluminescent organisms and carbon supply in the mesopelagic and bathypelagic environment. To do so, we could use the previously described mooring line (PMTs, UVP, LISS, CTD, current meter at different depths). Time series analysis using statistically adapted and developed mathematical methods is a crucial point to access to a better ecological understanding. Indeed, such time series are statistically defined as non-linear and non-stationary. If the Fourier method has been traditionally used for time-series decomposition it is not well adapted to non-stationary data. Since the 80's two different methods named the wavelet and the Huang-Hilbert decomposition methods have been developed to counteract this gap \citep{Huang1998,Torrence1998}. These decompositions methods are defined in the time-frequency domain. From the ANTARES time-series analysis, using both the wavelets and Hilbert-Huang methods, bioluminescence intensity has been found to be clearly sea-current-intensity dependent but also highly correlated with modification of deep-water masses (using salinity and temperature as proxies) \citep{martini2013b}. These mathematical methods represent an appropriate way to provide new insight from time-series and would be applied in future work.\\


\subsubsection*{Bioluminescence spectra analysis and automatic classification of organisms}

Using this instrumented-mooring line, a perspective of ecological interest in the mesopelagic and bathypelagic could also be approached. As already described, bioluminescence is a phenomenon produced by very wide range of organisms. We focused on links between biology and the environmental changes aiming at the definition of populations or organisms involved. Bioluminescence spectra analysis is a way to determine what kind of organisms are present close to the mooring line and, eventually, to detect some shifts in populations over time. Indeed, different organisms have already been described to emit different light pattern (see Figure \ref{proposal}, \citealp{widder1989,nealson1986}).\\

\begin{figure}[!h]
\linespread{1} 
\centering
\includegraphics[scale=0.6]{proposal.pdf}
\caption[Example of bioluminescence emission patterns. A) From the ANTARES observatory, B) In the literature.]{Example of bioluminescence emission patterns. A) From the ANTARES observatory, B) In the literature \citep{hastings1991}.}
\label{proposal}
\end{figure} 
 
These signatures can be used to describe populations crossing the mooring line. Bioluminescence spectra as a signature of organisms has been approached by \cite{nealson1986}, which only used ten species. In this part, the proposed approach is the use of mathematical unsupervised classification for population detection. Emission intensity can be recorded as curves for individuals crossing the detector. The shape, length and area of the curves are several potential descriptors that can be used for the classification method. Depending on the variability between these criteria, each bioluminescence emission curve is attributed to one class. At the end of the process, each class defines similar curves with the same pattern of light emission, potentially, the same class of organisms.\\

This classification method is already used into several fields such as cytometry \citep{malkassian2011} or hydrological prediction \citep{nerini2000}. In a final aim, crossing this automatic classification method for bioluminescence intensity (recorded at high-frequency sampling rates) with the UVP, we should define an image of organisms for each class defined. In order to recognize organisms that have been classified with video camera and bioluminescence spectra, collaborations with international experts in bioluminescent organisms have to be envisaged. This support would be of great importance to determine higher level identification for observed organisms.\\ 




%\chapter[Complementary data]{Complementary data
\chaptermark{Complementary data}}
\chaptermark{Complementary data}
\minitoc

\section{Bacterial strain from ANTARES station}
Conjointly to the isolation of \textit{Photobacterium phosphoreum} ANT-2200, a second bacterial strain was isolated. It has been described as \textit{Acinetobacter lwoffii} (Kingdom:Bacteria, Phylum:Proteobacteria, Class: Gammaproteobacteria, Order:Pseudomonadales, Family:	Moraxellaceae). \\
In Chapter 4, describing the diversity in samples from March and October 2011, \textit{Acinetobacter} have been detected up to 1.9 and 2.3\% respectively. \\  

\begin{figure}[!h]
\begin{small}
\begin{center}
\includegraphics[scale=0.5]{contam.pdf}
\caption{Phylogenetic tree for \textit{Acinetobacter lwoffii} isolated at the ANTARES station. }
\label{}
\end{center}
\end{small}
\end{figure} 

\section{Oxygen time series}

Oxygen time series has been analyzed in a similar way as bioluminescence, temperature, salinity and current speed time series. However, due to a pronounced trend in the data with no clear explanation, as well as no pertinent results this variable has been taken off this study. Figure \ref{oxyIMF} represents the Hilbert-Huang decomposition filtered into 10 IMFs and Figure \ref{oxyHHG} is the spectrogram crossing both oxygen and bioluminescence Hilbert-Huang frequency decomposition.\\

\begin{figure}[!h]
\linespread{1} 
\centering
\includegraphics[width=15cm]{oxyIMF.jpeg}
\caption[Hilbert-Huang decomposition for the oxygen signal into 10 Intrinsic Mode Functions.]{Hilbert-Huang decomposition for the oxygen signal into 10 intrinsic mode functions, from C1 to C10. These Intrinsic Mode Functions are first order stationary. }
\label{oxyIMF}
\end{figure}

\begin{figure}[!h]
\linespread{1} 
\centering
\includegraphics[width=12cm]{oxy_hhg.pdf}
\caption[Cross spectrogram between bioluminescence and oxygen.]{Cross spectrogram between bioluminescence and oxygen. The cross correlation coefficients (represented between -0.2 and 0.8) are plotted as isolines and color scale for bioluminescence with oxygen. X and y-axis represent frequencies for each variable.}
\label{oxyHHG}
\end{figure}

%\section{communiqu�s de presse}
%\begin{figure}[!h]
%\linespread{1} 
%\centering
%\includegraphics[width=15cm]{compress.pdf}
%\caption[]{}
%\label{compress}
%\end{figure} 


%\pdfbookmark[0]{Annexes}{Annexes}
\chapter{Annexes}

\section{Oxygen percentage attributed to the bioluminescence reaction}
\label{oxypercent}
A partir des donn�es acquises par experimentation en fermenteur, un ratio th�orique de respiration de l'O$_2$ transform� en CO$_2$ ($\frac{CO_{2resp}}{O_{2resp}}$) ainsi qu'un ratio experimental prenant en compte la part d'O$_2$ intervenant dans la r�action de bioluminescence ($\frac{CO_{2resp}}{O_{2resp}+O_{2biolu}}$) peuvent �tre d�crit comme suit:\\
\begin{flushleft}

\[ \left\{
\begin{array}{rcr}
\frac{CO_{2resp}}{O_{2resp}}&=& 0.86\\
\frac{CO_{2resp}}{O_{2resp}+O_{2biolu}}&= &0.82\\
\end{array}
\right. \]

D'o�: 

\[ \left\{
\begin{array}{rcr}
CO_{2resp}=0.86 \times O_{2resp}\\
\frac{0.86-0.82}{O_{2resp}}=O_{2biolu}\\
\end{array}
\right. \]

\[ \left\{
\begin{array}{rcr}
CO_{2resp}&=&0.86 \times O_{2resp}\\
O_{2biolu}&= &0.048 \times O_{2resp}\\
\end{array}
\right. \]

\end{flushleft}
On d�duit donc que pr�s de 4.8 \% de l'O$_2$ consomm� est d�di� � la reaction de bioluminescence.

\section{Quantum yield computation}
\label{quantumyield}

\subsection*{Computation example from \cite{makemson1986}}
\begin{center}
\textbf{Quantum yield}: \textit{le quantum yield \textit{in vivo} de la bioluminescence est le nombre de photons �mis par mol�cule d'O$_2$ utilis� par la lucif�rase.}\\
\end{center}

\textit{V. harveyi} consomme 80 nmol d'O$_2$ min$^{-1}$ 10$^{9}$ cellules et, dans cette �tude, environ 12\% de cette consommation est d�di� � la r�action de bioluminescence, soit 9.6 nmol d'O$_2$ min$^{-1}$ 10$^{9}$ cellules. Par mol�cule d'oxyg�ne on obtient:
\begin{center}
9.6 $\times$ 10$^{-9}$ $\times$ 6.022 $\times$ 10$^{23}$ = 5.6 $\times$ 10$^{15}$ mol�cules d'O$_2$ min$^{-1}$ 10$^{9}$ cellules
\end{center}  
Ces cellules vont �mettre environ 1000 photons $s^{-1}$ cellule soit:
\begin{center}
1000 $\times$ 10$^9$ $\times$ 60 = 6 $\times$ 10$^{13}$ photons min$^{-1}$ \\
 \end{center}
 
Le quantum yield pour cette souche bioluminescente serait donc de:
\begin{center}
(6 $\times$ 10$^{13}$) / (5.6 $\times$ 10$^{15}$) $\sim$ \textbf{0.017}\\
\end{center}

Ce quantum yield est de 0.026 pour \textit{V. fischeri} (bas� sur 3000 photons s$^{-1}$ 10$^9$ cellules).\\

\subsection*{Bioreactor quantum yield computation}
A partir des exp�rimentations men�es en fermenteur, une consommation de 19.1 mM d'O$_2$ a �t� mesur�e. Le pourcentage de consommation d'oxyg�ne utilis� par la r�action de bioluminescence, a �t� estim� au cours de ces exp�rimentations � 4.8 \%. Ce qui donne une consommation d'O$_2$ de 0.82 mM d�di� � la r�action de bioluminescence. En m�me temps, 4.1 $\times$ 10$^9$ photons ont �t� �mis et int�gr�s sur cette m�me p�riode. Ces donn�es permettent donc de d�finir successivement: \\
\begin{center}0.82 $\times$ 10$^{-3}$ $\times$ 6.022 $\times$ 10$^{23}$ = 4.9 $\times$ 10$^{20}$ mol�cules d'O$_2$ totales\\
\end{center}
avec 6.022 $\times$ 10$^{23}$ le nombre d'Avogadro.  \\

Et par cons�quent:\\
\begin{center}$\frac{4.1 \times 10^9}{4.9 \times 10^{20}} = 8.4 \times 10^{-12}$ photons par molecule d'O$_2$.
\end{center}

\subsection*{Quantum yield computation}
\textbf{High pressure, 22 MPa}\\
A partir des exp�rimentations men�es en HPBs, une consommation d'O$_2$ de 120 $\mu$mol L$^{-1}$ a �t� mesur�e. Le pourcentage de consommation d'oxyg�ne utilis� par la r�action de bioluminescence, a �t� estim� au cours des exp�rimentations en fermenteur � 4.8 \%, d�terminant ainsi une consommation d'O$_2$ de 5.76 $\times$ 10$^{-6}$ mol L$^{-1}$ d�di�e � la r�action de bioluminescence. De plus, 3.1 $\times$ 10$^9$ photons ont �t� �mis et int�gr�s sur cette m�me p�riode dans 500 mL de culture. Ces donn�es permettent donc de d�finir successivement: \\
\begin{center}5.76 $\times$ 10$^{-6}$ $\times$ 6.022 $\times$ 10$^{23}$ = 3.5 $\times$ 10$^{18}$ mol�cules d'O$_2$ totales
\end{center} avec 6.022 $\times$ 10$^{23}$ le nombre d'Avogadro. Et par cons�quent:\\
\begin{center}$\frac{3.1 \times 10^9 \times 2}{3.5 \times 10^{18}} = 1.8 \times 10^{-9}$  photons par molecule d'O$_2$.
\end{center}

\textbf{Atmospheric pressure, 0.1 MPa}\\
A partir des exp�rimentations men�es en HPBs, une consommation de 10 $\mu$mol L$^{-1}$ d'O$_2$ a �t� mesur�e. Le pourcentage de consommation d'oxyg�ne utilis� par la r�action de bioluminescence, a �t� estim� au cours des exp�rimentations en fermenteur � 4.8 \%. Ce qui donne une consommation d'O$_2$ de 4.8 $\times$ 10$^{-7}$ $\mu$mol L$^{-1}$ destin� � la r�action de bioluminescence. En m�me temps, 1.8 $\times$ 10$^9$ photons ont �t� �mis et int�gr�s sur cette m�me p�riode dans 500 mL. Ces donn�es permettent donc de d�finir successivement: \\
\begin{center}$4.8 \times 10^{-7} \times 6.022 \times 10^{23} = 2.9 \times 10^{17}$ mol�cules d'O$_2$ totales \end{center} avec 6.022 $\times$ 10$^{23}$ le nombre d'Avogadro. Et par cons�quent:\\
\begin{center}$\frac{1.8 \times 10^9 \times 2}{2.9 \times 10^{17}} = 1.2 \times 10^{-8}$ photons par mol�cule d'O$_2$.\end{center}

\section{Photon emission per cell}

L'ensemble des donn�es pr�sent�es ci-apr�s correspondent au pic de bioluminescence maximale �mise � un temps T$_n$, ainsi qu'� la valeur de DO$_{600nm}$ associ�e � T$_n$, d'apr�s le mod�le logistique. En fermenteur, le pic de bioluminescence donne l'�mission de 2.5 $\times$ 10$^7$ photons correspondant � une DO$_{600nm}$ de 4.0. D'apr�s l'�quation:\\
 \begin{center}
Number of DAPI-stained cells mL$^{-1}$ = 6.7 $\times$ 10$^8$ $\times$ OD$_{600nm}$ - 2.3 $\times$ 10$^7$ (1)\\
$(R^2 = 0.79, N = 14)$\\
\end{center}
on obtient donc 9.6 $\times$ 10$^{-3}$ photons cell$^{-1}$.\\

De la m�me fa�on, les quantifications du nombre de photons par cellule ont �t� calcul�s et sont pr�sent�s dans le tableau \ref{summary}.

% Notations


% ================================== BIBLIOGRAPHIE =============================
\baselineskip=12pt 

%% Choix du style 
%% En franais
\bibliographystyle{elsarticle-harv}
\bibliography{martinietal}
%\bibliographystyle{apalike-fr}
%\bibliographystyle{astron}
%\bibliographystyle{elsart-harv}
%\bibliographystyle{alpha-fr} % style alphabtique en franais
%% En anglais
%\bibliographystyle{alpha} % style alphabtique en anglais
%\bibliographystyle{unsrt} % style numrot en anglais
%% Il y a plein d'autres possibilits
%% Fabrication de la biblio
 
% pour afficher la biblio
% utilise le fichier bibliothese.bib

% Remarque : l'ajout de la biblio  la table des matires se fait par le
% paquet tocbibind (car la commande addcontentsline ne fabrique pas le bon
% numro de page)
%%\ThisLRCornerWallPaper{1}{cov2.pdf}
\pdfbookmark[0]{}{}
    \pagecolor{}
\chapter*{}
%\markboth{R�sum�-Abstract}{}
%\newpage
\textcolor{white}{\section*{}}
%%\ThisLRCornerWallPaper{1}{cov2.pdf}
\pdfbookmark[0]{R�sum�}{resume}
    \pagecolor{black}
    \thispagestyle{empty}

%\chapter*{}
%\markboth{R�sum�-Abstract}{}

\textcolor{white}{\section*{Abstract}}
\vspace{5mm}

\normalsize{ \textcolor{white}{Bioluminescence is the emission of light by living organisms. In the bathypelagic waters, where darkness is one of the main characteristic, this phenomenon seems to play a major role for biological interactions and in the carbon cycle. This work aims to determine if bioluminescence can be considered as a proxy of biological activity in the deep sea. Two axes have been studied\string: 
 \textit{(i)} in the deep sea, we attempt to describe and explain the bioluminescence variability over time \textit{(ii)} the part of bacterial bioluminescence is investigated in the light signal \textit{in situ}. This multidisciplinary study develops both \textit{in situ} and laboratory approaches.}}\\
 % Observations at microscopic, local and regional scales are taken into account. Moreover, several ecological levels are covered from population, to community and to ecosystem.}} \\

\normalsize{\textcolor{white}{The ANTARES telescope immersed in the Mediterranean Sea at 2,475 m depth has been used as an oceanographic observatory recording bioluminescence as well as environmental variables at high frequency. This time series analysis, defined as non linear and non stationary, highlighted two periods of high bioluminescence intensity in 2009 and 2010. These events have been explained by convection phenomena in the Gulf of Lion, indirectly impacting the bioluminescence sampled at this station. In the laboratory, bacterial bioluminescence has been described using a piezophilic bacterial model isolated during a high-bioluminescence-intensity event. Hydrostatic pressure linked to the \textit{in situ} depth (22 MPa) induces a higher bioluminescence activity than under atmospheric pressure (0.1 MPa). Then, the survey of the deep prokaryotic communities has been done at the ANTARES station, over the year 2011. This survey shows the presence of about 0.1 to 1\% of bioluminescent bacteria even during a low-bioluminescence-activity period. These cells were mainly actives.}}\\
\vspace{10mm}

\textcolor{white}{\textbf{Key words:} Bioluminescence, Bathypelagic environment, Mediterranean Sea, Time series analysis, Hydrostatic pressure, Bacteria, \textit{In situ} observatories}\\

\newpage
\textcolor{white}{\section*{R�sum�}}
% pour ne pas avoir de numro de page sur la page de garde -- le compteur de
% page est cependant  1, c'est--dire que la numrotation commence  partir
% de la page de garde

\vspace{5mm}

\normalsize{ \textcolor{white}{La bioluminescence est l'�mission de lumi�re par des organismes vivants. En milieu bathyp�lagique, o� l'absence de lumi�re est une caract�ristique majeure, ce ph�nom�ne semble avoir un r�le �cologique primordial dans les interactions biologiques ainsi que dans le cycle du carbone. Ce travail cherche � d�terminer si la bioluminescence peut �tre d�finie comme un proxy de l'activit� biologique en milieu profond. Deux axes sont �tudi�s: \textit{(i)} en milieu profond, la bioluminescence est d�crite et expliqu�e au cours du temps \textit{(ii)} la part de bioluminescence bact�rienne est estim�e dans l'�mission de luminescence enregistr�e \textit{in situ}. Cette �tude multidisciplinaire d�veloppe � la fois une approche \textit{in situ} et en laboratoire.}}\\
% Des �chelles d'observation microscopiques, locales et r�gionales sont prises en compte. De m�me, diff�rents niveaux �cologiques sont abord�s: populationnel, communautaire et �cosyst�mique.}}\\

\normalsize{\textcolor{white}{Le t�lescope ANTARES, immerg� en M�diterran�e, � 2475 m de profondeur, a jou� le r�le d'un observatoire oc�anographique enregistrant la bioluminescence ainsi que les variables environnementales � haute fr�quence. L'analyse de ces s�ries temporelles, non-lin�aires et non-stationnaires a permis de mettre en �vidence deux p�riodes de forte activit� de bioluminescence en 2009 et 2010. Ces �v�nements ont �t� expliqu�s par des ph�nom�nes de convection dans le Golfe du Lion, impactant indirectement la bioluminescence enregistr�e sur ANTARES. En laboratoire, la bioluminescence bact�rienne a �t� d�crite sur une souche mod�le piezophile, isol�e au cours d'un �v�nement de forte bioluminescence. La pression hydrostatique li�e � la profondeur \textit{in situ} (22 MPa) induit une plus forte bioluminescence qu'� pression atmosph�rique (0.1 MPa). Enfin, le suivi des communaut�s procaryotiques profondes a �t� r�alis�, sur le site ANTARES, au cours de l'ann�e 2011. Ce suivi a montr� la pr�sence de 0.1 � 1\% de bact�ries bioluminescentes dans une p�riode enregistrant une faible activit� de bioluminescence. Ces cellules ont �t� d�finies comme majoritairement actives.}}\\

\vspace{10mm}
%\begin{table}
%\begin{tabular}{c}
 \textcolor{white}{\textbf{Mots cl�s:} Bioluminescence, Milieu bathyp�lagique, Mer M�diterran�e, Analyse de s�ries temporelles, Pression hydrostatique, Bact�ries, Observatoire \textit{in situ}}
%\end{table}
%\end{tabular}


% =============================== INDEX DES NOTATIONS ==========================

\listoffigures
\listoftables
% numro de page)
\backmatter
%\ThisLRCornerWallPaper{1}{cov2.pdf}
\pdfbookmark[0]{R�sum�}{resume}
    \pagecolor{black}
    \thispagestyle{empty}

%\chapter*{}
%\markboth{R�sum�-Abstract}{}

\textcolor{white}{\section*{Abstract}}
\vspace{5mm}

\normalsize{ \textcolor{white}{Bioluminescence is the emission of light by living organisms. In the bathypelagic waters, where darkness is one of the main characteristic, this phenomenon seems to play a major role for biological interactions and in the carbon cycle. This work aims to determine if bioluminescence can be considered as a proxy of biological activity in the deep sea. Two axes have been studied\string: 
 \textit{(i)} in the deep sea, we attempt to describe and explain the bioluminescence variability over time \textit{(ii)} the part of bacterial bioluminescence is investigated in the light signal \textit{in situ}. This multidisciplinary study develops both \textit{in situ} and laboratory approaches.}}\\
 % Observations at microscopic, local and regional scales are taken into account. Moreover, several ecological levels are covered from population, to community and to ecosystem.}} \\

\normalsize{\textcolor{white}{The ANTARES telescope immersed in the Mediterranean Sea at 2,475 m depth has been used as an oceanographic observatory recording bioluminescence as well as environmental variables at high frequency. This time series analysis, defined as non linear and non stationary, highlighted two periods of high bioluminescence intensity in 2009 and 2010. These events have been explained by convection phenomena in the Gulf of Lion, indirectly impacting the bioluminescence sampled at this station. In the laboratory, bacterial bioluminescence has been described using a piezophilic bacterial model isolated during a high-bioluminescence-intensity event. Hydrostatic pressure linked to the \textit{in situ} depth (22 MPa) induces a higher bioluminescence activity than under atmospheric pressure (0.1 MPa). Then, the survey of the deep prokaryotic communities has been done at the ANTARES station, over the year 2011. This survey shows the presence of about 0.1 to 1\% of bioluminescent bacteria even during a low-bioluminescence-activity period. These cells were mainly actives.}}\\
\vspace{10mm}

\textcolor{white}{\textbf{Key words:} Bioluminescence, Bathypelagic environment, Mediterranean Sea, Time series analysis, Hydrostatic pressure, Bacteria, \textit{In situ} observatories}\\

\newpage
\textcolor{white}{\section*{R�sum�}}
% pour ne pas avoir de numro de page sur la page de garde -- le compteur de
% page est cependant  1, c'est--dire que la numrotation commence  partir
% de la page de garde

\vspace{5mm}

\normalsize{ \textcolor{white}{La bioluminescence est l'�mission de lumi�re par des organismes vivants. En milieu bathyp�lagique, o� l'absence de lumi�re est une caract�ristique majeure, ce ph�nom�ne semble avoir un r�le �cologique primordial dans les interactions biologiques ainsi que dans le cycle du carbone. Ce travail cherche � d�terminer si la bioluminescence peut �tre d�finie comme un proxy de l'activit� biologique en milieu profond. Deux axes sont �tudi�s: \textit{(i)} en milieu profond, la bioluminescence est d�crite et expliqu�e au cours du temps \textit{(ii)} la part de bioluminescence bact�rienne est estim�e dans l'�mission de luminescence enregistr�e \textit{in situ}. Cette �tude multidisciplinaire d�veloppe � la fois une approche \textit{in situ} et en laboratoire.}}\\
% Des �chelles d'observation microscopiques, locales et r�gionales sont prises en compte. De m�me, diff�rents niveaux �cologiques sont abord�s: populationnel, communautaire et �cosyst�mique.}}\\

\normalsize{\textcolor{white}{Le t�lescope ANTARES, immerg� en M�diterran�e, � 2475 m de profondeur, a jou� le r�le d'un observatoire oc�anographique enregistrant la bioluminescence ainsi que les variables environnementales � haute fr�quence. L'analyse de ces s�ries temporelles, non-lin�aires et non-stationnaires a permis de mettre en �vidence deux p�riodes de forte activit� de bioluminescence en 2009 et 2010. Ces �v�nements ont �t� expliqu�s par des ph�nom�nes de convection dans le Golfe du Lion, impactant indirectement la bioluminescence enregistr�e sur ANTARES. En laboratoire, la bioluminescence bact�rienne a �t� d�crite sur une souche mod�le piezophile, isol�e au cours d'un �v�nement de forte bioluminescence. La pression hydrostatique li�e � la profondeur \textit{in situ} (22 MPa) induit une plus forte bioluminescence qu'� pression atmosph�rique (0.1 MPa). Enfin, le suivi des communaut�s procaryotiques profondes a �t� r�alis�, sur le site ANTARES, au cours de l'ann�e 2011. Ce suivi a montr� la pr�sence de 0.1 � 1\% de bact�ries bioluminescentes dans une p�riode enregistrant une faible activit� de bioluminescence. Ces cellules ont �t� d�finies comme majoritairement actives.}}\\

\vspace{10mm}
%\begin{table}
%\begin{tabular}{c}
 \textcolor{white}{\textbf{Mots cl�s:} Bioluminescence, Milieu bathyp�lagique, Mer M�diterran�e, Analyse de s�ries temporelles, Pression hydrostatique, Bact�ries, Observatoire \textit{in situ}}
%\end{table}
%\end{tabular}

% =============================== INDEX DES NOTATIONS ==========================
%%\ThisLRCornerWallPaper{1}{cov2.pdf}
\pdfbookmark[0]{R�sum�}{resume}
    \pagecolor{black}
    \thispagestyle{empty}

%\chapter*{}
%\markboth{R�sum�-Abstract}{}

\textcolor{white}{\section*{Abstract}}
\vspace{5mm}

\normalsize{ \textcolor{white}{Bioluminescence is the emission of light by living organisms. In the bathypelagic waters, where darkness is one of the main characteristic, this phenomenon seems to play a major role for biological interactions and in the carbon cycle. This work aims to determine if bioluminescence can be considered as a proxy of biological activity in the deep sea. Two axes have been studied\string: 
 \textit{(i)} in the deep sea, we attempt to describe and explain the bioluminescence variability over time \textit{(ii)} the part of bacterial bioluminescence is investigated in the light signal \textit{in situ}. This multidisciplinary study develops both \textit{in situ} and laboratory approaches.}}\\
 % Observations at microscopic, local and regional scales are taken into account. Moreover, several ecological levels are covered from population, to community and to ecosystem.}} \\

\normalsize{\textcolor{white}{The ANTARES telescope immersed in the Mediterranean Sea at 2,475 m depth has been used as an oceanographic observatory recording bioluminescence as well as environmental variables at high frequency. This time series analysis, defined as non linear and non stationary, highlighted two periods of high bioluminescence intensity in 2009 and 2010. These events have been explained by convection phenomena in the Gulf of Lion, indirectly impacting the bioluminescence sampled at this station. In the laboratory, bacterial bioluminescence has been described using a piezophilic bacterial model isolated during a high-bioluminescence-intensity event. Hydrostatic pressure linked to the \textit{in situ} depth (22 MPa) induces a higher bioluminescence activity than under atmospheric pressure (0.1 MPa). Then, the survey of the deep prokaryotic communities has been done at the ANTARES station, over the year 2011. This survey shows the presence of about 0.1 to 1\% of bioluminescent bacteria even during a low-bioluminescence-activity period. These cells were mainly actives.}}\\
\vspace{10mm}

\textcolor{white}{\textbf{Key words:} Bioluminescence, Bathypelagic environment, Mediterranean Sea, Time series analysis, Hydrostatic pressure, Bacteria, \textit{In situ} observatories}\\

\newpage
\textcolor{white}{\section*{R�sum�}}
% pour ne pas avoir de numro de page sur la page de garde -- le compteur de
% page est cependant  1, c'est--dire que la numrotation commence  partir
% de la page de garde

\vspace{5mm}

\normalsize{ \textcolor{white}{La bioluminescence est l'�mission de lumi�re par des organismes vivants. En milieu bathyp�lagique, o� l'absence de lumi�re est une caract�ristique majeure, ce ph�nom�ne semble avoir un r�le �cologique primordial dans les interactions biologiques ainsi que dans le cycle du carbone. Ce travail cherche � d�terminer si la bioluminescence peut �tre d�finie comme un proxy de l'activit� biologique en milieu profond. Deux axes sont �tudi�s: \textit{(i)} en milieu profond, la bioluminescence est d�crite et expliqu�e au cours du temps \textit{(ii)} la part de bioluminescence bact�rienne est estim�e dans l'�mission de luminescence enregistr�e \textit{in situ}. Cette �tude multidisciplinaire d�veloppe � la fois une approche \textit{in situ} et en laboratoire.}}\\
% Des �chelles d'observation microscopiques, locales et r�gionales sont prises en compte. De m�me, diff�rents niveaux �cologiques sont abord�s: populationnel, communautaire et �cosyst�mique.}}\\

\normalsize{\textcolor{white}{Le t�lescope ANTARES, immerg� en M�diterran�e, � 2475 m de profondeur, a jou� le r�le d'un observatoire oc�anographique enregistrant la bioluminescence ainsi que les variables environnementales � haute fr�quence. L'analyse de ces s�ries temporelles, non-lin�aires et non-stationnaires a permis de mettre en �vidence deux p�riodes de forte activit� de bioluminescence en 2009 et 2010. Ces �v�nements ont �t� expliqu�s par des ph�nom�nes de convection dans le Golfe du Lion, impactant indirectement la bioluminescence enregistr�e sur ANTARES. En laboratoire, la bioluminescence bact�rienne a �t� d�crite sur une souche mod�le piezophile, isol�e au cours d'un �v�nement de forte bioluminescence. La pression hydrostatique li�e � la profondeur \textit{in situ} (22 MPa) induit une plus forte bioluminescence qu'� pression atmosph�rique (0.1 MPa). Enfin, le suivi des communaut�s procaryotiques profondes a �t� r�alis�, sur le site ANTARES, au cours de l'ann�e 2011. Ce suivi a montr� la pr�sence de 0.1 � 1\% de bact�ries bioluminescentes dans une p�riode enregistrant une faible activit� de bioluminescence. Ces cellules ont �t� d�finies comme majoritairement actives.}}\\

\vspace{10mm}
%\begin{table}
%\begin{tabular}{c}
 \textcolor{white}{\textbf{Mots cl�s:} Bioluminescence, Milieu bathyp�lagique, Mer M�diterran�e, Analyse de s�ries temporelles, Pression hydrostatique, Bact�ries, Observatoire \textit{in situ}}
%\end{table}
%\end{tabular}

 

% numro de page)

% =============================== INDEX DES NOTATIONS ==========================

 % pour afficher l'index
% pour afficher la biblio
% utilise le fichier bibliothese.bib

% Remarque : l'ajout de la biblio  la table des matires se fait par le
% paquet tocbibind (car la commande addcontentsline ne fabrique pas le bon
% numro de page)

% =============================== INDEX DES NOTATIONS ==========================

%\printindex % pour afficher l'index


\end{document}
