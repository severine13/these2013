\ThisLRCornerWallPaper{1}{cov2.pdf}

\thispagestyle{empty}
% pour ne pas avoir de numro de page sur la page de garde -- le compteur de
% page est cependant  1, c'est--dire que la numrotation commence  partir
% de la page de garde


\textcolor{white}{\textbf{Abstract:}}
\vspace{5mm}

\normalsize{ \textcolor{white}{Bioluminescence is the emission of light by living eukaryotic and prokaryotic organisms. In the bathypelagic environment, where darkness is one of the main characteristic, this phenomenon seems to play a major role for biological interactions and in the carbon cycle. This thesis aims at determine if bioluminescence is a proxy of biological activity in the deep sea. Two axes have been studied\string: 
 \textit{(i)} In the deep sea does it exist variability of light intensity over time and how to explain it ? \textit{(ii)} What is the part of bacterial bioluminescence in the light signal \textit{in situ} ? This multidisciplinary study develops both \textit{in situ} and in the laboratory approaches. Observations at microscopic, local and regional scales are taken into account. Moreover, several ecological levels are covered from population, to community and to ecosystem.}} \\

\normalsize{\textcolor{white}{The ANTARES telescope immersed in the Mediterranean Sea at 2,475 m depth has been used as an oceanographic observatory recording bioluminescence as well as environmental variables at high frequency. This time series analysis, defined as non linear and non stationary, highlighted two periods of high bioluminescence intensity in 2009 and 2010. These events have been explained by convection phenomena in the Gulf of Lion, indirectly impacting the bioluminescence sampled at this station. In the laboratory, bacterial bioluminescence has been described using a piezophilic bacterial model isolated during a high-bioluminescence-intensity event. Hydrostatic pressure linked to the \textit{in situ} depth (22 MPa) induces a higher bioluminescence activity than under atmospheric pressure (0.1 MPa). Then, the survey of the deep prokaryotic communities has been done at the ANTARES station, over the year 2011. This survey shows the presence of about 0.1 to 1\% of bioluminescent bacteria even during a low-bioluminescence-activity period. These cells were mainly actives.}}\\

\textcolor{white}{\textbf{Key words:} Bioluminescence, Bathypelagic environment, Mediterranean Sea, Time series analysis, Hydrostatic pressure, Bacteria, \textit{In situ} observatories}\\


