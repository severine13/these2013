\chapter[Introduction and objectives]{Introduction and objectives
\chaptermark{Introduction and objectives}}
\chaptermark{Introduction and objectives}
\minitoc

\newpage
\section{Marine bioluminescence}

The luminescence describes the production of cold light, by opposition to the incandescence. This term includes the phosphorescence, the fluorescence as well as the bioluminescence. The first two phenomena are an emission of light following the absorption of photons while the bioluminescence is the production of photons after a chemical reaction by living organisms.\\

The bioluminescence phenomenon is known since Antiquity. Aristote (348-322 BC) already reports light emission by dead fish. Thereafter, in the \textit{Naturalis Historia}, Pline (23-79 AD. JC) also describes various bioluminescent organisms such as glowing worms, "mushroom", or cnidarians. Moreover, in the History, marine bioluminescence, sometimes, played an unexpected role. Indeed, during the First World War, the submarine U-34, was the last German submarine sunk by the allies, identified by its luminescent wake. During the Second War, these wakes sometimes allowed the pilots to return to their aircraft carrier at night. Finally, during the Gulf War, american soldiers had to change the path followed by vessels to avoid the shoals of bioluminescent Dinoflagellates that might betray their presence.\\ 

\subsection{Phylogenetic diversity of bioluminescent organisms}
\label{groupe}
In terrestrial environments, few organisms have been described as bioluminescent. Conversely, in marine environments, nearly 90\% of organisms are estimated able of bioluminescence emission \citep{widder2010} and are extensively described in the literature (see Figure \ref{taxa}). Some non-planktonic groups (moving independently of water-mass movements), such as cephalopods (210 bioluminescent species) or telosteens (273 bioluminescent species), are well represented among bioluminescent organisms \citep{poupin1999,herring1987,herring1985}.\\

\begin{figure}[p]
\linespread{1} 
\centering
\includegraphics[width=14.5cm]{taxa.pdf}
\caption{Phylogenetic representation of the diversity of bioluminescent organisms, from \cite{Haddock2010}.}
\label{taxa}
\end{figure} 

\newpage
Planktonic phylogenetic groups, for which the largest number of bioluminescent species have been described, are\string: Bacteria, Dinoflagellates, Radiolarians, Cnidarians and Ctenophora, Crustaceans (Copepods, Amphipods, Euphausiids, Decapods), Chaetognathes and Tunicates. Bioluminescent species are also described within the Molluscs (Gastropods), Annelids (Polychaetes) and Echinodermata (Holothuroidea) (see Figure \ref{taxa}). Amongst prokaryotes\footnote{procaryote est un terme non-phylog�n�tique regroupant Bacteria et Archaea}, only Bacteria have been described as able to emit light. Archaea are, to our knowledge, not described as bioluminescent.\\

Bioluminescent bacterial strains have been described, from the beginnings of microbiology, with the work of B. Fischer in 1887, Mr. Beijerinck in 1889 and ZoBell in 1946. Luminescent bacteria\footnote{In the following manuscript, the term "bacteria" will be used for procaryotes} are Gammaproteobacteria, Gram-negative, non-spore forming, chemoorganotrophe, heterotrophic, and mostly aerobic. The known bioluminescent species belong to the genera \textit {Vibrio}, \textit{Photobacterium}, \textit{Shewanella} and \textit{Photorhabdus}. Amongst these genera, \textit{Photorhabdus} is the only terrestrial one, the others being present in marine environments.\\

To illustrate this diversity, \citep{poupin1999} proposed a non-exhaustive list in the Iroise sea, based on the literature. About 500 bioluminescent planktonic species, including six bacterial species, are described. Phylogenetically, the faculty of bioluminescence is randomly distributed in the living world \citep{herring1987}. Phylogenetic analyzes suggested that bioluminescence is a genetic trait that evolved independently between 30 and 50 times \citep{Haddock2010}. However, the authors of these phylogenetic syntheses, bringing together observations of bioluminescent organisms, highlight the difficulties of establishing such inventory. On the one hand, the techniques of \textit{in situ} observations of this phenomenon have changed in a few decades. These observations are constraint by the difficulty of capturing these bioluminescent organisms. On the other hand, the physiological state of the organisms determines the observation of light emission for some species. The photon emission sometimes decreases, due to the capture, to \textit{in situ} stimulation or to changing conditions (pressure, temperature ... ) after sampling \citep{herring1976}.\\

Upon bioluminescence emission, the observation and measurement of emission spectrum in marine organisms revealed a wide range of colors in the visible wavelength (\citealp{widder1983,haddock1999}, see Figure \ref{couleurbiol}). However, in the marine environment, where the blue wavelength penetrates more deeply, the majority of organisms emits at about 475 nm. For example, all bioluminescent bacteria emit at 490 nm. For coastal and benthic species, a shift was noticed in the emission wavelength to the green one. Other wavelengths of bioluminescence emission, from violet to red, have also been observed but less frequently.\\

\begin{figure}[!h]
\linespread{1} 
\centering
\includegraphics[scale=0.6]{spectre_biolu.pdf}
\caption[Classification of the number of bioluminescent species depending on their wavelength emission for marine organisms.]{Classification of the number of bioluminescent species depending on their wavelength emission for marine organisms. Main bioluminescent species are emitting for wavelength between 450 and 520 nm, in blue-green emission, from \cite{widder2010}}
\label{couleurbiol}
\end{figure}

\subsection{Chemical reaction producing light}
\subsubsection*{Eukaryotic reaction}

The first experiments on bioluminescence are associated to R. Boyle (1627-1691), an English chemist, member of the "invisible college", and R. Hooke (1635-1702). These scientists observed that bioluminescence can not take place in the absence of air (the gaseous composition of the air, and in particular the presence of oxygen, was not known at this time). Subsequently, in 1885, the French biologist Raphael Dubois used, for the study of bioluminescence, beetles of the family \textit{Elateridae} and gender \textit{Pyrophorus} (Illiger, 1809), from Central America. He took two bright thoracic organs of an individual and grounded them. After some time, the light turned off. The second body was immersed in boiling water and the light turned off suddenly. When R. Dubois grounded together the two bodies, the mass became luminous. The phenomenon was then explained by the presence, in the organs, of a substance, that he called luciferin, emitting light until its complete oxidation, when the reaction is activated by a diastase enzyme (luciferase). The luciferin-luciferase reaction is an enzyme-substrate type. In the presence of oxygen, luciferin will react with the enzyme luciferase producing a molecule of oxyluciferin and light (see Figure \ref{oxyluc}; \citealp{shimomura2012}). Luciferin and luciferase are generic terms. The composition of these molecules may differ depending on the species. Within the diversity of living systems, 5 luciferin-luciferase couples have been differentiated \citep{Haddock2010,wiles2005}.\\

\begin{figure}[!h]
\linespread{1} 
\centering
\includegraphics[scale=0.25]{reaclux.pdf}
\caption[Schematic reaction inducing bioluminescence.]{Schematic reaction inducing bioluminescence. The luciferin substrate interacts with ATP and is modified into luciferyl-adenylate. The second step of the reaction is the oxydation (using molecular O$_2$) of luciferyl-adenylate into oxyluciferin by the luciferase enzymatic action. This excited molecule get back to a stable state with photon emission (bioluminescence).}
\label{oxyluc}
\end{figure}

\subsubsection*{Bacterial reaction}
In Bacteria, the genes involved in the bioluminescence reaction are organized in cluster in which the \textit{lux} ICDABFEG genes are organized in operon (called \textit{lux} operon, see Figure \ref{complexlux}).\\

\begin{figure}[!h]
\linespread{1} 
\centering
\includegraphics[scale=0.45]{lux-gene.pdf}
\caption[\textit{Lux}-gene organization for the bioluminescent \textit{Photobacterium phosphoreum} bacterial strain.]{\textit{Lux} genes organization for the bioluminescent \textit{Photobacterium phosphoreum} bacterial strains. Arrows represent the direction of transcription. Modified from \cite{dunlap2006}.}
\label{complexlux}
\end{figure} 

\textit{lux}A and \textit{lux}B genes encode the two subunits of the bacterial luciferase, an heterodimer of 77 kDa (see \citealp{stabb2005}, Figure \ref{luxbact}). The \textit{lux}C, \textit{lux}D and \textit{lux}E genes encode the reductase complex allowing the regeneration of the aldehyde, oxidized during the enzyme-substrate reaction (\citealp{ruby1976,meighen1988,meighen1991}). The \textit{lux}F gene encodes a 23-kDa flavoprotein. The sequence of this protein has about 40\% homology to the carboxy terminus of the subunit \textit{lux}B. So it seems that this gene comes from gene duplication of \textit{lux}B \citep{meighen1993,soly1988}. The \textit{lux}F gene was only identified in marine bioluminescent bacterial species from meso- to bathypelagic environments. Although its function is not yet established, the \textit{lux}F gene appears not to be essential in the bioluminescence reaction \citep{sung2004}. Finally, the \textit{lux} operon is regulated by the \textit{lux}I and \textit{lux}R genes. The \textit{lux}I gene synthesizes an autoinducer, the acetyl-homoserine lactone (AHL) which will bind to the product of the \textit{lux}R gene. This \textit{lux}I / \textit{lux}R system will then activate the operon expression \citep{miller2001}. These genes are involved in the quorum-sensing response, for a low cell density, LuxI is produced at a basal level. When the population increases and the concentration of LuxI is high enough, it will bind to LuxR to activate the operon transcription and therefore, the production of bioluminescence (see Figure \ref{qs2} and Figure \ref{quorum sensing} B  \citealp{meighen1993}).\\

Bacterial reaction (see Figure \ref{luxbact}) induces the oxidation of one molecule of flavin mononucleotide reduced (FMNH$_2$) and the reduction of a long chain of aldehyde (RCHO). This reaction will produce a molecule of flavin mononucleotide oxidized (FMN) and a long fatty-acid chain (RCOOH) with the production of light \citep{hastings1986,meighen1988}. Other proteins such as LuxC and LuxD are responsible for the regeneration of the aldehyde. LuxG protein transfer the electron from NAD(P)H to FMN in order to regenerate FMNH$_ 2$.\\

\begin{figure}[!h]
\linespread{1} 
\centering
\includegraphics[scale=0.5]{stabb2005.pdf}
\caption[Biochemistry and physiology of reaction inducing \textit{Vibrio fischeri} bioluminescence and \textit{lux} genes involved into reactions are also ]{Chemical reaction inducing \textit{Vibrio fischeri} bioluminescence. Lux AB sequentially binds FMNH$_2$, O$_2$ and an aldehyde (RCHO) that are converted into an acid, FMN and water. Energy stored as ATP is consumed in regenerating the aldehyde substrate. Then, they are released from the enzyme with the concomitant production of light. From \cite{stabb2005}. }
\label{luxbact}
\end{figure} 
\vspace{10mm}
A breakthrough in the study of this phenomenon was discovered by \cite{shimomura1962} with the purification of a protein (aequorin) from a species of Cnidarian (\textit{Aequorea}). This protein, with the addition of Ca$^{2 +}$, also causes an intramolecular reaction with bioluminescence emission and without the necessary presence of oxygen. The amount of emitted photons is proportional to the protein concentration. A similar reaction has been observed in \textit{Cheatopterus} with the addition of Fe$^{2+}$. These new types of proteins do not correspond to the luciferin and luciferase previously described. Therefore, the generic term of photoprotein has been proposed (\citealp{Shimomura1969,Prasher1992}, see Table \ref{avanc�es}).\\

\captionof{table}{Progress in research on bioluminescence, modified from \cite{shimomura2012} } 
\label{avanc�es} 
\begin{tabular}{ll}
\textbf{Date} & \textbf{Advanced}\\
\hline
1885 & Discovery of the luciferin-luciferase system\\
1947 & ATP requirement for the firefly bioluminescence\\
1954 & FMNH$_2$ requirement for bacterial bioluminescence\\
1962 & Aequorine discovery\\
1966 & Photoprotein concept\\
1974 & Identification of a long chain of aldehyde in bioluminescent bacteria\\
1975 & Coelenterazine discovery\\
1981 & Discovery of the autoinducer structure in bacterial luminescence\\
1984-1985 & Cloning firefly luciferase\\
1985-1986 & Cloning bacterial luciferase\\
1996 & Bacterial luciferase structure\\
2005 & Firefly luciferase structure\\
\end{tabular}
\vspace{5mm}

\subsection{Bioluminescence \textit{in situ} observation}
\label{introbiolu}

\cite{bradner1987} classify bioluminescent organisms into two main groups. The first group is composed by bacteria capable of producing a constant light emission without response to external stimulation. Bioluminescent bacteria emit light when the growing conditions are favorable leading to activation of quorum-sensing phenomenon and in presence of oxygen (see paragraph \ref{quorsens}). This bacterial bioluminescence is not detectable using the bioluminescence sensors developed so far.\\

The second group of bioluminescent organisms defined by \citealp{bradner1987} includes a large phylogenetic diversity of individuals capable of emitting flashes. These organisms are luminescent "naturally" as a result of biological stimulation or only after mechanical stimulation. Within this group, for most multicellular species, luminescence is controlled by nerve. On the opposite, in unicellular organisms, such as Dinoflagellates or Radiolarians, the bioluminescence is triggered by a pressure differential, resulting in a  cell-surface deformation. Mechanical-transducer processes are not fully known. However, it is likely that the mechanical stimulus activates mechano-receptors causing, thereafter, a potential action to the tonoplast, leading to the acidification of the cytoplasm (due to the proton-flux vacuole). This pH reduction directly activates the chemical reaction of bioluminescence \citep{Fritz1990}. The instrumentation for measuring marine bioluminescence, which has greatly expanded from the 60s to the present day (see Table \ref{tab}), uses this mechanical-stimulation property to detect bioluminescent organisms.\\

\begin{sidewaystable}
%\begin{table}
\center
\caption{Bathyphotometers developed for \textit{in situ} bioluminescence measurements using mechanical stimulation. Modified from \cite{herren2005}. D\string: diameter, V\string: volume. NA\string: Non Available value.} \label{tab} 
%\captionof{table}{Bathyphotom�tres d�velopp�s pour la mesure de la bioluminescence par excitation des organismes, modifi� de \citep{herren2005} } \label{tab\string: title} 
%\rotatebox{90}{
\begin{tabular}{lllllll}
\textbf{Source} & \textbf{Deployment} & \textbf{Excitation} & \textbf{flux} & \textbf{D (cm)} & \textbf{V (L)}\\
\hline
\cite{clark1965} & profiler($\rightarrow$ 2,000 m) & impeller & 0.37 L s$^{-1}$ & 2.5 & NA\\
\cite{soli1966} & shallow profiler & impeller, detector & variable & 2.54 & 0.1\\
\cite{seliger1970} & towed & impeller & 0.2 L s$^{-1}$ & 1.3 & NA\\
\cite{hall1978} & profiler ($\rightarrow$ 200 m) & turbulence, pump & NA & NA & 0.025\\
\cite{aiken1984} & profiler ($\rightarrow$ 1,000 m) & turbulence & 1-5 dm$^{3}$ s$^{-1}$ & 2.8 & 0.02\\
&&& � 5 m s$^{-1}$&\\
\cite{greenblatt1984} & profiler & turbulence & 1.1 L s$^{-1}$ & 1.6 & 0.025\\
\cite{nealson1985} & profiler ($\rightarrow$ 300 m) & 1 L s$^{-1}$ & 2.5 & 0.1\\
\cite{swift1985} & profiler & impeller & 0.25 L $s^{-1}$ & 1.4 & NA\\
\cite{buskey1992} & profiler & inlet grid & 6.3 L s$^{-1}$ & NA & 4.7\\
\cite{widder1993} & profiler & inlet grid & 16-44 L s$^{-1}$ & 12 & 11.3\\
\cite{neilson1995} & Sea mooring & inlet propeller/surge & 1-12 L s$^{-1}$ & 12.7 & 5\\
\cite{fucile1996} & profiler (2 m s$^{-1}$) & inlet grid & 15.7 L s$^{-1}$ & 10 & 2\\
\cite{geistdoerfer1999} & profiler ($\rightarrow$ 600 m) & inlet grid & 0.5 L s$^{-1}$ & 1.7 & 0.19\\
\cite{mcduffey2002} & shipboard & turbulence & 1 L s$^{-1}$ & 1.3 & 0.049\\
\cite{bivens2002} & profiler & turbulence & 1 L s$^{-1}$ & 1.5 & 0.025\\
\cite{herren2005} & multiplateform & impeller & 0.5 L s$^{-1}$ & 3.2 & 0.5\\
\end{tabular}
\end{sidewaystable}
%\end{table}

\subsubsection*{Quantification of eukaryotic bioluminescence}

The term 'non-stimulated' or 'spontaneous' bioluminescence \citep{widder1989} was replaced by 'natural', referring to a bioluminescence reaction generated by stimulus of biological organisms \citep{craig2011}. Visual sensors (cameras) are used for the automatic detection of this bioluminescence called 'natural'. The majority of observations and the most common assumptions in the literature estimate very low frequency of these bioluminescence events \citep{priede2006,widder2002}. However, the frequency of observations seems controversial and dependent on the instrumentation developed. For example, \cite{gillibrand2007} measure the frequency of this spontaneous bioluminescence at about 1 event h$^{-1}$ between 2,000 and 3,000 m depth. Other studies have estimated the natural bioluminescence to the order of 0.12 event h$^{-1}$  at 2,400 m depth. However, recently, \cite{vacquie2012} determined a significantly higher frequency at about 13-25 events min$^{-1}$, between 600 and 1,000 m depth. This frequency was unexpectedly measured by photomultipliers, together with the ARGOS system, that were used in this study during dives of elephant seals (\textit{Mirounga leonina}). From their observations, \cite{craig2011} provide a linear relationship between the depth and the number of events per minute. This relationship estimated between 1,500 and 2,750 m depth, shows, however, extreme variability maybe due to sampling in spatially heterogeneous environment. Finally, the literature remains poor in the estimation of the non-stimulated bioluminescence in benthic and pelagic environments. Therefore, its importance is relatively unknown. This lack of information seems to be directly connected to the instrumentation used to estimate such bioluminescence.\\

Marine bioluminescence is widely observed from the coast to the open sea and from the surface to the deep, where bioluminescence emission was observed up to 7,500 m \citep{priede2006}. In the bathypelagic environment (deeper than 1,000 m), bioluminescence is the only source of visible light, giving it a major role in the detection of meso- and bathypelagic organisms. On a vertical profile of the water column in the Atlantic, the observed bioluminescence decreases linearly up to 2,500 m then, staying at a stable intensity up to 4,000 m \citep{geistdoerfer2001}. According to \cite{rudyakov1989}, beyond 1,000 m depth, most of the bioluminescence would be emitted by mesoplankton (0.2 to 20 mm of diameter).\\

\subsubsection*{Quantification of potentially bioluminescent bacteria}

According to \cite{yetinson1979}, in the eastern Mediterranean, the amount of cultivable bioluminescent bacteria (Unit Forming Colony UFC) along the coast and at different seasons is estimated constant. In contrast, the bioluminescent bacterial diversity in the water column varies. Potentially bioluminescent marine bacterial species are\string: \textit{Vibrio}, \textit{Photobacterium}, and \textit{Shewanella}, all belonging to the subclass of Gammaproteobacteria. Amongst these species, \textit{Photobacterium phosphoreum} (Cohn 1878) would  be the most represented in the Mediterranean \citep{gentile2009}. It is ranked among the gram-negative Bacteria, rod-shaped, chemoorganotrophe, non-sporulating, heterotrophic and mobile with 1-3 flagella \citep{dunlap2006}. According to \cite{hastings1977,dunlap1984} and \cite{makemson1986}, the emission of bioluminescence by this bacterial strain is estimated between 10$^3$ and 10$^4$ photons s$^{-1}$ cell$^{-1}$. For all bioluminescent bacterial species, these values may vary from 1 to 10$^5$ photons s$^{-1}$ cell$^{-1}$ according to the same authors, or 10$^{2}$ to 10$^4$ photons s$^{-1}$ according to \cite{bose2008}. It depends on the strain, the environment or the \textit{lux}-gene activation.\\

\cite{ruby1980} estimated that, from 100 to 1,000 m in the Atlantic, between 0.4 and 30 UFC, belonging to \textit{P. phosphoreum}, were found in 100 mL. They described little seasonal variation and few other bacterial species associated. From 4,000 to 7,000 m depth, few bacterial cells are observable (<0.1 UFC). In the Mediterranean, in the Strait of Sicily (Ionian Sea), \textit{P. phosphoreum} represents nearly 87\% of bioluminescent bacteria while \textit{Vibrio} and \textit{Shewanella} spp. are occasionally encountered. \cite{gentile2009} estimated that in the Tyrrhenian Sea, the isolated bacteria up to 500 m depth are mainly associated to \textit{P. phosphoreum}, whereas, up to 2,750 m depth, isolated bacteria belong only to \textit{Photobacterium kishitanii} (taxonomically close to \textit{P. Phosphoreum}).\\

\textit{P. phosphoreum} is described as a bacterial strain found almost only in combination with fish, mainly in the intestines, and not in the organs devoted to luminescence \citep{herring1982}. Hastings and Marechal (unpublished, see \citealp{nealson1979}) grew on Petri dish bacteria from the gut of Sicily-deep-water fish (Messina). Amongst these, 40 to 100 \% of UFC are luminescent and represented by the single species \textit{P. phosphoreum}. An hypothesis to explain the abundance of \textit{P. phosphoreum} in the water column is that this bacterium is the major organism in the fish digestive tract. The constant rejection of feces, in the environment, will lead to an increase, in the water column, in free-living or attached to the particles bacteria.\\

\subsubsection*{Effective bacterial bioluminescence, communication "cell-to-cell"}
\label{baithyp}
The presence of potentially bioluminescent bacterial strains in the sea does not indicate a bioluminescent activity. Indeed, the bioluminescence reaction is controlled by the autoinduction of genes. Each cell produces a quantity of molecules, called autoinducers (N-Acylated homoserine lactone or AHL for \textit{Vibrio fischeri}) and cross cell membranes. The accumulation of autoinducers will allow transcription of \textit{lux} gene, the synthesis of
luciferase and, thereafter, the emission of bioluminescence (\citealp{nealson1970,eberhard1972}, see Figures \ref{quorum sensing} and \ref{qs2}). This phenomenon, called "quorum sensing" has been described historically in \textit{Vibrio fischeri}, in the 1970s. This form of communication "cell-to-cell" is widely described and assimilated to the one of a multi-cellular organism.\\

\begin{figure}[!h]
\linespread{1} 
\centering
\includegraphics[scale=0.5]{qs2.pdf}
\caption[Genetics and quorum sensing, from \cite{stabb2005}. ]{Genetics and quorum sensing, from \cite{stabb2005}. \textit{lux}I and \textit{lux}R are involved into the quorum sensing regulation. The autoinducer (AI) LuxI interacts with LuxR and leading to the stimulation of \textit{lux} genes.}
\label{qs2}
\end{figure} 

In the marine environment, these autoinducers are quickly diluted and the cell concentration necessary to the light emission (estimated between 10$^8$ and 10$^9$ cells mL$^{-1}$) is rarely achieved. However, in the light organs of some species, bioluminescent bacteria can reach a concentration of 10$^{11}$ cells mL$^{-1}$. Such concentration can also be reached by
bacterial colonization of organic particles \citep{hmelo2011}, or marine snow \citep{azam1998,alldredge1987} leading to a sufficient concentration of autoinducer and consequently the emission of bioluminescence (see Figure \ref{quorum sensing}).\\

\begin{figure}[!h]
\linespread{1} 
\centering
\includegraphics[scale=0.4]{hmelo.pdf}
\caption[Quorum sensing representation modified from \cite{hmelo2011}. ]{Quorum-sensing representation. 1\string: Bacteria are attached to sinking particles. 2\string: Population grows and the quorum-sensing signal increases. 3\string: Quorum-sensing signal reaches a threshold concentration. 4\string: Bacteria initiate a coordinated expression (bioluminescence for example). Modified from \cite{hmelo2011}.}
\label{quorum sensing}
\end{figure} 

Finally, all the informations in these studies, for both Eukaryotes and bioluminescent bacteria, remain dependent on instrumentation or methodology. Indeed, so far, only the stimulated bioluminescence (second group organization described in paragraph \ref{groupe}, \citealp{bradner1987}) was quantifiable automatically using \textit{in situ} instrumentation.  On the contrary, the quantification of bioluminescent bacteria has to be carried out through the discrete sampling of sea water, allowing only the quantification of cultivable bacteria (estimated to be about 1\% of total bacteria).\\

\setlength{\fboxsep}{5mm}
\fbox{\begin{minipage}{15cm }
\textbf{The introduction of automated detectors for non-stimulated bioluminescence, the monitoring of biological activity into the deep sea and the description of the variability over time remain to be explored.}
\end{minipage}}

\subsection{Sensitivity of bioluminescence to environmental variables}

\textbf{Turbulence and current}\\
Many studies focused on the action of mechanical stimulation on the bioluminescent planktonic organisms (mainly Dinoflagellates, considered as the most abundant phylogenetic group in bioluminescent coastal waters). \cite{cussatlegras2005} describe the effect of mechanical flow stimulation of water. \cite{latz1994} describe the effect of a laminar flow. Then \cite{rohr1998} describe effects of the turbulence created by the swimming dolphins on the luminescence emitted by these organisms. Unfortunately, few studies have quantified mechanical stimulation that is necessary to stimulate bioluminescence in deep-sea animals. However, \cite{hartline1999} measure the minimum force required to elicit bioluminescence reaction from \textit{Pleuromamma xiphias}, a mesopelagic copepod species.\\

Bioluminescence emission for Dinoflagellates has been described in \cite{cussatlegras2006}. These organisms are mechanically stimulated using accelerations or pressure from fluids. A turbulent flow is efficient to stimulate bioluminescent organisms and various bathyphotometers use this principle \citep{losee1985, widder1993} for the light measurement. Laminar flux can also stimulate organisms. For bathyphotometers, most of the time, a grid stimulates organisms by generating a uniform and isotropic turbulence \cite{cussatlegras2006}. Under mechanical stimulation, Dinoflagellates have been observed to emit light within less than 20 ms and with the light emission of a flash lasting 100 ms to about 250 ms. Copepods flash emission was measured during 50 to 150 ms.\\

It is worth noting that the bioluminescence in bacteria is not influenced by mechanical stimulation, current or turbulence \citep{bradner1987}.\\

\textbf{Temperature}\\
The joint action of temperature associated with mechanical stimulation was apprehended by \cite{han2012}. In a large volume of seawater, these authors were unable to determine the relationship between temperature, ranging from 15.8 to 19.2\degre C, and bioluminescence emitted by \textit{Noctiluca} sp.. \cite{olga2012} tested the effect of temperature on two species of Ctenophora. The author observes that the amplitude and duration of bioluminescence, chemically or mechanically stimulated, is also affected by the medium temperature with an optimum at 22 \degres C and 26 \degres C, for \textit{Beroe ovata} and \textit{Mnemiopsis leidyi}, respectively.\\

For bioluminescent bacteria, temperature has, by itself, an influence on the intensity of bioluminescence emission. However, this link is sometimes indirect since temperature will act on the growth of microorganisms and therefore impact the measure of bioluminescence. Temperatures limiting the emission of light are variable, depending on the studied bacterial strains \citep{harvey1952}. However, the chemical bioluminescence reaction is limited by the inactivation temperature of luciferase, between 30 and 35\degres C \citep{dorn2003}.\\

\textbf{pH}\\
For bacterial bioluminescence, the study of \cite{dorn2003} shows that temperature and pH of the medium can justify 98.1\% of the intensity variation for bacteria grown on a salicylate substrate. consequently, these two variables have a major importance. The pH optima are variable depending on the bacterial strains. It is worth noting that the activation of luciferase is effective only between 6.0 and 8.5 pH units, outside this range, this system will not be activated. Moreover, \cite{dorn2003} show that a small variation, of about 0.2 pH units, can impact bioluminescence.\\

\textbf{Salinity}\\
For marine bacterial strains, culture-medium salinity is generally based on NaCl concentration similar to environmental conditions \citep{lee2001,eley1972}. The intensity of bioluminescence increases at this concentration (30 g L$^{-1}$) compared to a concentration of about 10 g L$^{-1}$. Indeed, with the ionic concentration too low, the osmotic pressure can not be maintained, leading to a disruption of the cell membrane \citep{vitukhnovskaya2001,nunes2003}. Furthermore, the increase in atomic weight of halogen anions, such as KCl , KI and KBr, causes lowering of the bacterial bioluminescence \citep{gerasimova2002,kirillova2007}.\\

\textbf{Hydrostatic pressure}\\
In the marine environment, the hydrostatic pressure plays an important role with an increase of 0.1 MPa every 10 m. However, very few studies have examined the effect of hydrostatic pressure on bioluminescence. This parameter could strongly influence planktonic bioluminescent organisms in nychtemeral variations, or when water-masses movements occur.\\

Amongst the few studies, \cite{strehler1954} show that the bioluminescence, emitted by an extract of \textit{Achromobacter fischeri} or on living cells of \textit{Photobacterium phosphoreum}, is related to the combined effect of temperature and pressure. An increase in pressure, ranging from 0.1 to 55 MPa, will increase bioluminescence activity for a temperature higher than the optimum. In contrast the same variation in pressure will inhibit the bioluminescence activity at lower temperatures (Figure \ref{ueda}).\\

More recently, \cite{ueda1994} were interested in the joint effect of the temperature and pressure of the firefly luciferase (Figure \ref{ueda} B). In these experiments, a mixture of luciferase, luciferin and ATP is used. These authors have shown that an increase in pressure up to 40 MPa, increases bioluminescence at a temperature above the optimal temperature (22.5\degres C) and decreases it at temperatures below the optimum value.\\ 

\begin{figure}[!h]
\linespread{1} 
\centering
\includegraphics[scale=0.45]{HP_biolu.pdf}
\caption[Effects of temperature and pressure conjointly on A) \textit{Achromobacter fischeri} extract and \textit{Photobacterium phosphoreum} living cells, from \cite{strehler1954} and B) firefly luciferase, modified from \cite{ueda1994}.]{Effects of temperature and pressure conjointly on A) \textit{Achromobacter fischeri} extract and \textit{Photobacterium phosphoreum} living cells, from \cite{strehler1954} and B) firefly luciferase, modified from \cite{ueda1994}.}
\label{ueda}
\end{figure} 

In a completely different approach, \cite{watanabe2011} estimate the effect of hydrostatic pressure on the Dinoflagellate \textit{Pyrocystis lunula} to determine the effect of breaking waves on bioluminescent organisms. The authors apply, to simulate the effect of increasing pressure, a water jet on a wall. They demonstrate that the imposed maximum pressure leads to a maximum bioluminescence.\\

\textbf{Bioluminescent bacteria and oxygen}\\
\label{inhib}
In the chemical reaction of bioluminescence, oxygen plays a major role of electron donor (see Figure \ref{oxyluc}). The concentration of dissolved oxygen in the medium interacts with the light emission of these organisms. Oxygen would be assigned, at first, to the respiratory chain and then to the light emission. As an example, \cite{grogan1984} shows that when the oxygen system is inhibited, there was a drop in bioluminescence and that this decrease is attenuated when the respiratory system is blocked with inhibitors. For species in symbiosis, with limited possibility to increase in biomass, energy from catabolic reactions is converted into light \citep{bourgois2001}. Finally, the system, using luciferase to emit bioluminescence, is considered as an alternative electron transport. \cite{lloyd1985} observed that after a long period in anaerobiosis, the aeration of the medium will result in a bioluminescence peak for a few seconds. This peak is due to the accumulation of luciferase-FMNH$_2$ complex and is not observed when aeration medium is progressive. According to \cite{nealson1979} the amount of luciferase produced is identical between anaerobic and aerobic conditions.\\

\cite{makemson1986} estimates the oxygen consumption by the bioluminescent bacterial strains to be 50-120 nmol O$_2$ min$^{-1}$ 10$^{9}$ \textit{Vibrio fischeri} cells and 80-120 nmol O$_2$ min$^{-1}$ for 10$^{9}$ \textit{Vibrio harveyi} cells. These values range from 10 to 15 nmol O$_2$ min$^{-1}$ for 10$^{9}$ cells \citep{karl1980} and can reach 120 to 300 nmol O$_2$ min$^{-1}$ for 10$^{9}$  cells in other publications \citep{watanabe1975}.\\

The use of chemicals permit to discriminate the oxygen consumption attributed to the formation of bacterial growth and oxygen consumption dedicated to the issue of light. Amongst these compounds, the CCCP (Carbonyl Cyanide m-Chlorophenylhydrazone), the KCN or the cyanide were commonly used \citep{grogan1984,karl1980,makemson1986}. From these experiments, the proportion of oxygen consumed for bioluminescence is estimated to be about 11 to 17\% with no difference between the various bacterial species tested. This percentage falls to 0.007\% for non-symbiotic species. \cite{makemson1986} and \cite{nealson1979} estimate this percentage to be 12\% for \textit{Vibrio harveyi} and 20\% for \textit{Vibrio fischeri}. However, \cite{dunlap1984} estimates this percentage as low as 3.4 \%. \cite{hastings1975} estimate the production of 0.0001 to 0.1 photon per molecule of oxygen consumed. This production is also called 'quantum yield' \footnote{\textbf{Quantum yield:} the quantum yield \textit{in vivo} of bioluminescence is the number of photons emitted by O$_2$ molecule and used by the luciferase.}.\\
%, the calculation of which is detailed in Appendix \ref{quantumyield}.\\

\textbf{\textit{In situ} environmental variables}\\
\textit{In situ}, few studies have attempted to correlate the intensity of measured bioluminescence with environmental variables. \cite{lapota1989,cussatlegras2001} and \cite{craig2010} show a correlation between the \textit{in situ} bioluminescence and the chlorophyll concentration, using the later as a proxy for photosynthetic-Dinoflagellates biomass. Quantification of bioluminescent bacteria was correlated with certain environmental variables such as temperature, depth, salinity, nutrient limitations or sensitivity to photo-oxidation \citep{dunlap2006}. However, the characterization of the relationship between environmental variables and bioluminescence intensity, seems very different from one study to another, with a strong dependence on the depth of the studied site. Similarly, the spatial scale of observation of these correlations is also very variable.\\

\setlength{\fboxsep}{5mm}
\fbox{\begin{minipage}{15cm }
\textbf{The links between the bioluminescence activity and environmental variables still need to be studied in a controlled environment laboratory or \textit{in situ}. The establishment of \textit{in situ} bioluminescence sensors, suitable for sampling the 'natural' bioluminescence, combined with sensors, dedicated to environmental variables, will determine the dynamics of these organisms and their sensitivity to the ecosystem changes.}
\end{minipage}}

\subsection{Ecological roles in the marine environment}
\subsubsection*{Communication between Eukaryotes}
\label{quorsens}
Various light organs were identified in the bioluminescent eukaryotes\string: the scintillons (Dinoflagellates), the photocytes (Cnidaria, Ctenophora and Appendicularia) or the secretory cells (Ostracods) \citep{fogel1972,desa1968}. The photocytes can be distributed all over the body, or regrouped in light organs called photophores. In Dinoflagellates, the scintillons are cortical vesicles. These organelles migrate from the cytoplasm to the vacuole in which they discharge their luciferin and luciferase, responsible for the bioluminescence reaction. Some species are not bioluminescent by themselves but are symbiotes with bioluminescent bacteria, that they accumulate in specialized organs \citep{ruby1976,dunlap2009,rader2012}.\\

The emission of bioluminescence is an energy consuming biological reaction for organisms. Indeed, the light emission as flashes have duration time from few milliseconds to several seconds. This light is even emitted continuously for bioluminescent bacteria. A major question in the study of this phenomenon is to understand its role and the benefits obtained by bioluminescent organisms \citep{stabb2005}. The light production has been proposed as being necessary to communication, predation, protection, and detection \citep{Haddock2010,widder2010,rivers2012}, Figure \ref{roles}). Bioluminescence can alternatively be used for several or all of these functions depending on the circumstances \citep{mesinger1992,fleisher1995,roithmayr1970}. \\

\begin{figure}[!h]
\linespread{1} 
\centering
\includegraphics[width=11.8cm]{roles.pdf}
\caption{Ecological roles of bioluminescence activity, from \cite{Haddock2010}}
\label{roles}
\end{figure} 

A different role in bioluminescence is directly related to the detection of this light in marine environment. Indeed, it is surprising to see the extent of the light emitted by these organisms, in an environment where the darkness is a major feature \citep{gillibrand2007}. According to \cite{priede2006}, bioluminescence associated with 'food falls' could be observed up to ten meters away. If this sounds relatively low across the ocean, in an oligotrophic environment, it is still a significant increase in the probability of finding a source of nutrition, visible to a large number of 'scavenger' organisms \citep{warrant2004,turner2009}. A recent study also demonstrates the role of bioluminescence. Indeed, \cite{vacquie2012} found that the dives of elephant seals for food are positively correlated with recordings of the light emitted by organisms. These results have been demonstrated \textit{in situ} and up to 1,050 m depth, showing that the presence of bioluminescence is actually an indicator of potential prey for predators in the deep sea.\\

\subsubsection*{Communication in Bacteria}
\label{quorsens}
Regarding the role of bioluminescence for symbiotic bacteria with certain organisms (fish, squid ...) \citep{ruby1976}, the gain related to the light emission seems clear for each symbiont. Bacteria provide the host the necessary light for attracting preys or partners or to escape predators \citep{dunlap2009}. The bacterial host provides an environment more favorable to their growth (nutrient sources, temperature ...). In the case of non-symbiotic bioluminescent bacteria, the role of bacterial bioluminescence is less obvious.\\

The major hypothesis concerning the role of bioluminescent bacteria is directly related to the carbon cycle and called "bait hypothesis" \citep{hastings1977,robison1977,ruby1979,andrews1984}. Indeed, bioluminescent bacteria colonize particles falling through the water column and the fecal pellets. The bioluminescence emitted would lead to a higher probability of visual detection of nutrient sources and therefore to the ingestion of these particles by zooplankton. The fecal pellets are more concentrated in essential minerals needed for the growth of organisms with a C\string: N\string: P ratio of 22\string: 2.8\string: 1, showing a high concentration of phosphorous components \citep{geesey1984}.\\

This hypothesis was observed by \cite{andrews1984} and recently demonstrated by \cite{zarubin2012}. \cite{andrews1984} found that fecal pellets from copepods and particulate materials are luminescent. In \cite{zarubin2012}, the authors described and demonstrated each step of this theory. First of all, bioluminescence was tested as an attractor to the zooplankton. Then, the luminescence of zooplankton following ingestion of these bacteria was demonstrated. Finally, the consumption of the newly bioluminescent zooplankton was quantified. Once in the digestive tract of zooplankton and fish, microorganisms found favorable conditions to their development, as well as a means of effective dispersion.\\

\begin{figure}[!h]
\linespread{1} 
\centering
\includegraphics[scale=0.45]{zarubin.pdf}
\caption[A) Fecal pellets produced after feeding on small colony fragments of the bioluminescent bacterium. B) Glow of zooplankton (\textit{Artemia salina}) after contacting and ingesting small particles broken off colonies of the bioluminescent bacterium \textit{Photobacterium leiognathi}.]{A) Fecal pellets of \textit{Artemia salina} produced after swimming and feeding on small colony fragments of the bioluminescent bacterium \textit{Photobacterium leiognathi} (visible in the background) photographed in room light (Left) and in darkness (Right) B) Glow of zooplankton (\textit{Artemia salina}) after contacting and ingesting small particles broken off colonies of the bioluminescent bacterium \textit{Photobacterium leiognathi}. The photograph on the left was taken in room light, and the photograph on the right was taken in darkness using long exposure (30 s). Scale bar\string: 1 cm. From \cite{zarubin2012}.}
\label{zarubin}
\end{figure} 

A second hypothesis suggests that the role of bacterial bioluminescence is related to the response of the luciferase in the transport of oxygen. On the one hand, \citep{hastings1985,meighen1993} show that, for bacterial symbionts in organs, metabolic function of bioluminescence is an alternative to the electron transport in this low oxygen medium. On the other hand, \cite{baltar2013} indicate that an oxidative stress caused by H$_2$O$_2$ might affects prokaryotic growth and hydrolysis of specific components of the organic matter pool. To counteract this process, the bioluminescence reaction could detoxifies molecular oxygen by its reduction \cite{timmins2001}.\\
\vspace{0.5mm}
\section{Afterthought and objectives of the study}
\subsection{Multidisciplinary objectives}

The objectives of this thesis can be defined following two scientific questions\string: \\

\textbf{\textit{(i)} Bioluminescence is described as "weak" in the deep sea, compared to the ocean surface, but are there variations in light intensity over time and how to explain them?}\\

The ANTARES neutrino telescope was used as a sentinel, in the deep sea, for marine bioluminescence. This observation site, with about 850 photomultipliers, on a surface of nearly 0.1 km$^2$, allows the continuous detection of non-stimulated bioluminescence, at high frequency and in real time, since 2007. At immediate vicinity of this network, a mooring line is dedicated to the sampling of  environmental variables. Chapter 2 describes the multivariate dataset. The use of a biological indicator (bioluminescence) in the deep sea, sampled automatically and at high frequency, will be validated at the end of this Chapter. Chapter 3 provides two different statistical methods appropriated to the analysis of changes in time and frequency of environmental time series, defined as non-linear and non-stationary. The work proposed in these two chapters allows understanding for oceanographic data at the ecosystem scale. In this section, all bioluminescent organisms will be considered.\\

\textbf{\textit{(ii)} In the deep sea, what is the part of bacterial bioluminescence in the emission of light \textit{in situ}?}\\

The phenomenon of bioluminescence in Eukaryotes is widely studied. However, because of sampling difficulties and the lack of existing instrumentation, bacterial bioluminescence has only been poorly considered, so far. Indeed, the literature offers few studies characterizing or quantifying \textit{in situ} bioluminescent bacterial communities. In addition, these studies are often old with the use of isolation techniques or bacterial culture allowing only partial detection of potentially bioluminescent strains (about 1\% of the bacterial strains is defined as cultivable). In other studies, \cite{yetinson1979}, \cite{malave2010} and \cite{gentile2009} perform a one-time sampling in time and space while \cite{asplund2011} focus only on the ocean surface. Bacterial bioluminescence would be involved in the remineralization rate of organic matter in the deep sea according to the 'bait hypothesis' (see section \ref{baithyp}), existing for thirty years but only recently demonstrated. Its description and its quantification seem an interesting approach not enough developed.\\

To reach the understanding of the part of bacterial bioluminescence into light emission, two axes are proposed. First, environmental forcings, are they likely to influence the bacterial bioluminescence ? Chapter 4 develops the effects of environmental variables (hydrostatic pressure, temperature, and carbon concentration) on bacterial bioluminescence, at the population scale. This laboratory work has been performed in a controlled environment, and with the use of a model strain isolated at the study site (\textit{Photobacterium phosphoreum} ANT-2200). This is an intermediate step to answer the scientific objectives \textit{in situ}. Finally, Chapter 5 will approach the question of the bacterial part in the emission of bioluminescence, through the use of the preliminary results of a survey over the year 2011, near the site ANTARES, and at 2,000 m depth. Total prokaryotic communities are described and bioluminescent ones are quantified using molecular biology methods. The part of  bioluminescent bacteria within the bioluminescence signal detected by photomultipliers is also estimated and discussed.\\

The " Conclusions and Perspectives " part will give a critical summary of the objectives and contributions of this work and the prospects that will have to be developed after this work.\\

\subsection{A study at several scales}

The study of an ecosystem confronts with the choice of a characteristic observation scale. The observation of the community population dynamics and of the processes that are influencing them, requires to take into account variations in time and space. This will to the understanding of the observed phenomenon instantaneously, but also its spatio-temporal dynamics. The chosen scale will define variability associated with observed processes \citep{hewitt2007}. The studied ecosystem varying into three dimensions (time, vertical and horizontal spatial variations), each of these dimensions has an inherent variability. Ecological studies suggest the concept of multi-scale theory to take into account a set of connections at different scales \citep{legendre1997,anderson1998}. If the importance of the characteristic scales of studied processes in ecology is recognized, few studies take it into account in the measurements or in the result interpretation.\\

In this work, we try to keep a link between the different scales of studied processes. In Chapter 2 and 3, the spatial and temporal variations are taken into account at the ecosystemic level (annual observation and regional scale). Chapter 4 is devoted to a population level approach with the study of a bacterial strain (hourly observations and microscale). Finally, Chapter 5 provides an approach of the community level interacting in this ecosystem (daily observations and local scale).\\
